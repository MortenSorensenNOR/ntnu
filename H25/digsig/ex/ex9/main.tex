\documentclass[a4paper,11pt,norsk]{article}
\usepackage{packages}

\begin{document}

%Headingdel:---------------------------------------------
\topmargin -1.5cm
\makebox[\textwidth][s]{
    \begin{minipage}[c]{0.25\textwidth}
        \includegraphics[width=2.0cm]{Bilder/ntnu_logo.png}  
    \end{minipage}
    \begin{minipage}[c]{0.75\textwidth}
        \huge{\textbf{TFE4120  Electromagnetism}} \\
        \Large{Exercise 7  ---  Morten Sørensen, \large{\color{black!75!white}\today}}
    \end{minipage}
}
\vspace{0.75cm}
\normalsize


\section*{Problem 1.}
\begin{enumerate}
    \item 
        \begin{align*}
            h_d[n] &= \int_{-f_c}^{f_c} e^{j2\pi f n} df = \left[\frac{e^{j2\pi fn}}{j 2\pi n}\right]_{-f_c}^{f_c} = \frac{e^{j2\pi f_c n} - e^{-j2\pi f_c n}}{j2\pi n} = \frac{\sin(2\pi f_c n)}{\pi n} = 2 f_c \cdot \text{sinc}(2f_c n)
        \end{align*}

    \item Timeshifting and windowing gives
        \[
            h[n] = h_d[n - (N - 1)/2] \cdot w[n]
        \]

    \item Code for calculating coefficients in matlab
\begin{matlabcode}
fc = 0.2;
N  = 31;

function [coeffs] = fir_window_coeffs(N, fc, w)
    m = -(N-1)/2 : (N-1)/2;
    hd = 2 * fc * sinc(2 * fc * m);
    coeffs = hd(:) .* w(:);
end

w = hamming(N);
coeffs = fir_window_coeffs(N, fc, w);
[h, w] = freqz(coeffs, 1, 4096);

figure;
title("Hamming window magnitude response")
plot(w/(2*pi), abs(h));
saveas(gcf, "../Bilder/1d_hamming_win_mag.png")
\end{matlabcode}

    \item For the actual FIR coefficients 
        \begin{figure}[H]
            \centering
            \begin{subfigure}{.5\textwidth}
                \centering
                \includegraphics[width=\linewidth]{Bilder/1d_rect_win.png}
                \subcaption{Rectangular window}
            \end{subfigure}%
            \begin{subfigure}{.5\textwidth}
                \centering
                \includegraphics[width=\linewidth]{Bilder/1d_hamming_win.png}
                \subcaption{Hamming window}
            \end{subfigure}
        \end{figure}

        And the magnituderesponse
        \begin{figure}[H]
            \centering
            \begin{subfigure}{.5\textwidth}
                \centering
                \includegraphics[width=\linewidth]{Bilder/1d_rect_win_mag.png}
                \subcaption{Rectangular window}
            \end{subfigure}%
            \begin{subfigure}{.5\textwidth}
                \centering
                \includegraphics[width=\linewidth]{Bilder/1d_hamming_win_mag.png}
                \subcaption{Hamming window}
            \end{subfigure}
        \end{figure}

        It seems that the rectangular window has a longer impulse response and more ripple in both the pass and stopband, whilst
        the hamming window as a smoother, but longer transition band.

    \item Using fir1 I got
        \begin{figure}[H]
            \centering
            \begin{subfigure}{.5\textwidth}
                \centering
                \includegraphics[width=\linewidth]{Bilder/1e_fir1_rect_win_mag.png}
                \subcaption{fir1 + Rectangular window}
            \end{subfigure}%
            \begin{subfigure}{.5\textwidth}
                \centering
                \includegraphics[width=\linewidth]{Bilder/1e_fir1_hamming_win_mag.png}
                \subcaption{fir1 + Hamming window}
            \end{subfigure}
        \end{figure}

        Figures look the same (though I plotted up until 1.0 instead of 0.5 due to different scaling in fir1 vs. coeff implementation).
\end{enumerate}

\section*{Problem 2.}


\end{document}
