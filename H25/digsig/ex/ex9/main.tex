\documentclass[a4paper,11pt,norsk]{article}
\usepackage{packages}

\begin{document}

%Headingdel:---------------------------------------------
\topmargin -1.5cm
\makebox[\textwidth][s]{
    \begin{minipage}[c]{0.25\textwidth}
        \includegraphics[width=2.0cm]{Bilder/ntnu_logo.png}  
    \end{minipage}
    \begin{minipage}[c]{0.75\textwidth}
        \huge{\textbf{TFE4120  Electromagnetism}} \\
        \Large{Exercise 7  ---  Morten Sørensen, \large{\color{black!75!white}\today}}
    \end{minipage}
}
\vspace{0.75cm}
\normalsize


\section*{Problem 1.}
\begin{enumerate}
    \item 
        \begin{align*}
            h_d[n] &= \int_{-f_c}^{f_c} e^{j2\pi f n} df = \left[\frac{e^{j2\pi fn}}{j 2\pi n}\right]_{-f_c}^{f_c} = \frac{e^{j2\pi f_c n} - e^{-j2\pi f_c n}}{j2\pi n} = \frac{\sin(2\pi f_c n)}{\pi n} = 2 f_c \cdot \text{sinc}(2f_c n)
        \end{align*}

    \item Timeshifting and windowing gives
        \[
            h[n] = h_d[n - (N - 1)/2] \cdot w[n]
        \]

    \item Code for calculating coefficients in matlab
\begin{matlabcode}
fc = 0.2;
N  = 31;

function [coeffs] = fir_window_coeffs(N, fc, w)
    m = -(N-1)/2 : (N-1)/2;
    hd = 2 * fc * sinc(2 * fc * m);
    coeffs = hd(:) .* w(:);
end

w = hamming(N);
coeffs = fir_window_coeffs(N, fc, w);
[h, w] = freqz(coeffs, 1, 4096);

figure;
title("Hamming window magnitude response")
plot(w/(2*pi), abs(h));
saveas(gcf, "../Bilder/1d_hamming_win_mag.png")
\end{matlabcode}

    \item For the actual FIR coefficients 
        \begin{figure}[H]
            \centering
            \begin{subfigure}{.5\textwidth}
                \centering
                \includegraphics[width=\linewidth]{Bilder/1d_rect_win.png}
                \subcaption{Rectangular window}
            \end{subfigure}%
            \begin{subfigure}{.5\textwidth}
                \centering
                \includegraphics[width=\linewidth]{Bilder/1d_hamming_win.png}
                \subcaption{Hamming window}
            \end{subfigure}
        \end{figure}

        And the magnituderesponse
        \begin{figure}[H]
            \centering
            \begin{subfigure}{.5\textwidth}
                \centering
                \includegraphics[width=\linewidth]{Bilder/1d_rect_win_mag.png}
                \subcaption{Rectangular window}
            \end{subfigure}%
            \begin{subfigure}{.5\textwidth}
                \centering
                \includegraphics[width=\linewidth]{Bilder/1d_hamming_win_mag.png}
                \subcaption{Hamming window}
            \end{subfigure}
        \end{figure}

        It seems that the rectangular window has a longer impulse response and more ripple in both the pass and stopband, whilst
        the hamming window as a smoother, but longer transition band.

    \item Using fir1 I got
        \begin{figure}[H]
            \centering
            \begin{subfigure}{.5\textwidth}
                \centering
                \includegraphics[width=\linewidth]{Bilder/1e_fir1_rect_win_mag.png}
                \subcaption{fir1 + Rectangular window}
            \end{subfigure}%
            \begin{subfigure}{.5\textwidth}
                \centering
                \includegraphics[width=\linewidth]{Bilder/1e_fir1_hamming_win_mag.png}
                \subcaption{fir1 + Hamming window}
            \end{subfigure}
        \end{figure}

        Figures look the same (though I plotted up until 1.0 instead of 0.5 due to different scaling in fir1 vs. coeff implementation).
\end{enumerate}

\section*{Problem 2.}
\begin{enumerate}
    \item The filter has magnitude response 
        \[
            |H_a(j\omega)| = \left| \frac{1}{RC j\omega + 1} \right| = \sqrt{\frac{1}{1 + (RC)^2 \omega^2}}
        \]
        Cutoff frequency when magnitude is euqal to $1/\sqrt{2}$, so
        \[
            \frac{1}{\sqrt{2}} = \sqrt{\frac{1}{1 + (RC)^2 \omega^2}} \implies \omega = \frac{1}{RC}
        \]
    \item 
        Start with substituing in for $s$ and $z$ and get
        \begin{align*}
            s = j\Omega = \frac{2}{T} \frac{1 - z^{-1}}{1 + z^{-1}} &= \frac{2}{T} \frac{1 - e^{-j\omega}}{1 + e^{-j\omega}}
        \end{align*}
        Simplifying we get
        \[
            1 - e^{-j\omega} = e^{-j\omega/2}(e^{j\omega/2} + e^{-j\omega/2}) = e^{-j\omega/2} \cdot 2j\sin(\omega/2)
        \]
        and
        \[
            1 + e^{-j\omega} = e^{-j\omega/2}(e^{j\omega/2} + e^{-j\omega/2}) = e^{-j\omega/2} \cdot 2\cos(\omega/2)
        \]
        Which gives
        \[
            \frac{1 - e^{-j\omega}}{1 + e^{-j\omega}} = \frac{2j \sin(\omega/2)}{2\cos(\omega/2)} = j\tan(\omega/2)
        \]
        Combining with the earlier equation gives
        \[
            \Omega = \frac{2}{T} \tan(\omega/2)
        \]

    \item Resultant cutoff in discrete domain is (assuming $T = 1$)
        \[
            \Omega_c = \frac{2}{T} \cdot 0.3249 = 0.6498
        \]
        So the analog transform function is
        \[
            H_a(s) = \frac{0.6498}{s + 0.6498}
        \]
        Applying the bilinear transform
        \begin{align*}
            H(z) &= \frac{0.6498}{\frac{2(1 - z^{-1})}{1 + z^{-1}} + 0.6498} \\
                 &= \frac{0.6498 (1 + z^{-1})}{2(1 - z^{-1}) + 0.6498(1 + z^{-1})} \\
                 &= \frac{0.6498 (1 + z^{-1})}{2.6498(1 - 0.5097z^{-1})} \\
                 &= \frac{0.245 (1 + z^{-1})}{1 - 0.51z^{-1}}
        \end{align*}

        Plotting the magnitude response I got this
        \begin{figure}[H]
            \centering
            \includegraphics[width=0.85\textwidth]{Bilder/2c_mag.png}
        \end{figure}
\end{enumerate}

\section*{Problem 3.}
\begin{enumerate}
    \item Again finding at which point the magnitude is at $1/\sqrt{2}$. Start with finding the magnitude response
        \begin{align*}
            |H_a(j\Omega)| &= \left| \frac{1}{(j\Omega)^2 + \sqrt{2}j\Omega + 1} \right| \\
                           &= \left| \frac{1}{(1 - \Omega^2) + j\sqrt{2}\Omega} \right| \\
                           &= \frac{1}{\sqrt{(1 - \Omega^2)^2 + 2\Omega^2}} = \frac{1}{\sqrt{1 + \Omega^4}}
        \end{align*}
        We then get
        \[
            \frac{1}{\sqrt{1 + \Omega^4}} = \frac{1}{\sqrt{2}} \implies \Omega_c = 1
        \]
        The transfer function also follows the structure of a normalized Butterworth filter of degree 2.

    \item Get the following plot
        \begin{figure}[H]
            \centering
            \includegraphics[width=0.85\textwidth]{Bilder/3b_mag.png}
        \end{figure}

    \item Using matlab I found two poles at $-\frac{\sqrt{2}}{2} \pm j \frac{\sqrt{2}}{2}$.
\end{enumerate}

\section*{Problem 4.}
\begin{enumerate}
    \item The filter designer tool window is a bit broken on Ubuntu, so I was not able to specify filter order at all, and 
        therefore was not able to effectively use a windowing method at all (the input box for filter order did not show/was broken, so only the default value could be used, which of 
        course does not meet any of the specifications).

    \item Using the "optimal order" selection in the filter designer I got a filter order of 201.
        I got the following plot for the magnitude and phase
        \begin{figure}[H]
            \centering
            \includegraphics[width=0.85\textwidth]{Bilder/4b_mag_and_phase.png}
        \end{figure}
        and this plot for the passband 
        \begin{figure}[H]
            \centering
            \includegraphics[width=0.85\textwidth]{Bilder/4b_mag_passband.png}
        \end{figure}

    \item For the IIR filter I got a filter order of 13, with these plots
        \begin{figure}[H]
            \centering
            \includegraphics[width=0.85\textwidth]{Bilder/4c_mag_and_phase.png}
        \end{figure}
        and
        \begin{figure}[H]
            \centering
            \includegraphics[width=0.85\textwidth]{Bilder/4c_mag_passband.png}
        \end{figure}

    \item It seems that the IIR filter has a much stronger stopband performance than the FIR filter, though
        it does not have a linear phase response, which may be desired in some applications. As far as I am aware 
        in audio applications linear phase/good phase performance is not necesserily critical as the human ear is not 
        as sensitive to phase as it is to frequency and amplitude, so IIR may work quite well for this kind of application. It also has
        a much shorter filter structure, but will have a much longer impulse response (from the name infinite).

    \item After filtering the noise is gone :)
\end{enumerate}

\end{document}
