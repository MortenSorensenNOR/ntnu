\documentclass[a4paper,11pt,norsk]{article}
\usepackage{packages}

\begin{document}

%Headingdel:---------------------------------------------
\topmargin -1.5cm
\makebox[\textwidth][s]{
    \begin{minipage}[c]{0.25\textwidth}
        \includegraphics[width=2.0cm]{Bilder/ntnu_logo.png}  
    \end{minipage}
    \begin{minipage}[c]{0.75\textwidth}
        \huge{\textbf{TTT4260 ESDA}} \\
        \Large{Øving 3  ---  Morten Sørensen, \large{\color{black!75!white}\today}}
    \end{minipage}
}
\vspace{0.75cm}
\normalsize


\section*{Problem 1.}
\begin{enumerate}
    \item 
        For $a = 0.9$:
        \begin{figure}[H]
            \centering
            \includegraphics[width=0.55\textwidth]{Bilder/1a_1.png}
            \caption{Pole-Zero plot for $a = 0.9$}
        \end{figure}
        For $a = -0.9$:
        \begin{figure}[H]
            \centering
            \includegraphics[width=0.55\textwidth]{Bilder/1a_2.png}
            \caption{Pole-Zero plot for $a = -0.9$}
        \end{figure}
        For $a = 0.9$ the pole is on the positive side of the real axis, so this is a lowpass filter, and 
        oppositely for $a = -0.9$ gives a highpass filter.

    \item Plotting the frequency response in matlab we get for $a = 0.9$:
        \begin{figure}[H]
            \centering
            \includegraphics[width=0.75\textwidth]{Bilder/1b_1.png}
            \caption{Normalized frequency reponse plot for $a = 0.9$}
        \end{figure}
        and for $a = -0.9$:
        \begin{figure}[H]
            \centering
            \includegraphics[width=0.75\textwidth]{Bilder/1b_2.png}
            \caption{Normalized frequency reponse plot for $a = -0.9$}
        \end{figure}
        We see from the plots that the answer from (a) is correct.
\end{enumerate}

\section*{Problem 2.}
\begin{enumerate}
    \item
        \begin{align*}
            H^{-1}(z) = \frac{1}{H(z)} = (1 - 1/2z^{-1})(1 + 1/2z^{-1})
        \end{align*}
    \item The above transfer function can be expressed as $h^{-1}[n] = \delta[n] - 0.25\delta[n - 2]$, which means the system is 
        BIBO stable.
    \item Minimum-phase filter requires that both the filter and it's inverse are both causal and stable. I.e. all poles of $H(z)$ must lie 
        inside the unit circle, and inversely, for $H^{-1}(z)$ the zeros (poles turn into zeros when inverting the transfer function) must also 
        lie inside the unit circle. The zeros for the $H^{-1}(z)$ system are $\pm 2$, i.e. outside. I believe that the filter is therefore 
        not minimum-phase.
    \item So
        \[
            G(e^{j\omega}) = 1 - \frac{1}{4}e^{-2j\omega} = 1 - \frac{1}{4}\cos(2\omega) + j\left(\frac{1}{4}\sin(2\omega)\right)
        \]
        And the phase response would therefore be
        \[
            \phi(\omega) = \tan^{-1}\left(\frac{1/4\sin(2\omega)}{1 - 1/4\cos(2\omega)}\right)
        \]
        which I don't think can be expressed as a linear function $\alpha \omega + \beta$. Plotting the phase reponse in matlab gave this
        \begin{figure}[H]
            \centering
            \includegraphics[width=0.75\textwidth]{Bilder/2d.png}
        \end{figure}
        which does indeed not look linear. Therefore the inverse filter is not linear phase.
\end{enumerate}

\section*{Problem 3.}
\begin{enumerate}
    \item 
        Start by finding the frequency response of $A(z)$:
        \begin{align*}
            |A(e^{j\omega})| = \left| \frac{a - e^{-j\omega}}{1 - ae^{-j\omega}}\right| = \frac{\sqrt{ a^2 - a(e^{j\omega} + e^{-j\omega}) +1}}{\sqrt{ 1 - a(e^{j\omega} + e^{-j\omega}) + a^2 }} = 1
        \end{align*}
        so I guess the filter is allpass, meaning that the frequency response is entirely flat. It will however have a phase response that is non-flat.

    \item 
        The upper branch can be expressed as
        \[
            H_u(z) = \frac{1}{2}(1 + A(z))
        \]
        whilst the lower branch
        \[
            H_l(z) = \frac{K}{2}(1 - A(z))
        \]

        The following matlab code can be used to get the magnitude reponse plots
\begin{matlabcode}
a = 0.9;
K = 1;

a_coeff = [a -1];
b_coeff = [1 -a];
[H_A, w] = freqz(b_coeff, a_coeff);

H_upper = 1/2 * (1 + H_A);
H_lower = K/2 * (1 - H_A);

figure;
subplot(2,1,1);
plot(w/pi, abs(H_upper));
xlabel('Normalized Frequency (\times\pi rad/sample)');
ylabel('|H_{up}(e^{j\omega})|');
title('Upper branch magnitude (high-pass)');

subplot(2,1,2);
plot(w/pi, abs(H_lower));
xlabel('Normalized Frequency (\times\pi rad/sample)');
ylabel('|H_{low}(e^{j\omega})|');
title('Lower branch magnitude (low-pass)');
grid on;
\end{matlabcode}

        which produced this plot
        \begin{figure}[H]
            \centering
            \includegraphics[width=0.85\textwidth]{Bilder/3b.png}
        \end{figure}


    \item Listening to the different values of $\alpha$ it seems that simply more and more of the higher tones are cut out (maybe, kind if difficult to 
        compare when I can't quickly switch back and forth).
        \begin{figure}[H]
            \centering
            \includegraphics[width=0.85\textwidth]{Bilder/3c.png}
            \caption{Magnitude response for $\alpha=0.9$. The magnitude response significantly cuts out higher frequencies.}
        \end{figure}

        When altering the $K$ value it seems that for $K < 1$ the system is a high-pass filter, at $K=1$ we have a flat 
        response and at $K > 1$ it is a low-pass filter.

        Seems to me that $\alpha$ controls the "slope" of the magnitude response, i.e. how sharply it drops from the passband to the stopband, whilst $K$ determines what kind of filter 
        it is, as I said above.

\end{enumerate}

\end{document}
