\documentclass[a4paper,11pt,norsk]{article}
\usepackage{packages}

\begin{document}

%Headingdel:---------------------------------------------
\topmargin -1.5cm
\makebox[\textwidth][s]{
    \begin{minipage}[c]{0.25\textwidth}
        \includegraphics[width=2.0cm]{Bilder/ntnu_logo.png}  
    \end{minipage}
    \begin{minipage}[c]{0.75\textwidth}
        \huge{\textbf{TFE4120  Electromagnetism}} \\
        \Large{Exercise 7  ---  Morten Sørensen, \large{\color{black!75!white}\today}}
    \end{minipage}
}
\vspace{0.75cm}
\normalsize


\section*{Problem 1.}
\begin{enumerate}
    \item 
        \begin{figure}[H]
            \center
            \includegraphics[width=0.85\textwidth]{Bilder/1a.png}
        \end{figure}
    \item 
        \begin{figure}[H]
            \center
            \includegraphics[width=0.85\textwidth]{Bilder/1b.png}
        \end{figure}
    \item 
        \begin{figure}[H]
            \center
            \includegraphics[width=0.85\textwidth]{Bilder/1c.png}
        \end{figure}
    \item 
        \begin{figure}[H]
            \center
            \includegraphics[width=0.85\textwidth]{Bilder/1d.png}
        \end{figure}
    \item 
        \begin{figure}[H]
            \center
            \includegraphics[width=0.85\textwidth]{Bilder/1e.png}
        \end{figure}

    \item $x[n]$ can be expressed as
        \[
            x[n] = 5\delta[n] + 4\delta[n-1] + 3\delta[n-2] + 2\delta[n-3] + \delta[n-4]
        \]
    \item $y[n]$ expressed with unit step signal is
        \[
            y[n] = u[n - 2] - u[n - 5]
        \]
    \item The energy in the signal $x$, $E_x$, is 
        \[
            E_x = \sum_{n=-\infty}^{\infty}|x[n]|^2 = 5^2 + 4^2 + 3^2 + 2^1 + 1^2 = 55
        \]
\end{enumerate}

\section*{Problem 2.}
\begin{enumerate}
    \item The normalized frequency $f_1$ is something like $\frac{F_1}{F_s}$, but this would mean that $F_1$ is in the range $[0, \SI{6}{\kilo\hertz}]$, however,
        since $x(t)$ is a sampled analogue signal, I would assume that the normalized frequency would be such that we avoid the physical frequency being above the nyquist 
        frequency, so, I would therefore assume that $F_1 \in [0, \SI{3}{\kilo\hertz}]$, where $0 < f_1 \leq 0.5$.
    \item Python code to generate 4 seconds of $x[n]$
\begin{pythoncode}
import numpy as np
import matplotlib.pyplot as plt

time = 4  # seconds
Fs = 6000 # Hz

dt = 1 / (time * Fs)
t = np.arange(0, time * Fs)

f1 = 0.3
x  = np.cos(2 * np.pi * f1 * t)       
\end{pythoncode}
    \item As $F_s$ is increased, the sound seems to get higher in frequency.
    \item It sounds as if, when we increase the physical frequency from 3 kHz to 6 kHz, the frequency goes down. This is 
        likely because of aliasing, where $F_1/F_s = 0.75 \geq 0.5$, i.e. hearing an aliased $f_1 = 0.25$ instead of the higher frequency.
\end{enumerate}

\section*{Problem 3.}
\begin{enumerate}
    \item The system is causal (only index by $n$ or $n - k$), non-linear ($x^2$ term) and time-invariant.
    \item The system is causal (only index by $n$ or $n - k$), linear, but not time-invariant since we have a scaling by $n$ in the first term.
    \item The system is causal (same), linear and time-invariant.
    \item Not causal ($n + 3$ gets future value), linear and time-invariant.
\end{enumerate}

\section*{Problem 4.}
\begin{enumerate}
    \item For $y_1$ we have 
        \[
            h_1 = \delta[n] + 2\delta[n - 1] + \delta[n - 2]
        \]
        For $y_2$ we have that for $x[n] = \delta[n]$ that
        \[
            h_2[n] = -0.9h_2[n-1] + delta[n]
        \]
        Since they are causal, $h_2[n] = 0, n < 0$, and so $h_2[0] = 1$, and for $n \geq 1$
        \[
            h_2[n] = -0.9h_2[n-1]
        \]
        and as such, the entire system must look like
        \[
            h_2[n] = (-0.9)^n u[n]
        \]
    \item The first one is obviously FIR, as there is only three values of $n$ for which $h_1[n]$ is not 0. For $h_2[n]$
        we see that the impulse response is never actually zero, but rather decays towards 0 as $n \to \infty$, so IIR.
    \item Can find stability either through checking if the sum of $h[n]$ is less than infinity or through the Z-transform and if the ROC contains the unit circle. Will do the simple 
        sum though :)
        \[
            \sum_{n=-\infty}^{\infty} |h_1[n]| = 4 < \infty
        \]
        so system (2) is stable. 
        \[
            \sum_{n=-\infty}^{\infty} |h_2[n]| = \sum_{n=0}^{\infty} (-0.9)^n = \frac{1}{1 - 0.9} = 10 < \infty
        \]
        i.e. system (3) is also stable.
    \item System (2):
        \begin{figure}[H]
            \center
            \includegraphics[width=0.35\textwidth]{Bilder/y_1[n].png}
        \end{figure}
        System (3):
        \begin{figure}[H]
            \center
            \includegraphics[width=0.35\textwidth]{Bilder/y_2[n].png}
        \end{figure}
\end{enumerate}

\section*{Problem 5.}
\begin{enumerate}
    \item This python code: 
\begin{pythoncode}
import numpy as np
import matplotlib.pyplot as plt

t = np.arange(0, 20)
x = np.where((t>=0)&(t<=2), t+1, 0)

h1 = np.array([1, 1, 1])
h2 = 0.9 ** np.arange(0, 11)

y1 = np.convolve(x, h1, mode="full")
y2 = np.convolve(y1, h2, mode="full")

# Plot
plt.figure(figsize=(10,4))

plt.subplot(1, 2, 1)
plt.title(r"$y_1[n]$")
plt.stem(y1, basefmt=" ")
plt.xticks(np.arange(0, y1.shape[0]))

plt.subplot(1, 2, 2)
plt.title(r"$y_2[n]$")
plt.stem(y2, basefmt=" ")
plt.xticks(np.arange(0, y2.shape[0]))

plt.show()
\end{pythoncode}
        produced the following plot for $y_1[n]$ and $y_2[n]$ -- did not want to sketch by hand :)
        \begin{figure}[H]
            \center
            \includegraphics[width=\textwidth]{Bilder/5a.png}
        \end{figure}
    \item (see above)
    \item The signal length at the output is greater than that on the input because the length of the 
        sample response is appended to the end of the input signal.
    \item Swapping the order of the two systems we get the following singal plots 
        \begin{figure}[H]
            \center
            \includegraphics[width=\textwidth]{Bilder/5d.png}
        \end{figure}
        We see that the final output $y_2[n]$ remains the same, but the signal $y_1[n]$ is now obviously different.
        This is what we would expect from the behaviour of cascaded systems.
\end{enumerate}

\end{document}
