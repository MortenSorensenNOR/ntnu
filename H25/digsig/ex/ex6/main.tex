\documentclass[a4paper,11pt,norsk]{article}
\usepackage{packages}

\begin{document}

%Headingdel:---------------------------------------------
\topmargin -1.5cm
\makebox[\textwidth][s]{
    \begin{minipage}[c]{0.25\textwidth}
        \includegraphics[width=2.0cm]{Bilder/ntnu_logo.png}  
    \end{minipage}
    \begin{minipage}[c]{0.75\textwidth}
        \huge{\textbf{TFE4120  Electromagnetism}} \\
        \Large{Exercise 7  ---  Morten Sørensen, \large{\color{black!75!white}\today}}
    \end{minipage}
}
\vspace{0.75cm}
\normalsize


\section*{Problem 1.}
\begin{enumerate}
    \item Used this code to compute the DTFT of $x[n]$:
\begin{matlabcode}
N = 28;
n = 0:N-1;
x = 0.9.^(0:N-1);

f      = linspace(0, 1, 2000);
f(end) = [];
E      = exp(-1j*2*pi*(n.').*f);
X      = x * E;

% Shift frequencies so that the last half is the negative frequencies
Xs = fftshift(X);
fs = fftshift(f);
fs(fs >= 0.5) = fs(fs >= 0.5) - 1;

figure;
plot(fs, abs(Xs)), grid on
xlim([-0.5, 0.5])
\end{matlabcode}
        which gave this plot
        \begin{figure}[H]
            \centering
            \includegraphics[width=0.6\textwidth]{Bilder/1a.png}
        \end{figure}

    \item Using this code
\begin{matlabcode}
N = 28;
n = 0:N-1;
x = 0.9.^(0:N-1);

fft_points = [N/4, N/2, N, 2*N];

figure;
for i = 1:length(fft_points)
    X = fft(x, fft_points(i));
    X_shifted = fftshift(X);
    f = (-fft_points(i)/2 : fft_points(i)/2 - 1) / fft_points(i);

    subplot(2, 2, i);
    plot(f, abs(X_shifted));
    title(['|X(k)|, N_{DFT} = ', num2str(fft_points(i))]);
    xlim([-0.5 0.5]);
end
\end{matlabcode}
        I got this plot
        \begin{figure}[H]
            \centering
            \includegraphics[width=0.9\textwidth]{Bilder/1b.png}
        \end{figure}
        The end value seems to be cut off, so I did not get a frequency component at $k = 1$.

    \item The relationship is something like
        \[
            f = \frac{k}{L}
        \]
        where $L$ is the length of the DFT. For $k = 1$ above we simply have 
        \[
            f = \frac{1}{L}
        \]
        so, in order of appearance in (b) we have the following $f$'s:
        \begin{center}
            $f = 4/N_x\:\:\:\:$, $\:\:\:\:f = 2/N_x\:\:\:\:$ $\:\:\:\:f = 1 / N_x\:\:\:\:$ $\:\:\:\:f = 1/(2N_x)$
        \end{center}

    \item Plotting them together yielded this figure
        \begin{figure}[H]
            \centering
            \includegraphics[width=0.9\textwidth]{Bilder/1d.png}
        \end{figure}
        It's clear that lengths up until at least $N_x$ yields higher precision in comparison to the DTFT, though perhaps 
        there are some diminishing for $L > N_x$.

    \item For real signals the DFT is going to be symmetric around $\pi$ and $2\pi$ periodic, so computing for $f \in [0, 0.5]$ is sufficient for normalized frequency.
\end{enumerate}

\section*{Problem 2.}
\begin{enumerate}
\item
\begin{matlabcode}
Nx = 28;
n  = 0:Nx-1;
x  = 0.9.^(0:Nx-1);

Nh = 9;
h  = ones(Nh, 1);
y = conv(x, h);

figure;
stem(y); 
\end{matlabcode}

This gave
\begin{figure}[H]
    \centering
    \includegraphics[width=0.9\textwidth]{Bilder/2a.png}
\end{figure}

The length of $y[n]$ will be the sum of the length of the two convoluted signals minus 1, i.e. $N_y = N_x + N_h - 1 = 36$.

\item Doing in freqency domain I did this
\begin{matlabcode}
Nx = 28;
Nh = 9;
Ny = Nx + Nh - 1;

n  = 0:Nx-1;
x  = 0.9.^(0:Nx-1)';
X  = fft(x, Ny);

h  = ones(Nh, 1);
H  = fft(h, Ny);

disp(X);
disp(H);

Y  = X.*H;
y  = ifft(Y, Ny);

figure;
stem(y);
title("Convolution in frequency domain Y = X * H");
\end{matlabcode}

    I chose the length of the fft window to be $N_y$, because then the frequency domain signals for $X$ and $H$ are 
    the same size for the elementwise multiplication.

    Looking at the different sizes for Ny I got this plot
    \begin{figure}[H]
        \centering
        \includegraphics[width=0.9\textwidth]{Bilder/2b.png}
    \end{figure}
    We see that again for DFT lengths greater or equal to Ny we get the same signal out as the constructed signal, but 
    that for lower ones, we loose some of the higher frequency components (likely) in the output convoluted signal.
\end{enumerate}

\section*{Problem 3.}
\begin{enumerate}
    \item There will be two peaks, one at $7/40$ and one at $9/40$. My drawing skills are poor :/
    \item Using this matlab code:
\begin{matlabcode}
f1 = 7/40;
f2 = 9/40;

N = 100;
n = 0:N-1;
x = sin(2*pi*f1*n) + sin(2*pi*f2*n);

figure;
fft_sizes = [1024, 1000, 30, 10];
for i = 1:length(fft_sizes)
    X = fft(x, fft_sizes(i));
    X = fftshift(X);
    f = (-fft_sizes(i)/2 : fft_sizes(i)/2-1) / fft_sizes(i);
    
    subplot(2, 2, i);
    plot(f, abs(X));
    xlim([0.0 0.5]);
end
\end{matlabcode}
    I got this plot 
    \begin{figure}[H]
        \centering
        \includegraphics[width=0.9\textwidth]{Bilder/3b.png}
    \end{figure}

    We see that there are the two peaks we expected, but also som ripple due to the limited length of the signal $x[n]$ and 
    the fft window (I think).

    \item We now get
        \begin{figure}[H]
            \centering
            \includegraphics[width=0.9\textwidth]{Bilder/3c.png}
        \end{figure}

        I think the only difference is that the peaks are spread out a bit more in frequency, as well
        as having some more ripple for the other frequencies, i.e. more noise or something like that.
        Also, since these are powers of two, no extra padding is needed when doing the FFT.
\end{enumerate}

\section*{Problem 4.}
\begin{enumerate}
    \item The FFT is an efficient way of computing the DFT.
    \item It assumes that the number of samples N is a power of two, and repetedly divides the DFT into smaller 
        DFT's of N/2 samples, using divide and conquer, with complexity $O(N\log(N))$.
    \item DFT algorithm is $O(N^2)$.
\end{enumerate}

\end{document}
