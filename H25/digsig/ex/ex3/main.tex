\documentclass[a4paper,11pt,norsk]{article}
\usepackage{packages}

\begin{document}

%Headingdel:---------------------------------------------
\topmargin -1.5cm
\makebox[\textwidth][s]{
    \begin{minipage}[c]{0.25\textwidth}
        \includegraphics[width=2.0cm]{Bilder/ntnu_logo.png}  
    \end{minipage}
    \begin{minipage}[c]{0.75\textwidth}
        \huge{\textbf{TTT4260 ESDA}} \\
        \Large{Øving 3  ---  Morten Sørensen, \large{\color{black!75!white}\today}}
    \end{minipage}
}
\vspace{0.75cm}
\normalsize


\section*{Problem 1.}
\begin{enumerate}
    \item 
        Start with the RC filter.
        The current $I$ through the resistor and capacitor is the same, so
        \[
            v_R = RI
        \]
        and the capacitor
        \[
            I = C \frac{d y(t)}{dt}
        \]
        Therefore, since $x(t) = v_R + y(t)$ we have
        \[
            \frac{d y(t)}{dt} + \frac{1}{RC}y(t) = \frac{1}{RC}x(t)
        \]
        which can further be written as 
        \[
            y(t) = \frac{x(t)}{1 + RC \frac{d}{dt}}
        \]
        
        The transfer function can be expressed by 
        \[
            H = \frac{y(t)}{x(t)} = \frac{1}{1 + RC \frac{d}{dt}}
        \]
        or in laplace
        \[
            H(s) = \frac{1}{1 + sRC}
        \]

        For the RL filter we have that the current again through both components are the same,
        \[
            v_R = RI
        \]
        and
        \[
            y(t) = L \frac{dI}{dt}
        \]
        We therefore get that as a function of $x(t)$, 
        \[
            y(t) = \frac{L \frac{d}{dt}}{R + L \frac{d}{dt}} x(t)
        \]
        The transfer function can therefore be written as 
        \[
            H(s) = \frac{Ls}{R + Ls}
        \]
        
    \item We clearly see that as $s \to \infty$ the second filter get's approximate unit frequency reponse, i.e. is this a highpass filter, and 
        similarly for the first filter (RC) as $s \to 0$ we get unit frequency response, i.e. a lowpass filter.
                
    \item Using the inverse laplace transform we quickly get the unit impulse response.
        \[
            h_{RC}(t) = \mathcal{L}^{-1}(H_{RC}(s)) = \mathcal{L}^{-1}\left(\frac{1}{RC} \cdot \frac{1}{s + \frac{1}{RC}}\right) = \frac{1}{RC}e^{-t/RC}u(t)
        \]

        For the LC filter we have 
        \[
            h_{LC}(t) = \mathcal{L}^{-1}(H_{LC}(s)) = \mathcal{L}^{-1}\left(\frac{s}{s + a}\right)
        \]
        where $\alpha = R/L$. Since
        \[
            \frac{s}{s + \alpha} = 1 - \frac{\alpha}{s + \alpha}
        \]
        Using an inverse laplace transform table we get
        \[
            h_{RL}(t) = \delta(t) - \alpha e^{-\alpha t}u(t) = \delta(t) - \frac{R}{L}e^{-(R/L)t}u(t)
        \]
\end{enumerate}

\section*{Problem 2.}
\begin{enumerate}
    \item The ROC is $|z| > \frac{2}{3}$ and the unit impulse response is 
        \[
            h[n] = \frac{2}{3}^n u[n]
        \]
    \item The ROC is $|z| > 1$ and we have
        \[
            H(z) = \frac{2/3}{1 - z^{-1}} + \frac{1/3}{1 + \frac{1}{2}z^{-1}}
        \]
        So the unit impulse response is
        \[
            h[n] = \frac{2}{3}u[n] + \frac{1}{3}\left(-\frac{1}{2}\right)^n u[n]
        \]
    \item The filter is anti-causal, so the ROC is the interior of the circle thing, and so
        ROC is $|z| < \frac{3}{2}$. Do the same as in (b)
        \[
            H(z) = \frac{2/9}{1 - 3z^{-1}} - \frac{2/9}{1 + \frac{3}{2}z^{-1}}
        \]
        So
        \[
            h[n] = \frac{2}{9}\left[-3^n u[-n-1] + \left(-\frac{3}{2}\right]^n u[-n-1] \right) = \frac{2}{9}\left[\left(-\frac{3}{2}\right)^n - 3^n\right]u[-n-1]
        \]

    \item The first filter is causal and has the unit circle inside the ROC, so that is stable.
        The second is also causal, but does not have the unit circle inside the ROC (only just because $|z| > 1$ and not $\geq$), so it is not stable.
        The last is non-causal, but does include the unit circle inside the ROC cicle thing, so that is also stable.
\end{enumerate}


\end{document}
