\documentclass[a4paper,11pt,norsk]{article}
\usepackage{packages}

\begin{document}

%Headingdel:---------------------------------------------
\topmargin -1.5cm
\makebox[\textwidth][s]{
    \begin{minipage}[c]{0.25\textwidth}
        \includegraphics[width=2.0cm]{Bilder/ntnu_logo.png}  
    \end{minipage}
    \begin{minipage}[c]{0.75\textwidth}
        \huge{\textbf{TFE4120  Electromagnetism}} \\
        \Large{Exercise 7  ---  Morten Sørensen, \large{\color{black!75!white}\today}}
    \end{minipage}
}
\vspace{0.75cm}
\normalsize


\section*{Problem 1.}
\begin{enumerate}
    \item We have that 
        \[
            X(z) = W(z)H(z) = W(z) \frac{1}{1 + \frac{1}{2}z^{-1}}
        \]
        i.e. in time domain
        \[
            x[n] + \frac{1}{2}x[n-1] = w[n]
        \]
        which is an autoregressive process of order 1 with coefficient $a_1 = -\frac{1}{2}$.

    \item The optimal first order predictor for $x[n]$ is 
        \[
            \hat{x}[n|n-1] = \alpha_1 x[n-1]
        \]
        where 
        \[
            \alpha_1 = \frac{\gamma_{xx}(1)}{\gamma_{xx}(0)}
        \]
        Since system is AR(1) we have that
        \[
            \gamma_{xx}(1) = -a_1\gamma_{xx}(0)
        \]
        (found on wikipedia :) )
        So, we have that 
        \[
            \alpha_1 = \frac{1}{2}
        \]

        I believe that the optimal second order predictor for the AR(1) process won't make much 
        sense, as $\alpha_2$ will be zero.
\end{enumerate}

\section*{Problem 2.}
\begin{enumerate}
    \item The system is a moving average process. Can clearly bee seen as the white noise is effectively 
        put through a system function with only zeros (compared with only poles for the autoregressive process).

    \item Since $w[n]$ is white gaussian noise we have the property that 
        \[
            \gamma_{ww}(\ell) = \sigma_w^2 \delta[\ell] 
        \]
        Can write $x[n]$ as
        \[
            x[n] = b_0w[n] + b_1 w[n-1]
        \]
        with $b_0 = 1$ and $b_1 = -0.5$.
        Since we know there are only two taps for $x[n]$ with $n$ and $n-1$, we know that this 
        dirac-delta pulse only will be non-zero for $\ell \in \{-1, 0, 1\}$, so the autocorrelation of 
        $x$ becomes
        \[
            \gamma_{xx}(\ell) = E[x[n]x[n + \ell]] = \sum_{i=0}^1\sum_{j=0}^1 b_i b_j \cdot E[w[n-i]w[n+\ell-j]] = \sigma_w^2\sum_{i, j} b_i b_j \delta[\ell + i - j]
        \]

        Therefore we have that
        \[
            \gamma_{xx}(\ell) = \begin{cases}
                \sigma_w^2(b_0^2 + b_1^2)       = 1.25 & \text{ when } \ell = 0 \\
                \sigma_w^2(b_0 b_1) \:\:\:\:\:\:= -0.5 & \text{ when } \ell = \pm 1 \\
                0 & \text{ otherwise. }
            \end{cases}
        \]

        The PSD can be found with
        \[
            \Gamma_{xx}(e^{j\omega}) = \sum_{\ell=-\infty}^{\infty} \gamma_{xx}(\ell)e^{-j\omega \ell} = 1.25 - \cos(\omega)
        \]
\end{enumerate}

\section*{Problem 3.}
I did not end up recording my own vowel to do the transformation, but using the vowels given I was able to still transform 
one vowel into another. This proves that it would be possible also for my own vowels (I believe).

\begin{matlabcode}
% Parameters
Fs  = 8000;
p   = 10;
NFFT = 4096;

% Load data
matFile = 'vowels.mat';
S = load(matFile);
vowelCell = [];
for fn = fieldnames(S).'
    v = S.(fn{1});
    if iscell(v) && isequal(size(v), [1 9])
        vowelCell = v; break;
    end
end
x_dst = vowelCell{3}(:);

% Load my vowel
x_src = vowelCell{1}(:);

% LPC envelope of src and dst signals
[a_src,g_src] = lpc(x_src, p);
[a_dst,g_dst] = lpc(x_dst, p);

% Inverse-filter of source to get voice base thingy
vowel_flat = filter(a_src, 1, x_src) / sqrt(g_src);

% Synthesize new vowel with dst envelope
y = filter(1, a_dst, vowel_flat*sqrt(g_dst));

% Listen to the input's and the output
sound(x_src, Fs);
pause(0.5);
sound(x_dst, Fs);
pause(0.5);
sound(y, Fs);
\end{matlabcode}

Listening to the output I can very clearly hear that it did indeed work when using only the vowels provided. Since the method 
first normalizes the src vowel with the inverse of the LPC envelope, whatever src vowel is used, the \textbf{vowel\_flat}
signal should represent the base of the voice in either case, and so the last process of synthesizing by the destination LPC envelope
filter thing should produce a good destination vowel for both mine and the provided vowels. (Too much hassle to record one of my vowels :/).

\end{document}
