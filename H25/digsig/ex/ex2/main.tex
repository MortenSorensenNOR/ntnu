\documentclass[a4paper,11pt,norsk]{article}
\usepackage{packages}

\begin{document}

%Headingdel:---------------------------------------------
\topmargin -1.5cm
\makebox[\textwidth][s]{
    \begin{minipage}[c]{0.25\textwidth}
        \includegraphics[width=2.0cm]{Bilder/ntnu_logo.png}  
    \end{minipage}
    \begin{minipage}[c]{0.75\textwidth}
        \huge{\textbf{TFE4120  Electromagnetism}} \\
        \Large{Exercise 7  ---  Morten Sørensen, \large{\color{black!75!white}\today}}
    \end{minipage}
}
\vspace{0.75cm}
\normalsize


\section*{Problem 1.}
\begin{enumerate}
    \item We have 
        \[
            x[n] = \begin{cases}
                2 & n = 0 \\
                1 & n = \pm 1 \\
                0 & \text{otherwise}
            \end{cases}
        \]

        So
        \[
            X(\omega) = \sum_{n = -\infty}^{\infty} x[n]e^{-j\omega n} = x[-1]e^{j\omega} + x[0] + x[1]e^{-j\omega} = 2 + 2\cos(\omega)
        \]
        with the identity
        \[
            \cos(\omega) = \frac{1}{2}(e^{j\omega} + e^{-j\omega})
        \]
        Sketching it for $\omega \in [-\pi, \pi]$ we get
        \begin{figure}[H]
            \center
            \includegraphics[width=0.9\textwidth]{Bilder/1a.png}
        \end{figure}

    \item 
        \begin{align*}
            Y(\omega) &= \sum_{n=-\infty}^{\infty} y[n]e^{-j\omega n} = \sum_{n=-M}^{M} e^{-j\omega n} = e^{j\omega M} \sum_{n=0}^{2M} e^{-j\omega n} \\
                      &= \frac{e^{j\omega M} \cdot e^{-j\omega\left(M + \frac{1}{2}\right)}}{e^{-\frac{j\omega}{2}}} \frac{\left(e^{j\omega\left(M + \frac{1}{2}\right)} - e^{-j\omega\left(M + \frac{1}{2}\right)}\right)}{\left(e^{\frac{j\omega}{2}} - e^{-\frac{j\omega}{2}}\right)} \\
                      &= \frac{\sin(\omega(M + \frac{1}{2}))}{\sin\left(\frac{\omega}{2}\right)}
        \end{align*}
        Sketching it for $\omega \in [-\pi, \pi]$ and $M = 10$ we get
        \begin{figure}[H]
            \center
            \includegraphics[width=0.9\textwidth]{Bilder/1b.png}
        \end{figure}

    \item Both $x[n]$ and $y[n]$ are even, real valued functions which result in real valued signals in the frequency domain.
    \item 
        For $N = 4$ we have
        \begin{figure}[H]
            \center
            \includegraphics[width=0.9\textwidth]{Bilder/1d.png}
        \end{figure}

        We know that $c_k$ can be calculated as 
        \[
            c_k = \frac{1}{N}\sum_{n=0}^{N-1}x[n] e^{-j2\pi(k)n/N}
        \]
        since $x[n]$ is $N$-periodic. So
        \begin{align*}
            c_k &= \frac{1}{N}\left(2e^{-j\frac{2\pi k}{N} \cdot 0} + 1e^{-j\frac{2\pi k}{N}} + 1e^{-j\frac{2\pi k}{N} \cdot (N - 1)}\right) \\
                &= \frac{1}{N}\left(2 + e^{-j\frac{2\pi}{N} + e^{j\frac{2\pi}{N}}}\right) = \frac{1}{N}\left(2 + 2\cos\left(\frac{2\pi k}{N}\right)\right)
        \end{align*}

        Plotting $c_k$ for $\omega$ in $[-\pi, \pi]$ we get
        \begin{figure}[H]
            \center
            \includegraphics[width=0.9\textwidth]{Bilder/1d_2.png}
        \end{figure}

    \item Asside from the fact that I plotted one continuous and one descrete I would guess that one is simply a normalized version of the other.
\end{enumerate}

\section*{Problem 2.}
\begin{enumerate}
    \item 
        \begin{align*}
            X_1(\omega) = \sum_{n=-\infty}^{\infty} x[n+3] e^{-j\omega n} = \sum_{m=-\infty}^{\infty} x[m]e^{-j\omega(m - 3)} = e^{3j\omega} X(\omega)
        \end{align*}
    \item
        \begin{align*}
            X_2(\omega) = \sum_{n=-\infty}^{\infty} x[-n] e^{-j\omega n} = \sum_{m=-\infty}^{\infty} x[m]e^{j\omega m} = X(-\omega)
        \end{align*}
    \item
        \begin{align*}
            X_3(\omega) = \sum_{n=-\infty}^{\infty} x[3-n] e^{-j\omega n} = \sum_{m=-\infty}^{\infty} x[m]e^{j\omega (3 - m)} = e^{-3j\omega}X(-\omega)
        \end{align*}
    \item 
        \begin{align*}
            X_4(\omega) &= \sum_{n=-\infty}^{\infty} x[n] * x[n] e^{-j\omega n} \\
                        &= \sum_{n=-\infty}^{\infty} \left(\sum_{k=-\infty}^{\infty} x[k]x[n - k] \right) \cdot e^{-j\omega n} \\
                        &= \sum_{k=-\infty}^{\infty} x[k] \cdot \left(\sum_{n=-\infty}^{\infty} x[n - k] e^{-j\omega n} \right)  \\
                        &= \sum_{k=-\infty}^{\infty} x[k] \cdot e^{-j\omega k} \cdot X(\omega) = X(\omega) \cdot X(\omega)
        \end{align*}
\end{enumerate}

\section*{Problem 3.}
\begin{enumerate}
    \item The frequency response $H(e^{j\omega})$ is given as 
        \[
            H(e^{j\omega}) = \frac{Y(e^{j\omega})}{X(e^{j\omega})} = \frac{\sum_{m=0}^{M} b_m e^{-j\omega m}}{\sum_{n=0}^{N} a_n e^{-j\omega n}}
        \]
        For system (1) we have
        \begin{align*}
            H_1(e^{j\omega}) &= 1 + 2e^{-j\omega} + e^{-2j\omega} = (1 + e^{-j\omega})^2 \\
                           &= \left(e^{-j\omega/2}(e^{j\omega/2} + e^{-j\omega/2}) \right)^2\\
                           &= \left(e^{-j\omega/2} \cdot 2\cos(\omega / 2)\right)^2 = e^{-j\omega} \cdot 4\cos^2(\omega/2)
        \end{align*}
        For system (2)
        \[
            H_2(e^{j\omega}) = \frac{1}{1 + 0.9e^{-j\omega}}
        \]

    \item For system (1) we already have it in the correct form, so the amplitude response is
        \[
            \left|H_1(e^{j\omega})\right| = 4\cos^2\left(\frac{\omega}{2}\right)
        \]
        and the phase response
        \[
            \angle H_1(e^{j\omega}) = -\omega
        \]

        For system (2) we have 
        \[
            H_2(e^{j\omega}) = \frac{1}{1 + 0.9e^{-j\omega}} = \frac{1 + 0.9e^{j\omega}}{|1 + 0.9e^{-j\omega}|^2}
        \]
        The amplitude response is therefore 
        \begin{align*}
            |H(e^{j\omega})| &= \frac{\sqrt{(1 + 0.9\cos \omega)^2 + (0.9 \sin \omega)^2}}{1 + 0.9^2 + 2 \cdot 0.9\cos \omega} \\
                             &= \frac{\sqrt{1.81 + 1.8 \cos\omega}}{1.81 + 1.8\cos\omega} = \frac{1}{\sqrt{1.81 + 1.8\cos\omega}}
        \end{align*}

        The phase response is 

        \[
            \angle H(e^{j\omega}) = \tan^{-1}\left(\frac{\Im(1 + 0.9e^{j\omega})}{\Re(1 + 0.9e^{j\omega})}\right)
        \]

        The frequency response for system (1) is an even function (cosine function multiplied by exponential is even). System (2) 
        is even in the reals and odd in the imaginary domain, so I would say that system two's frequency is odd.

    \item For system (1)
        \begin{figure}[H]
            \center
            \includegraphics[width=0.9\textwidth]{Bilder/3c_1.png}
        \end{figure}
        For system (2)
        \begin{figure}[H]
            \center
            \includegraphics[width=0.9\textwidth]{Bilder/3c_2.png}
        \end{figure}

    \item System (1) is a low-pass filter and (2) is a high-pass filter. Look at the graph.

    \item Since both (1) and (2) are LTI systems the output frequency will stay the same, the amplitude will be the amplitude of the 
        signal $x[n]$ and the system magnitude response, and the phase the phase of the signal combined with the phase response of the systems.

        So for system (1) the amplitude will be $2 \cos^2(\frac{\pi}{4})$ and the phase $-\pi/4$. For system (2) we have that 
        the amplitude is $\frac{1}{2\sqrt{1.81 + 1.8\sin \frac{\pi}{2}}}$ and the phase $\tan^{-1}\left(\frac{\Im(1 + 0.9e^{j\frac{\pi}{4}})}{\Re(1 + 0.9e^{j\frac{\pi}{4}})}\right)$, which I don't want to calculate :)
\end{enumerate}

\end{document}
