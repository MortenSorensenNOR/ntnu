\documentclass[a4paper,11pt]{article}
\usepackage{packages}

\begin{document}

%Headingdel:---------------------------------------------
\topmargin -1.5cm
\makebox[\textwidth][s]{
    \begin{minipage}[c]{0.25\textwidth}
        \includegraphics[width=2.0cm]{Bilder/ntnu_logo.png}  
    \end{minipage}
    \begin{minipage}[c]{0.75\textwidth}
        \huge{\textbf{TFE4120  Electromagnetism}} \\
        \Large{Exercise 7  ---  Morten Sørensen, \large{\color{black!75!white}\today}}
    \end{minipage}
}
\vspace{0.75cm}
\normalsize


\section*{Problem 1.}
\begin{enumerate}
    \item It's clear that the norm $\|\cdot\|_c$ follows the property of positivity, since both elements in the
        internal sum of the two norms $\|\cdot\|_a$ and $\|\cdot\|_b$ follow this property. Homogeneity can be shown with
        \begin{align*}
            \|\lambda u\|_c &= (\|\lambda u\|_a + \|\lambda u\|_b)^{1/2} \\
                            &= \lambda (\|u\|_a + \|u\|_b)^{1/2} = \lambda\| u\|_c
        \end{align*}
        Lastly, we need to show that the triangle inequality holds. We have that
        \[
            \|u + v\|_c^2 = \|u + v\|_a^2 + \|u + v\|_b^2
            \leq (\|u\|_a + \|v\|_a)^2 + (\|u\|_b + \|v\|_b)^2.
        \]
        Now we see that the last expression can be viewed as the square of a $2$-norm in $\mathbb{R}^2$,
        so we apply the triangle inequality for the Euclidean norm and get
        \[
            (\|u\|_a + \|v\|_a)^2 + (\|u\|_b + \|v\|_b)^2
            = \big\|(\|u\|_a, \|u\|_b) + (\|v\|_a, \|v\|_b)\big\|_2^2
            \leq (\|u\|_c + \|v\|_c)^2.
        \]
        Taking square roots on both sides, we obtain
        \[
            \|u + v\|_c \leq \|u\|_c + \|v\|_c.
        \]
        Therefore, the last property of the norm also holds, and $\|\cdot\|_c$ indeed defines a norm on $U$.

    \item \begin{proof} Want to show that there exists an $N \in \mathbb{N}$ for $\forall \epsilon > 0$ s.t.
        \[
            \|u_n - u\|_c < \epsilon \:\:\:\: \text{ whenever } n \geq N
        \]
        iff. $(u_n)_{n \in \mathbb{N}}$ converges to $u$ wrt. both $\|\cdot\|_a$ and $\|\cdot\|_b$.

        First, let $(u_n)_{n \in \mathbb{N}}$ converge to $u$ wrt. $\|\cdot\|_a$ and $\|\cdot\|_b$. We therefore have that for some 
        $\epsilon > 0$, we can choose a sufficiently large $N$ s.t. both
        \[
            \|u_n - u\|_a < \frac{\epsilon}{\sqrt{2}}
        \]
        and
        \[
            \|u_n - u\|_b < \frac{\epsilon}{\sqrt{2}}
        \]
        with $n \geq N$. We therefore have that
        \begin{align*}
            \|u_n - u\|_c = (\|u_n - u\|_a^2 + \|u_n - u\|_b^2)^{1/2} < \left(\frac{\epsilon^2}{2} + \frac{\epsilon^2}{2}\right)^{1/2} = \epsilon
        \end{align*}
        Therefore we have that $(u_n)_{n \in \mathbb{N}}$ converges to $u$ wrt. $\|\cdot\|_c$.

        Next, assume $\|u_n-u\|_c \to 0$. Then for every $\varepsilon>0$ there is $N$ such that for $n\ge N$,
        \[
            \|u_n-u\|_c<\varepsilon.
        \]
        Since $\|w\|_a\le \|w\|_c$ and $\|w\|_b\le \|w\|_c$ for all $w\in U$, we get
        \[
            \|u_n-u\|_a<\varepsilon\quad\text{and}\quad \|u_n-u\|_b<\varepsilon,
        \]
        so $u_n\to u$ with respect to both $\|\cdot\|_a$ and $\|\cdot\|_b$.
    \end{proof}
\end{enumerate}

\section*{Problem 2.}
\begin{proof}
Theorem 5.5 states that $T$ being continuous and $T$ being bounded are equivelent statements. Therefore, let's 
first assume that $T$ is not continuous. By extension that means that $T$ is an unbounded operator, so for every $n \in \mathbb{N}$
there exists some $v_n \in U$ where
\[
    \|Tv_n\|_V > n\|v_n\|_U
\]
Can then define
\[
    u_n = \frac{v_n}{\|v_n\|_U}
\]
This means that $\|u_n\|_U = 1$, and in turn we also get that
\[
    \|Tu_n\|_V = \frac{\|Tv_n\|_V}{\|v_n\|_U} > n
\]

Secondly, lets assume that there exists a sequence $u_n$ where $\|u_n\|_U = 1$ and $\|T(u_n)\|_V \geq n$ for all $n$.
There are now exists no uniform bound $C \geq 0$ s.t.
\[
    \|Tu\|_V \leq C\|u\|_U
\]
for all $u$ because letting $u = u_n$ would force $n \leq C$. Therefore $T$ is unbounded, and consequently not continuous.
\end{proof}

\section*{Problem 3.}
We have that 
\[
    |Tu| = |\langle u, v \rangle| \leq \|u\| \|v\|
\]
where the last step is due to Cauchy-Schwarz. Now, choosing $C = \|v\|$ we get that $T$ is a bounded linear operator with
\[
    |Tu| \leq C \|u\| \:\:\:\: \text{ for all } u \in U
\]

Using Lemma 5.14 we have that
\[
    \|T\| = \sup_{u \neq 0} \frac{|Tu|}{\|u\|_U} = \sup_{\|u\|_U = 1}|Tu| = \sup_{\|u\|_U = 1} |\langle u, v \rangle| \leq \|v\|
\]
which is achieved when $u = \frac{v}{\|v\|_U}$ with $v \neq 0$. Therefore $\|T\| = \|v\|$ for $v \neq 0$, and simply $\|T\| = 0$ when 
$v = 0$.

\section*{Problem 4.}
Need to check that
\begin{equation}
    \|T\| = \sup_{u \neq 0}\frac{\|Tu\|_V}{\|u\|_U}
    \label{eq:T_norm}
\end{equation}
defines a norm on $B(U, V)$. First, it is clear that the norm achieves positivity, as if $\|T\| = 0$, then $\|Tu\| \leq 0 \cdot \|u\| = 0$ and if $T = 0$, then $\|T\| = 0$.
Second, $\|T\|$ passes homogeneity as 
\[
    \|\lambda T\| = \sup_{u \neq 0} \frac{\|\lambda Tu\|_V}{\|u\|_U} = |\lambda| \sup_{u \neq 0} \frac{\|Tu\|_V}{\|u\|_U} = |\lambda| \|T\|
\]
Lastly, let $S, T \in B(U, V)$. Then
\[
    \frac{\|(S + T)u\|_V}{\|u\|_U} = \leq \frac{\|Su\|_V}{\|u\|_U} + \frac{\|Tu\|_V}{\|u\|_U} \leq \|S\| + \|T\|
\]
Taking the supremum over this as before we get that $\|S + T\| \leq \|S\| + \|T\|$ proving the triangle inequality. Therefore (\ref{eq:T_norm}) defines a 
norm on $B(U, V)$.

\section*{Problem 5.}
\begin{enumerate}
    \item Since $T$ is a linear operator on $C^\infty([0,1])$ we must have that $T$ is continuous iff. $T$ is a bounded linear operator. Therefore we have
        \[
            |Tf| = |\int_0^1 f(x) dx| \leq \int_0^1 |f(x)| dx \leq \max_{x\in [0, 1]}|f(x)| = \|f\|_\infty
        \]
        for all $f \in C^\infty([0, 1])$, and therefore bounded, and consequently continuous wrt. the infinity norm.

    \item This operator is not continuous. Take any trigonometric function as an example, say $\sin(2\pi n x)$. There is no $C$ s.t.
        \[
            \|f'\|_\infty \leq C\|f\|_\infty
        \]
        because, as $n \to \infty$, $C \to \infty$. $T$ is therefore not a continuous wrt. the infinity norm.

    \item Not continuous, since, for the bounded decision, as we let $x \to \infty$, $C$ must also grow to infinity, so therefore not bounded $\implies$ not continuous.

    \item Not bounded, though each element in the output of the mapping is bounded by definition of existing in $\ell^\infty$, though I don't see how we can construct some uniform constant $C$ s.t. 
        the entire mapping for all $f$ is bounded, therfore, not continuous (I think).
\end{enumerate}

\section*{Problem 6.}
\begin{proof}
Using Lemma 5.14 we have that
\[
    \max_{x \in \mathbb{K}^n \ \{0\}} \frac{\|Ax\|_\infty}{\|x\|_\infty} = \max_{\|x\|_\infty = 1} \|Ax\|_\infty = \max_{1 \leq i \leq m} \sum_{j=1}^n |a_{ij}|
\]
The last step is a consequence of $x$ being chosen to be infinity normed (perhaps a bit oversimplified, but I think it works).
\end{proof}

\end{document}
