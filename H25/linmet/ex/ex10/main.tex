\documentclass[a4paper,11pt]{article}
\usepackage{packages}

\begin{document}

%Headingdel:---------------------------------------------
\topmargin -1.5cm
\makebox[\textwidth][s]{
    \begin{minipage}[c]{0.25\textwidth}
        \includegraphics[width=2.0cm]{Bilder/ntnu_logo.png}  
    \end{minipage}
    \begin{minipage}[c]{0.75\textwidth}
        \huge{\textbf{TFE4120  Electromagnetism}} \\
        \Large{Exercise 7  ---  Morten Sørensen, \large{\color{black!75!white}\today}}
    \end{minipage}
}
\vspace{0.75cm}
\normalsize


\section*{Problem 1.}
\begin{enumerate}
    \item \begin{proof} We have that
        \[
            \|T\| = \sup_{\|x\|_1 \leq 1} \|Tx\|_1 = \sum_{k=1}^\infty \left|\frac{x_k}{k}\right| = \sum_{k=1}^\infty \frac{|x_k|}{k}
        \]
        The upper bound of $\|T\|$ we have that $1/k \leq 1, \forall k$, so
        \[
            \|Tx\|_1 \leq \sum_{k=1}^\infty |x_k| = \|x\|_1
        \]
        i.e. $\|T\| \leq 1$.

        Now we can choose $x = (1, 0, 0, \dots)$, meaning that 
        \[
            \|x\|_1 = 1 \implies \|Tx\|_1 = \frac{|1|}{1} = 1
        \]
        i.e. $\|Tx\|_1 = \|x\|_1$, meaning that 
        \[
            \|T\| = 1
        \]
    \end{proof}

    \item Let $x, y \in \ell^1$. Then if $Tx = Ty$, then $x = y$ implies injectivity.
        \[
            Tx = \left(\frac{x_k}{k}\right)_{k \in \mathbb{N}} = Ty = \left(\frac{y_k}{k}\right)_{k \in \mathbb{N}}
        \]
        then we have that $x_k = y_k$ for all $k \in \mathbb{N}$.

        If $T$ is surjective, then for some $y \in \ell^1$ there exists some $x \in \ell^1$ s.t. $Tx = y$.
        In other words 
        \[
            y_k = \frac{x_k}{k} \implies x_k = ky_k
        \]
        Since $x \in \ell^1$, we must check if the norm of $x$ is bounded. We have
        \[
            \|x\|_1 = \sum_{k=1}^\infty k|y_k|
        \]
        Now, we can let $y_k = \frac{1}{k^2}$, which is in $\ell^1$, but we have 
        \[
            \|x\|_1 = \sum_{k=1}^\infty \frac{1}{k} = \infty
        \]
        i.e. $x \notin \ell^1$, so $T$ is not surjective.

    \item We have that 
        \[
            \text{range}(T) = \{ (y_k) \in \ell^1 | \exists x \in \ell^1 \text{ s.t. } y_k = \frac{x_k}{k} \}
        \]
        From 1b we have that $T$ is not surjective, so we know that there are elements $y \in \ell^1$ not in the range of 
        $T$. I.e. we know that the range is a propper subset of $\ell^1$.
\end{enumerate}

\section*{Problem 2.}
\begin{enumerate}
    \item Since $(a_n)_{n \in \mathbb{N}}, (x_n)_{n \in \mathbb{N}} \in \ell^2$, we have that both
        \[
            \sum_{n=1}^\infty |a_n|^2 < \infty
        \]
        and
        \[
            \sum_{n=1}^\infty |x_n|^2 < \infty
        \]
        want to show that $Tx \in \ell^1, \forall x \in \ell^2$, i.e.
        \[
            \sum_{n=1}^\infty |a_n x_n| < \infty
        \]
        Using the cauchy schwarz inequality we have that 
        \[
            \sum_{n=1}^\infty |a_n x_n| \leq (\sum_{n=1}^\infty |a_n|^2)^{1/2} (\sum_{n=1}^\infty |x_n|^2)^{1/2} < \infty
        \]
        and since both $a_n$ and $x_n$ converge, the sum of all $|a_n x_n|$ will also converge. Therefore
        $Tx = (a_n x_n) \in \ell^1$.

    \item 
        $T$ maps from $\ell^2$ to $\ell^2$ through element wise multiplication between two sequences in $\ell^2$, so 
        $\text{range}(T)$ should be a subspace (could prove, but I'm tired).

        Can easily see that $\text{range}(T)$ is a propper subspace of $\ell^2$ by defining a 
        sequence $y = a_n x_n$, i.e. $x_n = \frac{y_n}{a_n}$ s.t. $\|x\| = \infty$. Take for instance 
        $y_n = |a_n|$, meaning that $x_n$ is a sequence of $\pm 1$, i.e. not convergent, so $y \notin \text{range}(T)$.

        Lastly, let $y \in \ell^2$ and let $y^{(N)} = (y_1, y_2, \dots, y_N, 0, 0, \dots)$ define a vector 
        that given some $\epsilon$ we may choose $N$ s.t the two sequences are arbitrarily close. The corresponding $x^(N)$ 
        will also live in $\ell^2$ since it has finitely many non-zero elements. Therefore, we can find an $N$ given some 
        $\epsilon$ where
        \[
            \|y - Tx^{(N)}\|_2 = \|y - y^{(N)}\|_2 \leq \epsilon
        \]
        $T$ is therefore a proper dense subspace of $\ell^2$.
\end{enumerate}

\section*{Problem 3.}
\begin{enumerate}
    \item Let $u, v \in U$. Assume $Su = Sv$, then
        \[
            Su = u - Tu = Sv = v - Tv
        \]
        therefore we have that 
        \[
            (1 - T)u = (1 - T)v
        \]
        i.e. $u = v$. $S$ is injective.

    \item No clue :/

    \item Assuming (b), then 
        \[
            \|S^{-1}v\| \leq \sum_{k=0}^\infty \|T\|^k \|v\| = \frac{1}{1 - \|T\|} \|v\|
        \]
        so the thing from the task holds.
\end{enumerate}

\section*{Problem 4.}
Want to show that for all $x \in c_0$, there exists a sequence $y \in \ell^1$ s.t.
\[
    f(x) = \sum_{k=1}^\infty x_k y_k
\]
is equivelent to $f$ being a bounded linear function on $c_0$.

First, assume $y \in \ell$. We have that
\[
    f(x) = \sum x_k y_k
\]
We know that the series converges for all $x$ since $|x_k|$ is bounded and $y_k$, by virtue of being 
in $\ell^1$ is summable. Therfore $f$ is also bounded and linear with
\[
    |f(x)|leq \|y\|_| \|x\|_\infty
\]
Therefore $f \in (c_0)^*$.

Conversely, let $f \in (c_0)^*$. If we now let 
\[
    y_k = f(e^{(k)})
\]
where $e^{(k)}$ is the $k-th$ standard basis vector. Now, we have $y = (y_k) \in \ell^1$ (since $f$ is bounded and $\|e^{(k)}\|_\infty = 1$) and
\[
    f(x) = \sum x_k y_k
\]
\end{document}
