\documentclass[a4paper,11pt,norsk]{article}
\usepackage{packages}

\begin{document}

%Headingdel:---------------------------------------------
\topmargin -1.5cm
\makebox[\textwidth][s]{
    \begin{minipage}[c]{0.25\textwidth}
        \includegraphics[width=2.0cm]{Bilder/ntnu_logo.png}  
    \end{minipage}
    \begin{minipage}[c]{0.75\textwidth}
        \huge{\textbf{TFE4120  Electromagnetism}} \\
        \Large{Exercise 7  ---  Morten Sørensen, \large{\color{black!75!white}\today}}
    \end{minipage}
}
\vspace{0.75cm}
\normalsize


\section*{Problem 1.}
\begin{enumerate}
    \item 
        We know that it holds for all odd functions that $\int_{-a}^{a} p_{o}(x) dx = 0$, so the vectors 
        \begin{center}
            $(0, 1, 0, 0, 0)\:\:\:\:$ and $\:\:\:\:(0, 0, 0, 1, 0)$
        \end{center}
        must be in the basis of $U$. We also know that all other polynomials that meet the general criteria
        \begin{align*}
            \int_{-1}^{1} p(x) dx &= \int_{-1}^{1} a_0 + a_1x + a_2x^2 + a_3x^3 + a_4x^4 dx\\
                                  &= \left[a_0x + \frac{a_1}{2}x^2 + \frac{a_2}{3}x^3 + \frac{a_3}{4}x^4 + \frac{a_4}{5}x^5\right]_{-1}^{\:\:\:1}\\
                                  &= 2a_0 + \frac{2a_2}{3} + \frac{2a_4}{5} = 0
        \end{align*}
        Let $s$ and $t$ be two independent variables. We therefore have that
        \[
            a_0 = -\frac{1}{3}s - \frac{1}{5}t
        \]
        where $a_2 = s$ and $a_4 = t$. We therefore have two more basis vectors for $U$. Choose $s = 1$ and $t = 0$ for one of the vectors
        and $s = 0$ and $t = 1$ for the other. Thus we get the vectors 
        \begin{center}
            $(-\frac{1}{3}, 0, 1, 0, 0)\:\:\:\:$ and $\:\:\:\:(-\frac{1}{5}, 0, 0, 0, 1)$
        \end{center}
        These vectors must be linearly independent of one another (cannot find a $\lambda$ such that $\lambda(-\frac{1}{3}, 0, 1, 0, 0) = (-\frac{1}{5}, 0, 0, 0, 1)$), and 
        are clearly independent of the other two basis vectors from earlier. Thus we may write the basis $B$ of $U$ as
        \[
            B = \left\{(0, 1, 0, 0, 0), (0, 0, 0, 1, 0), \left(-\frac{1}{3}, 0, 1, 0, 0\right), \left(-\frac{1}{5}, 0, 0, 0, 1\right)\right\}
        \]
    \item From the basis vectors in $B$ it's not possible to make the constant functions $p(x) = c$ (this would violate the 
        definition of $U$ as the integral of a constant function from -1 to 1 is non-zero). We can therefore extend the 
        basis with the vector $(1, 0, 0, 0, 0)$, giving us a basis of $\mathcal{P}_4$ with 5 elements, which matches the vector
        space's degree.
    \item The subspace $W = \span{1}$ of $\mathcal{P}_4$, that is all constant functions, is, together with U, directly summable 
        to $\mathcal{P}_4(\mathbb{R})$. We know from (b) that the basis of $U$ and $W$ together make up a bais for $\mathcal{P}_4$, 
        i.e. all elements in $\mathcal{P}_4$ can be written as a sum of elements in $U$ and $W$. Using Lemma 2.59 from the lecture notes 
        we have that the sum of $U$ and $W$ are only direct iff. $U \cap W = \{0\}$. Suppose we have $c \in W$, then
        \[
            \int_{-1}^{1} c = 2c = 0 \implies c = 0
        \]
        Therefore, we indeed have that $U \cap W = \{0\}$, and we have that
        \[
            U \oplus W = \mathcal{P}_4(\mathbb{R})
        \]
\end{enumerate}

\section*{Problem 2.}
\begin{enumerate}
    \item \begin{proof}
        Let $B_i$ denote a valid basis for each subspace $U_i$. We then define 
        \[
            B := \bigcup_{i=1}^{N} B_i
        \]
        which we know spans the sum of the subspaces $U_1 + \dots + U_n$. Therefore 
        \[
            \dim(U_1 + \dots + U_N) \leq |B| = \sum_{i} |B_i| = \sum_{i}\dim(U_i)
        \]
        \end{proof}
    \item \begin{proof}
        From (a) we have that 
        \[
            B := \bigcup_{i=1}^{N} B_i
        \]
        Proposition 2.67 from the lecture notes say's that if the sum of the subspaces $U_1 \oplus \dots \oplus U_N$ is direct,
        then $B$, as defined above, is a valid basis of $V$. Therefore we know that the dimension of the direct sum must indeed 
        be strictly equal to $|B|$, and therefore equal to the sum of the dimension of the subspaces. So
        \[
            \dim(U_1 + \dots + U_N) = |B| = \sum_i |B_i| = \sum_i \dim(U_i)
        \]

        Next, assume that
        \[
            \dim(U_1 + \dots + U_N) = \sum_i \dim(U_i)
        \]
        From (a) we know that $B$ spans $U_1 + \dots + U_N$, and that $|B| = \sum_{i=1}^N \dim(U_i)$. Since the size of the basis $B$ is equal to the dimension of the
        space it spans, then it implies that $B$ is linearly independent, and that the union of the bases $U_i$ is linearly independent. This again implies that the
        sum $U_i + \dots + U_N$ is direct per definition 2.57.

        We have therefore prooved that 
        \[
            \dim(U_1 + \dots + U_N) = \dim(U_1) + \dots + \dim(U_N)
        \]
        iff. the sum of the subspaces is direct.
    \end{proof}
\end{enumerate}

\section*{Problem 3.}
\begin{enumerate}
    \item We have that $T(p)$ for $p \in M$ is as follows
        \begin{align*}
            T(1)   = x \cdot 0 - 0        &=  \:\:\,\,0 \cdot 1 + 0 \cdot x + \:\,\,0 \cdot x^2 + \:\,\,0 \cdot x^3 + 0 \cdot x^4 + 0 \cdot x^5 \\
            T(x)   = x \cdot 1 - 0        &=  \:\:\,\,0 \cdot 1 + 1 \cdot x + \:\,\,0 \cdot x^2 + \:\,\,0 \cdot x^3 + 0 \cdot x^4 + 0 \cdot x^5 \\
            T(x^2) = x \cdot 2x   - 2     &=         -2 \cdot 1 + 0 \cdot x + \:\,\,2 \cdot x^2 + \:\,\,0 \cdot x^3 + 0 \cdot x^4 + 0 \cdot x^5 \\
            T(x^3) = x \cdot 3x^2 - 6x    &=  \:\:\,\,0 \cdot 1 - 6 \cdot x + \:\,\,0 \cdot x^2 + \:\,\,3 \cdot x^3 + 0 \cdot x^4 + 0 \cdot x^5 \\
            T(x^4) = x \cdot 4x^3 - 12x^2 &=  \:\:\,\,0 \cdot 1 + 0 \cdot x       -12 \cdot x^2 + \:\,\,0 \cdot x^3 + 4 \cdot x^4 + 0 \cdot x^5 \\
            T(x^5) = x \cdot 5x^4 - 20x^3 &=  \:\:\,\,0 \cdot 1 + 0 \cdot x + \:\,\,0 \cdot x^2       -20 \cdot x^3 + 0 \cdot x^4 + 5 \cdot x^5
        \end{align*}

        So the transformation matrix $\mathbb{A}$ for $T$ is 
        \[
            \mathbb{A} = \begin{bmatrix}
                0 & 0 & -2 &  0 &   0 &   0 \\
                0 & 1 &  0 & -6 &   0 &   0 \\
                0 & 0 &  2 &  0 & -12 &   0 \\
                0 & 0 &  0 &  3 &   0 & -20 \\
                0 & 0 &  0 &  0 &   4 &   0 \\
                0 & 0 &  0 &  0 &   0 &   5 
            \end{bmatrix}
        \]
    \item Since $\mathbb{A}$ is an upper triangular matrix, the eigenvalues are simply the values on the diagonal. That is, the eigenvalues of the transformation $T$ are
        $\boldsymbol{\lambda} = \{0, 1, 2, 3, 4, 5\}$.
\end{enumerate}

\section*{Problem 4.}
\begin{enumerate}
    \item Did not end up having time for these :/
    \item 
\end{enumerate}

\end{document}
