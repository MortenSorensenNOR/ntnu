\documentclass[a4paper,11pt]{article}
\usepackage{packages}

\begin{document}

%Headingdel:---------------------------------------------
\topmargin -1.5cm
\makebox[\textwidth][s]{
    \begin{minipage}[c]{0.25\textwidth}
        \includegraphics[width=2.0cm]{Bilder/ntnu_logo.png}  
    \end{minipage}
    \begin{minipage}[c]{0.75\textwidth}
        \huge{\textbf{TTT4260 ESDA}} \\
        \Large{Øving 3  ---  Morten Sørensen, \large{\color{black!75!white}\today}}
    \end{minipage}
}
\vspace{0.75cm}
\normalsize


\section*{Problem 1.}
In order to find the Moore-Penrose inverse of the matrix
\[
    A := \begin{pmatrix}
        1 & 1 \\ 
        0 & 0
    \end{pmatrix}
\]
we start by first finding the SVD decomposition of $A$. Start by finding $AA^H$.

\[
    AA^H = \begin{pmatrix} 1 & 1 \\ 0 & 0 \end{pmatrix} \begin{pmatrix} 1 & 0 \\ 1 & 0\end{pmatrix} = \begin{pmatrix} 2 & 0 \\ 0 & 0\end{pmatrix}
\]
Since $AA^H$ is upper triangular it has eigenvalues $\lambda_1 = 2$ and $\lambda_2 = 0$. We very clearly see that 
eigenvectors must be 
\begin{center}
    $v_1 = \begin{pmatrix}1 \\ 0\end{pmatrix}\:\:\:\:\:\:$ and $\:\:\:\:\:\:v_2 = \begin{pmatrix}0 \\ 1\end{pmatrix}$
\end{center}

We now have singular values $\sigma_1 = \sqrt{2}$ and $\sigma_2 = 0$, i.e. we get
\[
    \Sigma = \begin{pmatrix}
        \sqrt{2} & 0 \\
        0        & 0
    \end{pmatrix}
\]
Since we have orthonormal vectors we have that 
\[
    Q = \begin{pmatrix}
        1 & 0 \\ 0 & 1
    \end{pmatrix}
\]

Next we calculate $A^H A$.
\[
    A^HA = \begin{pmatrix} 1 & 0 \\ 1 & 0 \end{pmatrix} \begin{pmatrix} 1 & 1 \\ 0 & 0\end{pmatrix} 
         = \begin{pmatrix} 1 & 1 \\ 1 & 1\end{pmatrix}
\]

This has the same eigenvalues as $AA^H$, but with eigenvectors
\begin{center}
    $v_1 = \frac{1}{\sqrt{2}}\begin{pmatrix}1 \\ 1\end{pmatrix}\:\:\:\:\:\:$ and $\:\:\:\:\:\:v_2 = \frac{1}{\sqrt{2}}\begin{pmatrix}1 \\ -1\end{pmatrix}$
\end{center}
That gives the matrix 
\[
    P = \frac{1}{\sqrt{2}}\begin{pmatrix}
        1 &  1 \\
        1 & -1
    \end{pmatrix}
\]

Now, we can find the Moore-Penrose inverse as
\[
    A^{\dagger} = P\Sigma^{-1}Q^{H} 
                = \frac{1}{\sqrt{2}} \begin{pmatrix}1&1\\1&-1\end{pmatrix} 
                                     \begin{pmatrix}\frac{1}{\sqrt{2}} & 0 \\ 0 & 0\end{pmatrix}
                                     \begin{pmatrix}1 & 0 \\ 0 & 1\end{pmatrix}
                = \frac{1}{2}\begin{pmatrix}1 & 0 \\ 1 & 0\end{pmatrix}
\]

In order to show that it is not necessarily so that $(A^{\dagger})^2 = (A^2)^{\dagger}$, we first see that
\[
    A^2 = \begin{pmatrix} 1 & 1 \\ 0 & 0 \end{pmatrix} \begin{pmatrix} 1 & 1 \\ 0 & 0 \end{pmatrix} = \begin{pmatrix} 1 & 1 \\ 0 & 0 \end{pmatrix}
\]
which means that $(A^2)^{\dagger} = A^{\dagger}$. Next, we have that 
\[
    (A^\dagger)^2 = \frac{1}{4} \begin{pmatrix} 1 & 0 \\ 1 & 0\end{pmatrix} \begin{pmatrix} 1 & 0 \\ 1 & 0\end{pmatrix} = \frac{1}{4}\begin{pmatrix} 1 & 0 \\ 1 & 0\end{pmatrix}
\]
I.e. we have that 
\[
    (A^\dagger)^2 = \frac{1}{2} (A^2)^\dagger
\]

\section*{Problem 2.}
Start by computing $AA^H$ and $A^H A$.

\begin{align*}
    AA^H &= \begin{pmatrix} 10 & 10 \\ -1 & 7  \\ 5  & 5  \\ -2 & 14 \end{pmatrix} \begin{pmatrix} 10 & -1 & 5 & -2 \\ 10 & 7  & 5 & 14 \end{pmatrix} = 
            \begin{pmatrix} 200 &  60 & 100 & 120 \\ 60 &  50 &  30 & 100 \\ 100 &  30 &  50 &  60 \\ 120 & 100 &  60 & 200 \end{pmatrix} \\
    A^HA &= \begin{pmatrix} 10 & -1 & 5 & -2 \\ 10 & 7  & 5 & 14 \end{pmatrix} \begin{pmatrix} 10 & 10 \\ -1 & 7  \\ 5  & 5  \\ -2 & 14 \end{pmatrix} = 
            \begin{pmatrix} 130 &  90 \\ 90 & 370 \end{pmatrix}
\end{align*}

Since $A^HA$ is significantly smaller, we will find the eigenvalues on it.
\[
    \begin{vmatrix}130 - \lambda & 90 \\ 90 & 370 - \lambda\end{vmatrix} = (130 - \lambda)(370 - \lambda) - 90^2
\]
which gives $\lambda_1 = 100$ and $\lambda_2 = 400$, i.e. $\sigma_1 = 10$ and $\sigma_2 = 20$.
This means that 
\[
    \Sigma = \begin{pmatrix}
        10 & 0 \\
        0  & 20
    \end{pmatrix}
\]

Next, to find $P$, we find the eigenvectors of $A^H A$. For $\lambda_1$ we get $v_1 = \frac{1}{\sqrt{10}}\left(-3, 1\right)^T$, and for 
$\lambda_2$ we get $v_2 = \frac{1}{\sqrt{10}}(1, 3)^T$. Therefore we have that
\[
    P = \frac{1}{\sqrt{10}}
    \begin{pmatrix}
        -3 & 1 \\    
        1  & 3
    \end{pmatrix}
\]

Next, using that 
\[
    Q_i = \frac{1}{\sigma_i} Av_1
\]
where $Q_i$ is the $i$'th column of $Q$, we have that
\[
    Q = \frac{1}{\sqrt{10}}
    \begin{pmatrix}
        -2 & 2 \\   
         1 & 1 \\
        -1 & 1 \\
         2 & 2
    \end{pmatrix}
\]

Therefore the SVD decomposition of $A$ is 
\[
    A = Q\Sigma P^H = 
    \frac{1}{\sqrt{10}}
    \begin{pmatrix}
        -2 & 2 \\   
         1 & 1 \\
        -1 & 1 \\
         2 & 2
    \end{pmatrix}
    \begin{pmatrix}
        10 & 0 \\
        0  & 20
    \end{pmatrix}
    \frac{1}{\sqrt{10}}
    \begin{pmatrix}
        -3 & 1 \\    
        1  & 3
    \end{pmatrix}^H
\]

The Moore-Penrose inverse $A^\dagger$ is 
\[
    A^\dagger = P\Sigma^{-1} Q^H = 
    \frac{1}{\sqrt{10}} 
    \begin{pmatrix}
        -3 & 1 \\    
        1  & 3
    \end{pmatrix}
    \begin{pmatrix}
        1/10 & 0 \\
        0    & 1/20
    \end{pmatrix}
    \frac{1}{\sqrt{10}}
    \begin{pmatrix}
        -2 & 2 \\   
         1 & 1 \\
        -1 & 1 \\
         2 & 2
    \end{pmatrix}^H
\]

Since we have from theorem 3.57 in the lecture notes that the equation
\[
    u^\dagger = T^\dagger v
\]
solves the least squares problem, we simply here have that
\[
    x^\dagger = A^\dagger b
\]
i.e.
\[
    x^\dagger = \frac{1}{\sqrt{10}} 
    \begin{pmatrix}
        -3 & 1 \\    
        1  & 3
    \end{pmatrix}
    \begin{pmatrix}
        1/10 & 0 \\
        0    & 1/20
    \end{pmatrix}
    \frac{1}{\sqrt{10}}
    \begin{pmatrix}
        -2 & 2 \\   
         1 & 1 \\
        -1 & 1 \\
         2 & 2
    \end{pmatrix}^H \cdot \begin{pmatrix} 7 \\ -5 \\ 1 \\ 1 \end{pmatrix}
\]
which gives
\[
    x^\dagger = \begin{pmatrix}
        0.6 \\ 0
    \end{pmatrix}
\]

\section*{Problem 3.}
\begin{enumerate}
    \item 
        Let $\lambda \in \mathbb{K}$ be an eigenvalue of $T$ with eigenvector $u \in U$. Then
        \[
            \lambda ||u||^2 = \langle \lambda u, u \rangle = \langle Tu, u \rangle = \langle u, T^*u \rangle = \langle u, -Tu \rangle = \langle u, -\lambda u \rangle = \overline{-\lambda}||u||^2
        \]
        Therefore, since $||u|| \neq 0$, $\lambda = \overline{-\lambda}$, which only holds true if $\lambda \in \mathbb{C}$.
    \item
        Let $\lambda_i$ be an eigenvalue of $T$ with eigenvector $v_i$
        \[
            T^*Tv_i = T^*(Tv_i) = T^*(\lambda_i v_i) = -T(\lambda_i v_i) = -\lambda^2 v_i
        \]
        Now, since the singular values of $T$, $\sigma_i$, is equal to the square root of the eigenvalues of
        $T^*T$, we know that 
        \[
            \sigma_i = \sqrt{-\lambda_i^2}
        \]
        and since we know that all $\lambda_i$ of $T$ are purely imaginary
        \[
            \sigma_i = \sqrt{-\lambda_i^2} = |\lambda_i|
        \]
        Therefore the singular values of $T$ are exactly the absolute values of the eigenvalues of $T$.
\end{enumerate}


\section*{Problem 4.}
\begin{enumerate}
    \item
        We have that the matrix $M$ represents the transformation $T$. We wish to find the matrix $M^*$ that 
        represents the adjoint transformation of $T$, $T^*$. From Lemma 3.50 in the lecture notes we have that
        the matrix $M^*$ for $T^*$, where $T : U \to V$ can be expressed as
        \[
            M^* = G_U^{-1} M^H G_V
        \]
        where $G_U$, $G_V$ are the Gram matrices for the vector spaces $U$ and $V$ respectively.
        In our case $U = \text{Mat}_{m, n}$ and $V = \text{Mat}_{\ell, n}$. Though, because of the 
        inner product we are using, each Gram matrix is equal to the identity matrix (of correct size of course), so
        $M^* = M^H$.

        Therefore, we must have that we can represent
        \[
            T^* : \text{Mat}_{\ell, n} \to \text{Mat}_{m, n}
        \]
        as
        \[
            B \mapsto T^*(B) = M^H B
        \]
    \item 
        \begin{proof}
        By definition 3.22 in the lecture notes a matrix is Hermitian if $A^H = A$, and by
        definition 3.60 in the lecture notes $T$ is self-adjoint if $T^* = T$.
        First, it is trivial to see given the result in (a) that if the matrix $M$ is Hermitian that the transformation above is 
        self-adjoint, since we found that $M^* = M^H$, and therefore, by $M$ being Hermitian, $M^* = M$.
        Secondly, if we have that $T$ is self-adjoint, then $T^* = T$, and by (a) $M^* = M$, and therefore 
        $M^* = M^H = M$, so $T$ is Hermitian as well. Therefore $T$ is self-adjoint iff. $M$ is Hermitian.
        \end{proof}
    \item \begin{proof} 
        We have by Lemma 3.41 that 
        $T$ is unitary iff. 
        \[
            G_U = A^HG_VA
        \]
        where $A$ represents the transformation $T$ in matrix form. Therefore in problem 4. we have that 
        $T$ is unitary iff.
        \[
            M^H M = \text{Id}_m
        \]
        \end{proof}
\end{enumerate}

Realized I never really showed that the Gram matrices are the identity matrices for the inner product in problem 4.
Short proof for a given matrix $\text{Mat}_{m, n}(\mathbb{K})$:
\begin{proof}
(ish) Given the inner product 
\[
    \langle A, B \rangle = \text{tr}(B^H A)
\]
Want to find the Grahm matrix
\[
    G_{\text{Mat}_{m, n}} = \begin{pmatrix}
        \langle E_{11}, E_{11} \rangle & \dots & \langle E_{1n}, E_{11} \rangle \\
        \vdots  & \ddots & \vdots \\
        \langle v_{11}, v_{m1} \rangle & \dots & \langle E_{mn}, E_{mn} \rangle
    \end{pmatrix}
\]
wrt. the basis $(E_{11}, \dots, E_{mn})$ of $\text{Mat}_{m, n}(\mathbb{K})$ (or something like that),
where $E_{ij}$ is the matrix with a 1 in place $(i, j)$ and zero elsewhere.
Now, the inner product 
\[
    \langle E_{ij}, E_{kl} \rangle = \text{tr}(E^H_{kl}E_{ij}) = \text{tr}(E_{kl}E_{ij})
\]
where the last identity is true for all $\mathbb{K}$. Since the product above is zero unless $k = i$ and $l = j$,
we have that
\[
    \langle E_{ij}, E_{kl} \rangle = \delta_{ik} \delta_{jl}
\]
Therefore the basis of the vector space $\text{Mat}_{m, n}(\mathbb{K})$, $\{E_{ij}\}$ is orthonormal, and 
so the Gram matrix is the identity matrix of the correct size $G_{\text{Mat}_{m, n}} = \text{Id}_{m, n}$.
\end{proof}

\end{document}
