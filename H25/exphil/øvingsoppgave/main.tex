\documentclass[a4paper,11pt,norsk]{article}
\usepackage{packages}

\begin{document}

%Headingdel:---------------------------------------------
\topmargin -1.5cm
\makebox[\textwidth][s]{
    \begin{minipage}[c]{0.25\textwidth}
        \includegraphics[width=2.0cm]{Bilder/ntnu_logo.png}  
    \end{minipage}
    \begin{minipage}[c]{0.75\textwidth}
        \huge{\textbf{TFE4120  Electromagnetism}} \\
        \Large{Exercise 7  ---  Morten Sørensen, \large{\color{black!75!white}\today}}
    \end{minipage}
}
\vspace{0.75cm}
\normalsize


Kraft og energi er stadig et tema for debatt i Norge og rundtom i verden, spesielt i dagens tid av
bærekraftig omstilling, med økt fokus på fornybar energiproduksjon og overgangen til en elektrifisert verden 
med kraftbehovet det medfører. I ''Bygg mer vindkraft på land – nå!'' \cite{moberg} uttrykker Jan M. Moberg 
at Norge behøver å fokusere umiddelbart på utbyggingen av vindkraft på land for å imøtekomme etterspørselen i de kommende 
tiårene.

Denne oppgaven skal ta for seg en analyse av argumentasjonen til Moberg og deretter gi en vurdering av argumentasjonen i henhold
til metodene for argumentasjonsanalyse i \textit{Tenk!} \cite{tenk}.

Vi mener at argumentasjonen i ''Bygg mer vindkraft på land – nå!'' har vært effektiv med å fremme 
behovet for økt kraftproduksjon i Norge og at den umiddelbare utbyggingen primært burde bestå av vindkraft på land, dersom 
vi forutsetter at det rent faktuelle i argumentene stemmer. 

\subsection*{Argumentasjonsanalyse}
\begin{figure}[H]
    \centering
    \includegraphics[width=\textwidth]{Bilder/argumentasjonsdiagram.png}
    \caption{Argumentasjonsdiagram for ''Bygg mer vindkraft på land – nå!'' \cite{moberg}}
    \label{fig:argumentasjonsanalyse}
\end{figure}

Diagrammet i \ref{fig:argumentasjonsanalyse} representerer argumentasjonsstrukturen i teksten ''Bygg mer vindkraft på land – nå!'' \cite{moberg}
slik vi har tolket den. Standpunktet i en teksts argumentasjon er hovedpåstanden som det argumenteres for. I denne teksten er det tydelig at 
dette er umiddelbart økt utbygging av vindkraft på land, og kommer både tydelig frem i tittelen og det kommer veldig tydelig frem i teksten at 
forfatteren mener dette er det innlysende valget for videre utbygging av kraftproduksjon. I \ref{fig:argumentasjonsanalyse} ligger standpunktet i boks 1.1.

Fire standpunktargumenter underbygger standpunktet; 2.1, 2.2 og 2.3 danner sammen et argument for hvorvidt ny kraftutbygging er nødvendig og hvorfor akkurat 
vindkarft på land er det beste valget for fremtidig utbygging, mens 2.4 henviser til tredjepartiske motargumenter for hvorfor kjernekraft og ikke 
vindkraft burde være hovedfokuset for kraftutbyggingen i Norge. At 2.1-3 danner et sammensatt argument kommer av at hvert argument avhenger til del av de andre påstandene,
og hvorvidt de er riktige. Dersom det ikke var nødvendig at ny kraftutbygging er fornybar slik argument 2.2 sier, 
hadde det svekket argumentasjonen i 2.1 for at det er vindkraft på land vi trenger, versus f.eks. kullkraft eller olje og gass.
Likledes har vi for 2.1 og 2.3. Det siste standpunktargumentet, argument 2.4 er mer adskilt fra 2.1-3. Den tar opp en ekstern mening 
om at Norge heller burde satse på kjernekraft enn vindkraft på land. Dette argumentet er markert som et motargument til tross for at det ikke er en mening 
forfatteren selv fremmer, men er her heller indikert som et motargument for å fremme argumentene rundt hvorvidt vindkraft eller kjernekraft er best for 
fremtidig utbygging, slik det er diskutert i teksten. Det er derfor ment her å tolke de indirekte argumentene under 2.4, 3.6-8 som argumentasjon imot 2.4, og dermed indirekte
for standpunktet i 1.1.

\begin{itemize}
    
    \item \textbf{Snakke om indirekte argument for 2.1}
    \item \textbf{Kort om hvordan 2.2 er et normativt argument}
    \item \textbf{Litt om 2.3 (trenger ikke særlig mye, er basically bare kort om det sammensatte 3.4-5 argumentet)}
    \item \textbf{Forklare hvordan argumentasjonen rundt 2.4 er satt opp, hvordan forfatteren argumenterer mot en tredjepart, og på den måten fremmer vindkraft
                  som et bedre alternativ enn kjernekraft}
\end{itemize}

\subsection*{Vurdering av argumentasjonen}
\begin{itemize}
    \item \textbf{Hvorvidt er argumentasjonen relevant, både for 2.1, 2.2, 2.3 og 2.4}

    \item \textbf{Så vurdere hvorvidt argumentene er riktige}
\end{itemize}

\section*{Notater til hva som må/burde være med i besvarelsen fra eksempelbesvarelsen på BB}
\begin{itemize}
    \item forklare argumentene i diagrammet tilstrekkelig til at leseren hverken er nødt til å hele tiden referere tilbake til diagrammet, men 
        heller ikke gjennta alt som står i diagrammet da dette kan fort bli kjedelig og plasskrevende
    \item konsentrere seg om å begrunne og forklare analysen
    \item vurderingsdelen bør holde seg til å si noe om grader av relevans og tiltro til riktighet, og 
        å begrunne vurderingene gjort i diagrammet

    \item viktig for sensor å forsikre seg om at man har forstått begrepene, så ved både å forklare 
        begrepene og ved å vise til anvendelsen av dem viser man dette godt. eksempel gitt er:
        \begin{quote}
            Det er altså et argument som taler direkte for standpunktet. Det som gjør 1, 2 og 3 til et sammensatt argument er at relevansen til hver av påstandene avhenger av at de andre påstandene i argumentet er riktige. For eksempel får påstanden om at globale investeringer i fornybar energi kan ha negative innvirkninger på marginaliserte samfunn, svekket relevans for standunktet hvis vi benekter at bærekraft krever at alle får dekket sine behov. Dersom det ikke er slik at bærekraft krever at alle får dekket sine behov, viser påstanden i boks 2 bare at det grønne skiftet har noen problematiske sider, men ikke at det er en konflikt mellom det grønne skiftet og bærekraft.
        \end{quote}
        fra eksempelbesvarelse analyse og vurdering av argumentasjon

    \item dersom man gjør et kontroversielt valg i argumentasjonsdiagrammet er det verdt å kommentere på dette 
        for å gjøre sensor klar over at du er oppmerksom på innvendingene mot valget som er gjort, eksempelvis at man 
        har standpunktargument med flere påstander
    \item forventes at det noen steder gis konkrete begrunnelser for analysen at et argument underbygges av 
        indirekte argument, dog ikke bare generelle gjentakelser av hva det innebærer at noe er et indirekte argument
\end{itemize}

\section*{Vurdering av argumentasjon}
\begin{itemize}
    \item anbefales å gi en samlet vurdering etter at analysen er kommentert, da dette gir et bedre overblikk over vurderingen
    \item eksempler viser en konklusjon tidlig i seksjonen, som kan bidra til å gjøre argumentasjonen lettere å følge
        \begin{quote}
            Som det vil fremgå av vurderingen under,  lykkes argumentasjonen i artikkelen “Det grønne skiftet…” etter mitt syn kun delvis.
        \end{quote}
    \item starte med å vurdere relevans i argumentasjonen
    \item vurder så riktighet for argumentene

    \item når man vurderer riktighet for argumenter som ikke er gitt noen indirekte argumenter for 
        kan det være viktig i kortere tekster å heller prioritere å vise forståelse for det teoretiske stoffet over å 
        faktaskjekke påstandene som er gitt. slå heller fast hva man du forutsetter, da får man frem at man er klar over at det 
        ikke er gitt tilstrekkelig gode grunner til at det er forsvarlig å ta stilling til påstandens riktighet
    \item \textit{induktiv generalisering} $\leftarrow$ kan være fin å få inn :=)

    \item gi en samlet vurdering av argumentasjonen, fungerer den? fungerer den ikke? det er lov å 
        bruke personlige pronomen, men da med begrunnelser av det man selv påstår
\end{itemize}

\input{Bibliografi/bib_kode}

\end{document}
