\documentclass[a4paper,11pt,norsk]{article}
\usepackage{packages}

\newcommand{\nobibparens}[1]{\unskip\cite{#1}}

\begin{document}

%Headingdel:---------------------------------------------
\topmargin -1.5cm
\makebox[\textwidth][s]{
    \begin{minipage}[c]{0.25\textwidth}
        \includegraphics[width=2.0cm]{Bilder/ntnu_logo.png}  
    \end{minipage}
    \begin{minipage}[c]{0.75\textwidth}
        \huge{\textbf{TTT4260 ESDA}} \\
        \Large{Øving 3  ---  Morten Sørensen, \large{\color{black!75!white}\today}}
    \end{minipage}
}
\vspace{0.75cm}
\normalsize


\section*{Oppgave 1.}
\subsection*{Del 1 -- Teoretisk}
\begin{enumerate}
\item Vi har at systemfunksjonen er gitt som
    \[
        H(s) = \frac{\frac{1}{LC}}{s^2 + s * \frac{R}{L} + \frac{1}{LC}} = \frac{\omega_0^2}{s^2 + \omega_0 ^2 \cdot RC \cdot s + \omega_0^2}
    \]
    Som kan videre uttrykkes ved $\zeta$ som
    \[
        H(s) = \frac{\omega_0^2}{s^2 + 2 \omega_0 \zeta s + \omega_0^2}
    \]
    der $\zeta$ blir
    \[
        \zeta = \frac{RC \omega_0}{2}
    \]
    Setter vi inn komponentverdiene får vi at 
    \[
        \zeta = \frac{\SI{200}{\ohm} \cdot \SI{10}{\micro\farad}}{2} \cdot \frac{1}{\sqrt{\SI{100}{\milli\henry} \cdot \SI{10}{\micro\farad}}} = 1
    \]

    RLC kretsen er derfor kritisk dempet, da $\zeta = 1$.

\item Nei, $\zeta$ baserer ser på systemfunksjonen, og den er forskjellig for forskjellige andreordens system.

\item Dersom vi lar $H(s) = \frac{Q(s)}{P(s)}$ kan vi finne polene til $H(s)$ ved å finne røttene til $P(s)$.
    Vi har at 
    \[
        P(s) = s^2 + 2 \omega_0 \zeta s + \omega_0^2 = (s - p_1)(s - p_2)
    \]
    Siden $\zeta = 1$ har vi at 
    \[
        p_1 = p_2 = -\omega_0
    \]

\item Innsat får vi følgende røtter
    \[
        p_1 = p_2 = -\frac{1}{\sqrt{LC}} = -1000
    \]

    Når $R$ øker vil vi få høyere relativ dempningsfaktor, da også to reelle poler, 
    og for økende $R$ vil avstanden mellom polene på den reelle aksen øke, og vi får at $p_1$ vil gå mot 0 og
    $p_2$ vil gå mot $-R$.

\item Siden vi for underkritisk demping får komplekse røtter har vi at 
    ossileringen får en frekvens på 
    \begin{equation*}
        \beta = \frac{\omega_0 \sqrt{1 - \zeta^2}}{2\pi} \tag*{\nobibparens{lundheim}}
    \end{equation*}
\end{enumerate}

\subsection*{Del 2 -- Eksperimentell}
\begin{enumerate}
\item Sprangresponsen til systemet for $R = \SI{200}{\ohm}$, $|v_{in}| = \SI{1}{\volt}$
    \begin{figure}[H]
        \centering
        \includegraphics[width=0.8\textwidth]{Bilder/kritisk_demping_sprang.png}
    \end{figure}

\item For underkritisk demping valgte jeg $R = \SI{33}{\ohm}$ og fikk følgende sprangrespons
    \begin{figure}[H]
        \centering
        \includegraphics[width=0.8\textwidth]{Bilder/under_kritisk_demping_sprang.png}
    \end{figure}

    For overkritisk demping valgte jeg $R = \SI{300}{\ohm}$ og fikk
    \begin{figure}[H]
        \centering
        \includegraphics[width=0.8\textwidth]{Bilder/over_kritisk_demping_sprang.png}
    \end{figure}

    Og siden $R = \SI{200}{\ohm}$ gir kritisk demping, 
    \begin{figure}[H]
        \centering
        \includegraphics[width=0.8\textwidth]{Bilder/kritisk_demping_sprang.png}
    \end{figure}
    
\item Forventet svingefrekvens fra 1e er
    \[
        \beta = \frac{\omega_0 \sqrt{1 - \zeta^2}}{2\pi} \approx \SI{156.9}{\hertz}
    \]

    Ser vi på frekvensspekteret til svingningen får vi følgende plot
    \begin{figure}[H]
        \centering
        \includegraphics[width=0.8\textwidth]{Bilder/frekvensspekter_svingning.png}
    \end{figure}

    og vi ser at realisert svingefrekvens er $f_0 = \SI{163}{\hertz}$, svært nærme den forventede frekvensen.

\end{enumerate}

\subsection*{Del 3 -- Frekvensrespons med poler}
\begin{enumerate}
\item Fra kretsstrukturen er det lett å se at kretsen danner at lavpassfilter, da impedansen til kondensatoren styrer styrken
    på utgangssignalet, og den impedansen starter høy for lave frekvenser, og avtar mot høye frekvenser.
\item 
\item 
\item 
\item Fra del 1 har vi at en økende $R$ fil føre til at $p_1 \to 0$ og $p_2 \to -R$, den relative dempningsfaktoren
    $\zeta$ øker, og vi får et skarpere filter, altså at frekvensresponsen vil gi kraftigere dempning etter knekkfrekvensen sammenlignet 
    ned en lavere verdi for $R$.
\item Maksimal flat repsons skjer når dempningsfaktoren $\zeta = \sqrt{2}$, og er den flatest mulige 
    frekvensresponsen i pass-området til filteret. For dette systemet oppnår vi maksimal flat repsons omtrent 
    rundt $R \approx \SI{282}{\ohm}$.
\end{enumerate}

\section*{Oppgave 2.}
\subsection*{Del 1 -- Teoretisk}
\begin{enumerate}
\item Vi har at 
    \[
        H(s) = \frac{\frac{s}{RC}}{s^2 + \frac{s}{RC} + \frac{1}{LC}} = \frac{\frac{s}{RC}}{s^2 + 2 \omega_0 \zeta s + \omega_0^2}
    \]

    der 
    \[
        \zeta = \frac{1}{2R} \sqrt{\frac{L}{C}}
    \]

    Setter vi inn verdiene får vi at
    \[
        \zeta \approx 0.158
    \]
    altså er systemet kritisk underdempet, da $\zeta < 1$.

\item Vi har at $H(s) = \frac{Q(s)}{P(s)}$ gir poler for røttene til $P(s)$. Finner røttene $p_{1,2}$ ved
    \[
        p_{1,2} = -\omega_0 \zeta \pm \omega_0 \sqrt{\zeta^2 - 1}
    \
    og ettersom at $\zeta < 1$ får vi to komplekse røtter 
    \[
        p_{1,2} = -\omega_0 \zeta \pm j\omega_0 \sqrt{1 - \zeta^2}
    \]

\item De to polene blir 
    \begin{center}
        $p_1 = -500 + 3122.5j\:\:\:\:$ og $\:\:\:\:p_1 = -500 - 3122.5j$
    \end{center}
        
\end{enumerate}

\subsection*{Del 2 -- Praktisk}
\begin{enumerate}
\item Får følgende sprangrespons
    \begin{figure}[H]
        \centering
        \includegraphics[width=0.8\textwidth]{Bilder/oppgave_2_sprang.png}
    \end{figure}

    Dette ligner veldig på den analytiske løsningen funnet forrige uke.

\item Høyere $R$ gjør at ossileringene varer "lengre", altså at det leddet som fører til ossileringene henfaller saktere,
    og korresponderer til at polene beveger seg nærmere origo på realaksen, og ossileringene blir raskere, da
    imaginær delen vokser.
\end{enumerate}

\subsection*{Del 3 -- Frekvensrespons med poler}
\begin{enumerate}
    \item RLC kretsen blir et båndpassfilter.
    \item Fikk ikke tid til mer denne uken :)
\end{enumerate}


\bibliographystyle{plain}
\bibliography{references}

\end{document}
