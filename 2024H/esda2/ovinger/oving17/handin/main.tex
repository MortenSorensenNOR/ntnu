\documentclass[a4paper,11pt,norsk]{article}
\usepackage{packages}

\newcommand{\nobibparens}[1]{\unskip\cite{#1}}

\begin{document}

%Headingdel:---------------------------------------------
\topmargin -1.5cm
\makebox[\textwidth][s]{
    \begin{minipage}[c]{0.25\textwidth}
        \includegraphics[width=2.0cm]{Bilder/ntnu_logo.png}  
    \end{minipage}
    \begin{minipage}[c]{0.75\textwidth}
        \huge{\textbf{TTT4260 ESDA}} \\
        \Large{Øving 3  ---  Morten Sørensen, \large{\color{black!75!white}\today}}
    \end{minipage}
}
\vspace{0.75cm}
\normalsize


\section*{Oppgave 1.}
\begin{enumerate}
    \item Vi ser fra kretstegningen at signalet K kunn er høyt dersom både verdien fra den første flip-floppen, Q1, er høy og verdien på det andre 
        registeret, Q2, er lavt. Siden verdien til Q2 påtar seg verdien Q1 hadde forrige klokkesykel, vil K kunn være høy på andre positive klokkeflanke,
        altså den rett etter S går høyt, for så å gå lavt neste klokkeflanke.
    \item Har ikke veldig lyst å tegne tilstandsdiagrammet, men kan ta tilstandstabellen. Vi lar de to flip-floppene Q1 og Q2 være de to bitene 
        som representerer tilstanden til maskinen (State).
        \begin{table}[H]
            \centering
            \begin{tabular}{|l|l|l|l|}
                \hline
                \textbf{State (Q1 \& Q2)} & \textbf{S} & \textbf{Neste state} & \textbf{K} \\
                \hline
                00 & 0 & 00 & 0 \\
                00 & 1 & 10 & 0 \\
                10 & 0 & 01 & 1 \\
                10 & 1 & 11 & 1 \\
                01 & 0 & 00 & 0 \\
                01 & 1 & 10 & 0 \\
                11 & 0 & 01 & 0 \\
                11 & 1 & 11 & 0 \\
                \hline
            \end{tabular}
        \end{table}

    \item Har valgt å implementere dingsen på en Digilent Basys3 FPGA baser på en Xilinx FPGA. Basys3 FPGA-en har en ekstern oscillator med frekvens
        $\SI{100}{\mega\hertz}$. Dette er noe høyt for Analog Discovery-en å plukke opp, så starter med å generere en 10 MHz klokke.
        Bruker en av FPGA-ens MMCM enheter og sender utgangen inn i en klokkebuffer for å forhindre problemer med skew (ikke sannsynlig å være et problem 
        i dette designet, men er alltid greit å være på den sikre siden).
\begin{codebox}[verilog]
module clock_10Mhz (
    input logic clk_100m,    // 100 Mhz
    output logic clk_10m,
    output logic clk_10m_5x,
    output logic clk_10m_locked
    );

    localparam MULT_MASTER = 10;
    localparam DIV_MASTER = 1;  
    localparam DIV_5X = 20;     
    localparam DIV_1X = 100;    
    localparam IN_PERIOD = 10.0;

    logic feedback;
    logic clk_10m_unbuf;
    logic clk_10m_5x_unbuf;
    logic locked;

    MMCME2_BASE #(
        .CLKFBOUT_MULT_F(MULT_MASTER),
        .CLKIN1_PERIOD(IN_PERIOD),
        .CLKOUT0_DIVIDE_F(DIV_5X), 
        .CLKOUT1_DIVIDE(DIV_1X),   
        .DIVCLK_DIVIDE(DIV_MASTER)
    ) MMCME2_BASE_inst (
        .CLKIN1(clk_100m),
        .RST(0),
        .CLKOUT0(clk_10m_5x_unbuf),
        .CLKOUT1(clk_10m_unbuf),   
        .LOCKED(locked),
        .CLKFBOUT(feedback),
        .CLKFBIN(feedback)
    );

    // Buffer output clocks
    BUFG bufg_clk(.I(clk_10m_unbuf), .O(clk_10m));
    BUFG bufg_clk_5x(.I(clk_10m_5x_unbuf), .O(clk_1m_5x));

    // Synchronize the lock signal with clk_100m
    logic locked_sync_0;
    always_ff @(posedge clk_10m) begin
        locked_sync_0 <= locked;
        clk_10m_locked <= locked_sync_0;
    end
endmodule    
\end{codebox}
    
    Lager så en debounce modul (gjennbruk fra et tidligere prosjekt)
\begin{codebox}[verilog]
module debounce (
    input logic  clk,       // 1 MHz
    input logic  i_btn,     // Input btn som vi ønsker å debounce
    output logic o_btn      // Output fra debounce
    );

    parameter c_DEBOUNCE_LIMIT = 100000; // 10ms ved 10 MHz

    reg r_State = 1'b0;
    reg [$clog2(c_DEBOUNCE_LIMIT)-1:0] r_Count = 0;
    always @(posedge clk) begin
        if (i_btn !== r_State && r_Count < c_DEBOUNCE_LIMIT) begin
            r_Count <= r_Count + 1; // Counter
        end else if (r_Count == c_DEBOUNCE_LIMIT) begin
            r_Count <= 0;
            r_State <= i_btn;
        end else begin
            r_Count <= 0;
        end
    end
    assign o_btn = r_State;
endmodule
\end{codebox}
        modulen over har en teller som teller når knappen går fra lav til høyt (men kunn når signalet faktisk er høyt), og når den
        når en forhåndsdefinert verdi settes utgangen høyt også. Dette gjør så at raske endringer p grunn av bounce ikke kommer på utgangen. 10ms 
        er en god approksimert verdi for noe som er trygt å anta er tiden det tar før kanppen blir stabil.

        Lager så en modul som etterligner ønsket oppførsel med at vi kunn detekterer positiv flanke på knapp-signalet
\begin{codebox}[verilog]
module keypad (
    input logic clk,        // 1 Mhz
    input logic [3:0] btn,  // Input knappe trykk
    output logic [3:0] keys // Output knappe trykk puls
    );

    logic [3:0] Q1 = '0;
    logic [3:0] Q2 = '0;
    always_ff @(posedge clk) begin
        Q1 <= btn;
        Q2 <= Q1;
        keys[0] <= Q1[0] & ~Q2[0];
        keys[1] <= Q1[1] & ~Q2[1];
        keys[2] <= Q1[2] & ~Q2[2];
        keys[3] <= Q1[3] & ~Q2[3];
    end
endmodule
\end{codebox}

        Setter vi disse modulene sammen og ser på utgangen med Analog Discovery-en ser vi at vi får en puls på utgangen når vi trykker ned knappen 
        på eksakt $\SI{100}{\nano\second}$, som er det vi forventet. 
        \begin{figure}[H]
            \centering
            \includegraphics[width=\textwidth]{Bilder/key_press.png}
            \caption{Plot som viser K verdien for den tredje knappen (inngang D). K får en puls på eksakt 100ns som er lik klokkeperioden til FPGA-en.}
        \end{figure}
        Det samme ble gjort for de andre kanppene.
\end{enumerate}

\section*{Oppgave 2.}
\begin{enumerate}
    \item Siden vi har 5 states må vi ha et 3-bit register for å holde staten.
    \item Vi får følgende state transistion
        \begin{table}[H]
            \centering
            \begin{tabular}{|l|l|}
                \hline 
                \hline
                \textbf{Variabel} & \textbf{Logisk uttrykk} \\
                \hline
                F_0 & $B + C + D$ \\
                R_1 & $A\bar{B}\bar{C}\bar{D}$ \\
                F_1 & $\bar{B}(A + C + D)$ \\
                R_2 & $B\bar{A}\bar{C}\bar{D}$ \\
                F_2 & $\bar{B}(A + C + D)$ \\
                R_3 & $B\bar{A}\bar{C}\bar{D}$ \\
                F_3 & $\bar{A}(B + C + D)$ \\
                R_4 & $A\bar{B}\bar{C}\bar{D}$ \\
                F_4 & $A + B + C + D$ \\
                \hline
            \end{tabular}
        \end{table}
    \item $I = \bar{A}\bar{B}\bar{C}\bar{D}$.
\end{enumerate}

\section*{Oppgave 3.}
\begin{enumerate}
    \item Gjør dette som en kobinatorisk blokk
\begin{codebox}[verilog]
module G (
    input logic [2:0] current_state,
    input logic [3:0] keys,
    output logic [2:0] next_state
    )

    always_comb begin
        next_state = current_state;
        case (current_state)
            3'b000: begin
                if (keys[0] & ~keys[1] & ~keys[2] & ~keys[3]) begin
                    next_state = 3'b001;
                end
            end

            3'b001: begin
                if (~keys[0] & keys[1] & ~keys[2] & ~keys[3]) begin
                    next_state = 3'b010;
                end else if (~keys[1] & (keys[0] | keys[2] | keys[3])) begin
                    next_state = 3'b000;
                end
            end

            3'b010: begin
                if (~keys[0] & keys[1] & ~keys[2] & ~keys[3]) begin
                    next_state = 3'b011;
                end else if (~keys[1] & (keys[0] | keys[2] | keys[3])) begin
                    next_state = 3'b000;
                end
            end

            3'b011: begin
                if (keys[0] & ~keys[1] & ~keys[2] & ~keys[3]) begin
                    next_state = 3'b100;
                end else if (~keys[0] & (keys[1] | keys[2] | keys[3])) begin
                    next_state = 3'b000;
                end
            end

            3'b100: begin
            if (keys[0] | keys[1] | keys[2] | keys[3]) begin
                next_state = 3'b000;
            end

            default: begin
                next_state = 3'b000;
            end
        endcase
    end
endmodule
\end{codebox}
    \item Har ikke tid til resten denne gang.

\end{enumerate}

\end{document}
