\section{Testing}
\label{testing}

\subsection{LFSR}
LFSR-en ble testet med ved å koble utgangen $v[n]$ til oscilloskopet til en Analog Discovery 2. Figur \ref{fig:lfsr_plot} viser 
utgangssignalet til LFSR systemet $s(t)$ over tid og figur \ref{fig:lfsr_freq_plot} viser et plot av frekvensspekteret som ble målt 
av utgangssignalet.

\begin{figure}[H]
    \centering
    \includegraphics[width=0.775\textwidth]{Bilder/lfsr_signal.png}
    \caption{Utgangssignalet $s(t)$ fra LFSR systemet}
    \label{fig:lfsr_signal}
\end{figure}

\begin{figure}[H]
    \centering
    \includegraphics[width=0.775\textwidth]{Bilder/støy_freq.png}
    \caption{Frekvensspekteret til LFSR-et}
    \label{fig:lfsr_freq_plot}
\end{figure}

Vi ser at frekvensspekteret er relativt flatt, men at det er noen variasjoner i amplitude over spekteret,
så variasjonen i frekvens kunne vært bedre.

Dette kan f.eks. skyldes at den valgte konfigurasjonen kunn har 2 taps. En annen lengde $M$ som f.eks. 32
hadde mulig gitt en flatere spektrum, og da også produsert bedre hvitt støy.

Signalet ble også lyttet til og subjektivt produserte systemet et tydlig hvitt signal med mye observert støy.

\subsection{Delyiannis-Friend filter}
Filteret ble testet med Analog Discovery-ens Spectrum verktøy og produserte følgende amplituderesponser.

\begin{figure}[H]
    \centering
    \includegraphics[width=0.65\textwidth]{Bilder/bandpass_filter_freq_lang.png}
    \caption{Amplituderespons til filteret med $H_0 = 1$}
\end{figure}

\begin{figure}[H]
    \centering
    \includegraphics[width=0.65\textwidth]{Bilder/bandpass_freq_1h_close.png}
    \caption{Amplituderespons til filteret med $H_0 = 1$, men nærmere}
\end{figure}

\begin{figure}[H]
    \centering
    \includegraphics[width=0.65\textwidth]{Bilder/bandpass_freq_10h.png}
    \caption{Amplituderespons til filteret med $H_0 = 10$}
\end{figure}

Vi observerer at filteret ved $H_0 = 10$ gir en meget flat topp. Dette skyldes trolig at amplituderesponsen ble testet ved 
en inngangsspenning på $\SI{1}{\volt}$ og er trolig fordi at opampen clipper. En mulig forbedring av systemet hadde derfor vært 
å øke forsyningsspenningen til opampen.

Filteret ved $H_0 = 1$ oppnådde en båndbredde på $\SI{260}{\hertz}$ og en Q-faktor på $Q = 12.5$, mens 
$H_0 = 10$ gav en båndbredde på $\SI{690}{\hertz}$ og en Q-faktor på $4.5$. 

\subsection{Fullstendig system}
Det ferdige systemet ble så testet og fikk følgende frekvensspektrum.
\begin{figure}[H]
    \centering
    \includegraphics[width=0.65\textwidth]{Bilder/total_system_freq.png}
    \caption{Frekvensspektrum til ferdig implementert system}
\end{figure}

Utgangssignalet $y(t)$ ble også plottet.
\begin{figure}[H]
    \centering
    \includegraphics[width=0.65\textwidth]{Bilder/total_system_signal.png}
    \caption{Plot av utgangssignalet $y(t)$}
\end{figure}

Vi ser tydlig at frekvenskomponenten ved $f = \SI{3320}{\hertz}$ er godt isolert, og ved subjektiv høring av 
signalet var det tydlig en definert tone på utgangssignalet.
