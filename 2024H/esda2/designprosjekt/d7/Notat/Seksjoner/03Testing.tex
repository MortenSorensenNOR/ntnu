\section{Testing}
\label{testing}

For testing av kretsen er en Analog Discovery 2 benyttet, og er blitt brukt både for å å generere forsyningsspennigene $V^+$ og $V^-$, samt brukt som signalgenerator for inngangssignalet $v_1(t)$ og som oscilloskop for å måle utgangssignalet fra filteret.
Som nevnt var forsyningsspenningen satt til $\pm \SI{5}{\volt}$, og styrken på inngangssignalet $|v_1(t)| = \SI{100}{\milli\volt}$. Filteret ble så testet ved hjelp av Network funksjonaliteten til Analog Discovery-en. Amplituderesponsen til filteret, og for de to 
stagene av filteret (de to Sallen-Key strukturene) ble målt fra $\SI{50}{\hertz}$ til $f_s = \SI{9.1}{\kilo\hertz}$. Vi får så følgende resultater, plottet i figur \ref{fig:filter-freq-plot}

\begin{figure}[H]
    \centering
    \includegraphics[width=1\textwidth]{Bilder/filter_freq_plot.png}
    \caption{Amplituderespons til filteret og de to delstrukturene}
    \label{fig:filter-freq-plot}
\end{figure}

Vi ser fra plottet at filteret oppnår god flathet i passbandet slik spesifisert, og at målet om $|H(0.75 f_s / 2)| = -3$dB og $|H(f_s/2)| = -10$dB ble realisert perfekt, med målte verdier lik
\begin{center}
    $|H(\SI{3.4125}{\kilo\hertz})| = -3.012\text{dB}\:\:\:\:\:\:\:\:$ og $\:\:\:\:\:\:\:\:|H(\SI{4.55}{\kilo\hertz})| = -10.28\text{dB}$
\end{center}
noe som er godt innenfor spesifikasjonen.
