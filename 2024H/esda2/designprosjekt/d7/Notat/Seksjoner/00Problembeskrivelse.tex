\section{Problembeskrivelse}
\label{problemBeskrivelse}

For deskretisering av analoge signal er det viktig å ta hensyn til aliasing effekter som oppstår av at ADC-en har begrenset
sampling frekvens. Det er defor nødvendig å ta i bruk et anti-aliasing filter for å forsikre at Nyquist-kriteriet om at 
inngangssignalet til ADC-en ikke inneholder frekvenskoponenter over $f_s / 2$. Dette er helt klart ikke mulig å realisere 
med et fysisk filter, men dette designnotatet skal ta for seg en metode for å designe et aktivt analog Butterworth lavpass filter
som kan oppnå ønsket demping av de uønskede frekvenskomponentene. Det overordnede systemet er vist i figur \ref{system-figure}
\begin{figure}[H]
    \centering
    \includegraphics[width=0.75\textwidth]{Bilder/system.png}
    \caption{System med Anti-Aliasing filteret mellom signalkilden og ADC-en \cite{problembeskrivelse}}
    \label{system-figure}
\end{figure}

Videre krav satt til filteret er at 
\begin{enumerate}
    \item Filteret skal ha en knekkfrekvens $f_c \geq 0.75 \cdot f_s / 2$, altså $|H(f_s/2)|_{\text{[dB]}} \approx -3$dB
    \item Anti-Aliasing filteret skal påvirke frekvenskomponenter under $f_s/2$ minst mulig
    \item Filteret skal oppnå en dempning på minst 10dB ved frekvensen $f_B = f_s / 2$, da dette er i mange tillfeller 
        tilstrekkelig for å unngå det værste av aliasing problemene for en vanlig ADC.
\end{enumerate}

For dette notatet er $f_s = \SI{9.1}{\kilo\hertz}$ som gir 
\begin{center}
    $f_B \approx \SI{4.55}{\kilo\hertz}\:\:\:\:$ og $\:\:\:\:f_c \geq \SI{3.4125}{\kilo\hertz}$
\end{center}
samt at 
\[
    |H(\SI{4.55}{\kilo\hertz})|_{\text{[dB]}} = -10\text{dB}
\]
