\section{Realisering}
\label{realisering}
\subsection{Valg av orden $n$}
Første steg er å finne filter ordenen som trengs for å realisere den ønskede frekvensresponsen. Vi har først at vi 
ønsker en dempning på 10dB for nykvistfrekvensen
\[
    A_{\text{[dB]}} = 20 \log_{10}(A) = -10\text{dB} \implies A = 10^{-\frac{10}{20}} \approx 0.316
\]

Med $f = f_B \approx \SI{4.55}{\kilo\hertz}$ og $f_c \geq \SI{3.4125}{\kilo\hertz}$ får vi fra likning 2 at nødvendig 
verdi for $n$ er
\[
    n = \left\lceil\frac{1}{2}\frac{\ln{\left((0.316)^{-2} - 1\right)}}{\ln{\left(\frac{\SI{4.55}{\kilo\hertz}}{\SI{3.4125}{\kilo\hertz}}\right)}}\right\rceil = 4
\]

Kretsstrukturen til filteret blir dermed to andre ordens Sallen-Key strukturer i kaskade som vist i figur \ref{fig:filter-circuit}
\begin{figure}[H]
    \centering
    \includegraphics[width=0.95\textwidth]{Bilder/filter_circuit.png}
    \caption{Kretsdiagram}
    \label{fig:filter-circuit}
\end{figure}

\subsection{Utregning av komponentverdier}
Starter med å finne verdiene for $\zeta_i$. Vi har gitt fra formel 4 at
\[
    \zeta_i = \begin{cases}
        \cos\left(\frac{\pi}{n}i\right), & \text{ for $n$ odde,} \\
        \cos\left(\frac{\pi}{2n} + (i - 1)\frac{\pi}{n}\right), & \text{ for $n$ jevne} \\
    \end{cases}
\]

Vi får dermed at
\begin{center}

    $\zeta_1 \approx 0.9239\:\:\:\:\:$ og $\:\:\:\:\:\zeta_2 \approx 0.3827$
\end{center}

\subsubsection{Metode 1}
Følger vi metode 1 for utregning av komponentverdier i Sallen-Key strukturen fra \ref{sallen-key}, starter vi
med å sette en gunstig motstandsverdi $R$. Vi ønsker at kretsen skal produsere så lite støy som mulig,
og det er derfor lurt å velge en verdi for $R$ som gjør at operasjonsforsterkeren introduserer så lite støy
som mulig. En verdi for $R$ mellom c.a. $\SI{1}{\kilo\ohm}$ og $\SI{100}{\kilo\ohm}$ vil trolig funke bra 
for den valgte operasjonsforsterkeren, LF353 fra Texas Instruments \cite{opamp}.
Vi velger her $R = \SI{10}{\kilo\ohm}$.

Følger vi resten av utregingene fra \ref{sallen-key} får vi da følgende komponentverdier for den første 
andre ordens Sallen-Key struktur
\begin{center}
    $C_1 \approx \SI{5}{\nano\farad}\:\:\:\:\:\:\:\:$ og $\:\:\:\:\:\:\:\:C_2 \approx \SI{4.3}{\nano\farad}$
\end{center}

og følgelig for den andre Sallen-Key strukturen
\begin{center}
    $C_3 \approx \SI{12.2}{\nano\farad}\:\:\:\:\:\:\:\:$ og $\:\:\:\:\:\:\:\:C_2 \approx \SI{1.8}{\nano\farad}$
\end{center}

Siden kodensatorverdiene ikke er standard og mengden størrelser av kondensatorer for hånd var liten, ble 
metode 2 gjennom \url{https://tools.analog.com/en/filterwizard/} heller brukt, der kondensatorverdiene ble valgt 
til noe som var for hånd. Programvaren støtter også å kompansere for at operasjonsforsterkeren ikke er fullstendig 
ideell, men dette ble ikke benyttet. Det gav følgende komponentverdier

% Table showing component values R1, R2, C1, C2, R3, R4, C3, C4, with a seperator with for the two Sallen-Key structures, with a 
% line of bold text saying Sallen-Key Struktur 1: and Sallen-Key Struktur 2:
\begin{table}[h!]
    \centering
    \renewcommand{\arraystretch}{1.5}
    \begin{tabular}{|l|l|}
        \hline
        \multicolumn{2}{|l|}{\textbf{Sallen-Key Struktur 1}} \\
        \hline
        Komponent & Verdi \\
        \hline
        \( R_1 \) & \(\SI{3.54}{\kilo\ohm}\)\\
        \( R_2 \) & \(\SI{12.7}{\kilo\ohm}\)\\
        \( C_1 \) & \(\SI{22}{\nano\farad}\)\\
        \( C_2 \) & \(\SI{2.2}{\nano\farad}\)\\
        \hline
    \end{tabular}
    \vspace{1em}
    
    \begin{tabular}{|l|l|}
        \hline
        \multicolumn{2}{|l|}{\textbf{Sallen-Key Struktur 2}} \\
        \hline
        Komponent & Verdi \\
        \hline
        \( R_1 \) & \(\SI{12.2}{\kilo\ohm}\)\\
        \( R_2 \) & \(\SI{26.9}{\kilo\ohm}\)\\
        \( C_1 \) & \(\SI{3}{\nano\farad}\)\\
        \( C_2 \) & \(\SI{2.2}{\nano\farad}\)\\
        \hline
    \end{tabular}
    \vspace{1em}
\end{table}

Det gir følgende krets
\begin{figure}[H]
    \centering
    \includegraphics[width=0.9\textwidth]{Bilder/filter_circuit_final.png}
    \caption{Kretsdiagram med komponentverdier for hele filteret, kilde: \url{https://tools.analog.com/en/filterwizard/}}
    \label{fig:filter-circuit-final}
\end{figure}

\subsection{Realisering av kretsen}
Oppkobling av kretsen ble gjort på et koblingsbrett. Forsyningsspenningen valgt var $V^+ = \SI{5}{\volt}$ og 
$V^- = \SI{-5}{\volt}$, og ble koblet opp mot Op-Ampenes forsyningsspenninger.
\begin{figure}[H]
    \centering
    \includegraphics[width=0.8\textwidth]{Bilder/oppkobling.png}
    \caption{Realisering av kretsen på koblingsbrett}
    \label{fig:realisering-img}
\end{figure}
