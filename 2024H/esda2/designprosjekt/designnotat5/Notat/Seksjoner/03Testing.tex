\section{Testing}

Testing av systemet ble utført ved hjelp av en Analog Discovery 2 \cite{analog_discovery} oscilloskop og 
signalgenerator. Kilden $v_0$ ble generert som et sinussignal fra ett av oscilloskopets signalgeneratorer 
og ble satt til en frekvens $f = \SI{1.0}{\kilo\hertz}$ og en amplitude på $A_0 = \SI{500}{\milli\volt}$.

\subsection{Test av ideelt system}
Systemet ble først testet med ideelle verdier for kilde- og lastmotstanden. I figur \ref{fig:buffer_no_load_plot}
er kildespenningen $v_0$ plottet sammen med utgangsspenningen til bufferen $v_2$.
\begin{figure}[H]
    \centering 
    \includegraphics[width=0.7\textwidth]{Bilder/buffer_no_load_plot.png}
    \caption{Buffer test med ideell lastmotstand $R_L = \infty$ og kildemotstand $R_K = 0$}
    \label{fig:buffer_no_load_plot}
\end{figure}
Plottet viser tydelig at $v_0 \approx v_2$. De målte VRMS verdiene til de to signalene for denne testen var 
\begin{center}
    $v_0 = \SI{346.9}{\milli\volt}\:\:\:\:$ og $\:\:\:\:v_2 = \SI{333.3}{\milli\volt}$ 
\end{center}

Det gir en dempning på
\[
    A_\text{[dB]} = 20 \cdot \log{\left(\frac{v_2}{v_0}\right)} \approx -0.349 \text{dB}
\]

\subsection{Test med realistiske last- og kildemotstander}
Videre er systemet testet med lastmotstand (figur \ref{fig:buffer_load_no_res_plot}) og med last- 
og kildemotstand (figur \ref{fig:buffer_load_source_res_plot}).

\begin{figure}[H]
    \centering 
    \includegraphics[width=0.625\textwidth]{Bilder/buffer_load_no_res_plot.png}
    \caption{System test med lastmotstand $R_L = \SI{220}{\ohm}$ og ideell kildemotstand $R_K = \SI{0}{\ohm}$}
    \label{fig:buffer_load_no_res_plot}
\end{figure}

\begin{figure}[H]
    \centering 
    \includegraphics[width=0.625\textwidth]{Bilder/buffer_load_source_res.png}
    \caption{System test med lastmotstand $R_L = \SI{220}{\ohm}$ og kildemotstand $R_K = \SI{1.2}{\kilo\ohm}$}
    \label{fig:buffer_load_source_res_plot}
\end{figure}

Vi ser nå at vi får et større avvik mellom $v_2$ og den ønskede spenningen $v_0$. 

For systemtesten med kunn lastmotstand $R_L = \SI{220}{\ohm}$ oppnådde systemet følgende VRMS verdier:
\begin{center}
    $v_0 = \SI{346.2}{\milli\volt}\:\:\:\:$ og $\:\:\:\:v_2 = \SI{326.8}{\milli\volt}$ 
\end{center}

og følgende for systemtesten med både kildemotstanden $R_K = \SI{1.2}{\kilo\ohm}$ og
lastmotstanden $R_L = \SI{220}{\ohm}$

\begin{center}
    $v_0 = \SI{346.8}{\milli\volt}\:\:\:\:$ og $\:\:\:\:v_2 = \SI{303.2}{\milli\volt}$ 
\end{center}

Når kunn lastmotstanden er tilkoblet oppnår bufferen en demping på 
\[
    A_\text{[dB]} = 20 \cdot \log{\left(\frac{v_2}{v_0}\right)} \approx -0.500 \text{dB}
\]

og for både lastmotstanden og kildemotstanden oppnår bufferen en demping på
\[
    A_\text{[dB]} = 20 \cdot \log{\left(\frac{v_2}{v_0}\right)} \approx -1.166 \text{dB}
\]

Vi ser derfor at bufferen er langt mer sensitiv for endring i kildemotstanden $R_K$ enn lastmotstanden
$R_L$ når vi sammenligner dempningen i de to tilfellene med dempningen uten last og kildemotstand. 
Dette kan tyde på at inngangsmotstanden til bufferen ikke er tilstrekkelig. Høyere verdier for 
$R_{B1}$ og $R_{B2}$ ble testet men resulterte ikke i bedre bufferkarakteristikk, mulig fordi det har 
negative innvirkninger på transistorens karakteristikk.

Vi ser derimot som nevnt at introduksjonen av en lastmotstand ikke hadde store konsekvenser for 
bufferkarakteristikken, da dempningen kunn økte med $0.150$ dB, noe som tyder på at vi har valgt en 
god verdi for emittermotstanden $R_E$, og som konsekvens, et godt arbeidspunkt for emitterstrømmen.

Bedre bufferkarakteristikk kan trolig oppnås dersom flere stadier, og dermed også flere BJT-transistorer,
hadde blitt tatt i bruk i en mer avansert kretstopologi.

\subsection{Test av maksimal kildeamplitude $A_0$}
Maksimal kildeamplitude $A_0$ ble også testet. Amplituden til sinussignalet generert ble gradvis økt frem til 
forvrengninger i utgangssignalet ble tydelige. For den realiserte bufferkretsen gav amplituder omlag $\SI{850}{\milli\volt}$ 
tydlige forvrengninger, som vist i figur \ref{fig:buffer_disturb_850mv_plot}.

\begin{figure}[H]
    \centering 
    \includegraphics[width=0.625\textwidth]{Bilder/buffer_disturb_850mv_plot.png}
    \caption{Utgangsspenning $v_2(t)$ ved inngangsspenning $v_i(t)$ lik $\SI{850}{\milli\volt}$}
    \label{fig:buffer_disturb_850mv_plot}
\end{figure}

\subsection{Test av frekvensrespons}
Bufferens frekvensrespons ble også testet. Network Analyser funksjonen i 
Analog Discoveryens \textit{Waveforms} applikasjon ble benyttet for spektrumsanalysen. Frekvenser mellom
$\SI{10}{\hertz}$ og $\SI{4}{\mega\hertz}$ ble testet. Frekvensresponsen til systemet er vist i figur \ref{fig:buffer_frekvens_respons}.
\begin{figure}[H]
    \centering 
    \includegraphics[width=0.625\textwidth]{Bilder/buffer_frekvens_respons.png}
    \caption{Frekvensresponsen til buffer systemet}
    \label{fig:buffer_frekvens_respons}
\end{figure}

Systemet oppnådde en -3dB knekkfrekvens $f_0 = \SI{23.6}{\hertz}$. Dette er langt under den forventede 
frekvensen til kildesignalet $f = \SI{1000}{\hertz}$.

\subsection{Inn- og utgangsmotstand til bufferkretsen}
Til slutt ble inn- og utgangsmotstanden til bufferkretsen målt. Oppsettet for måling av inn og utgangsmotstand 
er vist i figur \ref{fig:in_out_res_measure_setup}.

\begin{figure}[H]
    \centering
    \includegraphics[width=0.6\textwidth]{Bilder/in_out_res_setup.png}
    \caption{Oppsett for måling av inn- og utgangsmotstand til bufferen}
    \label{fig:in_out_res_measure_setup}
\end{figure}

Fra figuren over finner vi at vi kan måle inngangsmotstanden $R_{inn}$ ved å se på spenningsdelingen 
over den kjente motstanden $R'$. Vi får da følgende formel for inngangsmotstanden
\[
    R_{inn} = \frac{v_i}{v_{0} - v_i} \cdot R'
\]

Vi måler spenningene $v_i$ og $v_0$ i VRMS og får følgende verdier
\begin{center}
    $v_i = 329$ mVRMS $\:\:\:\:v_s = 350$ mVRMS $\:\:\:\:R' = \SI{1.0}{\kilo\ohm}$
\end{center}

Det gir da en inngangsmotstand til bufferen på
\[
    R_{inn} = \frac{\SI{329}{\milli\volt}}{\SI{350}{\milli\volt} - \SI{329}{\milli\volt}} \cdot \SI{1.0}{\kilo\ohm} \approx \SI{15.6}{\kilo\ohm}
\]

Vi ser her at vi til tross for at $R_{B1} = R_{B2} = \SI{100}{\kilo\ohm}$ at det realiserte kretsens 
inngansmotstand kunn er omlag 13x større en kildemotstanden $R_K$. Dette forklarer hvorfor kildemotstanden
hadde stor påvirkning på bufferens dempning.

Videre kan vi finne et uttrykk for bufferkretsens utgangsmotstand $R_{ut}$
\[
    R_{ut} = \frac{v_E - v_2}{v_2} \cdot R_L
\]
der $v_E$ er emitter spenningen.

Får følgende verdier etter måling
\begin{center}
    $v_E = 0.0$ mVRMS $\:\:\:\:v_2 = 335.6$ mVRMS $\:\:\:\:R_L = \SI{1.0}{\kilo\ohm}$
\end{center}
