\section{Prinsipiell løsning}
\label{prinsipiellLoesning}

For å oppnå at designets mål at $v_2 \approx v_1 \approx v_0$ må bufferen designes på en sånn måte
at påvirkningen fra kildemotstanden $R_K$ og lastmotstanden $R_L$ blir minst mulig. Vi ønsker altså at mest 
mulig av spenningen $v_0$ skal ligge over inngangen på bufferen, og at minst mulig av det genererte 
spenningsnivået internt i bufferen skal ligge over bufferens utgangsmotstand. Dette er mulig dersom 
vi gjør inngangsmotstanden til bufferen stor, og utgangsmotstanden til bufferen liten.

For å oppnå at $v_2 \approx v_1$ kan en bufferkrets basert op en BJT Emitter follower krets benyttes. 
Kretsens oppbygning er vist i figur \ref{fig:bjt_voltage_follower}.

\begin{figure}[H]
    \centering 
    \includegraphics[width=0.55\textwidth]{Bilder/bjt_follower.png}
    \caption{BJT Emitter Follower \cite{bjt_follower}}
    \label{fig:bjt_voltage_follower}
\end{figure}

Arbeidspunktet til kretsen er gitt ved verdiene $V_B$, $V_E$ og $I_E$. Vi ønsker å velge et arbeidspunkt 
$V_B$ s.a. bufferen git størst mulighet for inngangssignalet $v_1$ å varriere uten å oppleve at bufferen 
går i mettning. Vi kan derfor velge at $V_B = \frac{V_{CC}}{2}$. Ettersom at en BJT transistor er brukt i kretsen 
vil det være naturlig å anta at emitterspenningen $V_E$ vil ligge omlag $\SI{0.7}{\volt}$ under basespenningen.
Vi har da at $V_E = V_B - \SI{0.7}{\volt}$. 

Til slutt kan en emitterstrøm $I_E$ velges. Emmitterstrømmen må velges utifra databladet til transistoren 
og ønsket emittermotstand. Siden emittermotstanden har stor påvirkining på utgangsmotstanden til 
bufferen $R_{ut}$ er det naturlig å velge en så høy strøm som mulig, da dette vil gi en lavere 
verdi for $R_E$ og dermed også lav utgangsmotstand. Verdiene til motstandene $R_{B1}$ og $R_{B2}$ må velges 
slik at $R_{B1} = R_{B2}$, da dette vil gi riktig arbeidspunkt for basespenningen $V_B$. Størrelsen på 
motstandene må velges til å være store, gjerne 10-100 kilo ohm, da dette vil gi en stor inngangsmotstand til 
bufferen $R_{inn}$. 

Til slutt kan kondensatorverdiene $C_1$ og $C_2$ velges. Ettersom at kondensatoren i lag med 
inngangsmotstanden $R_{inn}$ danner et høypassfilter, er det naturlig å velge en kondensatorverdi som 
er slik at knekkfrekvensen $f_0 << f$, der $f$ er den forventede frekvensen til inngangssignalet $v_0$.
Større kodensatorverdier har liten negativ påvirkning annet enn at de tar lenger tid å lade opp, og derfor 
forlenger tiden før kretsen oppnår de ønskede egenskapene fra kretsen først blir aktivert.
