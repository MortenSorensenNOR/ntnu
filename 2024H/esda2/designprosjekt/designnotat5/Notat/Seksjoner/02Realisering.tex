\section{Realisering}
\label{realisering}

For realiseringen, og senere testing av systemet, har en $V_{cc} = \SI{6}{\volt}$, 
$R_K = \SI{1.2}{\kilo\ohm}$ og $R_L = \SI{220}{\ohm}$ blitt valgt. Transistoren valgt for 
realiseringen av bufferen er en BC547B BJT transistor \cite{bjt}.

Arbeidspunktet til bufferen blir da 
\begin{center}
    $V_B = \frac{V_{CC}}{2} = \SI{3}{\volt}\:\:\:\:$ $\:\:\:\:V_E = V_B - \SI{0.7}{\volt} = \SI{2.3}{\volt}$
\end{center}
og valgte arbeidspunktet til emitterstrømmen til å være 
\[
    I_E = \SI{30}{\milli\ampere}
\]
da dette virket rimelig fra databladet og ettersom at det gir en lav emittermotstand
\[
    R_E = \frac{\SI{2.3}{\volt}}{\SI{30}{\milli\ampere}} = \SI{76.6}{\ohm}
\]
En høyere emitterstrøm er trolig mulig, men det kan påvirke egenskapene til bufferen ved høyere 
amplituder $A_0$ for kildesignalet $v_0$.
For å oppnå arbeidspunktet $V_B$ ble $R_{B1}$ og $R_{B2}$ begge valgt til en verdi på $\SI{100}{\kilo\ohm}$.
Til slutt ble $C_1 = C_2 = \SI{41}{\micro\farad}$ valgt, hver realisert ved å koble to 
polariserte $\SI{82}{\micro\farad}$ kodensatorer i serie med de negative polene koblet sammen, da dette 
ville produsere en mindre, upolarisert kodensator. Verdien på kodensatorene var trolig langt større enn 
nødvendig, men gav ikke store problemer under testingen av kretsen.

Den realiserte kretsen er fremstilt med komponentverdier i figur \ref{fig:buffer_realisering}

\begin{figure}[H]
    \centering
    \includegraphics[width=0.6\textwidth]{Bilder/bjt_buffer_realisering.png}
    \caption{Realisering av bufferen med komponentverdier}
    \label{fig:buffer_realisering}
\end{figure}
