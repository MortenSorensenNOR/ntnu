\section{Problembeskrivelse}
\label{problemBeskrivelse}

Innen analog elektronikk er det ofte at en signalkilde ikke klarer å levere nok effekt, og dermed strøm til 
lasten den forsøker å drive. I slike tilfeller kan en \textit{buffer} benyttes, der utgangsspenningen fra 
bufferen $v_2$ er omtrent lik inngangsspennigen $v_1$, og utgangseffekten er langt større enn inngangseffekten.
Bruken av en slik buffer er visualisert i figur \ref{fig:buffer_bruk}, der $v_0$ er spenningen kilden 
leverer uten last.

\begin{figure}[H]
    \centering 
    \includegraphics[width=0.85\textwidth]{Bilder/buffer_bruk.png}
    \caption{Kilde, buffer og last \cite{problem_stilling}}
    \label{fig:buffer_bruk}
\end{figure}

Vi ønsker altså at bufferen skal ha den egenskapen at
\[
    v_2 \approx v_1 \approx v_0
\]

I tillegg ønsker vi at bufferens egenskaper skal være mest mulig uavhengig av kildens utgangsmotstand 
$R_K$ og lastmotstanden $R_L$. For å kunne levere tilstrekkelig effekt er det ønskelig å designe 
en buffer ved hjelp av diskrete komponenter, altså transistorer, motstander og kondensatorer. 
Dette designet skal basere seg på BC547 BJT transistoren, og skal vurdere designet utifra:
\begin{enumerate}
    \item Avviket i amplituden $A_2$ til $v_2$ når $v_0$ er et sinussignal med frekvens $f = \SI{1000}{\hertz}$ og 
        amplitude $A_0 = \SI{500}{\milli\volt}$
    \item Maksimal amplitude til $v_0$ før forvrengningen til $v_2$ blit synlig
    \item Frekvensresponsen til bufferen, samt nedre $3$ dB knekkfrekvens
    \item Inn- og utgangsmotstanden til det realiserte systemet.
\end{enumerate}
\newpage
