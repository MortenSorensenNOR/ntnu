\section{Welcome to VLSI}

\subsection{MOS Transistors}
Silicon in its pure form creates a three dimensional lattice of atoms. In this form, all its valence electrons are 
involved in chemical covalent bonds, making it a poor conductor. However, impurities can be introduced into pure
silicon crystals, called \textit{doping} the silicon, with materials such as arsenic which gives the lattice 
structure a free electron, making the material a better conductor at room temperature. This forms an \textit{n-type}
semiconductor, because the free carriers are negatively charged electrons. 

Similarly, a material such as boron can be introduced into the silicon, a material which has only three valence electrons.
This creates a "hole" in the lattice structure, and consequently "borrows" an electron from a neighboring silicon atom, 
propagating the deficiency. The hole acts as a positive carrier, so this type of semiconductor is called a \textit{p-type} 
semiconductor. 

A junction between n-type and p-type silicon creates a \textit{diode}. When voltage is applied to the p-type \textit{anode}, 
and is raised above the n-type \textit{cathode}, the diode is said to be \textit{forward-biased} and current flows.
When the anode voltage is less than or equal to the cathode voltage, the diode is \textit{reverse-biased} 
and very little current flows.

A \textit{MOS} or Metal-Oxide-Semiconductor can be created by superinposing several layers of conducting and insulating materials 
to form a sandwich-like structure. The manifacture of such MOS structures can be done by a series of chemical processing steps 
involving oxidation, selective introduction of dopants, and position and etching of metal wires and contacts.

CMOS technology provides two types of transistors: n-type transistors (\textit{nMOS}) and p-type transistors (\textit{pMOS}).
The operation of the transistors is controlled by electric fields so the devices are also called 
Metal Oxide Semiconductor Field Effect Transistors (MOSFETs). A crossection of these transistors are shown in figure \ref{fig:nmos_pmos}.

\begin{figure}[H]
    \centering
    \includegraphics[width=0.7\textwidth]{Bilder/chapter1/nmos_pmos.jpg}
    \caption{Cross-section of nMOS (left) and pMOS (right) transistors}
    \label{fig:nmos_pmos}
\end{figure}

Each transistor consists of a stack of a conducting \textit{gate}, an insulating layer of silicon dioxide, and the silicon wafer, called the \textit{substrate}, 
\textit{body} or \textit{bulk}. The gates in early transistors were made of metal, i.e. the name metal-oxide-semiconductor, however, sice the 1970s these gates have been 
made from a \textit{polysilicon}.

An nMOS transistor is built with a p-type body and has regions of n-type silicon adjecent to the gate called the \textit{source} and \textit{drain}. These are physically
identical and will be regarded as interchangeable. The body of the transistor is typically grounded. The pMOS is the exact opposite, 
consisting of p-type source and drain regions with an n-type body. In a CMOS technology with both glavors of transistors, the substrate is either 
n-type or p-type silocon. The other flavor of transistor must be built in a special \textit{well} in which dopant atoms have been added to form the body of the 
opposite type.

The gate is a control input: It affects the flow of electrical current between the source and drain. Consider an nMOS transistor. The body is generally grounded so the p-n 
junctions of the source and drain to body are reverse-biased. If the gate is also grounded, no current flows through the reverse-biased junctions. Hence, we say that the
transistor is OFF. If the gate voltage is raised, it creates an electric field that starts to attract free electrons to the underside of the Si-SiO$_2$ interface. If the voltage 
is raised enough, the electrons outnumber the holes and a thin region under the gate called the \textit{channel} is inverted to act as an n-type semiconductor.
Hence, a conducting path of electron carriers is formed from source to drain and current can flow. We then say that the transistor is ON.

For a pMOS transistor, the situation is reversed. The body is held at a positive voltage. When the gate is also at a positive voltage, the source and drain
junctions are reverse-biased and no current flows, so the transistor is OFF. When the gate voltage is lowered, positive charges are attracted to the underside of the Si-SiO$_2$ 
interface. A sufficiently low gate volgate invertes the channel and a conducting path of positive carriers is formed from source to drain, so the transistor is ON. 
Notice that the symbol for the nMOS transistor yas a bubble on the gate, indicating that the transistor behavior is the opposite of the nMOS.

The positive voltage is usually called $V_{DD}$ or POWER and represents a logic 1 value in digital circuits. In popular logic families of the 1970s and 1980s, $V_{DD}$
was set to 5 volts. Smaller, more recent transistors are unable to withstand such high voltages and have used supplies of 3.3 V, 2.5 V, 1.8 V, 1.5 V, 1.2 V, 1.0 V, and so forth.
The low voltage is called GROUND (GND) or $V_{SS}$ and represents a logic 0 value. It is normally 0 volts.

\subsection{CMOS Logic}

\subsubsection{The Inverter}

Figure \ref{fig:cmos_inverter} shows the schematic and symbol for a CMOS inverter or NOT gate using one nMOS transistor and one pMOS transistor. When 
$A$ is 0, the nMOS transistor is OFF and the pMOS transistor is ON, so the output $Q$ is pulled up to a 1 because it is connected to $V_{DD}$ but not GND ($V_{SS}$).
Conversely, when $A$ is 1, the nMOS transistor is ON and the pMOS is OFF, resulting in the output $Q$ being pulled down to 0. This is 

\begin{figure}[H]
    \centering
    \includegraphics[width=0.5\textwidth]{Bilder/chapter1/cmos_inverter.jpg}
    \caption{Inverter symbol and truth table (left) and schematic (right)}
    \label{fig:cmos_inverter}
\end{figure}

\subsection{CMOS Fabrication and Layout}

\subsection{Design Partitioning}

\subsection{Example: A Simple MIPS Microprocessor}

\subsection{Logic Design}

\subsection{Circuit Design}
Circuit design is concerned with arranging transistors to perform a particular logic function. Given a circuit design, we can estimate the delay and power. The circuit
can be represented as a scematic, or in textual form as a netlist. Common transistor level netlist formats include Verilog and SPICE. Verilog netlists are used for functional
verification, while SPICE netlists have more detail necessary for delay and power simulations. Because a transistor gate is a good insulator, it can be modeled as a capacitor,
$C$. When the transistor is ON, some current flow $I$ flows between source and drain. Both the current and capacitence are proportional to the transistor width.

The delay of a logic gate is determined by the current that it can deliver and the capacitance that it is driving, as shown in figure \ref{tbd} (FIGURE OF INVERTER DRIVING 
ANOTHER IDENTICAL INVERTER). The capacitance is carged or discharged according to the constitutive equation
\[
    I = C \frac{dV}{dt}
\]

If an average current $I$ is applied, the time $t$ to switch between 0 and $V_{DD}$ is 
\[
    t = \frac{C}{I}V_{DD}
\]

Hence, the delay increases with the load capacitance and decreases with the drive current. To make these calculations, we will have to delve below the switch-level moel of a 
transistor. Chapter 2 develops more detailed models of transistors accounting for the current and capacitance. One of the goals of circuit design is to choose transistor 
widths to meet delay requirements. Methods for doing so are discussed in chapter 3.

Energy is required to charge and discharge the load capacitance. This is called the dynamic power because it is consumed only when the circuit is actively switching. The 
dynamic power consumed when a capacitor is charged and discharged at a frequency of $f$ is 
\[
    P_{\text{dynamic}} = CV^2_{DD}f
\]

Even when the gate is not switching, it draws some static power. Because an OFF transistor is leaky, a small amount of current $I_{\text{static}}$ flows between power and
ground, resulting in a static power disipation of
\[
    P_{\text{static}} = I_{\text{static}}V_{DD}
\]

Factors such as the fan-in and fan-out of each gate in a function, the gates used and the width of each transistor influences the capacitance and current, and 
hence the speed and power of the circuit, as well as area and cost.

\subsection{Physical Design}

\subsection{Design Verification}
