\section{Devices}
\subsection{Introduction}

\subsection{Long-Channel I-V Characteristics}
MOS transistors have three regions of operation:
\begin{itemize}
    \item Cutoff or subthreashold region
    \item Linear region
    \item Saturation region
\end{itemize}

A model can be derived relating the current and voltage (I-V) for an nMOS transistor in each of these regions. The model assumes that the 
channel length is long enough that the lateral electric field (the field between source and drain) is reletively weak, which is no longer the case in nanometer devices.
This model is variously known as the \textit{long-channel}, \textit{ideal}, \textit{first-order}, or \textit{Shockley} model.  

The long-channel model assumes that the current through an OFF transistor is 0.
When a transistor turns ON ($V_{gs} > V_t$), the gate attracts carriers (electrons) to form a channel. The electrons drift from source 
to drain at a rate proportional to the electric field between these regions. This, we can compute currents if we know the amount of charge in the channel and 
the rate at which it moves. We know that the charge on each plate of a capacitor is $Q = CV$. Thus, the charge in the channel $Q_{\text{channel}}$ is 
\begin{equation}
    Q_{\text{channel}} = C_g \left(V_{gc} - V_t\right)
\end{equation}

where $C_g$ is the capacitance of the gate to the channel and $V_{cg} - V_t$ is the amount of voltage attracting charge to the channel beyond the minimum 
required to invert from p to n. The gate voltage is referenced to the channel, which is not grounded. If the source is at $V_s$ and the drain is at $V_d$, 
the average is $V_c = (V_s + V_d)/2 = V_s + V_{ds} / 2$. Therefore, the mean difference between the gate and channel potentials $V_{gc}$ is $V_g - V_t = V_{gs} - V_{ds}/2$.

We can model the gate as a parallel plate capacitor with capacitance proportional to area over thickness. If the gate has length $L$ and width $W$ and the 
oxide thickness is $t_{ox}$, the capacitance is 
\begin{equation}
    C_g = k_{ox} \epsilon_{0} \frac{WL}{t_{ox}} = \epsilon_{ox} \frac{WL}{t_{ox}} = C_{ox} WL
\end{equation}

where $\epsilon_{0}$ is the permittivity of free space, $8.85 \times 10^{-14}$ F/cm, and the permittivity of SiO$_2$ is $k_{ox} = 3.9$ times as great. Often, the 
$\epsilon_{ox}/t_{ox}$ term is called $C_{ox}$, the capacitance per unit area of the gate oxide.

\subsection{C-V Characteristics}

\subsection{Nonideal I-V Effects}

\subsection{DC Transfer Characteristics}
