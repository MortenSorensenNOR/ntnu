\documentclass[a4paper,11pt,norsk]{article}
\usepackage{packages}

\begin{document}

%Headingdel:---------------------------------------------
\topmargin -1.5cm
\makebox[\textwidth][s]{
    \begin{minipage}[c]{0.25\textwidth}
        \includegraphics[width=2.0cm]{Bilder/ntnu_logo.png}  
    \end{minipage}
    \begin{minipage}[c]{0.75\textwidth}
        \huge{\textbf{TTT4260 ESDA}} \\
        \Large{Øving 3  ---  Morten Sørensen, \large{\color{black!75!white}\today}}
    \end{minipage}
}
\vspace{0.75cm}
\normalsize


\section*{Task 1}
\begin{enumerate}
    \item Here is the state diagram for detecting a sequence of 101 on the input x.
        \begin{figure}[H]
            \centering
            \includegraphics[width=0.7\textwidth]{Bilder/state_diagram.jpg}
            \caption{State diagram for detecting a sequence of 101 on the input x. Here y = 1 when in S3, and y = 0 otherwise.}
            \label{fig:Task1_1}
        \end{figure}
    \item The corresponding state table is as follows:
        \begin{table}[H]
            \centering
            \begin{tabular}{|c|c|c|c|}
                \hline
                \textbf{Current State} & \textbf{Input} & \textbf{Next State} & \textbf{Output} \\ \hline
                IDLE & 0 & IDLE & 0 \\ \hline
                IDLE & 1 & S1   & 0 \\ \hline
                S1   & 0 & S2   & 0 \\ \hline
                S1   & 1 & S1   & 0 \\ \hline
                S2   & 0 & IDLE & 0 \\ \hline
                S2   & 1 & S3   & 0 \\ \hline
                S3   & 0 & IDLE & 1 \\ \hline
                S3   & 1 & IDLE & 1 \\ \hline
            \end{tabular}
            \caption{State table for detecting a sequence of 101 on the input x.}
            \label{tab:Task1_2}
        \end{table}
    \item Using the state table above and the following state encoding
        \begin{table}[H]
            \centering
            \begin{tabular}{|c|c|}
                \hline
                \textbf{State} & \textbf{Encoding} \\ \hline
                IDLE & 00 \\ \hline
                S1   & 01 \\ \hline
                S2   & 10 \\ \hline
                S3   & 11 \\ \hline
            \end{tabular}
            \caption{State encoding for the state diagram in Figure \ref{fig:Task1_1}.}
            \label{tab:Task1_3}
        \end{table}
        We get that the equation for the next state, and therefore the input equation for the two edge triggered D flip-flops, is
        \begin{align*}
            NS_1 &= \overline{S_{11}} \cdot x \\
            NS_0 &= S_{01} \cdot \overline{x} + S_{10} \cdot x
        \end{align*}
        where $NS_0$ and $NS_1$ are the next state bits for the flip-flops, and $S_{ij}$ is the current state, with bits $i$ and $j$ currently set.
        The output equation is then consequently
        \begin{equation*}
            y = S_{11}
        \end{equation*}
    \item The combinational circuit can be implemented as follows
\begin{codebox}[verilog]
module sequence_detector(
    input wire x,
    input wire [1:0] current_state,
    output reg [1:0] next_state,
    output reg y
);
    // Determine next_state[0] and output y
    wire T0;

    nand U0 (T0, current_state[0], current_state[1]);
    and U1 (next_state[0], T0, x);

    not U2 (y, T0);

    // Determine next_state[1]
    wire T1, T2, T3, T4, T5;

    nor U3 (T1, current_state[1], x);
    and U4 (T2, current_state[0], T1);

    and U5 (T3, current_state[1], x);
    not U6 (T4, current_state[0]);
    and U7 (T5, T3, T4);

    or U8 (next_state[1], T1, T5);
endmodule
\end{codebox}

    \item The total number of transistors needed to implement the combinational logic using the following conversion from gate to transistor count
        \begin{table}[H]
            \centering
            \begin{tabular}{|l|c|}
                \hline
                \textbf{Gate} & \textbf{Transistor count} \\ \hline
                NOT & 2 \\ \hline
                NAND & 4 \\ \hline
                NOR  & 4 \\ \hline
                AND & 6 \\ \hline
                OR  & 6 \\ \hline
            \end{tabular}
        \end{table}
        we get that the total number of transistors needed is
        \begin{align*}
            \text{NAND} + 4 \cdot \text{AND} + 2 \cdot \text{NOT} + \text{NOR} + \text{OR} &= 4 + 4 \cdot 6 + 2 \cdot 2 + 4 + 6 = 42
        \end{align*}
\item
    \item Another logic implementation style other than static CMOS could domino logic. 
    \item One upside of domino logic is that it can achieve higher speeds than static CMOS, though at the cost of higher dynamic power consumption, 
        needing to utilize static CMOS inverters in order to implement all logic functions, and and circuit complexity, amongst others. (Also it needs a clock,
        which is not ideal for combinational).
    \item In order to simulate and verify the testbench, since the number of states combined with number of possible input signals are quite small,
        it's possible to verify the testbench by simply trying all possible combinations of inputs and states, and check that the output is as expected. 
        Basically all possible combinations of three bits, which is $2^3 = 8$ possible combinations.
\end{enumerate}

\section*{Task 2}
\begin{enumerate}
    \item In order to design an unskewed static CMOS inverter the two transconductances be equal, i.e. 
        $\beta_n = \beta_p$. We may use $L_n = L_p = L = W_n = \SI{45}{\nano\meter}$. 
        That gives 
        \begin{align*}
            \mu_n C_{ox} \frac{W_n}{L} = \mu_p C_{ox} \frac{W_p}{L} \implies W_p = \frac{\mu_n}{\mu_p} W_n
        \end{align*}
        For this technology $\frac{\mu_n}{\mu_p} = 4$, so we therefore have an unskewed inverter if 
        we choose $W_p = 4 W_n$.
    \item We have a symmetric inverted if both $V_{tn} = |V_{tp}|$ is true for the technology we are using, 
        and if the designed circuit has $\beta_n = \beta_p$. The first condition rules out the $\SI{0.8}{\micro\meter}$ 
        and $\SI{0.35}{\micro\meter}$ technologies.
    \item In a symmetric inverter the rise and fall times are the symmetric, which is ideal because it means 
        for consistent propagation delay, and can be quite important for high-speed circuits.
    \item We have that the paracitic capacitances at the midpoint are given by
        \[
            R_n = \frac{1}{\beta_n (V_{DD} - V_{tn})}
        \]
        and 
        \[
            R_p = \frac{1}{\beta_p (V_{DD} - |V_{tp}|)}
        \]

        We therefore have that $R_n = R_p \approx \SI{4.76}{\ohm}$.
    \item We have that for the fall-times, the time $t$ it takes for some voltage $V_{out}(t)$
        to be present can be expressed as
        \[
            t = \tau_n \ln{\left(\frac{V_{DD}}{V_{out}}\right)}
        \]
        Therefore we get that the fall-time for the symmetric inverter when 10\% and 90\% of the full voltage 
        swing are used as limits is given as 
        \begin{align*}
            t_f &= \tau_n \ln{\left(\frac{V_{DD}}{0.1 V_{DD}}\right)} - \tau_n \ln{\left(\frac{V_{DD}}{0.9 V_{DD}}\right)} \\
                &= \tau_n \ln{(10)} - \tau_n \ln{(1.11\bar{1})} \\
                &= \tau_n \ln{(9)} \approx 2.2 \tau_n = \underline{\underline{2.2 \cdot R_n \cdot C_{out}}}}}
        \end{align*}
        We have a similar scenario for the rise-time, where by symmetry
        \[
            t_r = 2.2 \tau_p
        \]
    \item If $C_{out} = \SI{5}{\femto\farad}$, then the rise- and fall-times for the symmetric inverter above is
        \[
            t_f = 2.2 \cdot \SI{4.76}{\ohm} \cdot \SI{5}{\femto\farad} \approx \SI{52.4}{\femto\second}
        \]
        and 
        \[
            t_r = t_f = \SI{52.4}{\femto\second}
        \]
    \item We have that
        \[
            f_{max} = \frac{1}{t_r + t_f} \approx \SI{9.55}{\tera\hertz}
        \]
    \item I do not have the book so I don't quite know what they define it as.
    \item 
    \item 
    \item A latch changes it's value whenever the input changes value, whilst a flip-flop changes
        it's value only on the edge (either positive or negative edge) of a control signal, usually a clock.
\end{enumerate}

\end{document}
