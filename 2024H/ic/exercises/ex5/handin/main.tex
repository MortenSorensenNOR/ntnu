\documentclass[a4paper,11pt,norsk]{article}
\usepackage{packages}

\begin{document}

%Headingdel:---------------------------------------------
\topmargin -1.5cm
\makebox[\textwidth][s]{
    \begin{minipage}[c]{0.25\textwidth}
        \includegraphics[width=2.0cm]{Bilder/ntnu_logo.png}  
    \end{minipage}
    \begin{minipage}[c]{0.75\textwidth}
        \huge{\textbf{TTT4260 ESDA}} \\
        \Large{Øving 3  ---  Morten Sørensen, \large{\color{black!75!white}\today}}
    \end{minipage}
}
\vspace{0.75cm}
\normalsize


\section*{Task 1}
\begin{enumerate}
    \item Here is the state diagram for detecting a sequence of 101 on the input x.
        \begin{figure}[H]
            \centering
            \includegraphics[width=0.7\textwidth]{Bilder/state_diagram.jpg}
            \caption{State diagram for detecting a sequence of 101 on the input x. Here y = 1 when in S3, and y = 0 otherwise.}
            \label{fig:Task1_1}
        \end{figure}
    \item The corresponding state table is as follows:
        \begin{table}[H]
            \centering
            \begin{tabular}{|c|c|c|c|}
                \hline
                \textbf{Current State} & \textbf{Input} & \textbf{Next State} & \textbf{Output} \\ \hline
                IDLE & 0 & IDLE & 0 \\ \hline
                IDLE & 1 & S1   & 0 \\ \hline
                S1   & 0 & S2   & 0 \\ \hline
                S1   & 1 & S1   & 0 \\ \hline
                S2   & 0 & IDLE & 0 \\ \hline
                S2   & 1 & S3   & 0 \\ \hline
                S3   & 0 & IDLE & 1 \\ \hline
                S3   & 1 & IDLE & 1 \\ \hline
            \end{tabular}
            \caption{State table for detecting a sequence of 101 on the input x.}
            \label{tab:Task1_2}
        \end{table}
    \item Using the state table above and the following state encoding
        \begin{table}[H]
            \centering
            \begin{tabular}{|c|c|}
                \hline
                \textbf{State} & \textbf{Encoding} \\ \hline
                IDLE & 00 \\ \hline
                S1   & 01 \\ \hline
                S2   & 10 \\ \hline
                S3   & 11 \\ \hline
            \end{tabular}
            \caption{State encoding for the state diagram in Figure \ref{fig:Task1_1}.}
            \label{tab:Task1_3}
        \end{table}
        We get that the equation for the next state, and therefore the input equation for the two edge triggered D flip-flops, is
        \begin{align*}
            NS_1 &= \overline{S_{11}} \cdot x \\
            NS_0 &= S_{01} \cdot \overline{x} + S_{10} \cdot x
        \end{align*}
        where $NS_0$ and $NS_1$ are the next state bits for the flip-flops, and $S_{ij}$ is the current state, with bits $i$ and $j$ currently set.
        The output equation is then consequently
        \begin{equation*}
            y = S_{11}
        \end{equation*}
    \item The combinational circuit can be implemented as follows
\begin{codebox}[verilog]
module sequence_detector(
    input wire x,
    input wire [1:0] current_state,
    output reg [1:0] next_state,
    output reg y
);
    // Determine next_state[0] and output y
    wire T0;

    nand U0 (T0, current_state[0], current_state[1]);
    and U1 (next_state[0], T0, x);

    not U2 (y, T0);

    // Determine next_state[1]
    wire T1, T2, T3, T4, T5;

    nor U3 (T1, current_state[1], x);
    and U4 (T2, current_state[0], T1);

    and U5 (T3, current_state[1], x);
    not U6 (T4, current_state[0]);
    and U7 (T5, T3, T4);

    or U8 (next_state[1], T1, T5);
endmodule
\end{codebox}

    \item The total number of transistors needed to implement the combinational logic using the following conversion from gate to transistor count
        \begin{table}[H]
            \centering
            \begin{tabular}{|l|c|}
                \hline
                \textbf{Gate} & \textbf{Transistor count} \\ \hline
                NOT & 2 \\ \hline
                NAND & 4 \\ \hline
                NOR  & 4 \\ \hline
                AND & 6 \\ \hline
                OR  & 6 \\ \hline
            \end{tabular}
        \end{table}
        we get that the total number of transistors needed is
        \begin{align*}
            \text{NAND} + 4 \cdot \text{AND} + 2 \cdot \text{NOT} + \text{NOR} + \text{OR} &= 4 + 4 \cdot 6 + 2 \cdot 2 + 4 + 6 = 42
        \end{align*}
\item
    \item Another logic implementation style other than static CMOS could domino logic. 
    \item One upside of domino logic is that it can achieve higher speeds than static CMOS, though at the cost of higher dynamic power consumption, 
        needing to utilize static CMOS inverters in order to implement all logic functions, and and circuit complexity, amongst others. (Also it needs a clock,
        which is not ideal for combinational).
    \item In order to simulate and verify the testbench, since the number of states combined with number of possible input signals are quite small,
        it's possible to verify the testbench by simply trying all possible combinations of inputs and states, and check that the output is as expected. 
        Basically all possible combinations of three bits, which is $2^3 = 8$ possible combinations.
\end{enumerate}

\section*{Task 2}
\begin{enumerate}
\item
\item
\item
\item
\item
\item
\item
\item
\item
\item
\item
\item
\end{enumerate}

\end{document}
