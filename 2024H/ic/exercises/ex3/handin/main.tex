\documentclass[a4paper,11pt,norsk]{article}
\usepackage{packages}

\begin{document}

%Headingdel:---------------------------------------------
\topmargin -1.5cm
\makebox[\textwidth][s]{
    \begin{minipage}[c]{0.25\textwidth}
        \includegraphics[width=2.0cm]{Bilder/ntnu_logo.png}  
    \end{minipage}
    \begin{minipage}[c]{0.75\textwidth}
        \huge{\textbf{TFE4120  Electromagnetism}} \\
        \Large{Exercise 7  ---  Morten Sørensen, \large{\color{black!75!white}\today}}
    \end{minipage}
}
\vspace{0.75cm}
\normalsize


\section*{Task 1}
\begin{enumerate}
    \item The circuit in the layout in figure 1 is a simple common drain amplifier. 
        We can observe the two NMOS transistors at the bottom constituting a simple current mirror, used as an active load with a biasing current of $I_{\text{bias}}$.
        We can also observe the PMOS transistor having the input signal $V_{\text{in}}$ connected to it's gate and the output signal $V_{\text{out}}$ connected to it's 
        source together with the current mirror output.
    \item In this circuit, if the transistors in the current mirror are not matched, the buffer will no longer work as intended, 
        and the desired property $v_{out} \approx v_{in}$ is no longer met.
    \item The circuit in figure 2 is a common source amplifier. The name comes from the source being connected to a "common" reference voltage for both the input and output voltage to the amplifier.
    \item The circuit in figure 2 is most commonly used to increase the signal strength due to it's relatively high gain.
    \item The gain of the common-source amplifier is given by
        \[
            A_v = \frac{v_{\text{out}}}{v_{\text{in}}} = -g_{m1} \cdot (r_{ds1} \:\:||\:\: r_{ds2})
        \]
        This can be found from the small-signal model:
        \begin{figure}[H]
            \centering
            \includegraphics[width=0.6\textwidth]{Bilder/common_source_small_signal.png}
            \caption{Small-signal model from CJM}
        \end{figure}
        Where we observe that the output voltage is only dependent on the current $g_{m1}v_{gs1}$ and the 
        output resitance $r_{out} = r_{ds1} \:\:||\:\: r_{ds2}$.
    \item 'Matching' in the context of integrated circuits refers to trying to make two or more transistors,
        such as the PMOS transistors in a current mirror, as identical with resprect to their electrical properties as possible.
        This is desireable, because if the transistors are not matched, the current on the output will differ from the input. This may cause 
        the gain of for instance the common-source amplifier to be off, or for the common-drain amplifier, for the 
        amplification to no longer be unity.
\end{enumerate}

\section*{Task 2}
\begin{enumerate}
\item 
Assuming that both transistors are working in the active reagion, we have that 
\[
    V_{eff} = V_{GS,M1} - V_{tn} = V_{eff} = V_x - V_{tn} \implies V_{x} = V_{eff} + V_{tn}
\]    

\item 
Assuming both transistors are in the active reagion we have that
\[
    I_D = \frac{1}{2} \mu_n C_{ox} \frac{W}{L} (V_x - V_{tn})^2
\]

Exchanging in the known values we get that
\[
    \SI{10}{\micro\ampere} = \frac{1}{2} \SI{250}{\micro\ampere\per\volt^2} \cdot 5 \cdot (V_x - \SI{0.45}{\volt})^2
\]

Which gives 
\[
    V_x \approx \SI{0.577}{\volt}
\]

\item 
We want $I_D \approx \SI{10}{\micro\ampere}$. We have the following expression for the current 
\[
    I_D = \frac{V_{DD} - V_x}{R} \implies R = \frac{V_{DD} - V_x}{I_D}
\]

Which gives
\[
    R \approx \SI{122}{\kilo\ohm}
\]

The following AimSpice circuit file can be used to simulate the operating point of the
current mirror:
\begin{tcolorbox}[colback=white!95!blue, title= AimSpice circuit file, colframe=gray!85!black]
\begin{minted}{python}
.include p18_cmos_models_tt.inc

vdd2 3 0 dc 1.8
vdd 2 0 dc 1.8

r1 2 1 122k

m1 1 1 0 0 NMOS l=1u w=5u
m2 3 1 0 0 NMOS l=1u w=5u
\end{minted}
\end{tcolorbox}
The simulation gave 
\begin{figure}[H]
    \centering
    \begin{tabular}{|c|c|}
        \hline
        \textbf{Corner} & \textbf{$I_{D,M1}$} \\
        \hline
        TT & $\SI{10.2}{\micro\ampere}$ \\
        FF & $\SI{11.1}{\micro\ampere}$ \\
        SS & $\SI{9.43}{\micro\ampere}$ \\
        \hline
    \end{tabular}
\end{figure}

\item
The process corners TT, FF, SS, SF and FS describe the variations in transistor characteristics 
during fabrication. The reason to simulate these different corners is to get an image of how 
the performance of a circuit may variate as a function of changes in fabrication, and to make sure that 
the circuit still operates as it should even if the transistors operate below their typical performance, 
such as the SS corner.

\item
Using the same simulation file as in 2c we get
\begin{figure}[H]
    \centering
    \begin{tabular}{|c|c|}
        \hline
        \textbf{Corner} & \textbf{$I_{D,M2}$} \\
        \hline
        TT & $\SI{11.7}{\micro\ampere}$ \\
        FF & $\SI{12.9}{\micro\ampere}$ \\
        SS & $\SI{10.06}{\micro\ampere}$ \\
        \hline
    \end{tabular}
\end{figure}

\item 
In order to get the output impedance $r_{ds,M2}$ we add the folliwng line to the AimSpice file:
\begin{tcolorbox}[colback=white!95!blue, title= AimSpice circuit file, colframe=gray!85!black]
\begin{minted}{python}
.plot rds(m2)
\end{minted}
\end{tcolorbox}
Doing a DC sweep from $\SI{1.7}{\volt}$ to $\SI{1.9}{\volt}$ we get 
\begin{figure}[H]
    \centering
    \begin{tabular}{|c|c|}
        \hline
        \textbf{Corner} & \textbf{$r_{ds,M2}$} \\
        \hline
        TT & $\sim\SI{365}{\kilo\ohm}$ \\
        FF & $\sim\SI{280}{\kilo\ohm}$ \\
        SS & $\sim\SI{480}{\kilo\ohm}$ \\
        \hline
    \end{tabular}
\end{figure}

\item
Process variations affecting the output impedance may alter the gain of an amplifier as 
\[
    A = g_m \cdot r_{out} = g_m \cdot r_{ds}
\]

\end{enumerate}
\end{document}
