\documentclass[a4paper,11pt]{article}
\usepackage{packages}

\begin{document}

%Headingdel:---------------------------------------------
\topmargin -1.5cm
\makebox[\textwidth][s]{
    \begin{minipage}[c]{0.25\textwidth}
        \includegraphics[width=2.0cm]{Bilder/ntnu_logo.png}  
    \end{minipage}
    \begin{minipage}[c]{0.75\textwidth}
        \huge{\textbf{TTT4260 ESDA}} \\
        \Large{Øving 3  ---  Morten Sørensen, \large{\color{black!75!white}\today}}
    \end{minipage}
}
\vspace{0.75cm}
\normalsize


\section*{Task 1.}
\begin{alphalist}
    \item There are two types are \textit{free} and \textit{bound} charges. Free charges are charges not related to the polarization $\mathbf{P}$, such 
        as charges from ions and other impurities inside the dielectric, or other free charges present in the material. These charges are free to move around the 
        material. A bound charges on the other hand, are charges that are bound to their respective atom, not free to move around the material. Due to polarization, say from an
        applied electric field, the positive and negative charges in the atom are pulled slightly away from one another; the protons in the core pulled along the field, and the 
        electrons pulled in the other direction. Though this creates an induced electric field, this field only really exists at the edges of the material. This is because interally,
        the skewed electrons of one atom cancle out with the skewed protons of a neighbouring atomic core. This way only the atoms at the edge of the material really have an impact on the overall
        induced electric field of the material. The charges here are called "bound charges", as they are still bound to the respective atoms, but contribute to some induced field.
    \item If we let the overall charge density in a dielectric material 
        \[
            \rho = \rho_f + \rho_b
        \]
        where $\rho_f$ is the charge density of the free charges and $\rho_b$ is the charge density of the bound charges, then Gauss' law becomes
        \[
            \nabla \cdot \mathbf{E} = \frac{1}{\epsilon_0} \cdot (\rho_f + \rho_b)
        \]
        If we have some polarization $\mathbf{P}$, then using $\rho_b = -\nabla \cdot \mathbf{P}$, we may rewrite this as
        \[
            \epsilon_0 \cdot \nabla \cdot \mathbf{E} = -\nabla \cdot \mathbf{P} + \rho_f
        \]
        Shuffeling around and letting $\mathbf{D} = \epsilon_0 \mathbf{E} + \mathbf{P}$, we have that
        \[
            \nabla \cdot \mathbf{D} = \rho_f
        \]
\end{alphalist}

\section*{Task 2.}
\begin{alphalist}
    \item Let the length of the wire be $L$, and the thickness $r < R$. If we use intergal version of the electric displacement 
        verison of Gauss' law we have that
        \[
            \oint \mathbf{D} \cdot d\mathbf{a} = Q_{enc}
        \]
        This means that
        \[
            2\pi r L \cdot D = L \cdot 2\lambda
        \]
        Rearranging we get
        \[
            D = \frac{\lambda}{\pi r} \implies \mathbf{D} = \frac{\lambda}{\pi r}\mathbf{\hat{r}}
        \]
        The displacement will not be the same inside and outside the rubber, because outside the rubber $\mathbf{P} = 0$, so 
        $\mathbf{E} = \frac{1}{\epsilon}\mathbf{D}$.
    \item As above, outside the rubber
        \[
            \mathbf{E} = \frac{1}{\epsilon}\mathbf{D} = \frac{\lambda}{\pi r \epsilon_0}\mathbf{\hat{r}}
        \]
        For inside the rubber we would have to know what $\mathbf{P}$, which is not provided.

\end{alphalist}

\section*{Task 3.}
\begin{alphalist}
    \item Linear dielectrics are materials that, provided as strong enough external electric field, obeys the formula
        \begin{equation}
            \mathbf{P} = \epsilon_0 \chi_e \mathbf{E}
            \label{eq:sus}
        \end{equation}
        where $\chi_e$ is a diemtionless constant for the material, called the materials electric susceptibility. Here $\mathbf{E}$ is the entire field, induced
        polarized field included.
    \item Kind of explained that above :)
    \item Not sure how to find the nonlinear ones. Have not been to the lectures, so perhaps it was covered there. Assuming, based on the phrasing of the 
        question that the polarization is an equation \ref{eq:sus} that the linear term is the dominant one, with a lot of nonlinear terms following, though what
        shape they take I do not know.
\end{alphalist}

\section*{Task 4.}
\begin{alphalist}
    \item From equation \ref{eq:sus} and using the formula for electric displacement from earlier, we have that
        \[
            \mathbf{D} = \epsilon_0 \mathbf{E} + \mathbf{P} = \epsilon_0(1 + \chi_e) \mathbf{E}
        \],
        Here the proportionality between the electric field $\mathbf{E}$ and the electric displacement $\mathbf{D}$ is the \textit{permittivity} $\epsilon$ of the material. 
        If we devide out the permittivity of free space $\epsilon_0$, we get the \textit{relative permittivity} of the material 
        \[
            \epsilon_r = 1 + \chi_e = \frac{\epsilon}{\epsilon_0}
        \]
    \item Relative permittivity measures the ratio between the absolute permittivity of the material in relation to free space. $\epsilon = \epsilon_0 \cdot \epsilon_r$.
    \item Air has a relative permittivity very close to 1, whilst dielectrics like paper have values in the range of 1.4, and a mouse matt, containing quite a bit of air, has a value 
        around 1.2 (Lab 1). Not quite sure about the refractive index, but I do assume that if you take the relative permitivity of two materials and the angle of intersection for say some light, we 
        get Snell's law out.
    \item Dielectrics have relative permitivity slightly above 1, whilst metals have values waaaay greater than 1. Relative permittivity can also be less than 1, because the way the material is 
        structured resists the applied field more than what vacuum does. As for negative relative permittivity, a quick google search says yes, but I do not think I am able to explain why it happens or why it makes sense.
\end{alphalist}

\section*{Task 5.}
\begin{alphalist}
    \item
        \begin{romanlist}
            \item 
            \item 
        \end{romanlist}
    \item
\end{alphalist}

\end{document}
