\documentclass[a4paper,11pt]{article}
\usepackage{packages}

\begin{document}

%Headingdel:---------------------------------------------
\topmargin -1.5cm
\makebox[\textwidth][s]{
    \begin{minipage}[c]{0.25\textwidth}
        \includegraphics[width=2.0cm]{Bilder/ntnu_logo.png}  
    \end{minipage}
    \begin{minipage}[c]{0.75\textwidth}
        \huge{\textbf{TFE4120  Electromagnetism}} \\
        \Large{Exercise 7  ---  Morten Sørensen, \large{\color{black!75!white}\today}}
    \end{minipage}
}
\vspace{0.75cm}
\normalsize


\section*{Task 1.}
\begin{alphalist}
    \item The following are the definitions of current in 1D, 2D and 3D:
    \begin{romanlist}
        \item 1D:
            \[
                \mathbf{I} = \lambda \mathbf{v}
            \]
            where $\lambda$ is the mobile line charge.
        \item 2D:
            \[
                \mathbf{K} = \sigma \mathbf{v}
            \]
            where $\sigma$ is the surface charge density.
        \item 3D:
            \[
                \mathbf{J} = \rho \mathbf{v}
            \]
            where $\rho$ is the volume charge density.
    \end{romanlist} 

    \item The Lorentz force for each of the cases 1D, 2D and 3D are:
    \begin{romanlist}
        \item 1D:
            \[
                \mathbf{F_{\text{mag}}} = \int I(d\mathbf{l} \times \mathbf{B})
            \]
        \item 2D:
            \[
                \mathbf{F_{\text{mag}}} = \int (\mathbf{K} \times \mathbf{B})
            \]
        \item 3D:
            \[
                \mathbf{F_{\text{mag}}} = \int (\mathbf{J} \times \mathbf{B})
            \]
    \end{romanlist}

    \item Don't quite understand the question, but I guess one can say that the currents are simply some stationary charges/charge distributions ($\lambda$, $\sigma$, etc.) 
        with some movement with velocity $\mathbf{v}$.
\end{alphalist}

\section*{Task 2.}
\begin{alphalist}
    \item From the equation
        \[
            I = \int_{\mathcal{S}} \mathbf{J} \cdot d\mathbf{a}
        \]
        which, using the divergence theorem we get
        \[
            \oint_{\mathcal{S}} \mathbf{J} \cdot d\mathbf{a} = \int_{\mathcal{V}} (\nabla \cdot \mathbf{J}) d\tau
        \]
        Since charge cannot simply be created or destroyed, what leaves the surface $\mathcal{S}$ must come from the volume inside, so
        \[
            \int_{\mathcal{V}} (\nabla \cdot \mathbf{J}) d\tau = -\frac{d}{dt} \int_{\mathcal{V}} \rho d\tau = -\int_{\mathcal{V}} \frac{\partial \rho}{\partial t} d\tau
        \]
        Since the area of integration is the same, the integrands must be equal, so 
        \[
            \nabla \cdot \mathbf{J} = -\frac{\partial \rho}{\partial t}
        \]
        which is the \textbf{continuity equation}. The physical meaning is that charge is conserved,
        meaning that a change in charge density must escape through the surface of the volume.
    
    \item For magnetostatics we have steady currents. That means that the change in volume charge density
        \[
            \frac{\partial \rho}{\partial t} = 0
        \]
        Therefore the continuity equation for magnetostatics simplifies to
        \[
            \nabla \cdot \mathbf{J} = 0
        \]
\end{alphalist}

\section*{Task 3.}
For a line current we have that Biot-Savart's Law is defined as
\[
    \mathbf{B}(\mathbf{r}) = \frac{\mu_0}{4\pi} I \int \frac{d\mathbf{l'} \times \hat{\boldscriptr}}{\scriptr^2}
\]
where $\mu_0$ is the permiability of free space.

For surface currents
\[
    \mathbf{B}(\mathbf{r}) = \frac{\mu_0}{4\pi} \int \frac{\mathbf{K(r')} \times \hat{\boldscriptr}}{\scriptr^2} da'
\]

and volume currents
\[
    \mathbf{B}(\mathbf{r}) = \frac{\mu_0}{4\pi} \int \frac{\mathbf{J(r')} \times \hat{\boldscriptr}}{\scriptr^2} da'
\]

Biot-Savart's law is applicable for steady currents, and may not me used for individual point charges as this does not constitute 
a steady current.

\section*{Task 4.}
\begin{alphalist}
    \item Firstly, we see from the figure that $d\mathbf{l'} \times \hat{\boldscriptr}$ points out of the page with a magnitude of 
        $dl' \sin(\alpha) = dl \cos(\theta)$. We also know that $l' = s \cdot \tan(\theta)$, so
        \[
            dl' = \frac{s}{\cos^2(\theta)} d\theta
        \]
        and since $s = \scriptr \cos(\theta)$
        \[
            \frac{1}{\scriptr^2} = \frac{\cos^2(\theta)}{s^2}
        \]
        
        We can therefore find the magnetic field considering two angles $\theta_1$ and $\theta_2$ as
        \begin{align*}
            B &= \frac{\mu_0 I}{4\pi} \int_{\theta_1}^{\theta_2} \frac{\cos^2(\theta)}{s^2} \cdot \frac{s}{\cos^2(\theta)} \cos(\theta) d\theta \\
              &= \frac{\mu_0 I}{4\pi} \left(\sin(\theta_2) - \sin(\theta_1)\right)         
        \end{align*}

        If we now consider the contribution from the entire infinite wire we get that $\theta$ ranges from $-\pi/2$ to $\pi/2$, so
        \[
            B = \frac{\mu_0 I}{2\pi s}
        \]
        and since it curls around the wire in accordance with the right-hand rule we get
        \[
            \mathbf{B} = \frac{\mu_0 I}{2\pi s} \hat{\mathbf{\phi}}
        \]

    \item The field from the first wire experienced at the second wire simply becomes
        \[
            B = \frac{\mu_0 I}{2\pi d}
        \]

    \item The Lorentz force between the two wires becomes 
        \[
            F = 3I \cdot \frac{\mu_0 I}{2\pi d} \int dl
        \]
\end{alphalist}

\section*{Task 5.}
\begin{alphalist}
    \item If we have some wire carrying a current $I$ the line integral of the magnetic field $\mathbf{B}$ around some 
        loop enclosing the wire is
        \[
            \oint \mathbf{B} \cdot d\mathbf{l} = \frac{\mu_0I}{2\pi} \oint \frac{1}{s} d\phi = \frac{\mu_0 I}{2\pi} \int_0^{2\pi} d\phi = \mu_0 I
        \]
        Now the current $I$ can be expressed in terms of the volume charge density $\mathbf{J}$, where
        \[
            I = \int \mathbf{J} \cdot d\mathbf{a}
        \]
        Stokes theorem on the line integral over the magnetic field therefore gives that
        \[
            \int \left(\nabla \times \mathbf{B}\right) \cdot d\mathbf{a} = \mu_0 \int \mathbf{J} \cdot d\mathbf{a}
        \]
        The integrands have to be equal, so
        \[
            \nabla \times \mathbf{B} = \mu_0 \mathbf{J}
        \]
        which is Ampère's law in differential form.
\end{alphalist}

\end{document}
