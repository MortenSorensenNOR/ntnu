\documentclass[a4paper,11pt]{article}
\usepackage{packages}

\begin{document}

%Headingdel:---------------------------------------------
\topmargin -1.5cm
\makebox[\textwidth][s]{
    \begin{minipage}[c]{0.25\textwidth}
        \includegraphics[width=2.0cm]{Bilder/ntnu_logo.png}  
    \end{minipage}
    \begin{minipage}[c]{0.75\textwidth}
        \huge{\textbf{TFE4120  Electromagnetism}} \\
        \Large{Exercise 7  ---  Morten Sørensen, \large{\color{black!75!white}\today}}
    \end{minipage}
}
\vspace{0.75cm}
\normalsize


\section*{Task 1.}
For a given radius $r \neq R$ we have that
\begin{align*}
    \oint E \cdot d\mathbf{a} = \oint |E| da = E \cdot 4\pi r^2 = \frac{1}{\epsilon_0} Q_{enc}
\end{align*}
For $r < R$ we get $E = 0$ since no charges are enclosed within, since all charges are distributed along the shell outside it.
For $r > R$ we get 
\[
    E = \frac{\sigma R^2}{\epsilon_0 r^2}\mathbf{\hat{r}}
\]

\section*{Task 2.}
Again we have for a radius $r$ away from the center of the cyliner
\[
    \oint E \cdot d\mathbf{a} = \oint |E| da = E \cdot 2\pi r \cdot l = \frac{1}{\epsilon_0}Q_{enc} = \frac{2 \pi \sigma R \cdot l}{\epsilon_0}
\]
For $r < R$, $Q_{enc} = 0$, so $E = 0$. For $r > R$ we get
\[
    E = \frac{2\pi \sigma R}{\epsilon_0 r}
\]

\section*{Task 3.}
\begin{romanlist}
    \item For a sphere we imagine a larger sphere with radius $r > R$. Since all the chages are now enclosed within this new 
        large sphere, we may extract $E$ due to symmetry. This makes the integral easy.
    \item For a cylinder we imagine a larger cylinder around it. From here the same logic applies as for the sphere.
    \item For an infinite plane we imagine a small box with some upper and lower area $A$, where, since the field only points normally 
        to the plane, the sides of the box do not matter. The total enclosed charge is proportional to the surface area of the top and bottom 
        side $A$, so we may extract $\int E \cdot da = 2A|E|$.
    \item We do the same as for the solid sphere.
\end{romanlist}

\section*{Task 4.}
\begin{romanlist}
    \item We have
        \[
            \int E \cdot d\mathbf{a} = 2A |E| = \frac{1}{\epsilon}\sigma A
        \]
        This gives us
        \[
            E = \frac{\sigma}{2\epsilon_0} \mathbf{\hat{n}}
        \]
    \item Again we have that
        \[
            \int E \cdot d\mathbf{a} = 4A |E| = \frac{2\sigma A}{\epsilon}
        \]
        This still gives
        \[
            E = \frac{\sigma}{2\epsilon_0} \mathbf{\hat{n}}
        \]
\end{romanlist}
The field of the infinite plane is independent of the distance from the plane.

\section*{Task 5.}
\begin{romanlist}
    \item Electric potential is a measure of the electric field strength along a line from some predefined reference point 
        to a point $\mathbf{r}$. The potential difference between two points therefore is the difference of the potentials of the two points.
        It stems from the fact that the electrif field $E$ has no curl, i.e. $\oint E \cdot d\mathbf{l} = 0$, therefore the path from the
        reference point to the point $\mathbf{r}$ does not depend upon the path $\mathbf{l}$, so therefore we may define the 
        potential as we have.
    \item \hphantom{}
        \begin{alphalist}
            \item As found previously, the field strength inside the sphere shell is 0, and outside is
                \[
                    E = \frac{1}{4\pi\epsilon_0}\frac{2q}{r^2}\mathbf{\hat{r}}
                \]
                for some $r > R$. We therefore have that outside the sphere ($r > R$) we have that
                \[
                    V(r) = -\int_{\mathcal{O}}^{\mathbf{r}} E \cdot d\mathbf{l} = \left-\frac{1}{4\pi\epsilon_0}\frac{2q}{r'}\right|_{\infty}^{r} = \frac{1}{4\pi \epsilon_0}\frac{2q}{r}
                \]
                Inside the sphere ($r < R$) we have two integrals, one from infinity to $R$, and the other one from $R$ to $r$.
                \[
                    V(r) = -\frac{1}{4\pi \epsilon_0}\int_{\infty}^{R} \frac{2a}{{r'^2}}dr' - \int_{R}^{r} 0 dr' = \frac{1}{4\pi \epsilon_0}\frac{2q}{R}
                \]
            \item 
        \end{alphalist}
\end{romanlist}

\section*{Task 6.}
\begin{romanlist}
    \item 
    \item 
\end{romanlist}

\end{document}
