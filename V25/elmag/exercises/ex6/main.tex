\documentclass[a4paper,11pt]{article}
\usepackage{packages}

\begin{document}

%Headingdel:---------------------------------------------
\topmargin -1.5cm
\makebox[\textwidth][s]{
    \begin{minipage}[c]{0.25\textwidth}
        \includegraphics[width=2.0cm]{Bilder/ntnu_logo.png}  
    \end{minipage}
    \begin{minipage}[c]{0.75\textwidth}
        \huge{\textbf{TFE4120  Electromagnetism}} \\
        \Large{Exercise 7  ---  Morten Sørensen, \large{\color{black!75!white}\today}}
    \end{minipage}
}
\vspace{0.75cm}
\normalsize


\section*{Task 1.}
\begin{alphalist}
    \item If there were to be an electric field inside the conductor, this field would promote electrons to redistribute themselfs
        within the conductor, thereby canceling out the electric field.
    \item From Gauss' law $\nabla \cdot E = \rho/\epsilon_0$. If the field is zero inside the conductor, then the charge density also must be 0.
    \item If the charge $+q$ is somewhere within the sphere's cavity, then this would induce 
        a charge $-q$ to surround the cavity in order to cancel out the charge, leaving a net zero effect from the cavity. This, though, in turn leaves 
        a net positive charge on the rest of the conductor of $+q$, which is now free to distribute itself uniformly on the surface of the sphere (see 1a for why only surface).
        Since it is uniformly distributed on the surface, we may state that the field outside the conductor is independent of the shape and size of the cavity.
\end{alphalist}

\section*{Task 2.}
Remember from an earlier exercise that the field strength $E$ from an infinite charged plane with charge density $\sigma$ is given as
\[
    E = \frac{\sigma}{2\epsilon_0}\mathbf{\hat{n}}
\]
Here the field strength is independent of the distance to the surface, where $\mathbf{\hat{n}}$ points away from each of the planes, i.e.
for the positive plane, in region (i) $\mathbf{\hat{n}}$ points upwards and in region (ii) and (iii) it points downwards.
The field from the two planes will therefore point away from each other in region (i) and (iii), having equal strength, meaning the net 
field strenght in those regions being 0. In region (ii) they both point downwards, with a field strength of
\[
    E = \frac{\sigma}{\epsilon_0}\mathbf{\hat{n}}
\]
where $\mathbf{\hat{n}}$ points downwards.

\section*{Task 3.}
\begin{alphalist}
    \item The charge densities are $+\sigma = \frac{+Q}{A}$ and $-\sigma = \frac{-Q}{A}$.
    \item We may, as in task 2., assume that the field outside the two conductive plates is 0. The potential difference between the two plates
        therefore is simply equal to the distance $d$ times the field strength inbetween the plates, i.e.
        \[
            E = \frac{Q}{A\epsilon}
        \]
        so the potential difference becomes
        \[
            V = \frac{Qd}{A\epsilon_0}
        \]
        The exact same question was part of the lab preperation work, so doing the integral felt unnecessary.
    \item The capacitance $C = \frac{Q}{V}$, so the capacitance for the two conductors is
        \[
            C = \frac{A\epsilon_0}{d}
        \]
\end{alphalist}

\section*{Task 4.}
\begin{alphalist}
    \item The field between the conductors is given by Gauss' law for a given length $L$ of cable with charge $Q$ on the inner conductor is
        \[
            \int E \cdot d\mathbf{a} = E \cdot 2\pi s \cdot L = \frac{Q}{\epsilon_0}
        \] 
        This implies that 
        \[
            E = \frac{Q}{2\pi L \epsilon_0}\frac{1}{s} \mathbf{\hat{s}}
        \]
    \item The potential differnece between the conductors is 
        \[
            V(b) - V(a) = -\int_{a}^{b} E \cdot d\mathbf{l} = -\frac{Q}{2\pi L\epsilon_0} \int_{a}^{b}\frac{1}{s} ds = -\frac{Q}{2\pi L\epsilon_0} \ln\left(\frac{b}{a}\right)
        \]
        We may drop the negative sign here.
    \item Since $C = \frac{Q}{V}$ we have that
        \[
            C = \frac{2\pi\epsilon_0}{\ln\left(\frac{b}{a}\right)}
        \]
        per unit length $L$.
    \item The ratio becomes
        \[
            \frac{b}{a} = e^{\frac{2\pi\epsilon_0}{C}} \approx 16
        \]
\end{alphalist}

\end{document}
