\documentclass[a4paper,11pt]{article}
\usepackage{packages}

\begin{document}

%Headingdel:---------------------------------------------
\topmargin -1.5cm
\makebox[\textwidth][s]{
    \begin{minipage}[c]{0.25\textwidth}
        \includegraphics[width=2.0cm]{Bilder/ntnu_logo.png}  
    \end{minipage}
    \begin{minipage}[c]{0.75\textwidth}
        \huge{\textbf{TFE4120  Electromagnetism}} \\
        \Large{Exercise 7  ---  Morten Sørensen, \large{\color{black!75!white}\today}}
    \end{minipage}
}
\vspace{0.75cm}
\normalsize


\section*{Task 1.}
We have the following pairing
\[
    ( A, v_{ii} ), ( B, v_{iii} ), ( C, v_{i} ), ( D, v_{iv} )
\]

\section*{Task 2.}
\begin{enumerate}
    \item We have given the function $\mathbf{v} = xy\mathbf{\hat{x}} + (x^2 + 1)\mathbf{\hat{y}}$. For path
        (1) we have that the line integral evaluates to
        \[
            \int_{a}^{b}{\mathbf{v} \cdot d\mathbf{l}} = \int_{1}^{2}{x dx} + \int_{1}^{3}{5 dy} = 1 - \frac{1}{2} + 15 - 5 = 10.5
        \]
        For path (2) we have that $y = 2x - 1$, so
        \[
            \int_{a}^{b}{\mathbf{v} \cdot d\mathbf{l}} = \int_{1}^{2}{2x^2 - x + x^2 + 1 dx} = \left[x^3 -\frac{x^2}{2} + x \right]_{1}^{2} = 6.5
        \]
        Lastly along (3) we have
        \[
            \int_{a}^{b}{\mathbf{v} \cdot d\mathbf{l}} = \int_{1}^{3}{2 dy} + \int_{1}^{2}{3x dy} = 6 - 2 + 6 - \frac{3}{2} = \frac{17}{2}
        \]
    \item For (I + II) we calculate the closed-loop integral as the line integral along the initial path minus the return path. So for (I) we have
        that 
        \[
            \oint_{a}^{b}{\mathbf{v} \cdot d\mathbf{l}} = 10.5 - 6.5 = 4
        \]
        and for (II) we get
        \[
            \oint_{a}^{b}{\mathbf{v} \cdot d\mathbf{l}} = 8.5 - 6.5 = 2
        \]
\end{enumerate}

\section*{Task 3.}
\begin{enumerate}
    \item Flux is a measure of how much is flowing through some imaginary surface, like water flowing through some crosssection of a pipe.
    \item We have 
        \[
            \mathbf{f} = 2z\mathbf{\hat{x}} + 3x\mathbf{\hat{y}} + 5x^2yz\mathbf{\hat{z}}
        \]
        In order to find the flux through the surfaces we consider one surface at a time. In order to aid in calculating 
        all the fluxes, we numerate the surfaces as in figure (A) and let that persist for (B) and (C). The surface 
        pointed at by the red arrow in figure (A) therefore would be surface (vi). We start by calculating the flux
        through each surface. \\
        (i):
        \[
            \Phi_{i} = \int{\mathbf{f} \cdot d\mathbf{a}} = \int_{0}^{2}\int_{0}^{2} 2z dy dz = \int_{0}^{2}4z dz = 8
        \]

        (ii):
        \[
            \Phi_{ii} = \int{\mathbf{f} \cdot d\mathbf{a}} = \int_{0}^{2}\int_{0}^{2} -2z dy dz = \int_{0}^{2}-4z dz = -8
        \]

        (iii):
        \[
            \Phi_{iii} = \int{\mathbf{f} \cdot d\mathbf{a}} = \int_{0}^{2}\int_{0}^{2}10x^2y dy dx = \int_{0}^{2}20x^2 dx = \frac{20}{3} \cdot 8 = \frac{160}{3}
        \]

        (iv):
        \[
            \Phi_{iv} = \int{\mathbf{f} \cdot d\mathbf{a}} = \int_{0}^{2}\int_{0}^{2}0 dy dx = 0
        \]

        (v):
        \[
            \Phi_{iv} = \int{\mathbf{f} \cdot d\mathbf{a}} = \int_{0}^{2}\int_{0}^{2}3x dz dx = \int_{0}^{2}6x dx = \left3x^2 \right|_{0}^{2} = 12
        \]

        (vi):
        \[
            \Phi_{iv} = \int{\mathbf{f} \cdot d\mathbf{a}} = \int_{0}^{2}\int_{0}^{2}-3x dz dx = \int_{0}^{2}-6x dx = \left-3x^2 \right|_{0}^{2} = -12
        \]

        That means that the total flux for each of the cases is as follows: \\
        (A): $\Phi = 8 - 8 + \frac{160}{3} + 0 + 12 = \frac{196}{3}$ \\
        (B): $\Phi = -8 + \frac{160}{3} + 0 + 12 - 12 = \frac{136}{3}$\\
        (C): $\Phi = 8 - 8 + 0 + 12 - 12 = 0$

\end{enumerate}

\section*{Task 4.}
The volume of a sphere can be calculated as
\begin{align*}
    V = \int{d\tau} &= \int_{0}^{2\pi}\int_{0}^{\pi}\int_{0}^{R}{r^2 \sin\theta dr d\theta d\phi} \\
                    &= \int_{0}^{2\pi}\int_{0}^{\pi}{\frac{R^3}{3} \sin\theta d\theta d\phi} \\
                    &= \int_{0}^{2\pi}{\left-\frac{R^3}{3} \cos\theta \right|_{0}^{\pi} d\phi} \\
                    &= \int_{0}^{2\pi}{\frac{2R^3}{3} d\phi} \\
                    &= \left\frac{2R^3 \cdot \phi}{3} \right|_{0}^{2\pi} = \frac{4\pi R^3}{3}
\end{align*}


\section*{Task 5.}
We first convert the function $\rho$ from cartesian coordinates to cylindrical coordinates as
\[
    \rho(s, \phi, z) = (s\cos(\phi))^2 + (s\sin(\phi))^2 = s^2
\]
We then have that in cylindrical coordinates $d\tau = s ds d\phi dz$, so the volume integral of the charge 
density $\phi$ for this cylinder becomes
\begin{align*}
    \int_{\frac{\pi}{2}}^{2\pi}\int_{0}^{5}\int_{0}^{2}s^2 \cdot s ds dz d\phi &= \int_{\frac{\pi}{2}}^{2\pi}\int_{0}^{5} 4 dz d\phi = \int_{\frac{\pi}{2}}^{2\pi}20 d\phi\\
                                                                               &= 20 \cdot \left(2\pi - \frac{\pi}{2}\right) = 30\pi
\end{align*}

\end{document}
