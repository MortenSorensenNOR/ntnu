\documentclass[a4paper,11pt]{article}
\usepackage{packages}

\begin{document}

%Headingdel:---------------------------------------------
\topmargin -1.5cm
\makebox[\textwidth][s]{
    \begin{minipage}[c]{0.25\textwidth}
        \includegraphics[width=2.0cm]{Bilder/ntnu_logo.png}  
    \end{minipage}
    \begin{minipage}[c]{0.75\textwidth}
        \huge{\textbf{TFE4120  Electromagnetism}} \\
        \Large{Exercise 7  ---  Morten Sørensen, \large{\color{black!75!white}\today}}
    \end{minipage}
}
\vspace{0.75cm}
\normalsize


\section*{Task 1.}
\begin{enumerate}
    \item If $\mathbf{B} = \mathbf{0}$ then 
        \begin{align*}
            F = q(\mathbf{E} + \mathbf{v} \times \mathbf{B}) = q \mathbf{E}
        \end{align*}
        I.e. the direction of $\mathbf{F}$ is equal to the direction of $\mathbf{E}$, or pointing 
        directly anti-parallel to it, depending on the sign of $q$.
    \item If $\mathbf{E} = \mathbf{0}$, then
        \begin{align*}
            F = q(\mathbf{E} + \mathbf{v} \times \mathbf{B}) = q (\mathbf{v} \times \mathbf{B})
        \end{align*}
        meaning that $\mathbf{F}$ points perpendicular to $\mathbf{v}$ and $\mathbf{B}$ in either direction
        depending on the sign of $q$.
\end{enumerate}

\section*{Task 2.}
We have $A = B + \lambda C$. That gives
\[
    A \times C = (B + \lambda C) \times C = B \times C + \lambda C \times C = B \times C
\]
since $C \times C = \|C\| \cdot \|C\| \cdot \sin(\theta) \mathbf{n}$, and since the vectors are equal, $\theta = 0$.

\section*{Task 3.}
\begin{enumerate}
    \item
        \[
            (1, 0, 1) \cdot (0, 1, 1) = 1 = \sqrt{2} \cdot \sqrt{2} \cdot \cos(\theta) \implies \theta = \cos^{-1}\left(\frac{1}{2}\right) = \frac{\pi}{3}
        \]
    \item 
        \[
            (1, 1, 0) \cdot (1, 1, 1) = 2 = \sqrt{2} \cdot \sqrt{3} \cdot \cos(\theta) \implies \theta = \cos^{-1}\left(\frac{\sqrt{2}}{\sqrt{3}}\right) \approx 35.26^{\circ}
        \]
    \item 
        \[
            (1, 1, -1) \cdot (1, 1, 1) = 1 = \sqrt{3} \cdot \sqrt{3} \cdot \cos(\theta) \implies \theta = \cos^{-1}\left(\frac{1}{3}\right) \approx 70.55^{\circ}
        \]
\end{enumerate}

\section*{Task 4.}
\begin{enumerate}
    \item We create two new vectors $\mathbf{A}$ and $\mathbf{B}$ where $\mathbf{A} = (0, -1, 2)^{T}$ 
        and $\mathbf{B} = (3, -1, 0)^{T}$. We then have that the normalvector $\mathbf{n}$ is given as 
        \[
            \mathbf{A} \cross \mathbf{B} = (2, 6, 3)^{T}
        \]
        the length of which is $\sqrt{2^2 + 6^2 + 3^2} = 7$, so
        \[
            \mathbf{\hat{n}} = \left(\frac{2}{7}, \frac{6}{7}, \frac{3}{7}\right)^{T}
        \]
    \item For the first vector $\mathbf{\hat{n}_i}$ it is trivially equal to $(0, 0, 1)^{T}$.
        We know from cylindrical coordinates that the unit vector pointing along the radial is given as
        \[
            \mathbf{\hat{n}_{ii}} = (\cos{\theta}, \sin{\theta}, 0)
        \]
        for $\theta$ equal to the angle away from the x-axis.
\end{enumerate}

\section*{Task 5.}
\begin{enumerate}
    \item We get 
        \begin{itemize}
            \item 
                \[
                    \nabla v_1 = (3x^2y^2z, 2yx^3z, y^2x^3)^{T}
                \]
            \item 
                \[
                    \nabla v_2 = (3x^2 + y, x, 4z^3)^{T}
                \]
            \item 
                \[
                    \nabla v_3 = (4x, z, y)^{T}
                \]
        \end{itemize}
    \item For the divergence we get
        \begin{itemize}
            \item 
                \[
                    \nabla \cdot v_1 = zy^3
                \]
            \item 
                \[
                    \nabla \cdot v_2 = 2x + 2z
                \]
            \item 
                \[
                    \nabla \cdot v_2 = 1
                \]
        \end{itemize}
\end{enumerate}

\section*{Task 6.}
\begin{enumerate}
    \item 
        \[
            \nabla \times v_1 = (-2z, xy^3, z^3-3y^2xz)^{T}
        \]
    \item 
        \[
            \nabla \times v_2 = (3, 0, 0)^{T}
        \]
    \item 
        \[
            \nabla \times v_3 = (10xy-2, -5y^2, 0)^{T}
        \]
\end{enumerate}

\end{document}
