\documentclass[a4paper,11pt]{article}
\usepackage{packages}

\begin{document}

%Headingdel:---------------------------------------------
\topmargin -1.5cm
\makebox[\textwidth][s]{
    \begin{minipage}[c]{0.25\textwidth}
        \includegraphics[width=2.0cm]{Bilder/ntnu_logo.png}  
    \end{minipage}
    \begin{minipage}[c]{0.75\textwidth}
        \huge{\textbf{TFE4120  Electromagnetism}} \\
        \Large{Exercise 7  ---  Morten Sørensen, \large{\color{black!75!white}\today}}
    \end{minipage}
}
\vspace{0.75cm}
\normalsize


\section*{Task 1.}
\begin{alphalist}
    \item Using Gauss' Law with electric displacement we have that 
        \[
            \oint_{V} \mathbf{D} \cdot d\mathbf{a} = Q_{\text{enc}} = q
        \]
        So, for $r < R$ we have that
        \[
            D_{r} \cdot 4\pi r^2 = q \implies D_r = \frac{q}{4\pi r^2}
        \]
        We know that, since this situation is spherically symmetric, that
        \[
            \mathbf{D} = \epsilon \mathbf{E}
        \]
        i.e.
        \[
            \mathbf{E} = \frac{q}{4\pi \epsilon r^2}\mathbf{\hat{r}}
        \]
        For the field outside the dielectric $\epsilon$ becomes $\epsilon_0$ (i think).
    \item We have that for a linear dielectric material
        \[
            \mathbf{P} = \epsilon_0 \chi_e \mathbf{E} = \frac{\epsilon_0 \chi_e q}{4\pi \epsilon r^2}\mathbf{\hat{r}}
        \]
    \item The average bound volume charge in the dielectric is
        \[
            \rho_b = - \nabla \cdot \mathbf{P} = - \frac{\epsilon_0 \chi_e q}{4\pi \epsilon} \cdot \left(\nabla \cdot \left(\frac{\mathbf{\hat{r}}}{r^2}\right)\right) \stackrel{\mathclap{\normalfont\mbox{(1.99)}}}{=\joinrel=\joinrel=} -\frac{\chi_e q}{\epsilon_r} \delta^3(\mathbf{r})
        \]
        The average surface charge is 
        \[
            \sigma_b = \mathbf{P} \cdot \mathbf{\hat{n}} = \frac{\epsilon_0 \chi_e q}{4\pi \epsilon R^2}
        \]
    \item The total bound volume charge is
        \[
            q_{\text{vb}} = \oint_{\mathcal{V}} \rho_b d\tau = -\frac{\chi_e q}{e_r} \oint \delta^3(\mathbf{r}) d\tau = -\frac{\chi_e q}{\epsilon_r}
        \]
        This charge is located at the very center of the sphere - due to the dirac-delta.

        The surface charge 
        \[
            q_{\text{sb}} = \oint_{\mathcal{A}} \sigma_b d\mathbf{a} = 4\pi R^2 \sigma_b = \frac{\chi_e q}{\epsilon_r}
        \]
\end{alphalist}

\section*{Task 2.}
\begin{alphalist}
    \item Electrostatic energy is the measure of potential energy stored in some charge configuration, whilst the total energy, 
        also includes the energy of actually polarizing the dielectric material. The total energy can be written as
        \begin{equation}
            W = \frac{1}{2} \int \mathbf{D} \cdot \mathbf{E} \:d\tau
            \label{eq:total_energy}
        \end{equation}
        whilst the electrostatic (from previous exercises) is
        \begin{equation}
            W = \frac{\epsilon_0}{2}\int E^2 \:d\tau
            \label{eq:static_energy}
        \end{equation}
    \item Start by finding $\mathbf{D}$ and $\mathbf{E}$ both inside - $r < R$ - and outside - $r > R$ - the dielectric.
        We have that 
        \[
            \oint \mathbf{D} \cdot d\mathbf{a} = Q_{\text{enc}}
        \]
        so
        \[
            D_r \cdot 4\pi r^2 = Q_{\text{enc}}
        \]
        For $r < R$ we have that
        \[
            Q_{\text{enc}} = \int \rho_f \:d\tau = \int_{0}^{r} \rho_f \cdot 4\pi r'^2 \:dr' = \frac{4\pi \rho_f r^3}{3}
        \]
        so
        \[
            \boxed{\mathbf{D}(r) = \frac{\rho_f}{3}\mathbf{r}}
        \]
        For $r > R$ we have 
        \[
            Q_{\text{enc}} = \int_{0}^{R} \rho_f \cdot 4\pi r^2 \:dr = \frac{4\pi \rho_f R^3}{3}
        \]
        so
        \[
            \boxed{\mathbf{D}(r) = \frac{\rho_f R^3}{3r^2}\mathbf{\hat{r}}}
        \]

        For the electric field $\mathbf{E}$ we use $\mathbf{D} = \epsilon \mathbf{E}$, so
        we have that for $r < R$
        \[
            \boxed{\mathbf{E}(r) = \frac{\rho_f}{3\epsilon_0\epsilon_r}\mathbf{r}}
        \]
        and $r > R$
        \[
            \boxed{\mathbf{E}(r) = \frac{\rho_f R^3}{3\epsilon_0 r^2}\mathbf{\hat{r}}}
        \]
        (for $r > R$ the effect of polarization disappears and we no longer have the $\epsilon_r$).

        Now we can find the electrostatic energy using equation (\ref{eq:static_energy})
        \begin{align*}
            W_{\text{static}} &= \frac{\epsilon_0}{2} \cdot \left[\left(\frac{\rho_f}{3\epsilon_0\epsilon_r}\right)^2 \int_0^R r^2 \cdot 4\pi r^2 dr
            + \left(\frac{\rho_f R^3}{3\epsilon_0}\right)^2 \int_R^{\infty} \frac{1}{r^4} \cdot 4\pi r^2 \:dr \right] \\
                              &= \frac{2\pi}{9\epsilon_0}\rho_f^2 R^5 \left(\frac{1}{5\epsilon_r^2} + 1\right)
        \end{align*}
        If we now insert $\rho_f = 3k$ we get that the electrostatic energy in the system is
        \[
            W_{\text{static}} = \frac{2\pi k}{\epsilon_0} R^{5}\left(\frac{1}{5\epsilon_r^2} + 1\right)
        \]

        Now the total energy can be calculated using equation (\ref{eq:total_energy})
        \begin{align*}
            W_{\text{total}} &= \frac{1}{2} \cdot \left[\left(\frac{\rho_f}{3}\right) \cdot \left(\frac{\rho_f}{3\epsilon_0\epsilon_r}\right) \cdot \int_0^R r^2 \cdot 4\pi r^2 \:dr 
            + \left(\frac{\rho_f R^3}{3}\right) \cdot \left(\frac{\rho_f R^3}{3\epsilon_0}\right) \cdot \int_{R}^{\infty} \frac{1}{r^4} \cdot 4\pi r^2 \:dr\right] \\
                             &= \frac{2\pi}{9\epsilon_0} \rho_f^2 R^5 \left(\frac{1}{5\epsilon_r} + 1\right)
        \end{align*}
        Again using $\rho_f = 3k$ we get
        \[
            W_{\text{total}} = \frac{2\pi k}{\epsilon_0} R^5 \left(\frac{1}{5\epsilon_r} + 1\right)
        \]

\end{alphalist}

\section*{Task 3.}
\begin{alphalist}
    \item A stationary charge produces an electric field $\mathbf{E}$, whilst a moving charge also produces a magnetic field $\mathbf{B}$.
    \item The Lorentz force $\mathbf{F}_{\text{mag}}$ for when
        \begin{romanlist}
            \item only a magnetic field is applied can be defined as
                \[
                    \mathbf{F}_{\text{mag}} = \mathbf{Q}(\mathbf{v} \times \mathbf{B})
                \]
                for a charge $\mathbf{Q}$ moving with a velocity $\mathbf{v}$ through a magnetic field $\mathbf{B}$.
            \item both a magnetic field and an electric field is applied can be defined as
                \[
                    \mathbf{F}_{\text{mag}} = \mathbf{Q}(E + \left(\mathbf{v} \times \mathbf{B}\right))
                \]
        \end{romanlist}
\end{alphalist}

\section*{Task 4.}
\begin{alphalist}
    \item The needle will point north, becuase there is unlikely to be any strong local magnetic field sources stronger than that of the earths
        magnetic field.
    \item Considering each case of current direction separetly
        \begin{romanlist}
            \item For point A the needle will point into the page and for B out of it. For C no direction will be stronger than any other, so
                perhaps pointing north?
            \item For point A the needle will point out of the page and for B into the page. For C the same argument follows as before.
        \end{romanlist}
    \item The Lorentz force will here point to the left for the left wire and to the right for the right one. They will therefore repell one another.
\end{alphalist}

\section*{Task 5.}
The work done by the magnetic field on an electron following some path $x(t)$ is defined as
\[
    W = \int \mathbf{F} \cdot d\mathbf{l} = \int \mathbf{F} \cdot \dot{x}(t) \:dt = \int Q(\mathbf{v} \times \mathbf{B}) \cdot \mathbf{v} \:dt
\]
Now since $(\mathbf{v} \times \mathbf{B})$ is perpendicular to both $\mathbf{v}$ and $\mathbf{B}$, the dot product
$(\mathbf{v} \times \mathbf{B}) \cdot \mathbf{v} = 0$, so the work done by the magnetic field is also 0.

\section*{Task 6.}
\begin{alphalist}
    \item 
    \item 
\end{alphalist}

\end{document}
