\documentclass[a4paper,11pt]{article}
\usepackage{packages}

\begin{document}

%Headingdel:---------------------------------------------
\topmargin -1.5cm
\makebox[\textwidth][s]{
    \begin{minipage}[c]{0.25\textwidth}
        \includegraphics[width=2.0cm]{Bilder/ntnu_logo.png}  
    \end{minipage}
    \begin{minipage}[c]{0.75\textwidth}
        \huge{\textbf{TTT4260 ESDA}} \\
        \Large{Øving 3  ---  Morten Sørensen, \large{\color{black!75!white}\today}}
    \end{minipage}
}
\vspace{0.75cm}
\normalsize


\section*{Task 1.}
\begin{itemize}
    \item Gradient represents the direction of most significant change in some scalar field.
    \item Divergence is the measure of expantion of some vectorfield.
    \item Curl is a measure of how much, and about which axis, a vectorfield rotates.
\end{itemize}

\section*{Task 2.}
\begin{alphalist}
    \item  Gauss' theorem states that for the vector field $F$
        \[
            \int_{S} F \cdot d\mathbf{S} = \int_{V} (\nabla \cdot F) d\mathbf{V}
        \]
        If we let $\mathbf{F} = \nabla \times \mathbf{v}$, then
        \[
            \int_{S} (\nabla \times v) \cdot D\mathbf{S} = \int_{V} \nabla \cdot (\nabla \times \mathbf{v}) d\mathbf{V} = 0
        \]
        since $\nabla \cdot (\nabla \times \mathbf{v}) = 0$.
    \item Stokes theorem states that
        \[
            \oint_{S}(\nabla \times \mathbf{u}) \cdot d\mathbf{a} = \oint_{\partial S} \mathbf{u} \cdot d\mathbf{l} \overset{def}{=} 0
        \]
        since $S$ is a closed surface, and the boundry therefore does not exist.
\end{alphalist}

\section*{Task 3.}
The electric field $E$ at a point $\mathbf{r}$ above the line is given as
\[
    E(\mathbf{r}) = \frac{1}{4\pi\epsilon_0}\int \frac{\lambda}{\mathcal{R}^2}\mathbf{\hat{\mathcal{R}}} dl
\]
where $\mathbf{\mathcal{R}} = \mathbf{r} - \mathbf{r}'$ is the separation vector between the segments $dl$ and the field point $P$,
which here becomes 
\[
    \mathbf{\mathcal{R}} = Z\hat{\mathbf{z}} - x\hat{\mathbf{x}}
\]
Therefore 
\[
    \mathcal{R} = \sqrt{Z^2 + x^2}
\]
and
\[
    \hat{\mathbf{\mathcal{R}}} = \frac{Z\hat{\mathbf{z}} - x\hat{\mathbf{x}}}{\sqrt{Z^2 + x^2}}
\]
We get 
\begin{align*}
    E(\mathbf{r}) &= \frac{1}{4\pi\epsilon_0} \int_{-L/2}^{L/2} \frac{\lambda}{Z^2 + x^2}\frac{Z\hat{\mathbf{z}} - x\hat{\mathbf{x}}}{\sqrt{Z^2 + x^2}} dx \\
                  &= \frac{\lambda}{4\pi\epsilon_0}\left[Z\hat{\mathbf{z}} \int_{-L/2}^{L/2} \frac{1}{(Z^2 + x^2)^{3/2}}dx - \int_{-L/2}^{L/2} \frac{x\mathbf{\hat{x}}}{(Z^2 + x^2)^{3/2}}dx\right]  \\
                  &= \frac{\lambda}{4\pi\epsilon_0} \left[ z\mathbf{\hat{z}} \cdot \left \frac{x}{Z^2 \cdot \sqrt{Z^2 + x^2}} \right|_{-L/2}^{L/2} + \left \frac{\hat{\mathbf{x}}}{\sqrt{Z^2 + x^2}}\right|_{-L/2}^{L/2}  \right] \\
                  &= \frac{\lambda}{4\pi\epsilon_0} \frac{\lambda L}{z\sqrt{Z^2 + \frac{L^2}{4}}} \mathbf{\hat{z}}
\end{align*}

\section*{Task 4.}
\begin{alphalist}
    \item The electric potential $V(\mathbf{r})$ at the point $\mathbf{r}$ is the measure of the strength of an electric field $E$ along some line from a reference point $\mathcal{O}$ to
        the point $\mathbf{r}$. Since the electric field is conservative, the path taken does not matter.
    \item 
    \begin{romanlist}
        \item The field strength inside a hollow sphere shell is 0, and outside is
            \[
                E(\mathbf{r}) = \frac{1}{4\pi\epsilon_0}\frac{2q}{r^2}\mathbf{\hat{r}}
            \]
            as we have found previously. Outside the sphere shell ($r > R$) we therefore have that, for a reference point 
            at infinity, the potential is
            \[
                V(\mathbf{r}) = -\int_{\mathcal{O}}^{\mathbf{r}} E \cdot d\mathbf{l} = -\left \frac{1}{4\pi\epsilon_0}\frac{2q}{r'} \right|_{\infty}^{r} = \frac{1}{4\pi\epsilon_0}\frac{2q}{r}
            \]
            Inside the sphere ($r < R$) we get that the potential is
            \[
                V(\mathbf{r}) = -\int_{\infty}^{R}\frac{2q}{r'^2}dr' - \int_{R}^{r} 0\: dr' = \frac{1}{4\pi\epsilon_0}\frac{2q}{R}
            \]
        \item In the case of the solid sphere the electric field outside when $r > R$ is simply the field of gaussian surface enclosing the 
            sphere as
            \[
                E(\mathbf{r}) = \frac{1}{4\pi\epsilon_0}\frac{2q}{r^2}
            \]
            For inside the sphere, the enclosed charge will be proportional to $\frac{r^3}{R^3}, so
            the field strength becomes
            \[
                E(\mathbf{r}) = \frac{1}{4\pi\epsilon_0}\frac{2qr}{R^3}
            \]

            We may then calculate the potetial, first for $r > R$:
            \[
                V(\mathbf{r}) = -\int_{\infty}^{r} \frac{1}{4\pi\epsilon_0}\frac{2q}{r'^2} dr' = -\left \frac{1}{4\pi\epsilon_0}\frac{2q}{r'} \right|_{\infty}^{r} = \frac{1}{4\pi\epsilon_0}\frac{2q}{r}
            \]
            Then for $r < R$:
            \begin{align*}
                V(\mathbf{r}) &= -\int_{\infty}^{R} \frac{1}{4\pi\epsilon_0}\frac{2q}{r'^2} dr' - \int_{R}^{r} \frac{1}{4\pi\epsilon_0}\frac{2qr'}{R^3} dr' \\
                              &= \frac{1}{4\pi\epsilon_0}\frac{2q}{R} - \frac{1}{4\pi\epsilon_0}\frac{2q(r^2 - R^2)}{2R^3} = \frac{2q}{4\pi\epsilon_0} \cdot \left[\frac{1}{R} - \frac{r^2 - R^2}{2R^3}\right]
            \end{align*}
    \end{romanlist}
\end{alphalist}

\section*{Task 5.}
\begin{alphalist}
    \item We have that $W = qV$ (assuming that we start at infinity where $V = 0$), so we start by finding the voltage at the the point.
        \[
            V = \frac{1}{4\pi\epsilon_0} \cdot \left(\frac{q}{a} + \frac{-q}{\sqrt{2}a}\right)
        \]
        And the work to bring the new charge in becomes
        \[
            W = -q \cdot \frac{1}{4\pi\epsilon_0} \cdot \left(\frac{q}{a} + \frac{-q}{\sqrt{2}a}\right) = \frac{1}{4\pi\epsilon_0} \cdot \left(\frac{q^2}{\sqrt{2}a} - \frac{q^2}{a}\right)
        \]
    \item The fourth charge has a potential in the upper right corner of
        \[
            V = \frac{1}{4\pi\epsilon_0} \cdot \left(\frac{-q}{a} + \frac{-q}{a} + \frac{q}{\sqrt{2}a} \right) = \frac{q}{4\pi\epsilon_0 a}\left(-2 + \frac{1}{\sqrt{2}}\right)
        \]
        So the work to bring it in becomes
        \[
            W = qV = \frac{q^2}{4\pi\epsilon_0 a}\left(-2 + \frac{1}{\sqrt{2}}\right)
        \]
    \item The total work to configure it is going to be
        \[
            W = \frac{q^2}{4\pi\epsilon_0 a} \cdot \left(-1 + \frac{1}{\sqrt{2}} - 1 - 2 + \frac{1}{\sqrt{2}}\right)
        \]
        (First charge requires no work, second one only affected by the first one, and then sum the other two).
\end{alphalist}

\section*{Task 6.}
The electrostatic work $W$ for a volume $V$ with charge density $\rho$ is equal to
\[
    W = \frac{1}{2}\int \rho V d\tau
\]
We may rewrite the expression onto a form that is only dependent on $E$ by first setting
\[
    \rho = \epsilon_0 \cdot (\nabla \cdot E)
\]
This comes from Gauss' law, where 
\[
    \int_{V} \nabla \cdot E d\tau = \frac{1}{\epsilon_0}\int_{V} \rho d\tau
\]
We may then rewrite the work as
\[
    W = \frac{\epsilon_0}{2} \int (\nabla \cdot E) V d\tau
\]
If we then integrate by parts we get
\[
    W = \frac{\epsilon_0}{2} \left[ -\int_V E \cdot (\nabla V) d\tau + \oint_{S} EV \cdot d\mathbf{a} \right]
\]
We see now that $\nabla V = -E$, so
\[
    W = \frac{\epsilon_0}{2} \left[ \int_{V} E^2 d\tau + \oint_{S} EV \cdot d\mathbf{a} \right]
\]
The trick here is to see that the volume, and therefore the corresponding surface $S$, that we are integrating
over is independent of the work, so if we integrate over all space, then the surface integral goes to 0, and we are left with
\[
    \boxed{W = \frac{\epsilon_0}{2} \int E^2 \:d\tau}
\]

\section*{Task 7.}
We start with the equation derived above
\[
    W = \frac{\epsilon_0}{2} \int E^2 \:d\tau
\]
Let $E = E_1 + E_2$. We then have 
\begin{align*}
    W &= \frac{\epsilon_0}{2} \int E^2 \:d\tau \\ 
      &= \frac{\epsilon_0}{2} \int (E_1 + E_2)^2 \:d\tau \\
      &= \frac{\epsilon_0}{2} \int E_1^2 + E_2^2 + 2E_1 \cdot E_2 \:d\tau \\
      &= W_1 + W_2 + \epsilon_0 \int E_1 \cdot E_2 \:d\tau \neq W_1 + W_2
\end{align*}
Therefore the expression for work above does not follow the superposition principle. Coulumb's law on the
other hand does follow the superposition principle, as we have shown in earlier exercises.

\section*{Task 8.}
\begin{alphalist}
    \item The surface charge density of the metal sphere is given as
        \[
            \sigma_R = \frac{q}{4\pi R^2}
        \]

        Since the shell is supposed to be neutrally charged, the surface charge for the inner surface at a radius of $a$ must be 
        oppositly charged in order to keep the shell uncharged. So
        \[
            \sigma_a = \frac{-q}{4\pi a^2}
        \]
        Similarly, the outer surface of the shell must have a surface charge of
        \[
            \sigma_b = \frac{q}{4\pi b^2}
        \]
    
    \item The potential at the center of the sphere becomes
        \begin{align*}
            V(0) &= -\int_{\infty}^{0} E \cdot d\mathbf{l} = -\int_{\infty}^{b} \frac{1}{4\pi\epsilon_0}\frac{q}{r^2} dr - \int_{b}^{a} 0\: dr - \int_{a}^{R} \frac{1}{4\pi\epsilon_0}\frac{q}{r^2}\:dr - \int_{R}^{0} 0\: dr \\
                 &= \frac{q}{4\pi\epsilon_0} \cdot \left(\frac{1}{b} + \frac{1}{R} - \frac{1}{a}\right)
        \end{align*}

\end{alphalist}

\end{document}
