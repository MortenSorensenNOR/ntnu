\documentclass[a4paper,11pt]{article}
\usepackage{packages}

\begin{document}

%Headingdel:---------------------------------------------
\topmargin -1.5cm
\makebox[\textwidth][s]{
    \begin{minipage}[c]{0.25\textwidth}
        \includegraphics[width=2.0cm]{Bilder/ntnu_logo.png}  
    \end{minipage}
    \begin{minipage}[c]{0.75\textwidth}
        \huge{\textbf{TTT4260 ESDA}} \\
        \Large{Øving 3  ---  Morten Sørensen, \large{\color{black!75!white}\today}}
    \end{minipage}
}
\vspace{0.75cm}
\normalsize


\section*{Task 1.}
\begin{romanlist}
    \item Start by finding the gradient
        \[
            \nabla T = \begin{pmatrix} 5y^2, & 10xy, & 0 \end{pmatrix}
        \]

        Going along (i) and (ii) we get
        \[
            \int_{a}^{b} \nabla T \cdot d\mathbf{l} = \int_{0}^{2} 0\: dx + \int_{0}^{1} 20y\: dy = 10
        \]
        
        Using the theorem we get
        \[
            \int_{a}^{b} \nabla T \cdot d\mathbf{l} = T(B) - T(A) = 5\cdot 2 \cdot 1 - 0 = 10
        \]

        We see that the theorem holds.

    \item  Again we find the gradient
        \[
            \nabla T = \begin{pmatrix} 0, & 6y, & 0 \end{pmatrix}
        \]

        Along (i) and (ii)
        \[
            \int_{a}^{b} \nabla T \cdot d\mathbf{l} = \int_{0}^{2} 0\: dx + \int_{0}^{1} 6y\: dy = 3
        \]

        The theorem gives
        \[
            \int_{a}^{b} \nabla T \cdot d\mathbf{l} = T(B) - T(A) = 3 - 0 = 3
        \]
\end{romanlist}

\section*{Task 2.}
Start by finding $\nabla \cdot v$:
\[
    \nabla \cdot v = 1
\]

The volume integral becomes
\[
    \int \nabla \cdot v = \int_0^1 \int_0^1 \int_0^1 1 dx dy dz = 1
\]

\newpage
The surface integrals become
\begin{align}
    \int_0^1 \int_0^1 \hphantom{1 -}\:\:y\: dz dy = \hphantom{-}\frac{1}{2} \tag{i}     \\
    \int_0^1 \int_0^1 \hphantom{1  }-y\: dz dy =            -\frac{1}{2}    \tag{ii}    \\
    \int_0^1 \int_0^1 \hphantom{1 -}\:\:x\: dz dx = \hphantom{-}\frac{1}{2} \tag{iii}   \\
    \int_0^1 \int_0^1 \hphantom{1  }-x\: dz dx =            -\frac{1}{2}    \tag{vi}    \\
    \int_0^1 \int_0^1 \hphantom{1 -}\:\:x\: dy dx = \hphantom{-}\frac{1}{2} \tag{v}     \\
    \int_0^1 \int_0^1 1 - x\: dy dx = \hphantom{-}\frac{1}{2}               \tag{vi} 
\end{align}

So the total flux becomes $1$, therefore the theorem holds.

\section*{Task 3.}
Start by finding $\nabla \times v$, where $v = x^2 y \mathbf{\hat{y}} + (3zy + 1)\mathbf{\hat{z}}$:
\[
    \nabla \times v = \begin{pmatrix} 3z - 0, & 0 - 0, & 2xy - 0 \end{pmatrix} = \begin{pmatrix} 3z, & 0, & 2xy \end{pmatrix}
\]

\begin{alphalist}
    \item The surface integral gives us 
        \[
            \int \nabla \times v \cdot d\Sigma = \int_0^1 \int_0^1 3z\: dy dz = \frac{3}{2}
        \]

        The $\mathbf{\hat{y}}$ component of $v$ is independen of $z$, so the line integral over (i) and (iii) cancel out.
        The line integral for (ii) is:
        \[
            \int_0^1 3z + 1 dz = \frac{3}{2} + 1
        \]
        And (iv):
        \[
            \int_1^0 1 dz = -1
        \]

        So the sum of the line integrals is $\frac{3}{2}$, and Stokes theorem holds.

    \item We again have that
        \[
            \int \nabla \times v \cdot d\Sigma = \int_0^2 \int_0^2 3z\: dy dz = 12
        \]

        The cancelation of line segments (i) and (iii) still holds. We get that (ii) gives:
        \[
            \int_0^2 6z + 1 dz = 12 + 2
        \]
        And (vi):
        \[
            \int_0^2 -1 dz = -2
        \]

        So again Stokes theorem holds.
\end{alphalist}

\section*{Task 4.}
\begin{alphalist}
    \item Gauss' theorem tells us that the flux through any closed surface, say for instance a sphere or cylinder, is equal to 
        the total charge enclosed within. A charge outside the surface will not contribute because it's field lines will enter on one side 
        and exit on another, contributing nothing to the total flux.
    \item 
        Consider the vectorfield $\mathbf{v} = r^2 \mathbf{\hat{r}}$. The volume integral for a spehere with radius $R$ becomes
        \begin{align*}
            \int \nabla \cdot \mathbf{v} d\tau &= \int_0^{R} \int_0^\pi \int_0^{2\pi} \frac{1}{r^2} \frac{\partial (r^2 \cdot r^2)}{\partial r} \cdot r^2 \sin(\theta)\: d\phi d\theta dr \\
                                               &= \int_0^{R} \int_0^\pi \int_0^{2\pi} 4r \cdot r^2 \sin(\theta)\: d\phi d\theta dr \\
                                               &= \int_0^{R} \int_0^\pi 8 \pi r^3 \sin(\theta)\: d\theta dr \\
                                               &= \int_0^{R} 16 \pi r^3\: dr = 4 \pi R^4
        \end{align*}
        The surface integral over the same vectorfield becomes
        \[
            \oint \mathbf{v} \cdot da = \int_{0}^{2\pi} \int_{0}^{\pi} R^2 \cdot R^2 \sin(\theta) d\theta d\phi = 4\pi R^4
        \]
        We therefore have that the divergence theorem golds for spherical coordinates.
\end{alphalist}

\section*{Task 5.}
The delta function, or dirac delta function, is essentially an infinetly high and narrow spike with are 1. The function does not obey the divergence theorem,
as shown in the lecture.

Given $\mathbf{v_r} = \frac{1}{r^2}\mathbf{\hat{r}}$, applying the divergence theorem, the left hand side gives us
\begin{align*}
    \int \nabla \cdot \mathbf{v} d\tau &= \int_0^{R} \int_{0}^{2\pi} \int_{0}^{\pi} \left(\frac{1}{r^2} \frac{\partial \left(r^2 \cdot \frac{1}{r^2}\right)}{\partial r}\right) \cdot r^2 \sin(\theta)\: d\phi d\theta dr \\
                                       &= \int_0^{R} \int_{0}^{2\pi} \int_{0}^{\pi} \left(\frac{1}{r^2} \cdot 0\right) r^2 \sin(\theta)\: d\phi d\theta dr = 0 \\
\end{align*}

Then the right hand side gives
\[
    \oint \mathbf{v} \cdot da = \int_{0}^{2\pi} \int_{0}^{\pi} \frac{1}{r^2} \cdot r^2 \sin(\theta) d\phi d\theta = 4\pi
\]

Therefore we see that the divergence theorem does not hold for the dirac delta function. We may however define a function 
$\delta(\mathbf{x})$, where $\mathbf{x} \in \mathbb{R}^3$, such that
\[
    \int_{-\infty}^{\infty}\int_{-\infty}^{\infty}\int_{-\infty}^{\infty} \delta(\mathbf{x}) dxdydz = \int_{-\infty}^{\infty}\int_{-\infty}^{\infty}\int_{-\infty}^{\infty} \delta(x)\delta(y)\delta(z) dxdydz = 1
\]

\section*{Task 6.}
\begin{romanlist}
    \item A source charge is a charge for which we want to measure or calculate the electric field of. The test charge is an imagnined charge 
        of which we assume the charge is negligible, such that the measure of force upon it may be approximated as the strength of the field in the 
        position of the test charge. For electrostatics the source charge should not be moving and the test charge should have a charge that is 
        infinitesimally small. Presumably, for measuring the electric field strength, the magnetic field ought to be zero.
    \item Coulomb's law in vector form can be written as the force $\mathbf{F}$ acting between two charges $q$ and $Q$ as
        \[
            \mathbf{F} = \frac{1}{4\pi \epsilon_0}\frac{qQ}{n^2}\mathbf{\hat{n}}
        \]
        where $\mathbf{n}$ is the seperation vector between the charges and $\epsilon_0$ is the permittivity of free space.
        The law gives us an expression for the force that two charges act uppon the other.
    \item The principle of superposition in the context of electrostatics tells us that in a system with multiple charges
        the force felt by each charge can be calculated as the cumulative sum of the forces for each other charge on it's own. That is,
        with $k$ charges, the total force felt by any one charge will be
        \[
            \mathbf{F_i} = \sum_{i \neq k}^{k} F_{ik}
        \]
        where $F_i$ is the total force on charge $i$ and $F_{ik}$ is the force between charge $i$ and $k$. The same 
        notion of summin up individual contributions also holds for potential fields as well.
    \item In descrete form, the electric field $E(\mathbf{r})$ can be written as
        \[
            E(\mathbf{r}) = \frac{1}{4\pi \epsilon_0} \sum_{i = 1}^{n} \frac{q_i}{n_i^2}\mathbf{\hat{n_i}}
        \]
        where $\mathbf{r}$ is a position, say in $\mathbb{R}^3$, $\mathbf{n}$ is the seperation vector between $\mathbf{r}$ and the 
        charge position $\mathbf{r'}$, and, as before, $\epsilon_0$ is the permittivity of free space. We may understand the field as the force per 
        unit charge, that is, for a given charge $Q$ the force acted upon that charge in the electric field $E(\mathbf{r})$ is 
        \[
            \mathbf{F} = Q\mathbf{E}(\mathbf{r})
        \]
        where $\mathbf{r}$ is the position of the charge $Q$.
    \item No.
\end{romanlist}

\section*{Task 7.}
\begin{romanlist}
    \item Start by finding a common expression for the force in the vertical ($z$) direction. Since the two charges are equedistant in the $x$ direction
        to the point $P$, and given that the charges are equal but opposite, we have
        \[
            F_z = \frac{1}{4\pi \epsilon_0} \left(
                \frac{q}{n^2}\mathbf{\hat{z}} + \frac{-q}{n^2}\mathbf{\hat{z}}
            \right) = 0
        \]
        where $\mathbf{n}$ is the seperation vector from $q$ to $P$.
        For the horizontal we have
        \[
            F_x = \frac{1}{4\pi \epsilon_0} \left(
                \frac{q}{n^2} \cdot \mathbf{\hat{x}} + \frac{-q}{n^2} \cdot (-\mathbf{\hat{x}})
            \right) = \frac{1}{2\pi \epsilon_0} \frac{q}{n^2} \cdot \mathbf{\hat{x}} = \frac{1}{4\pi \epsilon_0} \frac{dq}{\left[z^2 + (d/2)^2\right]^{3/2}} \cdot \mathbf{\hat{x}}
        \]
    \item This yields the same result, but with the force component in the $x$-direction pointing the other way.
\end{romanlist}

\section*{Task 8.}
\begin{romanlist}
    \item 
    \begin{align*}
        E(P) &= \frac{1}{4\pi \epsilon_0} \int_{0}^{2\pi} \frac{\lambda z}{[r^2 + z^2]^{3/2}} \mathbf{\hat{z}} r d\theta \\
             &= \boxed{\frac{1}{2\epsilon_0} \frac{\lambda zr}{[r^2 + z^2]^{3/2}} \mathbf{\hat{z}}}
    \end{align*}

    \item 
    \begin{align*}
        E(P) &= \frac{1}{4\pi \epsilon_0} \int_{0}^{r} \frac{\sigma z}{[r'^2 + z^2]^{3/2}} 2\pi r' dr' \\
             &= \frac{2\pi \sigma z}{4\pi \epsilon_0} \int_{0}^{r} \frac{r'}{[r'^2 + z^2]^{3/2}} dr'
    \end{align*}

    Using $u = r^2 + z^2$, which means that $du = 2rdr$ (constant z), which gives
    \begin{align*}
        E(P) &= \frac{2\pi \sigma z}{4\pi \epsilon_0} \int_{0}^{r} \frac{1}{2u^{3/2}} du \\        
             &= \frac{2\pi \sigma z}{4\pi \epsilon_0} \cdot \left[-\frac{1}{\sqrt{r'^2 + z^2}}\right]_{0}^{r} \\
             &= \frac{2\pi \sigma z}{4\pi \epsilon_0} \cdot \left(\frac{1}{z} - \frac{1}{\sqrt{r^2 + z^2}}\right) \\
             &= \boxed{\frac{\sigma}{2\epsilon_0} \cdot \left(1 - \frac{z}{\sqrt{r^2 + z^2}}\right)}
    \end{align*}
\end{romanlist}

\end{document}
