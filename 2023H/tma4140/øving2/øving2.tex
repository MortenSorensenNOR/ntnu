\documentclass{article}

% Packages
\usepackage{braket}
\usepackage{amsmath}
\usepackage{amssymb}
\usepackage{amsfonts}
\usepackage{mathtools}
\usepackage[shortlabels]{enumitem}

% Margins
\topmargin=-2.5cm
\evensidemargin=0in
\oddsidemargin=0in
\textwidth=6.5in
\textheight=9.0in
\headsep=0.25in

% Commands
\makeatletter
\newcommand{\listintertext}{\@ifstar\listintertext@\listintertext@@}
\newcommand{\listintertext@}[1]{% \listintertext*{#1}
\hspace*{-\@totalleftmargin}#1}
\newcommand{\listintertext@@}[1]{% \listintertext{#1}
\hspace{-\leftmargin}#1}
\makeatother

\title{\huge{Øving 2\\ Diskret matematikk}}
\author{Morten Sørensen}
\date{\today}

\begin{document}
\maketitle

\subsection*{Oppgave 1.}
Lar $A = \set{1, 2, 3, 4}$, og betrakter relasjonen $R = \set{\langle 1, 2 \rangle, \langle 1, 3 \rangle, \langle 1, 4 \rangle}$. 
Skal finne alle egenskapene til relasjonen $R$.

\begin{enumerate}[(a)]
	\item $R$ er ikke refleksiv, ettersom at vi ikke har at ingen $\langle a, a \rangle \in R$ for noen $a \in R$. 
	\item $R$ er heller ikke symmetrisk siden vi ikke har at $\langle a, b \rangle \in R \implies \langle b, a \rangle \in R$ for noen $a, b \in A$.
	\item $R$ er ikke transitiv siden det ikke eksisterer $\langle a, b \rangle, \langle b, c \rangle \in R$.
	\item $R$ er ikke antisymmetrisk siden det ikke finnes noen $\langle a, b \rangle, \langle b, a \rangle \in R$.
	\item $R$ er irrefleksiv siden det ikke eksisterer noen $\langle a, a \rangle$ for alle $a \in A$.
\end{enumerate}

\subsection*{Oppgave 2.}
Lar $A = \set{1, 2, 3, 4}$.
\begin{enumerate}[(a)]
	\item {
		Skal finne egenskapene til relasjonen $R = \set{\langle x, y \rangle \mid x, y \in A \text{ \& } 2 \mid (x + y)}$. Vi har at $\langle x, y \rangle \in R$ dersom 
		$x, y$ enten er begge oddetall eller begge partall.
		\begin{enumerate} [1.]
		\item {
			$R$ er refleksiv. Hvis $a \in A$ er er et oddetall, vil $a + a$ være et partall, og hvis $a$ er et partall er også $a + a$ et partall. Altså er 
			$\langle a, a \rangle \in R$ for alle $a \in A$.
		}
		\item {
			$R$ er symmetrisk. Addisjon er kommutativt, og alle $\langle a, b \rangle \in R \implies \langle b, a \rangle \in R$.
		}
		\item {
			$R$ er ikke transitiv. Det er ikke nok partalls- eller oddetallselementer for å danne en transitiv relasjon på $A$.
		}
		\item {
			$R$ er ikke anti-symmetrisk. Vi har at $\langle 1, 3 \rangle, \langle 3, 1 \rangle \in R$, men $1 \neq 3$.
		}
		\item {
			$R$ er ikke irrefleksiv. Vi har at $R$ er refleksiv.
		}
		\end{enumerate}
	}
	\item {
		Skal finne egenskapene til relasjonen $R = \set{\langle x, y \rangle \mid x, y \in A \text{ \& } 2 \mid (x \cdot y)}$. Vi har at $\langle x, y \rangle \in R$ dersom 
		enten $x$ eller $y$ er partall, eller dersom både $x$ og $y$ er partall.
		\begin{enumerate} [1.]
		\item {
			$R$ er ikke refleksiv. $\langle 1, 1 \rangle \notin R$.
		}
		\item {
			$R$ er symmetrisk. Multiplikasjon er kommutativt, så $\langle x, y \rangle \in R \implies \langle y, x \rangle \in R$.
		}
		\item {
			$R$ er ikke transitiv. $\langle 1, 2 \rangle, \langle 2, 3 \rangle \in R$, men $\langle 1, 3 \rangle \notin R$.
		}
		\item {
			$R$ er ikke anti-symmetrisk. $\langle 1, 2 \rangle, \langle 2, 1 \rangle \in R$, men $1 \neq 2$.
		}
		\item {
			$R$ er ikke irrefleksiv. $\langle 2, 2 \rangle \in R$.
		}
		\end{enumerate}
	}
\end{enumerate}

\subsection*{Oppgave 3.}
\begin{enumerate}[(a)]
	\item {
		$f$ er injektiv, ettersom at to verdier $x_1, x_2 \in \mathbb{N}$ der $f(x_1) = f(x_2)$ gir 
		$x_1 + 2 = x_2 + 2 \implies x_1 = x_2$. \\
		$f$ er ikke surjektiv ettersom at det ikke $\exists a \in \mathbb{N}$ s.a. $\langle a, 1 \rangle \in f$.
	}
	\item {
		$f$ er injektiv. Hvis vi har at to elementer $x_1, x_2 \in \mathbb{N}$ oppfyller $f(x_1) = f(x_2)$, har vi følgende 
		scenarioer:
		\begin{enumerate}[1.]
			\item {
				Både $x_1$ og $x_2$ er partall. Da er $x_1 + 1 = x_2 + 1 \implies x_1 = x_2$.
			}
			\item {
				Både $x_1$ og $x_2$ er oddetall. Da er $x_1 - 1 = x_2 - 1 \implies x_1 = x_2$.
			}
			\item {
				$x_1$ er partall og $x_2$ er oddetall. Da er $x_1 + 1 = x_2 - 1$, altså 
				$x_1 - x_2 = 2$, noe som ikke er mulig, ettersom at differansen mellom et partall og et oddetall 
				aldri kan være et partall.
			}
		Altså har vi at $f(x_1) = f(x_2) \implies x_1 = x_2$, og $f$ er injektiv.
		\end{enumerate}
		$f$ er surjektiv. La $y = f(x)$ for en vilkårlig $x$. Da er $y \in \mathbb{N}$. Dersom $x$ er et partall 
		vil $y = x + 1$ være et oddetall, og dersom $x$ er et oddetall vil $y = x - 1$ være et partall. Altså, hvis 
		$y \in \mathbb{N}$ så er $x \in \mathbb{N}$, og funksjonen er derfor surjektiv.
	}
	\item {
		Injektivitet: \\
		For to vilkårlige $x_1, x_2 \in \mathbb{N}$ der $f(x_1) = f(x_2)$ har vi følgende scenarioer
		\begin{enumerate}[1.]
			\item {
				Både $x_1$ og $x_2$ er partall. Da har vi at $x_1/2 = x_2/2 \implies x_1 = x_2$.
			}
			\item {
				Både $x_1$ og $x_2$ er oddetall. Da har vi at $(x_1-1)/2 = (x_2-1)/2 \implies x_1 = x_2$.
			}
			\item {
				$x_1$ er partall og $x_2$ er oddetall. Da har vi at 
				$$\frac{x_1}{2} = \frac{(x_2-1)}{2} \implies x_2 - x_1 = 1$$
				Altså er $x_1 \neq x_2$, og $f$ er derfor ikke injektiv.
			}
		\end{enumerate}
		Surjektivitet: Fikk ikke tid :(
		
	}
\end{enumerate}

\subsection*{Oppgave 4.}
\begin{enumerate}[(a)]
	\item {
		Ettersom at $g$ er surjektiv eksisterer det for enhver $c \in C$ en $b \in B$ s.a. $g(b) = c$,
		og ettersom at $f$ er surjektiv, eksisterer det for enhver $b \in B$ en $a \in A$ s.a. $f(a) = b$.
		Kan derfor vise at det eksisterer en $a \in A$ for enhver $c \in C$ s.a. $g(f(a)) = g(b) = c$.
		Altså må $g \circ f$ være surjektiv hvis både $g$ og $f$ er det.
	}
	\item {
		Skal bevise at dersom $g \circ f$ er injektiv, så må $f$ også være injektiv. Dersom vi først antar at 
		$f$ ikke er injektiv, så har vi at det eksisterer to elementer $x, y \in A$ s.a. $f(a) = f(b)$ uten at 
		$a = b$. Vi har da at $g \circ f(x) = g \circ f(y)$ uten at $a = b$, noe som er en motsigelse med at komposisjonen
		$g \circ f$ er injektiv. Altså må $f$ også være injektiv.
	}
	\item {
		Skal motbevise at injektivitet på $g \circ f$ impliserer injektivitet på $g$. Dersom vi lar 
		$g : \mathbb{Z} \to \mathbb{N}$ der 
		$$g(x) = x^2$$
		og $f : \mathbb{N} \to \mathbb{N}$ der 
		$$f(x) = x\text{,}$$
		så vil komposisjonen $g \circ f$ være injektiv, men funksjonen $g$ i seg selv vil ikke være det.
	}
\end{enumerate}

\subsection*{Oppgave 5.}
\begin{enumerate}[(a)]
	\item {
		Siden $M$ er en tellbar mengde finnes det en bijektiv funksjon $f : M \to \mathbb{N}$. Kan definere 
		en ny bijektiv funksjon $g : M \cup \set{42} \to \mathbb{N}$ der $g(42) = 0$ og $g(e) = f(e) + 1$ for $e \in M$.
	}
	\item {
		Ettersom at både $M$ og $N$ er tellbare mengder, eksisterer det to bijektive funksjoner $f : M \to \mathbb{N}$ og 
		$g : N \to \mathbb{N}$. Kan definere en ny funksjon $h : M \cup N \to \mathbb{N}$ s.a.
		$$h(e) = 
		\begin{cases}
			2 \cdot f(e) & e \in M \\
			2 \cdot g(e) + 1 & e \in N \\	
		\end{cases}$$
		Skal da vise at $h$ er injektiv. For to vilkårlige elementer $a, b \in M \cup N$,
		så har vi at hvis
		\begin{enumerate}[1.]
			\item {
				$a \in M$ og $b \in M$ så har vi at $h(a) = h(b)$ medfører at $2 \cdot f(a) = 2 \cdot f(b)$, og siden 
				$f$ er bijektiv, medfører dette at $a = b$.
			}
			\item {
				$a \in N$ og $b \in N$ så har vi at $h(a) = h(b)$ medfører at $2 \cdot g(a) + 1 = 2 \cdot g(b) + 1$, og siden 
				$g$ er bijektiv, medfører dette at $a = b$.
			}
			\item {
				$a \in M$ og $b \in N$ så har vi at $h(a) = h(b)$ medfører $2 \cdot f(a) = 2 \cdot g(b) + 1$, altså at 
				$f(a) - g(b)= \frac{1}{2}$, noe som er en motsigelse med at $f(a), g(b) \in \mathbb{N}$.
			}
			\item {
				$a \in N$ og $b \in M$ så har vi at $h(a) = h(b)$ medfører $2 \cdot f(b) = 2 \cdot g(a) + 1$, altså at 
				$f(b) - g(a)= \frac{1}{2}$, noe som er en motsigelse med at $f(b), g(a) \in \mathbb{N}$.
			}
		\end{enumerate}
		Vi har da at $h(a) = h(b)$ kunn hvis $a = b$, altså er $h$ injektiv og $A \cup B$ er tellbar.
	}
\end{enumerate}
\end{document}