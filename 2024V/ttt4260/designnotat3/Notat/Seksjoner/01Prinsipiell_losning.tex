\section{Prinsipiell løsning}
\label{prinsipiellLoesning}

Ettersom at vi vet at støysignalet $w(t)$ kan tilnærmes til èn spesifik, tidsinvariant frekvens $f_w$, kan den prinsipielle løsningen av 
problemet deles inn i to deler:
\begin{enumerate}
    \item Finne frekvensen $f_w$ ved spektralanalyse
    \item Designe et filter der mest mulig av spekteret til $s(t)$ slipper gjennom, og minst mulig av spekteret til $w(t)$ gjør det, altså ikke $f_w$
\end{enumerate}

\subsection{Spektralanalyse}
Gitt at musikken som blir spilt inn ikke er en håndfull rene sinustoner, vil den konstante, tidsinvariante forstyrrelsen $w(t)$ vise seg på 
spektralanalysen som et relativt høy spiss om frekvensen $f_w$. Det er derfor mulig å lese av frekvensen $f_w$ direkte fra spekteret. Det er også 
foretrukket å kjøre spektralanalysen på hele lydsignalet, siden $w(t)$ igjen er et tidsinvariant signal og de aller fleste av frekvenskomponentene ellers 
i lydsignalet ikke er det. Analysen bør hovedsakelig ta for seg frekvensområdet $\SI{20}{\hertz}$ til $\SI{20}{\kilo\hertz}$, ettersom 
at det er dette som er grensene for menneskelig hørsel, og derfor relevant for lydsignalet $x(t)$.

\subsection{Prinsipielt filter design}
Når frekvensen $f_w$ er kjent, kan vi designe et båndstop-filter for akkurat frekvensen til forstyrrelsen $w(t)$. 
Fordi vi kunn ønsker å filtrere ut signaler ved frekvensen $f = f_w$, ønsker vi en liten båndbredde. 
I tillegg ønsker vi at senteret av båndstopp området skal ligge i $f_w$, og vi ønsker  en høy dempning der.
Den ønskede amplituderesponsen til båndstop-filteret vil da være som i figur 2, her sentrert rundt $\SI{50}{\hertz}$.

\begin{figure}[H]
    \centering
    \includegraphics[width=0.5\textwidth]{Bilder/ideal_notch_filter.jpg}
    \caption{Et eksempel på et båndstopp-filter med liten båndbredde og høy kvalitetsfaktor}
\end{figure}

Flere filterdesign er mulige for å oppnå et de karakteristikkene vi er ute etter. Det fins både aktive og passive 
båndstoppfiltre. Vi skal ta for oss to forskjellige design av et passivt båndstoppfilter.

Passive båndstoppfiltre har fordelen at de er mye simplere enn aktive filtre, siden de kunn trenger motstander, spoler og kondensatorer for å fungere.
To mulige design av et passivt RLC båndstopp-filter er vist i figur 3.

\begin{figure}[H]
    \centering 
    \includegraphics[width=0.7\textwidth]{Bilder/prinsip_rcl.png}
    \caption{To mulige realiseringer av et passivt RLC båndstopp-filter}
\end{figure}

Filter 1 (venstre) ble valgt som som den prinsipielle løsningen, men filter 2 (høyre) skal vist nok gi en mindre båndbredde enn den til venstre.

%%%%% UTREGNING AV KOMPONENTVERDIER %%%%%
For å finne de rette komponentverdiene for $R$, $L$ og $C$ er det viktig å forstå hvordan filteret fungerer.
Når frekvensen $f$ er lik resonansfrekvensen til $LC$ seriekoblingen vil utgansspennigen $s(t)$ gå mot 0. Altså 
er det ønskelig at resonansfrekvensen $f_0 = f_w$. Vi får da at 
\[
    f_w = \frac{1}{2\pi \sqrt{LC}} \implies LC = \frac{1}{\left(2\pi f_w\right)^2} 
\]

Dersom vi velger en verdi for $L$ får vi en verdi for $C$, og motsatt. Motstandsverdien $R$ blir valgt for å justere båndbredden
til filteret og hvor kraftig amplituderesponsen til filteret blir om $f_w$. 
I dette designet har et potensiometer blitt valgt, s.a. forskjellige motstandsverdier kan utforskes.

For å oppnå en mindre båndbredde og høyere kvalitetsfaktor er det også mulig å lage et filter ved å 
kaskadekoble flere RLC filtre etter hverandre slik som i figur 4, dersom inngansimpedansen til system $H_n$ er mye større enn 
utgangsimpedansen til det foregående systemet $H_{n-1}$ i kaskaden.

\begin{figure}[H]
    \centering
    \includegraphics[width=0.65\textwidth]{Bilder/kaskade.png}
    \caption{System ved kaskadekopling av filtre $H_i$}
\end{figure}

Vi har da at totalresponsen er produktet av delresponsene, altså at 
\[
    H_{tot}(f) = H_1(f) H_2(f) \cdot\cdot\cdot H_n(f)
\]
En mulig måte å oppnå de ønskede inngans- og utgansimpedansene er ved å benytte en buffer som i figur 5.
\begin{figure}[H]
    \centering
    \includegraphics[width=0.55\textwidth]{Bilder/buffer.png}
    \caption{System ved kaskadekopling av filtre $H_i$}
\end{figure}

\subsection{Digitalt filter}
Et annet mulig design av filter er et digitalt IIR filter (Infinite Impulse Response filter).
Filteret benytter en algoritme for å beregne fourier transformen til innganssignalet for diskrete tidsintervall, og filtrerer signalet i
frekvensdomenet, og transformerer det filtrerte frekvenssignalet tilbake til et så filtrert lydsignal. Ettersom at filtreringen blir gjort 
i frekvensdomenet, vil den effektive amplituderesponsen til systemet kunne ha en effektiv båndbredde lik oppløsningen til den diskrete fouriertransformasjonen,
og ha at $|H(f_w)| \approx -\infty$. Et slikt filter er mulig å implementere både i Python og i hardware, f.eks. på en FPGA.
