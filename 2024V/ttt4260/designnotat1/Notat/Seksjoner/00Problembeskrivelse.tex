\section{Problembeskrivelse}
\label{problemBeskrivelse}

%Her kan du skrive om din problembeskrivelse
Innen signalbehandling og kretsdesign er det ofte nyttig å kunne regulere spenningsnivået til et signal, og å kunne
variere reguleringen etter behov. Dette designnotatet skal derfor ta for seg et design av en variabel nivåregulator utelukkende 
for demping av signaler.

Det overordnede systemet er illustrert i figur 1.

\begin{figure}[H]
    \centering
    \includegraphics[width=0.85\textwidth]{Bilder/nivaaregulator.png}
    \caption{Generell nivåregulator \cite{teknot}}
    \label{fig:nivaaregulator}
\end{figure}

En spenningskilde med utgangsmotstand $R_k$ sender et signal $v_1(t)$ gjennom nivåregulatoren.
Nivåregulatoren leverer så et signal $v_2(t)$ til lastmotstanden $R_L$, der signalet $v_2(t)$ er på formen
\[
    v_2(t) = Av_1(t)
\]
Forholdstalet $A$ blir dermed den avgjørende faktoren for hvor mye utgangssignalet blir endret. For en $A > 1$ blir 
utgangssignalet forsterket, og for $A < 1$ blir signalet dempet. For evalueringen av hvorvidt $A$ følger kravspesifikasjonene 
blir desibel brukt:
\[
    A_\text{[dB]} = 20\lg{(A)}
\]

Designet skal gi en variabel demping av et vilkårlig signal etter kravspesifikasjner om 
maksimal og minimal demping, $[-A_{min}, -A_{max}]$dB,
der dempestyrken $A$ skal kunne reguleres gjennom dreibar kontroll. Signalkilden er antatt å være 
et signussignal med frekvens $f = 1000$ Hz med utgangsmotstand $R_k \approx 0\Omega$, og lasten på systemet tiltenkt ideell, altså 
$R_L \approx \infty\Omega$. Målet for realiseringen av designet er et avvik fra kravspesifikasjonene på under $0.1$ dB.
