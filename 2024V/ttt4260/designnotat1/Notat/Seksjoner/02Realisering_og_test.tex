\section{Realisering og test}
\label{realiseringOgTest}

%Her kan du skrive om realisering og test
\subsection{Realisering}
For realiseringen av systemet er $A_{min} = -5\text{dB}$ og $A_{max} = -22\text{dB}$ valgt.
Setter vi $A_{min}$ og $A_{max}$ inn i likning (2) og (3) får vi følgende likningsett
\begin{align}
    &20 \cdot \lg{\left(\frac{R + R_2}{R + R_1 + R_2}\right)} = -5 \text{dB} \label{eq:eq1} \tag{I} \\
    &20 \cdot \lg{\left(\frac{R_2}{R_1 + R_2}\right)} = -22 \text{dB} \label{eq:eq2} \tag{II}
\end{align}

Velger først et potensiometer med målt maksverdi på $R \approx 10.65k\Omega$, og løser for $R_1$ og $R_2$.
Dette gir $R_1 \approx 8885\Omega$ og $R_2 \approx 767\Omega$. Fordi motstandene brukt i realiseringen 
hadde en toleranse på $\pm 5$\% ble følgende kombinasjon av motstander brukt etter eksperimentering og måling
\begin{align*}
    &R_1 = 2 \cdot 3.3k\Omega + 2 \cdot 1k\Omega + 220\Omega + 3 \cdot 33\Omega + 10\Omega = 8929\Omega \pm 5\% \\
    &R_2 = 2 \cdot 330\Omega + 100\Omega + \left(10 \mid\mid 10 \mid\mid 10 \mid\mid 10\right)\Omega = 762\Omega \pm 5\%
\end{align*}

Vi forventer da å få følgende målte $A_{min}$ og $A_{max}$ verdier:
\begin{align*}
    &A = A_{min} = (-5.0 \pm 0.2) \text{dB} \\
    &A = A_{max} = (-22.0 \pm 0.8) \text{dB} 
\end{align*}

\subsection{Testing}
For analyse av systemet har en Analog Discovery 2 \cite{discovery} blitt brukt. 
W1 generatoren til Discovery-en ble koblet på $v_1$ inngangen på nivåregulatoren,
mens CH1 og CH2 oscilloskopene var koblet til henholdsvis $v_1$ og $v_2$. Signalgeneratoren og 
de to oscilloskopene var alle koblet til felles ground.

Signalgeneratoren ble satt til å generere et sinussignal med amplitude på 1V og 
frekvens $f = 1000$Hz, og hadde en sampling rate på $4 \cdot 10^6$Hz og hver datalogging 
hadde en varighet på 2ms.

\begin{figure}[H]
    \centering
    \includegraphics[width=0.75\textwidth]{Bilder/voltage_plot.png}
    \caption{Volt/Tid plott fra oscilloskop inngang CH1 og CH2.}
    \label{fig:volt_plot}
\end{figure}

Forsøket ble utført ved $A = A_{min}$ og $A = A_{max}$. I figur 3. ser man 
spenningsplottet til utgangen av nivåregulatoren over tid. For $A_{min}$ 
var maks amplitude $0.566\text{V}$ som medfører at $A_{min} = -4.94\text{dB}$, 
og for $A_{max}$ var maks amplitude $0.080 \text{V}$ som tilsvarer at 
$A_{max} = -21.92 \text{dB}$. Altså er avviket fra ønsket verdi 
$\Delta A \leq 0.1\text{dB}$ i begge tilfellene.

Til tross for at realiseringen er innenfor toleransene når signalet er på sitt høyeste,
kan en se en relativt stor spredning i dataene, med et standardavvik på henholdsvis 
$0.42$dB og $1.10$dB for $A_{min}$ og $A_{max}$ (se figur 4). 

Det kan være flere grunner for det store spriket i $A$-verdier. I figur 5 er dempingen 
$A$ plottet over tid med samme tidsakse som i figur 3. Når $v_1$ og $v_2$ går mot 0V, divergerer 
dempingen fra den forventede dempningsverdien. Dette er trolig på grunn av begrenset presisjon 
i måleverkyøyet. Medianen til avviket fra spesifisert $A_{min}$ til målt data er på $0.03$dB, og 
for $A_{max}$ var medianen til avviket $0.17$dB.

\begin{figure}[H]
    \centering
    \includegraphics[width=0.75\textwidth]{Bilder/gauss_plot.png}
    \caption{Gaussfordeling og boksplott av demping av signalet.}
    \label{fig:gaus_plot}
\end{figure}

\begin{figure}[H]
    \centering
    \includegraphics[width=0.75\textwidth]{Bilder/desibel_plot.png}
    \caption{Desibel/Tid plott fra oscilloskop inngang CH1 og CH2.}
    \label{fig:gain_plot}
\end{figure}
