\section{Prinsipiell løsning}
\label{prinsipiellLoesning}
 

%Her kan du skrive om prinsipiell løsning
Systemet kan representeres slik som i figur 1. En signalkilde
med utgangsmotstand $R_k$ sender et signal $v_1(t)$ gjennom nivåregulatoren, som sender ut 
et dempet/forsterket signal $v_2(t)$ gjennom en last med lastmotstand $R_L$. Rent matematisk 
blir likningen til systemet
\[
    v_2(t) = Av_1(t)
\]
der $A > 1$ fører til en forsterkning av inngangssignalet $v_1$, 
og $A < 1$ fører til en dempning av signalet. Siden dette designet utelukkende tar for seg demping 
av signaler, er kunn $A < 1$ relevant.

\begin{figure}[H]
    \centering
    \includegraphics[width=0.2\textwidth]{Bilder/spenningsdeler.png}
    \caption{Variabel demping realisert med potensiometer \cite{teknot}}
    \label{fig:spenningsdeler}
\end{figure}

En mulig prinsipiell løsning for nivåregulatoren er ved bruk av en variabel spenningsdeler med potensiometer (figur 2.).
Ved å justere motstandsverdien $R$ til potensiometeret, endres spenningsfallet $v_2$ (likning 1), der $0V \leq |v_2| \leq |v_1|$.

\begin{equation}
    v_2 = \frac{R + R_2}{R + R_1 + R_2}v_1
\end{equation}

Fra likning 1. får vi at dempningsfaktoren
\[
    A = \frac{R + R_2}{R + R_1 + R_2}
\]
Designsesifikasjonen sier at dempningsfaktoren $A$ skal være slik at
$A_{max} \leq A \leq A_{min}$, oppgitt i desibel. Det følger da for de to
ekstremene at
\begin{align}
    &A = A_{min}: \hspace{2.25cm} A = 20 \cdot \lg{\left(\frac{R + R_2}{R + R_1 + R_2}\right)} \hspace{3.5cm}\\ 
    &A = A_{max}: \hspace{2.25cm} A = 20 \cdot \lg{\left(\frac{R_2}{R_1 + R_2}\right)} \hspace{3.5cm} 
\end{align}
