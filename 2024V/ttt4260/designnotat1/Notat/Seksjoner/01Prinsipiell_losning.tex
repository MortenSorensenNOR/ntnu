\section{Prinsipiell løsning}
\label{prinsipiellLoesning}
 

%Her kan du skrive om prinsipiell løsning
En mulig prinsipiell løsning for nivåregulatoren er ved bruk av en variabel spenningsdeler med potensiometer er vist i figur 
\ref{fig:spenningsdeler}.
Ved å justere motstandsverdien $R$ til potensiometeret, endres spenningsfallet $v_2$ (likning 1), der $0V \leq |v_2| \leq |v_1|$.
\begin{figure}[H]
    \centering
    \includegraphics[width=0.2\textwidth]{Bilder/spenningsdeler.png}
    \caption{Variabel demping realisert med potensiometer \cite{teknot}}
    \label{fig:spenningsdeler}
\end{figure}


\begin{equation}
    v_2 = \frac{R + R_2}{R + R_1 + R_2}v_1
\end{equation}

Fra likning 1. får vi at dempningsfaktoren
\[
    A = \frac{v_2}{v_1} = \frac{R + R_2}{R + R_1 + R_2}
\]
Designsesifikasjonen sier at dempningsfaktoren $A$ skal være slik at
$A_{max} \leq A \leq A_{min}$, oppgitt i desibel. Det følger da at 
vi får følgende når vi setter inn for minimum og maksimum demping
\begin{align}
    &A_{min} = 20 \cdot \lg{\left(\frac{R + R_2}{R + R_1 + R_2}\right)}\\ 
    &A_{max} = 20 \cdot \lg{\left(\frac{R_2}{R_1 + R_2}\right)}  
\end{align}

For å forhindre at en ikke-ideel lastmotstand påvirker dempningsfaktoren kan vi 
inkludere en buffer på utgangen av nivåregulatoren. Da får vi følgende system

\begin{figure}[H]
    \centering 
    \includegraphics[width=0.45\textwidth]{Bilder/notat1_opamp.png}
    \caption{Kretsskjema av system med buffer på output.}
\end{figure}
