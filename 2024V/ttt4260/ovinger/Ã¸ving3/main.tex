\documentclass[a4paper,11pt,norsk]{article}
\usepackage{packages}

\begin{document}

%Headingdel:---------------------------------------------
\topmargin -1.5cm
\makebox[\textwidth][s]{
    \begin{minipage}[c]{0.25\textwidth}
        \includegraphics[width=2.0cm]{Bilder/ntnu_logo.png}  
    \end{minipage}
    \begin{minipage}[c]{0.75\textwidth}
        \huge{\textbf{TTT4260 ESDA}} \\
        \Large{Øving 3  ---  Morten Sørensen, \large{\color{black!75!white}\today}}
    \end{minipage}
}
\vspace{0.75cm}
\normalsize


\section*{Oppgave 1.}
\begin{itemize}[label=\alph)]
    \item[a)] {
        \[
            A_1 = 5 \:\:\:\:\:\:\:\:\:\:\:\:\:\:\:\:\:\:\:\:\:\:\:\: A_2 = 38e^{j \cdot \pi/3} \:\:\:\:\:\:\:\:\:\:\:\:\:\:\:\:\:\:\:\:\:\: A_3 = e^{2j}
        \]
    } 
    \item[b)] {
        \begin{itemize}
            \item[i)] {
                \[
                    7 \cdot \cos{\left(\omega \cdot t + \pi\right)}
                \]
            }
            \item[ii)] {
                \[
                    3 \cdot \cos{\left(\omega \cdot t + 4.3\pi\right)}
                \]
            }
            \item[iii)] {
                \[
                    C \cdot \cos{\left(\omega \cdot t + \beta\right)}
                \]
            }
            \item[iv)] {
                \[
                    4+4j = 4\sqrt(2) \cdot \left(\cos{\left(\frac{\pi}{4}\right) + j \cdot \sin{\left(\frac{\pi}{4}\right)}}\right) = 4\sqrt{2}e^{j \cdot \frac{\pi}{4}}
                \]
            }
        \end{itemize}
    }
    \item[c)] {
        \[
            A = \frac{3}{2} \:\:\:\:\:\:\:\:\:\:\:\:\:\:\:\:\:\: B = 2e^{j \cdot \frac{\pi}{2}} \:\:\:\:\:\:\:\:\:\:\:\:\:\:\:\:\:\: C = \frac{1}{2}e^{-j \cdot \frac{\pi}{2}} \:\:\:\:\:\:\:\:\:\:\:\:\:\:\:\:\:\: D = \frac{1}{2}e^{-j \cdot \pi}
        \]
    }
\end{itemize}

\section*{Oppgave 2.}
Starter med å omgjøre $x(t)$ til det analytiske signalet $\tilde{x}(t)$
\begin{align*}
    \tilde{x}(t) &= \left(10e^{j \cdot 0.42} + 4.2e^{-j \cdot 1.3} - 6e^{j \cdot \left(0.38 - \frac{\pi}{2}\right)}\right) \cdot e^{j \omega t} \\ 
                 &= \left(8.03 + 5.6j\right)e^{j \omega t} \\ 
                 &= 9.79e^{0.61j} \cdot e^{j \omega t} = 9.79e^{j\left(\omega t + 0.61\right)} \\
                 &= 9.79(\cos{\left(\omega t + 0.61\right)} + j \cdot \sin{\left(\omega t + 0.61\right)})
\end{align*}
Får da at $x(t)$ er lik den reelle delen av $\tilde{x}(t)$, altså
\[
    x(t) = \underline{\underline{9.79\cos{\left(\omega t + 0.61\right)}}}
\]

\newpage
\section*{Oppgave 3.}
\begin{itemize}
    \item[a)] {
        Gitt et signal $x(t) = a \cos{\left(\omega t\right)}$ og en tidsforsinkelse på $\Delta t$ får vi at det tidsforsinkede signalet
        \[
            a \cos{\left(\omega (t - \Delta t)\right)} = a \cos{\left(\omega t - \omega \Delta t\right)} = a \cos{\left(\omega t + \phi\right)}
        \]
        der $\phi = -\Delta t$.
    }
    \item[b)] {
        Setter opp forsøket i falstad \cite{falstad}:
        \begin{figure}[H]
            \centering
            \includegraphics[width=0.875\textwidth]{Bilder/vplot.png}
        \end{figure}
        Måler så en tidsforsinkelse på $\SI{-0.53}{\milli\s}$ på R og en tidsforsinkelse på $\SI{0.715}{\milli\s}$ på
        kondensatoren C, og maksamplituder på henholdsvis $\SI{0.392}{\volt}$ og $\SI{0.311}{\volt}$. 
        
        Det gir komplekse amplituder
        \begin{align*}
            &V_1 = 0.392e^{j \cdot (\SI{0.53}{\milli\s} \cdot 2\pi \cdot \SI{200}{\hertz})} \\
            &V_2 =  0.392e^{-j \cdot (\SI{0.715}{\milli\s} \cdot 2\pi \cdot \SI{200}{\hertz})}
        \end{align*}

        Vi har fra Kirchoffs spenningslov at $V_0 = V_1 + V_2$. Legger vi resultatene ovenfra inn i GeoGebra får vi at 
        \begin{figure}[H]
            \centering
            \includegraphics[width=0.55\textwidth]{Bilder/sol.png}
        \end{figure}
        Altså stemmer Kirchoffs spenningslov.
    }
\end{itemize}

\section*{Oppgave 4.}
\begin{itemize}
    \item[a+b)] {
        Vi har at $V_S = 5$, $Z_1 = 100$, $Z_2 = \frac{1}{j\omega C} = \frac{1}{j \cdot 2\pi 10^4 \cdot 100 \cdot 10^{-9}}$ og til slutt at 
        $V_3 = j\omega L = j \cdot 2\pi 10^4 \cdot 10^{-3}$.

        Bruker regler for impedans i parallelkobling, og får at 
        \[
            Z_p = \frac{Z_2 \cdot Z_3}{Z_2 + Z_3}
        \]
        og bruker så Ohms lov til å få at den komplekse amplituden til spenningen over parrallelkoblingen (som er lik $V_1$) blir lik
        \[
            V_1 = V_s \cdot \frac{Z_p}{Z_1 + Z_p} \approx \underline{\underline{2.59 + 2.5j}}
        \]
        Vi kan da finne at spenningen $v_1(t)$ er lik
        \[
            v_1(t) = 3.6e^{j \cdot 0.77} \cdot e^{j \cdot 2\pi 10^4t}
        \]
    }
\end{itemize}

\newpage
\phantomsection
\addcontentsline{toc}{section}{Referanser}
\begin{thebibliography}{99}
%Bibliografi: Legg til flere elementer ved å legge til flere \bibitem:--------

\bibitem{bounce}
  Geeksforgeeks,
  \emph{Switch Debounce in Digital Circuits},
  20 november, 2019. \\
  URL: \url{https://www.geeksforgeeks.org/switch-debounce-in-digital-circuits/}

\bibitem{bouncedur}
  Dygma,
  \emph{Mechanical keyboard switch bounce and debounce delay},
  10 desember, 2018. \\
  URL: \url{https://dygma.com/blogs/stories/switch-bounce-and-debounce-delay}

\bibitem{goboard}
  Nandland,
  \emph{The Go Board}.
  URL: \url{https://nandland.com/the-go-board/}

\bibitem{ledcurrent}
  Cadence,
  \emph{Learn How to Limit Current to LED}.
  URL: \url{https://resources.pcb.cadence.com/blog/2022-learn-how-to-limit-current-to-led}

\end{thebibliography}

\end{document}
