\documentclass[a4paper,11pt,norsk]{article}
\usepackage{packages}

\begin{document}

%Headingdel:---------------------------------------------
\topmargin -1.5cm
\makebox[\textwidth][s]{
    \begin{minipage}[c]{0.25\textwidth}
        \includegraphics[width=2.0cm]{Bilder/ntnu_logo.png}  
    \end{minipage}
    \begin{minipage}[c]{0.75\textwidth}
        \huge{\textbf{TFE4120  Electromagnetism}} \\
        \Large{Exercise 7  ---  Morten Sørensen, \large{\color{black!75!white}\today}}
    \end{minipage}
}
\vspace{0.75cm}
\normalsize


\section*{Oppgave 1.}
\begin{enumerate}
    \item {
        Inngangssignalet blir dempet med omtrent $20\log{\left(\frac{\SI{0.5}{\volt}}{\SI{1}{\volt}}\right)} \approx \SI{-6}{\decibel}$, og 
        en faseforskyvning $\phi = -2\pi \cdot \SI{1}{\kilo\hertz} \cdot (-\SI{0.165}{\milli\s}) \approx 1.04[\text{rad}] = 59.59[\text{deg}]$.
     
        \begin{figure}[H]
            \centering
            \includegraphics[width=0.85\textwidth]{Bilder/1a_dt.png}
            \caption{Spenningsplott fra Analog Discovery 2 av $V_1$ (i gult) og $V_2$ (i blått).}
        \end{figure}
    }
    \item {
        Får følgende resultat fra å variere inngangsfrekvensen $f$
        \begin{figure}[H]
            \centering
            \begin{tabular}{| l | c | c |}
                \hline
                Frekvens (Hz) & Forsterkning $A$ & Faseforskyvning $\phi$ \\
                \hline
                $\SI{10}{\hertz}$ & $\sim 0.006$ & $90^{\circ}$? \\ 
                \hline
                $\SI{100}{\hertz}$ & $\sim 0.061$ & $\SI{2.5}{\milli\s} \approx 89.95^{\circ}$\\ 
                \hline
                $\SI{1}{\kilo\hertz}$ & $\sim 0.5$ & $59.59^{\circ}$ \\ 
                \hline
                Osv.. & & \\
                \hline
            \end{tabular} 
        \end{figure}

        Vi ser at dempningen minker når frekvensen øker, mens faseforskyvningen minker når frekvensen øker.
    }
    \item {
        Under er resultatet fra netverksanalysen.
        \begin{figure}[H]
            \centering
            \includegraphics[width=0.95\textwidth]{Bilder/network.png}
            \caption{Netverksanalyse av kretsen, dempning (over) og faseforskyvning (under), som funksjon av frekvens.}
        \end{figure} 
        Systemet funker som et lavpassfilter. For frekvenser under $\sim \SI{10}{\kilo\hertz}$ blir utgangssignalet en dempet versjon av inngangssignalet,
        og opplever en faseforskyvning på mellom 0 og 90 grader, mens over $\sim \SI{10}{\kilo\hertz}$ er utgangssignalet omtrent likt inngangssignalet.
    }
\end{enumerate}

\section*{Oppgave 2.}
\begin{enumerate}
\item {
    Systemet vil oppføre seg som et høypassfilter. For lave frekvenser vil kondensatoren oppføre seg som en åpen krets,
    og hele spenningsfallet kommer da over utgangssignalet $V_2$. For høye frekvenser vil kondensatoren oppføre seg som 
    en lukket krets, og vil derfor ha lavt spenningsfall over seg.
    \newpage
}
\item {
    Resultatet fra netverksanalysen:
    \begin{figure}[H]
        \centering
        \includegraphics[width=0.95\textwidth]{Bilder/network2.png}
        \caption{Systemet fungerer som et høypassfilter.}
    \end{figure}
}
\item {
    For et inngangssignal på forment $5\cos{\left(4000 \pi t\right)}$ får vi at amplituden er $\SI{5}{\volt}$ og 
    frekvensen $f = \SI{2}{\kilo\hertz}$. Leser vi av fra plottet over får vi at
    \[
        A = \SI{-3.85}{\decibel} \approx 0.64
    \]
    og
    \[
        \phi = -49.4^{\circ} = -0.87[\text{rad}]
    \]

    Vi får da at $v_2(t) = 5 \cdot 0.64 \cos{\left(4000 \pi t + (-0.87 + 2\pi)\right)} = 3.2 \cos{\left(4000 \pi t + 5.41\right)}$
}
\end{enumerate}

\newpage
\section*{Oppgave 3.}
Vi har gitt kretsen i figur 4.
\begin{figure}[H]
    \centering
    \includegraphics[width=0.55\textwidth]{Bilder/impedance_cir.png}
    \caption{Spenningsdeler med generelle impedanser}
\end{figure}

Ohms lov gir oss at 
\[
    V_2 = I \cdot Z_2 = \left(\frac{V}{Z_1 + Z_2}\right) \cdot  Z_2
\]

Finner vi frekvensresponsen til systemet får vi at 
\[
    H = \frac{V_2}{V} = \frac{1}{V} \cdot \left(\frac{V}{Z_1 + Z_2}\right) \cdot  Z_2 = \frac{Z_2}{Z_1 + Z_2} \:\:\: \blacksquare
\]

\section*{Oppgave 4.}
\begin{enumerate}
    \item Bruker formelen fra oppgave 3:
        \begin{enumerate}[label=\Roman*)]
            \item 
                \[
                    H(\omega) = \frac{j \omega L}{R + j \omega L}
                \]
            \item 
                \[
                    H(\omega) = \frac{\frac{1}{j \omega C}}{R + \frac{1}{j \omega L}} = \frac{1}{j\tau\omega + 1}
                \]
            \item 
                \[
                    H(\omega) = \frac{R}{R + \frac{1}{j \omega L}} = \frac{j\tau\omega}{j\tau\omega + 1}
                \]
        \end{enumerate}
    \item Får følgende amplituderesponser
        \usetagform{roman}
        \begin{equation}
            \left|H(\omega)\right| = \left|\frac{j \omega L}{R + j \omega L}\right| = \frac{\omega L}{\sqrt{L^2\omega^2 + R^2}}
        \end{equation}

        \begin{equation}
            \left|H(\omega)\right| = \left|\frac{1}{j\tau\omega + 1}\right| = \frac{1}{\sqrt{\tau^2\omega^2 + 1}}
        \end{equation}

        \begin{equation}
            \left|H(\omega)\right| = \left|\frac{j\tau\omega}{j\tau\omega + 1}\right| = \frac{\tau\omega}{\sqrt{\tau^2\omega^2 + 1}}
        \end{equation}

        \setcounter{equation}{0} 
        For å finne faseresponsen omformer vi først uttrykkene til ren kompleks form
        \begin{equation}
            \frac{j\omega L}{R + j\omega L} = \frac{(j\omega L)(R - j\omega L)}{R^2 + \omega^2L^2} = \frac{\omega^2L^2}{R^2+\omega^2L^2} + j\frac{R\omega L}{R^2 + \omega^2L^2}
        \end{equation}

        \begin{equation}
            \frac{1}{j\tau\omega + 1} = \frac{1-j\tau\omega}{1 + \tau^2\omega^2} = \frac{1}{1 + \tau^2\omega^2} - j\frac{\tau\omega}{1 + \tau^2\omega^2} 
        \end{equation}

        \begin{equation}
            \frac{j\tau\omega}{j\tau\omega + 1} = \frac{(j\tau\omega)(1-j\tau\omega)}{1 + \tau^2\omega^2} = \frac{\tau^2\omega^2}{1 + \tau^2\omega^2} + j\frac{\tau\omega}{1 + \tau^2\omega^2} 
        \end{equation}

        \setcounter{equation}{0} 
        Vi får da at 
        \begin{equation}
            \angle H(\omega) = \tan^{-1}{\left(\frac{R \omega L}{\omega^2L^2}\right)} = \tan^{-1}{\left(\frac{R}{\omega L}\right)}
        \end{equation}

        \begin{equation}
            \angle H(\omega) = \tan^{-1}{\left(-\frac{\tau\omega}{1}\right)} = \tan^{-1}{(-\tau\omega)}
        \end{equation}

        \begin{equation}
            \angle H(\omega) = \tan^{-1}{\left(\frac{\tau\omega}{\tau^2\omega^2}\right)} = \tan^{-1}{\left(\frac{1}{\tau\omega}\right)}
        \end{equation}
    \item Plotter vi funksjonene fra 4b får vi følgende plot for amplituderespons
        \begin{figure}[H]
            \centering
            \includegraphics[width=0.85\textwidth]{Bilder/amplituderespons.png}
        \end{figure}
        og følgende plot for faseresponss
        \begin{figure}[H]
            \centering
            \includegraphics[width=0.85\textwidth]{Bilder/faserespons.png}
        \end{figure}

        Vi ser at situasjon II samsvarer med høypassfilteret i oppgave 2 og situasjon III samsvarer med lavpassfilteret i
        oppgave 1.
\end{enumerate}


\section*{Oppgave 5.}
\begin{enumerate}
    \item Ser på uttrykket for amlituderesponsen i situasjon II:
        \[
            |H(\omega)| = \frac{1}{\sqrt{\tau^2\omega^2 + 1}}
        \]
        Setter $|H(\omega)| = 0.5$ ved $\omega = 2\pi \cdot \SI{275}{\hertz}$ og får at 
        $R \approx \SI{10}{\kilo\ohm}$.
\end{enumerate}

\section*{Oppgave 6.}
\begin{enumerate}
    \item For å få samme frekvens- og amplituderespons som systemet i oppgave 5 lager vi et 
        system slik som system I i oppgave 4a, men der spolen og motstanden er byttet om.
        Vi får da følgende uttrykk for frekvensresponsen til spenningen over motstanden
        \[
            H(\omega) = \frac{R}{R + j \omega L}
        \]
        Amplituderesponsen blir da
        \[
            |H(\omega)| = \left|\frac{R}{R + j \omega L}\right| = \frac{R}{\sqrt{L^2\omega^2 + R^2}} = \frac{1}{\sqrt{L^2\omega^2 + R^2}} \cdot \frac{1}{\sqrt{\left(\frac{1}{R}\right)^2}} = \frac{1}{\sqrt{\frac{1}{R^2}L^2\omega^2 + 1}}
        \]
        For å finne faseresponsen finner vi først uttrykket på rent komplekst form:
        \[
            \frac{R}{R + j \omega L} = \frac{R(R - j \omega L)}{R^2 + \omega^2L^2} = \frac{R^2}{R^2 + \omega^2L^2} - j \frac{R \omega L}{R^2 + \omega^2L^2}
        \]
        Finner så faseresponsen som
        \[
            \angle H(\omega) = \tan^{-1}{\left(-\frac{R \omega L}{R^2}\right)} = -\tan^{-1}{\left(-\frac{\omega L}{R}\right)}
        \]

        Vi ser da at for enkelte verdier av L og R vil systemet få samme respons som systemet i oppgave 5.
    \item For at systemet skal ha samme respons som oppgave 5 når $L = \SI{100}{\milli\henry}$, så må
        \[
            |H(2\pi \cdot \SI{275}{\hertz})| = 0.5
        \]
        altså må
        \[
            \frac{1}{\sqrt{\frac{1}{R^2} \cdot \SI{100}{\milli\henry}^2 \cdot (2\pi \cdot \SI{275}{\hertz})^2 + 1}} = 0.5
        \]
        Det gir en motstandsverdi $R \approx \SI{100}{\ohm}$.
    \item :)
    \item Hadde nok valgt kondensatorkretsen i praksis.
\end{enumerate}

\end{document}
