\documentclass[a4paper,11pt,norsk]{article}
\usepackage{packages}

\begin{document}

%Headingdel:---------------------------------------------
\topmargin -1.5cm
\makebox[\textwidth][s]{
    \begin{minipage}[c]{0.25\textwidth}
        \includegraphics[width=2.0cm]{Bilder/ntnu_logo.png}  
    \end{minipage}
    \begin{minipage}[c]{0.75\textwidth}
        \huge{\textbf{TTT4260 ESDA}} \\
        \Large{Øving 3  ---  Morten Sørensen, \large{\color{black!75!white}\today}}
    \end{minipage}
}
\vspace{0.75cm}
\normalsize


\section*{Oppgave 0.}
Forventet tidskonstant ved $R = \SI{1}{\kilo\ohm}$ og $L = \SI{100}{\milli\henry}$ er $\tau = \frac{\SI{100}{\milli\henry}}{\SI{1}{\kilo\ohm}} = \SI{0.1}{\milli\s}$.
Målt tidskonstant $\tau \approx \SI{0.110}{\milli\s}$. Målt motstand er $R \approx \SI{0.981}{\kilo\ohm}$.
Det betyr at den faktiske induktansen er $L = \tau \cdot R \approx \SI{0.110}{\milli\s} \cdot \SI{0.981}{\kilo\ohm} = \SI{107}{\milli\henry}$.

\section*{Oppgave 1.}
\begin{enumerate}
    \item Ved resonnansfrekvensen er den komplekse delen av den totale impedansen lik 0, og vi får da at 
        \[
            X_L = X_C \implies 2\pi f_0 L = \frac{1}{2\pi f_0 C}
        \]
        der $X_L$ er reaktansen til spolen, $X_C$ er reaktansen til kondensatoren og $f_0$ er resonnansfrekvensen til systemet.
        Dette gir at 
        \[
            f_0 = \frac{1}{2\pi\sqrt{LC}} = \frac{1}{2\pi\sqrt{ \SI{100}{\milli\henry} \cdot \SI{100}{\nano\farad}}} \approx \SI{1592}{\hertz}
        \]
    \item Lavest dempning av inngangssignalet var ved $f = \SI{1.59}{\kilo\hertz}$, som er innen $\pm 1\%$ av forventet verdi, altså komfortabelt innenfor feilmarginer i 
        målingen av tidskonstant og motstandsverdi.
        \begin{figure}[H]
            \centering
            \includegraphics[width=0.85\textwidth]{Bilder/rlc_bode.png}
            \caption{Bodeanalyse av RLC systemet. Utgangssignalet var målt over motstanden $R$}
        \end{figure}
    \item Når motstanden til potensiometeret synker, vil spenningen over spolen øke.
\end{enumerate}

\section*{Oppgave 2.}
\begin{enumerate}
    \item Krets (a) er et bådpassfilter, som vi så i oppgave 1b. Krets (b) er et høypassfilter fordi at ved lave frekvenser legger hele spenningsfallet seg over kondensatoren,
        mens ved høye frekvenser kommer hele spenningsfallet over spolen. Krets (c) er et lavpassfilter, av motsatt argument som for krets (b).
    \item Måling for krets (a) har vi gjort i oppgave 1b. Måling av krets (b) stemte overrens med at det er et høypassfilter (figur 2), og måling av krets (c) viser at det er et lavpassfilter (figur 3).
        \begin{figure}[H]
            \centering
            \includegraphics[width=0.85\textwidth]{Bilder/2b_highpass.png}
            \caption{Krets (b). Et høypassfilter}
        \end{figure}

        \begin{figure}[H]
            \centering
            \includegraphics[width=0.85\textwidth]{Bilder/2b_lowpass.png}
            \caption{Krets (c). Et høypassfilter}
        \end{figure}
        \newpage
    \item
        \hphantom{Hei}
        \begin{figure}[H]
            \centering
            \includegraphics[width=0.35\textwidth]{Bilder/shrug.png}
            \caption{:)}
        \end{figure}
    \item $\uparrow$ 
    \item Skal finne et uttrykk for amplituderesponsen til filtrene a-c. \\
        \textbf{Filter (a):} \\ 
        \[
            \left|H(\omega)\right| = \left|\frac{V_2}{V_1}\right| = \left|\frac{R}{R + j\omega L + \frac{1}{j\omega C}}\right| = \left|\frac{1}{1 + j\omega\tau -\frac{j}{\omega\tau}}\right|
        \]
        
        \textbf{Filter (b):} \\ 
        \[
            \left|H(\omega)\right| = \left|\frac{j\omega L}{j\omega L + R + \frac{1}{j\omega C}}\right| = \left|\frac{j\omega\tau}{1 + j\omega\tau + \frac{1}{\omega\tau}}\right|
        \]

        \textbf{Filter (c):} \\ 
        \[
            \left|H(\omega)\right| = \left|\frac{\frac{1}{j\omega C}}{\frac{1}{j\omega C} + R + j\omega L}\right| = \left|\frac{1}{1 + j\omega\tau - CL\omega^2}\right|
        \]


\end{enumerate}

\section*{Oppgave 3.}
Vi ønsker at antennen skal få tilført størst mulig effekt når v(t) er et sinussignal på formen
\[
    v(t) = A\sin{\left(\frac{R}{L}t\right)}
\]
Starter med å finne thevininekvivalenten til motstanden R og spolen L. Vi får at thevinin impedansen blir
\[
    z_{th} = \frac{R j \omega L}{R + j \omega L}
\]
og at thevinin spenningen blir
\[
    v_{th} = V \frac{j \omega L}{R + j \omega L}
\]
For å få impedanstilpasning i systemet må $Z_L = \overline{Z_{th}}$. Det gir at  
\[
    Z_L = \overline{\left(\frac{R j \omega L}{R + j \omega L}\right)} =  \overline{\left(\frac{(R j \omega L) \cdot (R - j \omega L)}{R^2 + \omega^2 L^2}\right)} = \frac{R \omega^2 L^2}{R^2 + \omega^2 L^2} - \frac{j R^2 L^2 \omega^2}{R^2 + \omega^2 L^2}
\]

\end{document}
