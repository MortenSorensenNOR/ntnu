\documentclass[a4paper,11pt,norsk]{article}
\usepackage{packages}

\begin{document}

%Headingdel:---------------------------------------------
\topmargin -1.5cm
\makebox[\textwidth][s]{
    \begin{minipage}[c]{0.25\textwidth}
        \includegraphics[width=2.0cm]{Bilder/ntnu_logo.png}  
    \end{minipage}
    \begin{minipage}[c]{0.75\textwidth}
        \huge{\textbf{TFE4120  Electromagnetism}} \\
        \Large{Exercise 7  ---  Morten Sørensen, \large{\color{black!75!white}\today}}
    \end{minipage}
}
\vspace{0.75cm}
\normalsize


\section*{Oppgave 1.}
\begin{figure}[htbp]
    \centering
    \begin{circuitikz}
        \draw
        (0,0) to[battery1,l=$V$] ++(0,2)
        to[switch,l=$S$] ++(2,0)
        to[R,l=$R$] ++(2.5,0) coordinate(node1)
        to[short ,-*] ++(1.5,0) node[below]{+}
        (node1) to[C,l=$C$,*-*] ++(0, -2) coordinate(node2)
        to [short,-*] ++(1.5,0) node[above]{-}
        ++(0,1) node[]{$v_C$}
        (node2) to [short] (0,0)
        ;
    \end{circuitikz}
    \caption{Krets for oppgave 1.}
\end{figure}

\begin{enumerate}[label=\alph*)]
    \item {
        Vi har fra definisjonen at
        \[
            \tau = R \cdot C = 1k\Omega \cdot 100\mu F = 100\text{ms}
        \]
        Dette gir at uttrykket for spenningen $v_C(t)$ er gitt ved
        \[
            v_C(t) = 5V(1 + -e^{-\frac{t}{\tau}})
        \]
        Følgende plot viser $v_C(t)$ ved flere multipler av $\tau$
        \begin{figure}[H]
                \centering
                \includegraphics[width=0.65\textwidth]{Bilder/tau.png}
                \caption{Spenningsplott for forskjellige multipler av $\tau$}
        \end{figure}
    }
    \item {
        Koblet opp kretsen fra med $R = \SI{1}{\kilo\ohm}$ og $C = \SI{100}{\micro\farad}$ og fikk målt tidskonstant
        $\tau = \SI{0.093}{\s}$.
        \begin{figure}[H]
                \centering
                \includegraphics[width=0.65\textwidth]{Bilder/wave.png}
                \caption{Målt spenning over kondensatoren $v_c$ + $\tau$}
        \end{figure}

    }
\end{enumerate}


\section*{Oppgave 2.}
\begin{figure}[htbp]
    \centering
    \begin{circuitikz}
        \draw
        (0,0) to [battery1,l=$V$] ++(0,2)
        to[opening switch,l=$S$] ++(2 ,0)
        to[R,l=$R_1$] ++(2.5,0) coordinate(node1)
        to[R,l_=$R_2$,*-*] ++(0,-2)
        (node1) to[short] ++(1,0) coordinate(node2)
        to[R,l=$R_3$,*-*] ++(0,-2)
        (node2) to[R,l=$R_4$,*-*] ++(2,0) coordinate(node3)
        to[short,-*] ++(1.5,0) node[below]{+}
        (node3) to[C,l=$C$,*-*] ++(0,-2) coordinate(node4)
        to[short,-*] ++(1.5,0) node[above]{-}
        ++(0,1) node[]{$v_C$}
        (node4) to[short] (0,0)
        ;
    \end{circuitikz}
    \caption{Krets for oppgave 2.}
\end{figure}
Regner ut thevenin-ekvivalent med hensyn på V
\[
    R_{th} = R_4 + \frac{R_3 \cdot R_{12}}{R_3 + R_{12}} = \left(300 + \frac{220 \cdot 140}{220 + 140}\right)\Omega \approx 385\Omega
\]
Thevenin-spenningen blir
\[
    V_{th} = \frac{\frac{470 \cdot 220}{470 + 220}}{\left(200 + \frac{470 \cdot 220}{470 + 220}\right)} \cdot 5V \approx 2.15V 
\]
For å finne tidskonstanten til kretsen bruker vi thevenin-ekvivalent på kretsen når bryteren er åpen med hensyn på kondensatoren. Får da at 
thevenin-mostanden $R_{th}$ er lik
\[
    R_{th} = R_4 + R_2 \mid \mid R_3 \approx 450\Omega
\]
Får da at tidskonstanten er lik
\[
    \tau = 450\Omega \cdot 10\text{nF} = 4.5\mu\text{s}
\]
Siden $v_C(0) = 2.15V$ får vi da at uviklingen er gitt ved
\[
    v_C(t) = 2.15e^{-\frac{t}{4.5\mu\text{s}}}V
\]
Får da følgende plott for spenningen over tid
\begin{figure}[H]
    \centering
    \includegraphics[width=0.65\textwidth]{Bilder/vc.png}
    \caption{Spenningsplott for $v_C(t)$}
\end{figure}


\section*{Oppgave 3.}
\begin{figure}[htbp]
    \centering
    \begin{circuitikz}
        \draw
        (0,0) to[battery1,l=$V$] ++(0,2)
        to[switch,l=$S_1$] ++(2,0)
        to[R,l=$R_1$] ++(2.5,0) coordinate(node1)
        to[R,l_=$R_2$ ,*-*] ++(0,-2)
        (node1) to[short] ++(1,0) node[below]{+} coordinate(vc)
        ++(0,-1) node[]{$v_C$}
        ++(0,-1) node[above]{$-$}
        (vc) to[short] ++(1,0) coordinate(node2)
        to[C,l=$C$,*-*] ++(0,-2)
        (node2) to[switch, l=$S_2$] ++(2,0)
        to[R,l=$R_3$] ++(0,-2)
        to[short] (0,0)
        ;
    \end{circuitikz}
    \caption{Krets for oppgave 3.}
\end{figure}
\begin{enumerate}[label=\alph*)]
    \item {
        Starter med å regne ut thevenin-spenningen for systemet sett fra kondensatoren.
        \[
            V_{th} = \frac{R_2}{R_1 + R_2} \cdot V = \frac{\SI{1}{\kilo\ohm}}{\SI{1}{\kilo\ohm} + \SI{1}{\kilo\ohm}} \cdot \SI{1}{\volt} = \SI{0.5}{\volt}
        \]
        Tidskonstanten til systemet blir 
        \[
            \tau = \SI{1}{\kilo\ohm} \cdot \SI{100}{\micro\farad} = \SI{100}{\milli\s}
        \]
        Får da at oppladningen følger 
        \[
            v_C(t) = 0.5\left(1 - e^{-\frac{t}{\SI{100}{\milli\s}}}\right)\text{V}
        \]
    }
    \item {
        Når $S_2$ lukkes når $t = 6 \cdot \tau$ blir spenningsplottet
        \begin{figure}[H]
            \centering
            \includegraphics[width=0.75\textwidth]{Bilder/dual_plot_b.png}
            \caption{Spenningsplott for $v_C(t)$}
        \end{figure}
    }
    \item {
        Når $S_2$ lukkes når $t = 0.5 \cdot \tau$ blir spenningsplottet
        \begin{figure}[H]
            \centering
            \includegraphics[width=0.75\textwidth]{Bilder/dual_plot_c.png}
            \caption{Spenningsplott for $v_C(t)$}
        \end{figure}
    }
\end{enumerate}

\section*{Oppgave 4.}
\begin{figure}[htbp]
    \centering
    \begin{circuitikz}
        \draw
        (0,0) to[square voltage source,l=$v_1$] ++(0,2)
        to[R,l=$R$] ++(2.5,0) coordinate(node1)
        to[short,-*] ++(1.5,0) node[below]{+}
        (node1) to[C,l=$C$,*-*] ++(0,-2) coordinate(node2)
        to[short,-*] ++(1.5,0) node[above]{-}
        ++(0,1) node[]{$v_2$}
        (node2) to[short] (0,0)
        ;
    \end{circuitikz}
    \caption{Krets for oppgave 4.}
\end{figure}
\begin{enumerate}[label=\alph*)]
    \item \hspace{-0.175cm}Tidskonstanten $\tau = \SI{1}{\kilo\ohm} \cdot \SI{10}{\nano\farad} = \SI{10}{\micro\farad}$.
    \item {
        Når $f = 5$kHz med tidskonstant $\tau = \SI{10}{\micro\farad}$ får vi følgende plott
        \begin{figure}[H]
            \centering
            \includegraphics[width=0.7\textwidth]{Bilder/square.png}
            \caption{Spenningsplott for $v_C(t)$}
        \end{figure}
    }
    \item {
        Når $f = 30$kHz med tidskonstant $\tau = \SI{10}{\micro\farad}$ får vi følgende plott
        \begin{figure}[H]
            \centering
            \includegraphics[width=0.7\textwidth]{Bilder/square2.png}
            \caption{Spenningsplott for $v_C(t)$}
        \end{figure}
    }
    \item {
        Her er resultatene for henholdsvis $f = \SI{5}{\kilo\hertz}$ og $f = \SI{30}{\kilo\hertz}$
        \begin{figure}[H]
            \centering
            \includegraphics[width=0.7\textwidth]{Bilder/5k.png}
            \caption{$f = \SI{5}{\kilo\hertz}$}
        \end{figure}
        \begin{figure}[H]
            \centering
            \includegraphics[width=0.7\textwidth]{Bilder/30k.png}
            \caption{$f = \SI{30}{\kilo\hertz}$}
        \end{figure}
    }
    \item {
        Vi har da at spenningsfunksjonen er gitt ved 
        \[
            v_C(t) = (1 - e^{-\frac{t}{R \cdot C}})
        \]
        der $v_C(1/5000) = \SI{0.8}{\volt}$. Setter vi $C = \SI{10}{\nano\farad}$ 
        får vi at $R \approx \SI{12.5}{\kilo\ohm}$.
        Dette viste seg å ikke fungere i testing. :)
    }
\end{enumerate}

\end{document}
