\documentclass[a4paper,11pt,norsk]{article}
\usepackage{packages}

\begin{document}

%Headingdel:---------------------------------------------
\topmargin -1.5cm
\makebox[\textwidth][s]{
    \begin{minipage}[c]{0.25\textwidth}
        \includegraphics[width=2.0cm]{Bilder/ntnu_logo.png}  
    \end{minipage}
    \begin{minipage}[c]{0.75\textwidth}
        \huge{\textbf{TTT4260 ESDA}} \\
        \Large{Øving 3  ---  Morten Sørensen, \large{\color{black!75!white}\today}}
    \end{minipage}
}
\vspace{0.75cm}
\normalsize


\section*{Oppgave 1.}
Her er Verilog implementeringen til systemet
\begin{verilogcode}
module Switch_To_LED
    (input i_Switch_1,
     output o_LED_1);

 assign o_LED_1 = i_Switch_1;

endmodule
\end{verilogcode}

\begin{figure}[H]
    \centering
    \includegraphics[width=0.85\textwidth]{Bilder/oppgave1.jpg}
    \caption{LED lyser når Switch 1 blir trykket}
\end{figure}

\section*{Oppgave 2.}
Her er Verilog implementeringen til systemet
\begin{verilogcode}
module PMOD_To_Board
    (input i_Switch_1,
    output io_PMOD_1);

 assign io_PMOD_1 = i_Switch_1;

endmodule
\end{verilogcode}
\begin{figure}[H]
    \centering
    \includegraphics[width=0.85\textwidth, angle=-90]{Bilder/oppgave2.jpg}
    \caption{LED lyser pga. signal fra PMOD}
\end{figure}

\newpage
\section*{Oppgave 3.}
Her er Verilog implementeringen til systemet
\begin{verilogcode}
module XOR_LED
    (input i_Switch_1,
     input i_Switch_2,
     output o_LED_1);

 assign o_LED_1 = i_Switch_1 ^ i_Switch_2;

endmodule
\end{verilogcode}

\section*{Oppgave 4.}
Her er sannhetstabellen for $f = A \cdot \overline{B} + \overline{A} \cdot B$
\begin{center}
    \begin{tabular}{|| c | c | c ||} 
        \hline
        A & B & f \\ [0.5ex] 
        \hline\hline
        0 & 0 & 0 \\ 
        \hline
        0 & 1 & 1 \\
        \hline
        1 & 0 & 1 \\
        \hline
        1 & 1 & 0 \\
        \hline
    \end{tabular}
\end{center}

\section*{Oppgave 5.}
Starter med følgende bools uttrykk:
\[
    f = \overline{A}B\overline{C} + A\overline{B}\overline{C} + AB\overline{C}
\]
Forkorter så uttrykket
\begin{align*}
    \overline{A}B\overline{C} + A\overline{B}\overline{C} + AB\overline{C} &\implies \\
    B\overline{C}(\overline{A} + A) + A\overline{B}\overline{C} &\implies \\ 
    B\overline{C} + A\overline{B}\overline{C} &\implies \\ 
    \overline{C}(A\overline{B} + B) &\implies 
    \underline{\underline{\overline{C}(A + B)}}
\end{align*}

\newpage
\section*{Oppgave 6.}
Her er Verilog implementeringen til systemet
\begin{verilogcode}
module Boolean_Expr
    (input i_Switch_1,
     input i_Switch_2,
     input i_Switch_3,
     output o_LED_1);

 assign o_LED_1 = (!i_Switch_3 & i_Switch_1) | (!i_Switch_3 & i_Switch_2);

endmodule    
\end{verilogcode}
\begin{figure}[H]
    \centering
    \includegraphics[width=0.6\textwidth, angle=-90]{Bilder/oppgave4.jpg}
    \includegraphics[width=0.6\textwidth, angle=-90]{Bilder/oppgave6.jpg}
    % \begin{subfigure}
    % \end{subfigure}
    % \begin{subfigure}
    % \end{subfigure}
    \caption{Switch 1 (A) og Switch 2 (B) fører til at lyset lyser. Hvis C (switch 3) blir aktivert, slutter lyster å lyse}
\end{figure}

\newpage
\section*{Oppgave 7.}
Her er Verilog implementeringen til systemet
\begin{verilogcode}
module Full_Adderer
    (input A,
     input B,
     input C_in,
     output S,
     output C_out);

 wire A_XOR_B = (A ^ B);

 assign S = A_XOR_B ^ C_in;
 assign C_out = (A & B) | (A_XOR_B & C_in);

endmodule
\end{verilogcode}

\newpage
\section*{Oppgave 8.}
Her er Verilog implementeringen til systemet
\begin{verilogcode}
module Full_Adderer_2bit
    (input i_Switch_1,
     input i_Switch_2,
     input i_Switch_3,
     input i_Switch_4,
     output o_LED_1,
     output o_LED_2,
     output o_LED_3);

 wire S1;
 wire S2;
 wire C_out1;
 wire C_out2;

 Full_Adderer Full_Adderer_Inst_1
    (.A(i_Switch_1),
     .B(i_Switch_3),
     .C_in(1'b0),
     .S(S1),
     .C_out(C_out1));

 Full_Adderer Full_Adderer_Inst_2
    (.A(i_Switch_2),
     .B(i_Switch_4),
     .C_in(C_out1),
     .S(S2),
     .C_out(C_out2));

 assign o_LED_1 = S1;
 assign o_LED_2 = S2;
 assign o_LED_3 = C_out2;

endmodule
\end{verilogcode}
\begin{figure}[H]
    \centering
    \includegraphics[width=0.415\textwidth, angle=-90]{Bilder/2bitadder.jpg}
    \includegraphics[width=0.415\textwidth, angle=-90]{Bilder/2bitadder2.jpg}
    \includegraphics[width=0.415\textwidth, angle=-90]{Bilder/2bitadder3.jpg}
    % \begin{subfigure}
    % \end{subfigure}
    % \begin{subfigure}
    % \end{subfigure} 
    % \begin{subfigure}
    % \end{subfigure}
    \caption{2-Bits full-addereren med forskjellige inputverdier. Tall 1 består av switch 1 og 2, mens tall 2 består av switch 3 og 4}
\end{figure}

\section*{Oppgave 9.}
Vi har fått gitt at $C = \SI{50}{\micro\farad}$ og en frekvens $f_c = \SI{1}{\hertz}$.
Skal finne $R$. Har da at 
\[
    \tau = RC = \frac{1}{\SI{1}{\hertz}} \implies R = \frac{1}{\SI{1}{\hertz} \cdot \SI{50}{\micro\farad}} \approx \underline{\underline{\SI{20}{\kilo\ohm}}}
\]

\end{document}
