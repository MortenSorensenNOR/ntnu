\section{Prinsipiell løsning}
\label{prinsipiellLoesning}

Den prinsipielle løsningen foreslått er som vist i figur \ref{fig:prinsip_sys}.

\begin{figure}[H]
    \centering
    \includegraphics[width=0.75\textwidth]{Bilder/sys_overview.png}
    \caption{Illustrering av den prinsipielle løsningens overordnet form}
    \label{fig:prinsip_sys}
\end{figure}

Signalet $x_1(t)$ sendes først inn i et ulineært system. Målet med det ulineære systemet 
er å generere overharmoniske frekvenser ved $2f$, $3f$, $4f$, osv. Ved å sende utgangssignalet 
$y(t)$ fra det ulineære systemet gjennom et båndpassfilter kan frekvenskomponenten ved $2f$ isoleres,
og dermed oppnå det ønskede utgangssignalet $x_2(t)$. Kvaliteten på båndpassfilteret er derfor den 
dominerende faktoren mht. distorsjon på utgangssignalet, og dermed hvorvidt $\hat{x}_2(t)$ blir en god
tilnærming av $x_2(t)$.
Et skarpere filter med større demping i båndstopp-området vil føre til at amplituden av de andre harmoniske komponentene 
blir minsket i utgangssignalet, og vil dermed gi et renere sinussignal og en større SDR.

\subsection{Det ulineære systemet}
For å oppnå den ønskede oppførselen til det ulineære systemet er designet i figur \ref{fig:ulin_sys} foreslått.

\begin{figure}[H]
    \centering
    \includegraphics[width=0.3\textwidth]{Bilder/ulinear_sys.png}
    \caption{Prinsipielt design av det ulineære systemet}
    \label{fig:ulin_sys}
\end{figure}

Siden inngangssignalet $x_1(t)$ ossilerer rundt 0, vil dioden føre til at signalet blir forvrengt, og vil derfor ikke 
lenger være sinus-formet. Dette kommer av Shockley diode likningen \cite{shockley}
\[
    I_D = I_S\left(e^{\frac{V_D}{n V_T}} - 1\right)\text{,}
\]
en modell for oppførselen til en diode, er ulineær.
Men siden utgangssignalet $y(t)$ fortsatt har samme periode som inngangssignalet,
vil utgangssignalet ha en grunnfrekvens $f$ i tillegg til overharmoniske frekvenskomponenter ved $2f$, $3f$, $4f$, osv., 
slik vi ønsket. For dette designet er $R1$ valgt stor.

\subsection{Båndpassfilter}
Ettersom at båndpassfilteret spiller en sentral rolle i kvaliteten på systemet er to metoder for 
båndpassfiltrering foreslått. De to prinsipielle løsningene er et analogt, passivt RLC båndpassfilter og et digitalt 
FIR\footnote{\textit{Finite Impulse Response}} båndpassfilter. Begge filtrene har sine fordeler og ulemper. RLC filteret 
er svært simpelt, energi- og kosteffektivt, men har dårligere filterkarakteristikker enn det digitale FIR filteret.

\subsubsection{Analog RLC båndpassfilter}
Prinsipielt er det analoge RLC båndpassfilteret simpelt: en spole, en kondensator og en motstand kobles i serie med ground
som vist i figur \ref{fig:rlc_idea}.

\begin{figure}[H]
    \centering
    \includegraphics[width=0.3\textwidth]{Bilder/rlc_bandpass.png}
    \caption{Prinsipielt design for analogt RLC båndpassfilter}
    \label{fig:rlc_idea}
\end{figure}

Utgangsspenningen til filteret blir dermed spenningsfallet over 
motstanden i relasjon til ground. 

For å kunne undersøke påvirkningen av båndpassfilteret på systemet kan 
vi undersøke fourierkoeffisientene til inn- og utgangssignalet.
Vi lar $c_k^2$ representere den $k$-te fourierkoeffisienten til 
utgangssignalet $x_2(t)$, $H(j\omega_k)$ være 
systemresponsen til båndpassfilteret ved den $k$-te multiplum av grunnvinkelfrekvensen
$\omega_1$ og $c_k^y$ være den $k$-te fourierkoeffisienten til $y(t)$. Vi har da at
\begin{equation}
    \label{eq:bandpass_fourier_coeff}
    c_k^2 = H(j\omega_k) \cdot c_k^y
\end{equation}

Vi kan finne frekvensresponsen til filteret ved å regne på spenningsdeling med impedansre, og får at 
\[
    H(j\omega_k) = \frac{R}{R + j\omega_k L + \frac{1}{j\omega_k C}}
\]

Vi ser dermed at for en gitt vinkelfrekvens $\omega_k$ vil få ressonnans i filteret og får da at
\[
    j\omega_k L + \frac{1}{j\omega_k C} = 0
\]
og $H(j\omega_k) = 1$. Filteret blir dermet et båndpassfilter, og ved å tilpasse $L$ og $C$ kan vi oppnå 
at $\omega_k = 2f \cdot 2\pi$, og dermed få ut den overharmoniske ved $2f$ og dempe resten av fourierkoeffisientene til 
utgangssignalet.

Men det er en kaviat; dette gjelder kunn dersom inngangsimpedansen til filteret er mye større enn utgangsimpedansen til det ulineære systemet.
Dette kan vi forsikre oss om ved å koble inn en buffer mellom de to systemene.
Vi lar så inn- og utgangsspenningen til Op-Ampen V+ og V- være større enn $A_1$ for å 
forhindre at den også introduserer nye overharmoniske inn i signalet.

Ser vi på systemet i sin helhet får vi kretsskjemaet i figur \ref{fig:anal_rlc_sys}.
\begin{figure}[H]
    \centering
    \includegraphics[width=0.75\textwidth]{Bilder/Analog_sys.png}
    \caption{Prinsipielt design av det totale systemet med et analogt RLC båndpassfilter}
    \label{fig:anal_rlc_sys}
\end{figure}

For å finne en formel for $L$ og $C$ slik at den ønskede filterkarakteristikken blir oppfylt kan vi se på 
impedanslikningen over (\ref{eq:bandpass_fourier_coeff}):
\[
    Z_L + Z_C = 0 \implies j \cdot 2\pi \cdot 2f \cdot L + \frac{1}{j \cdot 2\pi \cdot 2f \cdot C} = 0
\]

Løser vi likningen for $LC$ får vi at 
\[
    LC = \frac{1}{4\pi^2 \cdot (2f)^2}
\]

Altså kan vi ved å velge en verdi for enten spolen $L$ eller kodensatoren $C$, regne ut 
den andre ukjente komponentverdien vi trenger for filteret.

Videre kan vi gitt verdiene for $L$ og $C$ finne et uttrykk for motstanden $R_2$ gitt en ønsket 
kvalitetsfaktor $Q$ til filteret. Vi har at 

\[
    Q = \sqrt{\frac{L}{C R_2^2}} \implies R_2 = \sqrt{\frac{L}{C Q^2}}
\]

Vi har da alle verdiene vi trenger for å kunne implementere båndpassfilteret.

For å videre forbedre filterkarakteristikken til båndpassfilteret er en kaskadekobling av flere båndpassfiltere mulig \cite{design3}.
Dette kan oppnås ved å koble utgangen av et filter gjennom et buffer til det neste slik som i figur \ref{fig:cascade}. Den resulterende frekvensresponsen til 
systemet for $n$ sammenkoblede filtre blir da 
\[
    H_{tot}(s) = H_1(s) \cdot H_2(s) \cdot H_3(s) \cdot ... \cdot H_{n-1}(s) \cdot H_n(s)
\]

\begin{figure}[H]
    \centering
    \includegraphics[width=0.8\textwidth]{Bilder/buffer.png}
    \caption{Kaskadekobling av filtre \cite{design3}}
    \label{fig:cascade}
\end{figure}

\subsubsection{Digitalt FIR båndpassfilter}
\setcounter{equation}{0}
Et digitalt FIR filter er en type digitalt filter ofte brukt i signalbehandling. Denne prinsipielle løsningen 
skal ta for seg et diskret tid digitalt FIR filter. Prinsipiellt kan et FIR filter beskrives med følgende differensligning \cite{dsp_self}
\begin{equation}
    \label{eq:fir_diff}
    y[n] = \sum_{k=0}^{N} b_k \cdot x[n-k]
\end{equation}

der $y$ er utgangssignalet, $x$ er inngangssignalet, $b_k$ er filterkoeffisientene og $N$ er filterets orden.
Impulsresponsen til filteret blir da gitt ved 
\begin{equation}
    h(n) = \sum_{i=0}^{N} b_i \cdot \delta[n-i] = \begin{cases}
        b_n, &0 \leq n \leq N \\
        0, &\text{ellers}
    \end{cases}
\end{equation}

Vi kan så skrive om differenslikningen til en konvolusjon mellom impulsresponsen $h$ og inngangssignalet $x$
\begin{equation}
    \label{eq:fir_impulse}
    y[n] = x[n] \ast h[n]
\end{equation}

Visualiseringen i figur \ref{fig:fir_filter_overview} er en vanlig fremstilling av 
strukturen til et FIR filter, der operasjonen $z^{-1}$ er en delay operator på signalet, skrevet i Z-transform notasjon.

\begin{figure}[H]
    \centering
    \includegraphics[width=0.75\textwidth]{Bilder/FIR_Filter_overview.png}
    \caption{N-te ordens diskret tid FIR filter \cite{fir_wiki}}
    \label{fig:fir_filter_overview}
\end{figure}

For enkelthetens skyld kan man tenke på $z^{-1}$ som nettop en delay operator, eller forsinkelsesoperator, 
altså at $x[n-1] = z^{-1}x[n]$. Vi ser derfor at et finite impulse response filter er en type vektet sum av 
inngangssignalet $x$ over tid, med andre ord en konvolusjon av filterkoeffisentene $b_k$ over signalet $x$ 
slik vi så i (\ref{eq:fir_impulse}).

Når vi skal designe FIR filteret er hovedsakelig tre parametre vi må ta hensyn til:
\begin{enumerate}
    \item Samplingfrekvensen til systemet - $f_s$
    \item Filterets orden - $N$
    \item Filterkoeffisientene - $b_k$
\end{enumerate}

Samplingfrekvensen $f_s$ er som regel styrt av applikasjonen. For dette systemet må signalet gå fra 
kontinuerlig tid til diskret tid. Det betyr at systemet må ta høyde for aliasing\footnote{Frekvenser over Nyquist-frekvensen fremstår feilaktig som lavere frekvenskomponenter grunnet utilstrekkelig samplingrate \cite{aliasing}} 
når samplingfrekvensen blir valgt.
For å oppnå Nyquist-kriteriet \cite{nyquist} må samplingfrekvensen være \textit{minst} dobbelt så stor som den høyeste
frekvenskomponenten i signalet som blir målt. Dette er et krav for at man skal kunne rekonstruere det opprinnelige signalet 
nøyaktig fra diskrete målinger av et kontinuerlig signal. Dersom $f$ er frekvensdoblerens inngangsfrekvens, må vi for å kunne 
oppnå Nyquist kriteriet for den doble frekvensen $2f$, ha en samplingfrekvens på minst $4f$. Setter vi samplingfrekvensen $f_s \sim 20f$ ligger vi godt innenfor 
Nyquist frekvensen, selv for den doblede frekvensen til den første overharmoniske, 
i tillegg til at vi får god oppløsnig av signalet når vi senere skal analysere utgangssignalet. 
En lavere samplingfrekvens vil da også være mulig dersom applikasjonen krever det. Et argument imot høyere
samplingfrekvens (altså alt over $4f$) er at høyere samplingfrekvens fører til at filterets orden må økes 
for å kompansere. Høyere filterorden medfører større kompleksitet ifht. implementering, og vil øke energiforbruken 
til systemt. Både filterets orden og filterkoeffisentene er strengt avhengig av applikasjonen og hvilken
frekvens vi ønsker å designe båndpassfilteret for. 

For å videre forhindre aliasing når vi sampler signalet ønsker vi å kjøre signalet først gjennom et analogt 
lavpassfilter med cutoff-frekvens lik Nyquist-frekvensen ($f_s/2$).

For design av filterkoeffisientene er MatLab sin Filter designer et godt verktøy. Ved å velge samplingfrekvens, designmåte, filtertype 
og ønsket demping i båndpass og båndstoppområdet kan vi få ut den filterkarakteristikken vi er ute etter. Følgende MatLab script kan benyttes for 
å regne ut filterkoeffisientene.

\begin{matlabcode}
    Fs = 96000;  % Sampling frekvens

    Fstop1 = 3000;            % Første stoppbånd frekvens
    Fpass1 = 4500;            % Første Passband frekvens
    Fpass2 = 5500;            % Andre Passband frekvens
    Fstop2 = 7000;            % Andre Stopband frekvens
    Dstop1_dB = -80;          % Første Stopband Attenuation in dB
    Dpass_dB  = -0.5;         % Passband Ripple in dB
    Dstop2_dB = -80;          % Andre Stopband Attenuation in dB
    flag   = 'scale';         % Sampling Flag

    % Konverter dB tilbake til lineære tall
    Dstop1 = 10^(Dstop1_dB/20);
    Dpass  = 10^(Dpass_dB/20);
    Dstop2 = 10^(Dstop2_dB/20);

    % Beregn filterordenen med funksjonen KAISERORD.
    [N,Wn,BETA,TYPE] = kaiserord([Fstop1 Fpass1 Fpass2 Fstop2]/(Fs/2), 
    [0 ... 1 0], [Dstop1 Dpass Dstop2]);

    % Beregn filterkoeffisientene ved hjelp av FIR1 funksjonen
    b  = fir1(N, Wn, TYPE, kaiser(N+1, BETA), flag);
    Hd = dfilt.dffir(b);

    % Plot frekvensresponsen til filteret
    freqz(b, 1, 2024, Fs);
\end{matlabcode}

I eksemplet over blir et FIR båndpassfilter samplingfrekvens på $\SI{96}{\kilo\hertz}$ og passbånd fra
$\SI{4500}{\hertz}$ til $\SI{4500}{\hertz}$ valgt, med en demping på $-80$ desibel i båndstop området, designet.
Variabelen med navnet \textit{b} innholder informasjonen om filterkoeffisientene $b_n$.

To prinsipielle løsninger for hvordan FIR filteret kan benyttes er gitt:
\begin{enumerate}
    \item Programvare gjennom python kode
    \item Hardware implementering på FPGA med SystemVerilog
\end{enumerate}

\textbf{Python program}\\
Python programmet tar inn et parameter \textit{signal} og en kjerne \textit{kernel} - filterkoeffisientene - og benytter 
konvolusjon på signalet for å produsere et filtrert utgangssignal \textit{output}. 
Se \ref{code:fir_bandpass_python} for implementeringen.

Denne python programvaren kjører ikke i sanntid, og er derfor begrenset til applikasjoner der sanntidsaspektet av 
systemet ikke er nødvendig, men kan lett modifiseres til å kjøre i sanntid dersom det er nødvendig, eksempelvis 
på en mikrokontroller. 

\textbf{Hardware - SystemVerilog}\\
En måte å oppnå at filteret kjører i sanntid er ved å implementere filteret i hardware. For dette designet er FGPA 
teknologi valgt, siden det gir stor fleksibilitet for implementering og reprogramerbarhet. Språket valgt for utviklingen av 
RTL kode er SystemVerilog - et språk ofte brukt innen mikroelektronikk industrien, og som også har visse funksjoner for verifikasjon ikke til stede
i Verilog.

For å oppnå sanntidsfiltrering må systemet kunne ta inn en kontinuerlig datastrøm, her med frekvens $f_s$, og sende ut et filtrert signal 
med samme frekvens. For å tillate systemklokken å være uavhengig av samplingfrekvensen $f_s$ er et datasignal \textit{drdy}\footnote{Forkortning: Data ready}
valgt som et av inputparametrene til filteret sammen med datasignalet \textit{data\_in[11:0]} og systemklokke-signalet \textit{clk}. Filteret sender så ut 
et datasignal \textit{data\_out[11:0]}. 12-bit databredde er valgt for både inn- og utgangssignalet. Dette er fordi det gir en oppløsning på omlag $0.0244$\%,
og er antatt tilstrekkelig for de aller fleste signaler. Høyere databredde er mulig dersom applikasjonen krever det, men dette setter større krav til 
analog til digital og digital til analog omformeren. Til slutt er også et reset signal \textit{rst} valgt, noe som gir mulighet for en ekstern reset av filtersystemet.
Moduldeklarasjonen til filteret er vist under
\begin{verilogcode}
module FIR_FILTER (
    input                   clk,
    input   signed  [11:0]  data_in,
    input                   drdy,
    input                   rst,
    output  signed  [11:0]  data_out);
\end{verilogcode}

For å kunne gjøre multiplikasjon mellom inngangssignalet $x$ og filtercoeffisientene $h$ i likning
\ref{eq:fir_diff}, må vi først konvertere filterkoeffisientene til fikspunkt form \cite{fixed_point}. 
Det er antatt her at inngangsdatasignalet \textit{data\_in} allerede er på fikspunkt format.
Dette er slik at vi kan gjøre multiplikasjon med rasjonale tall. Dette kan vi få til med følgende MatLab kode:
\begin{matlabcode}
b_fixedpoint = fi(b, 1, antall_bits, brøkdel_bits) 
\end{matlabcode}
Her blir filterkoeffisientene fra matlabscriptet over konvertert til et toerkomplett fikspunkt tall 
med \textit{antall\_bits} bits, der antallet bits som går til brøkdelen av tallet er \textit{brøkdel\_bits} antall bits.
Det er her viktig å se på filterkoeffisientene før man velger antall bits for brøkdelen, ettersom at de resterende bits-ene vil representere 
heltalls delen av tallet. Om alle koeffisiente er brøker mellom $(-1, 1)$, kan vi sette alle bitsene til brøkdelen.

Når vi har alle koeffisientene på fikspunkt kan vi omforme de til et binært tall og legge de inn i filteret
\begin{verilogcode}
reg signed [23:0] coeff [0:N] = { b0, b1, ..., bN };
\end{verilogcode}
Over er 24 bits valgt for \textit{antall\_bits}.

For at filteret skal kunne operere i sanntid benytter vi et konsept som heter \textit{pipelining}\cite{pipelining}.
Vi deklarerer defor følgende registre 
\begin{verilogcode}
reg signed [11:0] shift_register [0:N];
reg signed [35:0] product [0:N];
reg signed [35:0] accumulate;
\end{verilogcode}

Når et nytt datapunkt kommer inn vil følgende kode kjøre 
\begin{verilogcode}
always @(posedge clk) begin // Shift register
    if (rst) begin
        for (i = 0; i < N; i = i + 1) begin
            shift_register[i] <= 0;
        end
        accumulate <= 36'd0;
    end else if (drdy) begin
        shift_register[0] <= data;
        for (j = 1; j < N; j = j + 1) begin
            shift_register[i] <= shift_register[i-1];
        end
    end
end
\end{verilogcode}
der dataverdien blir sendt gjennom et shiftregister slik at nyeste verdi alltid ligger først, etterfulgt 
av tidligere måledata. På den måten kan vi kjøre en konvulsjon over data-en som kommer inn.

Videre blir data-en multiplisert med filterkoeffisientene 
\begin{verilogcode}
always @(posedge clk) begin // Multiply
    for (k = 0; k < N; k = k + 1) begin
        product[k] <= shift_register[k] * coeff[k];
    end
end
\end{verilogcode}

og så blir resultatet akkumulert 
\begin{verilogcode}
always @(posedge clk) begin // Accumulate
    accumulate <= product[0] + product[1] + product[2] + ...
                  product[N-2]  + product[N-1]  + product[N];
end
\end{verilogcode}
Vi har nå oppnådd en konvulsjon mellom en innkommende datastrøm med filterkoeffisientene $b_n$.
Utgangsdataen blir så sendt ut ved å sette den lik de øvre 12 bits-ene av det akkumulerte tallet
\begin{verilogcode}
assign data_out = $signed(accumulate[35:24]);
\end{verilogcode}

For hele koden se \ref{code:fir_bandpass_sysverilog}.

Vi kan altså konstruere et system som vist i figur \ref{fig:fir_filter_system_overview} der spenningssignalet $y(t)$ fra det ulineære systemet 
blir først sendt gjennom et lavpassfilter med knekkfrekvens lik $f_s/2$, så gjennom en analog til digital omformer som kan måle både negative og positive spenninger,
gjennom det digitale FIR filteret, enten det er python programmet eller FPGA implementeringen, og så til slutt gjennom en digital til analog omformer som kan omforme til 
både negative og positive spenninger.
\begin{figure}[H]
    \centering
    \includegraphics[width=0.95\textwidth]{Bilder/FIR_FILTER_SYSTEM.png}
    \caption{Kretstegning av systemet med et FIR filter}
    \label{fig:fir_filter_system_overview} 
\end{figure}


\subsection{Signal-til-distorsjonsforholdet}
For å kunne gi en kvalitativ måling på hvor godt systemet klarer å produsere en tilnærming 
av et rent sinussignal ved den doble frekvensen er SDR foreslått. 

Vi kan tenke oss at det produserte signalet $\hat{x}_2(t)$ er summen av det ønskede utgangssignalet $x_2(t)$
og et forstyrringssignal $d(t)$. Ved å se på forholdet mellom det ønskede signalet $x_2(t)$ og forstyrringssignalet 
$d(t)$, kan vi få et mål på hvor godt signalet er. Men siden vi har at alle signalene stammer fra samme kilde med overharmoniske 
perioder som går opp i $T = \frac{1}{f}$, kan vi gjøre en forenkling og finne SDR verdien til $\hat{x}_2(t)$ som 
\[
    \text{SDR} = \frac{P_{x_2}}{P_{\hat{x}_2} - P_{x_2}} = \frac{V_{x_2}^2}{V_{\hat{x}_2}^2 - V_{x_2}^2}
\]
der $V_{x_2}$ og $V_{\hat{x}_2}$ er V-RMS verdien til henholdsvis den ønskede frekvenskomponenten ved $2f$ og til det totale utgangssignalet 
$\hat{x}_2(t)$ \cite{frekvens_mul}.

