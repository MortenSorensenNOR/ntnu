\section{Realisering}
\label{realiseringOgTest}

Realiseringen av systemet er blitt gjort ved å sende et generert sinussignal 
med frekvens $f = \SI{2600}{\hertz}$ og amplitude $A_1 = \SI{1}{\volt}$ gjennom systemet.
Signalet ble generert av an Analog Discovery 2 \cite{analog_discovery}. 

For evalueringen av hele systemet ble Analog Discovery-ens to oscilloskop brukt. Den første, Channel 1,
ble koblet til inngangssignalet $x_1(t)$. Den andre, Channel 2, ble koblet 
til utgangssignalet $x_2(t)$.
For hvert av de underliggende systemene - det ulineære systemet og båndpassfilteret - ble 
Channel 1 koblet til inngangssignalet og Channel 2 koblet til utgangssignalet av systemet.

\subsection{Det ulineære systemet}
For realiseringen av det ulineære systemet har en motstandsverdi $R_1$ på $\SI{100}{\kilo\ohm}$ blitt brukt. Denne komponentverdien 
ble valgt for å forhindre mulige ikke-idealiteter i den valgte dioden med henhold til diodens motstandsverdi.

\subsection{Båndpassfilter}

\subsubsection{RLC båndpass}
For realiseringen av det analoge RLC båndpassfilteret har en induktansverdi på $L = \SI{100}{\milli\henry}$ blitt valgt. To slike spoler ble 
målt, der begge ble målt til en verdi på $L = \SI{100}{\milli\henry} \pm \SI{1}{\milli\henry}$ over et spektrum fra $\SI{1}{\kilo\hertz}$ til $\SI{20}{\kilo\hertz}$. 
Får å oppnå den ønskede senterfrekvensen til systemet får vi at kondensatorverdien til 
RLC filteret blir 
\[
    C = \frac{1}{4\pi^2 \cdot (2f)^2 \cdot L} = \SI{9.37}{\nano\farad}
\]

For realiseringen av systemet har en parallellkobling av to $\SI{4.7}{\nano\farad}$
kondensatorer blitt brukt, slik vist i figur \ref{fig:impl_rcl_sys}.
Parallellkoblingen av kondensatorene ble målt til en kombinert kapasitansverdi på $C = \SI{9.265}{\nano\farad} \pm \SI{0.08}{\nano\farad}$ over et spektrum fra $\SI{1}{\kilo\hertz}$ til $\SI{20}{\kilo\hertz}$.
Dette gir en effektiv senterfrekvens til realiseringen av filteret på $f_c = \SI{5226}{\hertz}$.

\begin{figure}[H]
    \centering
    \includegraphics[width=0.75\textwidth]{Bilder/Analog_sys_val.png}
    \caption{Realisering av et RLC båndpassfilter med senterfrekvens $f_c = \SI{5.2}{\kilo\hertz}$}
    \label{fig:impl_rcl_sys}
\end{figure}

Til slutt er et $\SI{10}{\kilo\ohm}$ potensiometer valgt for $R_2$ for å kunne justere kvalitetsfaktoren
til filteret. 

Det endelige systemet ble valgt til å ha et andre ordens filter der et identisk filter til filteret i figur \ref{fig:impl_rcl_sys} ble 
kaskadekoblet med en buffer i mellom. Implementeringen av systemet er vist i figur \ref{fig:impl_rcl_sys_img}.

\begin{figure}[H]
    \centering
    \includegraphics[width=0.4\textwidth]{Bilder/Analog_sys_img.jpg}
    \caption{Realisering av et andre ordens RLC båndpassfilter med senterfrekvens $f_c = \SI{5.2}{\kilo\hertz}$}
    \label{fig:impl_rcl_sys_img}
\end{figure}

\subsubsection{FIR båndpass}
\begin{tcolorbox}[colback=white, colframe=white!55!black]
    \textbf{TODO:}
    \begin{enumerate}
        \item Lage testbench for HDL kode
        \item Forklare utregning av filterkoeffisienter - Matlab
            \begin{itemize}
                \item FIR bandstop 
                \item Sample frequency $\SI{50}{\kilo\hertz}$
                \item Equaripple
                \item Variere filter orden N
            \end{itemize}
    \end{enumerate}
\end{tcolorbox}

\begin{figure}[H]
    \centering
    \includegraphics[width=0.9\textwidth]{Bilder/Digital_sys.png}
    \caption{Realisering av system med FIR båndpassfilter og en samplingfrekvens $f_s = \SI{50}{\kilo\hertz}$}
    \label{fig:fir_sys}
\end{figure}

\begin{figure}[H]
    \centering 
    \includegraphics[width=0.9\textwidth]{Bilder/fir_filter_perf.png}
    \caption{Sammenlikning av første og andre ordens RLC båndpassfilter med senterfrekvens $f_c = \SI{5.2}{\kilo\hertz}$}
    \label{fig:test_fir_filter_perf}
\end{figure}

\begin{figure}[H]
    \centering 
    \includegraphics[width=0.9\textwidth]{Bilder/fir_filter_perf_250.png}
    \caption{Sammenlikning av første og andre ordens RLC båndpassfilter med senterfrekvens $f_c = \SI{5.2}{\kilo\hertz}$}
    \label{fig:test_fir_filter_perf_250}
\end{figure}

\begin{figure}[H]
    \centering 
    \includegraphics[width=0.9\textwidth]{Bilder/fir_filter_perf_175.png}
    \caption{Sammenlikning av første og andre ordens RLC båndpassfilter med senterfrekvens $f_c = \SI{5.2}{\kilo\hertz}$}
    \label{fig:test_fir_filter_perf_175}
\end{figure}

\begin{figure}[H]
    \centering 
    \includegraphics[width=0.5\textwidth]{Bilder/Zturn.jpg}
    \caption{Z-Turn V2 7Z020 FPGA board \cite{zturn}}
    \label{fig:zturn}
\end{figure}

\begin{figure}[H]
    \centering 
    \includegraphics[width=0.9\textwidth]{Bilder/Vivado_fir_filter.png}
    \caption{Blokk diagram fra Vivado; System implementering av FIR på FPGA}
    \label{fig:vivado_fir_block_diagram}
\end{figure}
