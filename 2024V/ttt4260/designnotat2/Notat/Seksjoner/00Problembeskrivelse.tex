\newpage
\section{Problembeskrivelse}
\label{problemBeskrivelse}

Dette designnotatet tar for seg et mulig design av en elektronisk terning. Terningen skal ha syv lysdioder av forskjellige farger, som til sammen 
danner et mønster av prikker tilsvarende de man finner på en ordinær terning. Terningen skal kunne "kastes" ved å holde inne en knapp, som, når man slipper 
knappen, fører til at et (tilsynelatende) tilfeldig tall mellom 1 og 6 vises på lysdiodene som nevnt. Et design av en slik elektronisk terning er vist 
i figur 1.

\begin{figure}[H]
    \centering
    \includegraphics[width=0.35\textwidth]{Bilder/terning_ex.png}
    \caption{Et eksempel på en elektronisk terning}
\end{figure}

Notatet skal ta for seg hvordan en FPGA av typen Lattice ICE40 FPGA kan benyttes for å realisere den elektroniske terningen.
I tillegg skal effektforbruket til terningen undersøkes, både når terningen "kastes" og når den viser terningens resultat.
