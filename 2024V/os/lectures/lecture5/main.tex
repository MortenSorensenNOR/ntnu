\documentclass[a4paper,11pt,english]{article}
\usepackage{packages}

\begin{document}

%Headingdel:---------------------------------------------
\topmargin -1.5cm
\makebox[\textwidth][s]{
    \begin{minipage}[c]{0.25\textwidth}
        \includegraphics[width=2.0cm]{Bilder/ntnu_logo.png}  
    \end{minipage}
    \begin{minipage}[c]{0.75\textwidth}
        \huge{\textbf{TFE4120  Electromagnetism}} \\
        \Large{Exercise 7  ---  Morten Sørensen, \large{\color{black!75!white}\today}}
    \end{minipage}
}
\vspace{0.75cm}
\normalsize


\textbf{\color{blue} }

\section*{Memory}
\begin{itemize}
    \item RECALL: a program must be brought from disk into \textbf{\color{blue} memory} and becomes a process 
\end{itemize}

\section*{Adress Space}
\begin{itemize}
    \item \textbf{\color{red} Why do we need an abstraction for memory?}
        \begin{itemize}
            \item \textbf{\color{red} Early computers did not have memory abstraction}
            \item \textbf{\color{red} Programs directly access physical memory}
        \end{itemize}
    \item \textbf{\color{blue} Uniprogramming}: Only one process runs at a time
    \item \textbf{\color{blue} Multiprogramming} (Now):
        \begin{itemize}
            \item Processes share hardware resources, CPU, memory
        \end{itemize}
    \item Instead of having only one process \textbf{\color{blue} multiple processes} reside in the memory
    \item Processes may \textbf{\color{red} interfere} with each other and suffer from access issues if \textbf{\color{blue} no memory abstraction (protection)}
    \item One process can overwrite another process
    \item It works with \textbf{\color{blue} Program Status Word (PSW)}, but has \textbf{\color{blue} relocation issue}
    \item \textbf{\color{blue} Adress space}
        \begin{itemize}
            \item An easy-to-use abstraction of physical memory
            \item The set of adresses that a process can use to address memory
            \item The running program's view of memory in the system
        \end{itemize}
    \item \textbf{\color{blue} Adress space} consists of \textbf{\color{blue} static} and \textbf{\color{blue} dynamic} components
        \begin{itemize}
            \item \textbf{\color{blue} Static}: code and some global variables
            \item \textbf{\color{blue} Dynamic}: stack and heap
        \end{itemize}
    \item 3 components (more than 3, but enough to get the idea)
        \begin{itemize}
            \item \textbf{\color{blue} Code}
            \item \textbf{\color{blue} Heap}
            \item \textbf{\color{blue} Stack}
        \end{itemize}
    \item Code will not be changed during execution, so it is static
    \item \textbf{\color{blue} Stack}
        \begin{itemize}
            \item Local variables, pass parameters, return values
            \item Last in first out
            \item Simple and efficient implementation: stack pointer
                \begin{itemize}
                    \item Add/Push: increment pointer
                    \item Free/Pop: decrement pointer
                \end{itemize}
        \end{itemize}
    \item \textbf{\color{blue} Heap}
        \begin{itemize}
            \item Explicitly dynamically-allocated, user-managed memory, such as malloc() in C or \textbf{\color{blue} new} in C++ and Java
            \item Shoudl be deallocated explicitly by programmers, free()
        \end{itemize}
    \item OS virtualizes memory by using \textbf{\color{blue} virtual adresses}
    \item \textbf{\color{blue} Adress translation/mapping}: virtual to physical
    \item Goals of virtual memory
        \begin{itemize}
            \item \textbf{\color{blue} Transparency}
                \begin{itemize}
                    \item Processes are unaware that memory is shared 
                \end{itemize}
            \item \textbf{\color{blue} Protection}
                \begin{itemize}
                    \item Cannot read data of other processes
                    \item Cannot corrupt OS or other processes
                \end{itemize}
            \item \textbf{\color{blue} Efficiency}
                \begin{itemize}
                    \item Time
                    \item Space
                \end{itemize}
        \end{itemize}
    \item Any adress that the programmer can see is a \textbf{\color{blue} virtual adress}
    \item Compiler/linker assumes that the process starts at \textbf{\color{blue} 0}
    \item When the OS loads the process, it allocates a contiguous segment of memory where the process fits
    \item How to relocate the adresses in a transparent way?
    \item Address translation methods:
        \begin{itemize}
            \item \textbf{\color{blue} Static relocation (no HW requirement)}:
                \begin{itemize}
                    \item At load time, the OS adjusts the addresses in a process to reflect its position in memory
                    \item Once a process is assigned a place in memory and starts executing it, the OS cannot move it
                    \item \textbf{\color{red} Issue: inflexibility, rewrite}
                \end{itemize}
            \item \textbf{\color{blue} Dynamic relocation (HW supported):}
                \begin{itemize}
                    \item Two registers for each process
                        \begin{itemize}
                            \item \textbf{\color{blue} Base register} indicates the starting adress
                            \item \textbf{\color{blue} Bound/limit register} determines the memory size
                        \end{itemize}
                    \item \textbf{\color{blue} Memory-Management Unit (MMU)}
                        \begin{itemize}
                            \item Translate virtual addresses to physical addresses
                            \item Such translation happens for every memory reference
                        \end{itemize}
                \end{itemize}
        \end{itemize}
\end{itemize}

\section*{Dynamic Relocation}
\begin{itemize}
    \item Every memory reference has the following procedure
        \begin{itemize}
            \item When a program starts running, the OS decides where in physical memory a process should be loaded. Set the base register
            \item Each memory reference is tranlated as follows
                \[
                    physical address = virtual adress + base
                \]
            \item Chech whether the adress is within the bound
        \end{itemize}
    \item Assume a process with its base register of 32KB and its size of 16KB
    \item fetch instruction \textbf{\color{blue} ldr} at \textbf{\color{blue} virtual adress} 128
    \item The physical adress of instruction \textbf{\color{blue} ldr} is $32896 = 128 + 32KB(base)$
    \item The largets address this process can access is $47KB = 16KB + 32KB$
    \item What is the \textbf{\color{red} problem} of dynamic relocation?
        \begin{itemize}
            \item \textbf{\color{red} Fragmentation (memory inefficiency)}
            \item Internal or external
        \end{itemize}
\end{itemize}

\section*{Memory Allocation}
\begin{itemize}
    \item Allocation \textbf{\color{blue} contiguous} memory spaces to different processes:
        \begin{itemize}
            \item Fixed partition
                \begin{itemize}
                    \item Physical memory is broken up into fixed partitions
                    \item Only one \textbf{\color{blue} base register} is needed for each process
                    \item \textbf{\color{blue} Advantage}: easy to implement and fast context switch
                    \item \textbf{\color{red} Problem: internal fragmentation}
                \end{itemize} 
            \item Variable partition
                \begin{itemize}
                    \item \textbf{\color{blue} Base register} and \textbf{\color{blue} limit register} (dynamic relocation)
                    \item \textbf{\color{blue} Advantage}: flexible
                    \item \textbf{\color{red} Problem: external fragmentation} 
                \end{itemize}
        \end{itemize}
\end{itemize}

\section*{Internal Fragmentation}
\begin{itemize}
    \item Each partition has the same size 
\end{itemize}

\section*{Variable Partition}
\begin{itemize}
    \item Each process is assigned a memory partition as it needs
    \item OS maintains a table of available memory space, also called \textbf{\color{blue} hole; holes of various size are scattered throughout memory}
    \item Allocation algorithms:
        \begin{itemize}
            \item \textbf{\color{blue} First-fit:} allocate the first hole that is big enough
            \item \textbf{\color{blue} Best-fit:} allocate the smalles hole that is big enough
            \item \textbf{\color{blue} Worst-fit:} allocate the largest hole
        \end{itemize}
    \item Each process is assigned a memory partition as it needs
    \item As processes are loaded to memory and removed from memory, the memory freed are fragmented into small \textbf{\color{red} holes}
    \item \textbf{\color{blue} Sufficient memory} to complete a request, but \textbf{\color{blue} not contigous}
\end{itemize}

\section*{External Fragmentation}
\begin{itemize}
    \item \textbf{\color{blue} Compaction} is a solution to external framentation
        \begin{itemize}
            \item Shuffle the memory contets to place all free memory together in one large block
        \end{itemize}
    \item \textbf{\color{blue} Concerns:} performance overhead
        \begin{itemize}
            \item Memory intensive
        \end{itemize}
\end{itemize}

\section*{Segmentation}
\begin{itemize}
    \item Instead of having only one pair of registers, divide the contigous address space into several \textbf{\color{blue} logic segments}
    \item If an \textbf{\color{red} illegal address} such as 7KB which is beyond the end of heap is referenced, the OS occurs \textbf{\color{red} segmentation fault}.
        \begin{itemize}
            \item The hardware detects that the address is \textbf{\color{red} out of bounds}
        \end{itemize}
    \item \textbf{Explicit approach}
        \begin{itemize}
            \item Break the address space into segments based on the \textbf{\color{blue} top few bits} of virtual address.
        \end{itemize}
    \item Stack grows \textbf{\color{blue} backward}
    \item \textbf{\color{blue} Extra hardware support} is needed.
        \begin{itemize}
            \item The hardware checks which way the segment grows
            \item 1: positive direction, 0: negative direction
        \end{itemize}
    \item Segment can be \textbf{\color{blue} shared} between address space
        \begin{itemize}
            \item \textbf{\color{blue} Code sharing} is still in use in systems today
            \item Need extra hardware support
        \end{itemize}
    \item \textbf{Hardware support: protection bits}
        \begin{itemize}
            \item \textbf{A few more bits} per segment to indicate \textbf{permissions} of \textbf{read}, write and \textbf{execute}
        \end{itemize}
\end{itemize}

\end{document}
