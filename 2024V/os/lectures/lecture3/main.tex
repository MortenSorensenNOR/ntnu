\documentclass[a4paper,11pt,english]{article}
\usepackage{packages}

\begin{document}

%Headingdel:---------------------------------------------
\topmargin -1.5cm
\makebox[\textwidth][s]{
    \begin{minipage}[c]{0.25\textwidth}
        \includegraphics[width=2.0cm]{Bilder/ntnu_logo.png}  
    \end{minipage}
    \begin{minipage}[c]{0.75\textwidth}
        \huge{\textbf{TTT4260 ESDA}} \\
        \Large{Øving 3  ---  Morten Sørensen, \large{\color{black!75!white}\today}}
    \end{minipage}
}
\vspace{0.75cm}
\normalsize


\section*{Scheduling}
\begin{itemize}
    \item \textbf{\color{blue} Scheduling} is an important problem in many contexts, e.g. logistics, airports, game schedule, etc.
    \item \textbf{\color{blue} Scheduling}: policies that the OS employs to determine the exectution order of ready processes
    \item \textbf{\color{blue} Scheduling alorithms/schedulers} have diverse objectives and demonstrate different effects on the system performance.
\end{itemize}

\subsection*{Terminologies}
\begin{itemize}
    \item Scheduling metrics and goals:
        \begin{itemize}
            \item \textbf{\color{blue} CPU utilization}: percentage of time CPU is busy executing jobs
            \item \textbf{\color{blue} Throughput}: the number of processes completed in a given amount of time
            \item \textbf{\color{blue} Fairness}: all processes are likely to be executed
            \item \textbf{\color{red} Turnaround time}: the time elapesed between the arrival and completion of a process
                \[
                    T_{turnaround} = T_{complete} - T_{arrival}
                \]
            \item \textbf{\color{red} Waiting time}: the time a process spends in the ready queue
            \item \textbf{\color{red} Response time}: the time elapsed between the process' arrival and its first output
        \end{itemize}
    \item \textbf{\color{blue} Maximize: CPU utilization, throughput, fairness}
    \item \textbf{\color{red} Minimize: Turnaround time, waiting time, and response time}
\end{itemize}

\subsection*{Assumptions For Workloads}
\begin{itemize}
    \item Each job runs for the same amount of time
    \item All jobs arrive at the same time
    \item All jobs only use the CPU (no I/O)
    \item Run-time of each job is known
    \item Once started, each job runs to completion (\textbf{\color{blue} Preemption})
\end{itemize}

\subsection*{First in First out (FIFO)}
\begin{itemize}
    \item A simple and basic scheduling algorithm
    \item First Come First Served (FCFS)
    \item Jobs are executed in \textbf{\color{blue} arrival time order}
    \item How about in therms of \textbf{\color{blue} turnaround}?
    \item $T_t = T_c - T_a$
    \item Convoy effect
        \begin{itemize}
            \item A scheduling phenomonon in which a number of jobs wait for one job to get off a core, causing overall device and CPU utilization to be suboptimal. 
        \end{itemize}
    \item Order of jobs matter for turnaround time
\end{itemize}

\subsection*{Shortest Job First (SJF)}
\begin{itemize}
    \item Policy: The jobs with the shortest execution time is scheduled first
    \item 
        \begin{itemize}
            \item All jobs arrive at the same time
            \item All jobs only use the CPU (no I/O)
            \item Run-time of each job is known
            \item Once started, each job runs to completion (\textbf{\color{blue} Preemption})
        \end{itemize}
    \item If the 4 assumptions all hold, SJF is an \textbf{\color{blue} optimal} scheduling algorithm in terms of \textbf{\color{blue} average waiting time}
    \newline
    \item Now remove the assumption that all jobs arrive at the same time
    \item How can we reduce the average turnaround time?
    \item We also remove the assumption that each job runs to completion 
\end{itemize}

\subsection*{Shortest Time-to-Complete First (STCF)}
\begin{tcolorbox}[colback=blue!15!white, colframe=white!50!blue]
    \begin{center}
        Once started, each job runs to completion (\textbf{\color{blue} Preemption})
    \end{center}
\end{tcolorbox}
\begin{itemize}
    \item This assumption indicates the concept of \textbf{\color{blue} preemption}
    \item FIFO and SJF are bot \textbf{\color{blue} non-preemtive} schedulers
    \item STCF policy: Always switch to jobs with the shortest completion time
    \item STCF is a \textbf{\color{blue} preemptive} scheduler
    \item STCF is optimal in terms of minimizing the average waitinig time, if the assumptions hold
    \item STCF may cause \textbf{\color{red} starvation} 
\end{itemize}

\subsection*{Response Time}
\begin{itemize}
    \item \textbf{\color{blue} Response time} is another important performance metric, especially for interactive applications
        \[
            R = \frac{1}{N}\sum_{i=0}^{N}T_{wait}(i)
        \]
\end{itemize}

\subsection*{Round-Robin (RR)}
\begin{itemize}
    \item A new concept of \textbf{\color{blue} time quantum/time slice/scheduling quantum}: a fixed and small amount of time units
    \item Each process executes for \textbf{\color{blue} a time slice}
    \item Switches to another one \textbf{\color{red} regardless whether it has completed its execution or not}
    \item If the job has not yet completed its execution, the incomplete job is added to the tail of the ready queue, FIFO queue
    \item If there are \textbf{\color{blue} n} processes in the ready queue and the time slice is \textbf{\color{blue} q}, no process waits more than \textbf{\color{blue} $(n-1) \cdot q$} time units
    \item RR is a good scheduler in terms of \textbf{\color{blue} response time}
    \item But poor in terms of \textbf{\color{red} turnaround time}
    \item The selection of time slice size should be carefullly considered (Usually \textbf{\color{blue} 10-100 milliseconds})
        \begin{itemize}
            \item Switching between processes comes at some overhead, i.e., context switching
            \item Turnaround time depends on the size of time slice
        \end{itemize}
    \item RR is a \textbf{\color{blue} starvation-free} scheduler
        \begin{itemize}
            \item Some processes never get scheduled, e.g., processes with long execution time in STCF
        \end{itemize}
    \item RR is fair, simple, and easy to implement and thus is used in modern OSs, such as Linux and MacOS
    \item XV6 implements a simple RR
\end{itemize}

\subsection*{XV6 Scheduler}
\begin{itemize}
    \item Scheduler is implemented as an individual \textbf{\color{blue} kernel thread}
        \begin{itemize}
            \item \textbf{\color{blue} To schedule a new task, it needs to first switch the scheduler thread}
        \end{itemize}
    \item Two functions
        \begin{itemize}
            \item \textbf{\color{blue} sched()}
                \begin{itemize}
                    \item Check various conditions in \textbf{\color{blue} yield()},\textbf{\color{blue} exit()} and \textbf{\color{blue} sleep()} 
                    \item Hand the control over to the scheduler function
                \end{itemize}
            \item \textbf{\color{blue} scheduler()}
                \begin{itemize}
                    \item Select a runnable process to execute
                    \item Per-CPU core tread, i.e., each CPU has one scheduler
                \end{itemize}
        \end{itemize}
    \item Context-switch
        \begin{itemize}
            \item \textbf{\color{blue} swtch(struct context\* old, struct context\* new)}
                \begin{itemize}
                    \item \textbf{\color{blue} Save the old context and load the new context}
                \end{itemize}
        \end{itemize}
\end{itemize}

\end{document}
