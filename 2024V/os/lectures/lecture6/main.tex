\documentclass[a4paper,11pt,english]{article}
\usepackage{packages}

\begin{document}

%Headingdel:---------------------------------------------
\topmargin -1.5cm
\makebox[\textwidth][s]{
    \begin{minipage}[c]{0.25\textwidth}
        \includegraphics[width=2.0cm]{Bilder/ntnu_logo.png}  
    \end{minipage}
    \begin{minipage}[c]{0.75\textwidth}
        \huge{\textbf{TTT4260 ESDA}} \\
        \Large{Øving 3  ---  Morten Sørensen, \large{\color{black!75!white}\today}}
    \end{minipage}
}
\vspace{0.75cm}
\normalsize


\section*{Paging}
\begin{itemize}
    \item Segmentation allows \textbf{\color{blue} non-contiguous} memory assignments
    \item Segmentation gets memory fragmented (\textbf{\color{red} external frament})
        \begin{itemize}
            \item Perfromance overhead (\textbf{\color{red} compaction})
        \end{itemize}
    \item Paging is another method that allows the physical address space of a process to be \textbf{\color{blue} non-contiguous}
    \item Divide virtual memory into blocks of the same size called \textbf{\color{blue} page frames} (or simply \textbf{\color{blue} pages})
    \item Flexible mapping pages to frames
    \item Keep a track of all free frames
    \item Set up a \textbf{\color{blue} page table} to translate
    \item Nedd both OS and Hardware
    \item No \textbf{\color{blue} external fragmentation}, but \textbf{\color{red} internal fragmentation} 

    \item \textbf{Flexibility}: Supporting the abstraction of address space effectibely
        \begin{itemize}
            \item Don't need assumption of how heap and stack grow and are used
        \end{itemize}
    \item \textbf{Simplicity}: ease of free-space management
        \begin{itemize}
            \item The page address space and the page frame are the same size
            \item Easy to allocate and keep a free list
        \end{itemize}
\end{itemize}

\section*{Page Translation}
\begin{itemize}
    \item A virtual address is split into \textbf{\color{blue} two parts}
        \begin{itemize}
            \item \textbf{\color{blue} Virtual page number} (or \textbf{\color{blue} Page number}) - used as an index into a page table
                which contains base address of each pag ein physical memory
                \begin{itemize}
                    \item High bits to indicate page number
                \end{itemize}
            \item \textbf{\color{blue} Offset} - combined with base address to define the physical memory address that is sent to the memory unit
                \begin{itemize}
                    \item Low bits to indicate offset
                \end{itemize}
        \end{itemize}
    \item Given virutual address space $2^m$ and page size $2^n$, \textbf{\color{blue} n} bits for offset and $\color{blue}m-n$ bits for page number
    \item Only need to translate the page number to determine where the physocal page is
    \item Offset remains the same
    \item \textbf{\color{blue} 32-bit} virtual address
    \item Page size \textbf{\color{blue} 4KB}
    \item Physical memory \textbf{\color{blue} 2GB}
    \item Then $\log{2^{12}} = \text{12 bits}$ for the offset is needed
    \item And for the page number $\color{blue} \text{2GB}/\text{4KB} = 2^{31-12} = 2^{19} = \text{19 bits}$ are needed
\end{itemize}

\section*{Page Table}
\begin{itemize}
    \item To keep track of the mapping of virtual to physical addresses
    \item A \textbf{\color{blue} per-process} data structure
    \item the page table data structure is kept in \textbf{\color{blue} main memory}
    \item \textbf{\color{blue} Page-table base register} indicates the starting address of page table
    \item The simples from a page table is a \textbf{\color{blue} linear page table}, an array
    \item \textbf{\color{blue} Page Table Entry (PTE)} includes
        \begin{itemize}
            \item Translation information
            \item Other information
        \end{itemize}
    \item \textbf{\color{blue} XV6 - 64 bits, each PTE consists of}
        \begin{itemize}
            \item \textbf{\color{blue} 44 bits (10 - 53) for physical page number}
            \item \textbf{\color{blue} 10 bits (0 - 9) for flags, other information}
        \end{itemize}
\end{itemize}

\section*{Address Translation}
\begin{itemize}
    \item XV6-Sv39
        \begin{itemize}
            \item \textbf{\color{blue} 39 bit} virtual address
            \item \textbf{\color{blue} 27 bit} to index PTE
            \item \textbf{\color{blue} 12 bit} to determine the address within a page frame
            \item 39 bits $\rightarrow$ 512 GB sufficient for RISC-V
        \end{itemize}
    \item XV6-SV48
        \begin{itemize}
            \item 48 bit virtual address
        \end{itemize}
\end{itemize}

\section*{Inverted Page TAbles}
\begin{itemize}
    \item Keeping a signle pag table that has an etry for \textbf{\color{blue} each physical page frame} of the system
    \item The entry tells us which process is using this page, and which virtual page of that process maps to this physical page
    \item \textbf{\color{blue} Pros: Memory saving}
    \item \textbf{\color{red} Cons: Long searching time, page sharing}
\end{itemize}

\section*{Page Sharing}
\begin{itemize}
    \item For per-process page table, let two pages point to the same frame
    \item For inverted page table, it is diffucult to implement page sharing
\end{itemize}

\section*{Address Translation}
\begin{itemize}
    \item The steps of each memory reference
        \begin{enumerate}
            \item Extract \textbf{\color{blue} virtual page number} from \textbf{\color{blue} virtual address} (\textbf{\color{red} memory access})
            \item Calculate address of \textbf{\color{blue} page table entry}
            \item Fetch \textbf{\color{blue} page table address}
            \item Extract \textbf{\color{blue} physical page frame number}
            \item Determine \textbf{\color{blue} physical address}
            \item Fetch \textbf{\color{blue} physicall address to register} \textbf{\color{red} (memory access)}
        \end{enumerate}
\end{itemize}

\section*{Probelms of Paging}
\begin{itemize}
    \item The issue of the ontroduced paging mechanism
        \begin{itemize}
            \item Page table in main memory
            \item \textbf{\color{red} Fetch the translation} from in-memory page table
            \item Explicit load/store access on a memory address
        \end{itemize}
    \item In this scheme every data/instruction access requires \textbf{\color{red} two} memory accesses
        \begin{itemize}
            \item One for page table
            \item And one for the data/instruction
        \end{itemize}
    \item The performance is affected by \textbf{\color{red} a factor of 2}!
\end{itemize}

\section*{Table Lookaside Buffer (TLB)}
\begin{itemize}
    \item To speed up the paging translatio, hardware helps and builds a cache to store some popular translations
    \item \textbf{\color{blue} Translation lookaside buffer (TLB)}
        \begin{itemize}
            \item Part of \textbf{\color{blue} Memory-management unit} (MMU)
            \item Hardware Cache
        \end{itemize}
    \item Memory reference with TLB
        \begin{itemize}
            \item \textbf{\color{blue} TLB hit: Virtual page number (VPN)} is in TLB and can be quickly accessed
            \item \textbf{\color{red} TLB miss}: VPN is \textbf{\color{red} not} in TLB. Access page table to get the translation, update the TLB entry with the translation
        \end{itemize}
    \item \textbf{\color{blue} Temporal Locality}
        \begin{itemize}
            \item An instruction or data item that has been recently accessed will likely be re-accessed soon in the future
        \end{itemize}
    \item \textbf{\color{blue} Spatial Locality}
        \begin{itemize}
            \item If a program accesses memory at address $\mathbf{x}$, it will likely soon access memory near $\mathbf{x}$
        \end{itemize}
    \item TLB is a \textbf{\color{blue} fully associative} cache
        \begin{itemize}
            \item Any given translation can be anywhere in the TLB
            \item Hardware searches entire TLB in parallel to find the target
        \end{itemize}
    \item A typical TLB entry - \textbf{VPN | PFN | other bits}
        \begin{itemize}
            \item Valid bit
            \item Protection bit
            \item Dirty bit
        \end{itemize}
\end{itemize}

\section*{TLB Issue: Context Switch}
\begin{itemize}
    \item GLB contains translations are only valid for \textbf{\color{blue} current running process}
    \item Switching from one process to another requires OS or hardware to do more work
    \item How to distinguish which process a TLB entry belongs to
    \item Solution 1: \textbf{\color{blue} Flush}
        \begin{itemize}
            \item OS flushes the whole TLB on context switch
            \item Flush operation sets all valid biti to 0
            \item \textbf{\color{red} Problem}: the overhead is too high if OS switches processes too frequently
        \end{itemize}
    \item Solution 2: \textbf{\color{blue} Address space identifier (ASID)}
        \begin{itemize}
            \item Some hardware systems provide an address space identifier (ASID) field in the TLB
            \item Think of ASID as a process identifier (PID)
        \end{itemize}
\end{itemize}

\end{document}
