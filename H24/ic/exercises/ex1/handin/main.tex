\documentclass[a4paper,11pt,norsk]{article}
\usepackage{packages}

\begin{document}

%Headingdel:---------------------------------------------
\topmargin -1.5cm
\makebox[\textwidth][s]{
    \begin{minipage}[c]{0.25\textwidth}
        \includegraphics[width=2.0cm]{Bilder/ntnu_logo.png}  
    \end{minipage}
    \begin{minipage}[c]{0.75\textwidth}
        \huge{\textbf{TTT4260 ESDA}} \\
        \Large{Øving 3  ---  Morten Sørensen, \large{\color{black!75!white}\today}}
    \end{minipage}
}
\vspace{0.75cm}
\normalsize


\section*{Task 1}
We find the built-in junction potential $\Phi_{0}$ as 
\[
    \Phi_0 = \frac{kT}{q} \ln{\left(\frac{N_A \cdot N_D}{n_i^2}\right)}
\]
That gives 
\[
    \Phi_0 = \frac{1.38 \cdot 10^{-23} \cdot 300}{1.602 \cdot 10^{-19}} \ln{\left(\frac{4 \cdot 10^{25} \cdot 2 \cdot 10^{21}}{(1.1 \cdot 10^{16})^2}\right)} \approx \SI{0.88}{\volt}
\]

\section*{Task 2}
\begin{enumerate}
\item Doping the silicon with arsenic would make it an n-type material as arsenic has 1 extra valence electron.
\item As the doped material is an n-type material, the number of negative carriers is practically equal to 
    the doping concentration, i.e., $n_n = N_D = 4 \cdot 10^{23}$ $\text{electrons}/m^3$. The positive carrier 
    concentration $p_n$ will then equal $p_n = n_i^2/N_D$, where $n_i$ is the intrinsic carrier concentration. We
    therefore get $p_n = \frac{(1.1 \cdot 10^{16})^2}{4 \cdot 10^{23}} \approx 0.303 \cdot 10^9$ positive carriers per 
    meter cubed.
\item We know that the number of intrinsic carriers $n_i$ doubles for every $\SI{11}{\celsius}$ increase in temperature.
    We therefore have that the $\SI{33}{\kelvin}$ increase in temperature equates to a $2^3 = 8$ fold increase in 
    intrinsic carriers per meter cubed. We consequently get a positive carrier consentration of 
    $p_n = \frac{(8 \cdot 1.1 \cdot 10^{16})^2}{4 \cdot 10^{23}} \approx 19.36 \cdot 10^9$ holes per cubic meter.
    Since the formula for the number of electrons per cube meter in an n-type material does not depend upon the 
    intrinsic carrier concentration, the electron concentration stays the same.
\item \hphantom{hei}
    \begin{itemize}
        \item Yes, as stated previosly $n_i$ increases with temperature.
        \item Also yes, but is less significant as the doping levels will likely contribute to the conductivity of the material to a much greater degree than the higher mobility of 
            higher energy bound electrons in the silicon.
    \end{itemize}
\end{enumerate}

\section*{Task 3}
\begin{enumerate}
\item The reverse leakage current will be equal to the number of electron-hole pairs generated per second in the depletion region of the diode multiplied by the charge of each electron, giving
\[
    I_{\text{leak}} = q \cdot n = 1.602 \cdot 10^{-19} \cdot 2.7 \cdot 10^{7} \approx \SI{4.325}{\pico\ampere}
\]
\item No?
\item This current will likely increase with temperature, as the number of free carriers in the intrinsic silicon $n_i$ is exponentially proportional with temperature (the number of carriers doubles for every 
    $\SI{11}{\celsius}$ increase in temperature).
\end{enumerate}

\section*{Task 4}
\begin{enumerate}
\item The following circuit description can be used to model the reverse biased diode from task 3 in Aim Spice
\begin{pythoncode}
.include C:\Users\morte\Downloads\DBSBdiode.mod

Vdd v+ 0 DC 4.8V
D1 anode cathode DSUB
.connect cathode v+
.connect anode 0

.plot I(Vdd)
\end{pythoncode}
\item The following is the produced I-V plot when sweeping the voltage from 0 to 4.8V:
    \begin{figure}[H]
        \centering
        \includegraphics[width=0.95\textwidth]{Bilder/leakage_current.png}
    \end{figure}
    It shows that there is a linear relationship between the bias voltage and leakage current for 
    bias voltages over, roughtly, the built in juction voltage of the diode of around 0.89V.
\item Using the DC Operating Point function gave a leakage current of $\SI{-9.857}{\pico\ampere}$ at 4.8V.
\item Because the area of the 50x50um diode is $(50/100)^2 = 1/4$ the area, the following code can be used to model the smaller diode 
\begin{pythoncode}
.include C:\Users\morte\Downloads\DBSBdiode.mod

Vdd v+ 0 DC 4.8V
D1 anode cathode DSUB 0.25
.connect cathode v+
.connect anode 0

.plot I(Vdd)
\end{pythoncode}
    This gives a leakage current of -6.06 pA at 4.8V. The relationship between size and leakage current is not quite linear. A quarter of 
    diode size did not give a quarter of the leakage current.
\item The built-in junction voltage would be the "VJ" parameter, which is set to 0.89V.
\item The following is a plot for the forward bias of the modeled diode. It is simulated from 0 to 1.2V
    \begin{figure}[H]
        \centering
        \includegraphics[width=0.95\textwidth]{Bilder/forward_bias.png}
    \end{figure}
    There does seem to be some correspondance between the found built-in junction voltage and 
    the voltage at which the current seems to start going exponential and increasing way above the leakage current.
\end{enumerate}

\section*{Task 5}
We can observe that the donor concentration $N_D$ is much greater than the acceptor concentration $N_A$, so we can use the 
following approximate formulas to find the depletion region depths:
\[
    x_n \approx \left[\frac{2K_s \epsilon_0 (\Phi_0 + V_R)N_A}{qN_D^2}\right]^{\frac{1}{2}}
\]
and
\[
    x_p \approx \left[\frac{2K_s \epsilon_0 (\Phi_0 + V_R)}{qN_A}\right]^{\frac{1}{2}}
\]

Looking at the values given and the resulting depletion region depths ($\SI{19.4}{\nano\meter}$ and $\SI{9.3}{\centi\meter}$) I assume that the units for the acceptor and donor concentrations are 
mistakenly given in $\text{cm}^{-3}$ instead of $\text{m}^{-3}$. I'm therefore going to interpret the values as concentration per meters cubed.

This gives the following values for the depletion region depths
\begin{center}
    $x_n \approx \underline{\underline{\SI{19.4}{\pico\meter}}}\:\:\:\:$ and $\:\:\:\:x_p \approx \underline{\underline{\SI{93.1}{\micro\meter}}}$
\end{center}

These values make more sense than the previous ones. Though, the difference in doping levels means that the depths are extremely different in scale, which I find somewhat odd.

\section*{Task 6}
\begin{enumerate}
\item  
\item Assuming the temperature of the diode is at room temperature (around $\SI{300}{\kelvin}$) we get the following equations for the 
small-signal resitance and capacitance:
\[
    r_d = \frac{V_T}{I_D} = \frac{kT}{q \cdot I_D} = \frac{1.38 \cdot 10^{-23} \text{ JK}^{-1} \cdot \SI{300}{\kelvin}}{1.602 \cdot 10^{-19} \text{ C } \cdot \SI{3}{\milli\ampere}} \approx \underline{\underline{\SI{8.6}{\ohm}}}
\]
and
\[
    C_d = \frac{\tau_T}{r_d} = \frac{\SI{40}{\pico\s}}{\SI{8.6}{\ohm}} \approx \underline{\underline{\SI{4.64}{\pico\farad}}}
\]
\end{enumerate}

\end{document}
