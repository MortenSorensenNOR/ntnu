\documentclass[a4paper,11pt,norsk]{article}
\usepackage{packages}

\begin{document}

%Headingdel:---------------------------------------------
\topmargin -1.5cm
\makebox[\textwidth][s]{
    \begin{minipage}[c]{0.25\textwidth}
        \includegraphics[width=2.0cm]{Bilder/ntnu_logo.png}  
    \end{minipage}
    \begin{minipage}[c]{0.75\textwidth}
        \huge{\textbf{TTT4260 ESDA}} \\
        \Large{Øving 3  ---  Morten Sørensen, \large{\color{black!75!white}\today}}
    \end{minipage}
}
\vspace{0.75cm}
\normalsize


\section*{Task 1}
\begin{enumerate}
\item Given that $V_{DS} = V_{eff}$ we can say that the region of operation is 
    in the active region.
\item Increasing $V_{DS}$ beyond $V_{eff}$ shortens the effective channel length, so the following 
    model can be used to calculate the incraesed drain source current
    \begin{align*}
        I_{DS} &= \frac{\mu_n C_{ox}}{2} \frac{W}{L} V_{eff}^2 \cdot (1 + \frac{\lambda \cdot L}{L} (V_{DS} - V_{eff})) \\
               &= \frac{\SI{270}{\micro\ampere\per\volt\squared}}{2} \frac{0.5}{0.2} \cdot (\SI{0.2}{\volt})^2 \cdot \left(1 + \frac{\SI{0.08}{\micro\meter\per\volt}}{\SI{0.2}{\micro\meter}} \cdot \SI{50}{\milli\volt}\right) \\
               &= \underline{\underline{\SI{13.77}{\micro\ampere}}}
    \end{align*}
\item Mobility degradation happens when $V_{eff} > \frac{1}{2\theta}$, so for the given transistor, that would 
    occur at 
    \[
        V_{eff} > \frac{1}{2 \cdot \SI{1.7}{\per\volt}} \approx \SI{0.294}{\volt}
    \]
\item The small signal output resistance $r_{ds}$ in the active region is given by the following relationship
    \[
        \frac{1}{r_{ds}} = \lambda \cdot \left(\frac{\mu_n C_{ox}}{2}\right)\left(\frac{W}{L}\right) V_{eff}^2
    \]
    That gives 
    \[
    \frac{1}{r_{ds}} = 5.4 \cdot 10^{-6} \left(\frac{1}{\Omega}\right) \implies r_{ds} = \underline{\underline{\SI{185.2}{\kilo\ohm}}}
    \]
\item If the ration between the width and the length were to be kept the same, but the length to be 3 times as long, the 
    small signal output resistance would be
    \[
        \frac{1}{r_{ds}} = 1.8 \cdot 10^{-6} \left(\frac{1}{\Omega}\right) \implies r_{ds} = \underline{\underline{\SI{555.6}{\kilo\ohm}}}
    \]
\item The small signal transconductance in the active region is given by
    \[
        g_m = \frac{2 I_D}{V_{eff}} = \frac{2 \cdot \SI{13.77}{\micro\ampere}}{\SI{0.2}{\volt}} \approx \underline{\underline{137.7 \cdot 10^{-6}}}
    \]
\item $g_s$ will be roughly twice the value of $g_m$ since the transconductance is given by the following relationship
    \[
        g_s = 2 g_m
    \]
    That gives 
    \[
        g_s = 2 \cdot 137.7 \cdot 10^{-6} = \underline{\underline{275.4 \cdot 10^{-6}}}
    \]
\item A lot of current.
\item Can easily be found in CJM, but the paracitic capacitances include the Gate-Drain, Gate-Source and Source-Bulk capacitances.
\item 
\item 
\item Subthreshold operation may be usefull for analog and digital designers for instance in the case of low power applications, where 
    power efficiency is important, including high gain to power consumption ratio which is useful to analog designers.
\end{enumerate}

\section*{Task 2}
I could not get aimspice or ngspice to cooporate with the model files provided. This may be my fault.

\section*{Task 3}
In order to reduce the the $V_t$ mismatch to 25\% we would have the following:
\[
    \sigma_{t}' = 0.25 \sigma_t
\]

That means that from the model for the variance given that the distance is negligible we have that
\begin{align*}
    \sigma_{t}' &= 0.25 \cdot \sqrt{\sigma_t^2} \\
                &= 0.25 \cdot A_p \cdot \sqrt{\frac{1}{WL}}
\end{align*}

This gives 
\[
    \sqrt{\frac{1}{W' L'}} = 0.25 \cdot \sqrt{\frac{1}{WL}} \implies W'L' = \frac{WL}{0.25^2} \approx \frac{WL}{0.0625}
\]

Which is the same as
\[
    \frac{W'L'}{WL} = \frac{1}{0.0625} = 16
\]

In other words increasing the width length product by 16 would yield a $V_t$ mismatch of 25\% of the original standard deviation.

\section*{Task 4}
The following is what I believe to be a valid transistor schematic for the circuit
the layout illustrated in Figure 2.
\begin{figure}[H]
    \centering
    \includegraphics[width=0.7\textwidth]{Bilder/task4.png}
\end{figure}
The schematic represents a 4-input OR-Gate.

\end{document}
