\documentclass[a4paper,11pt,norsk]{article}
\usepackage{packages}

\begin{document}

%Headingdel:---------------------------------------------
\topmargin -1.5cm
\makebox[\textwidth][s]{
    \begin{minipage}[c]{0.25\textwidth}
        \includegraphics[width=2.0cm]{Bilder/ntnu_logo.png}  
    \end{minipage}
    \begin{minipage}[c]{0.75\textwidth}
        \huge{\textbf{TFE4120  Electromagnetism}} \\
        \Large{Exercise 7  ---  Morten Sørensen, \large{\color{black!75!white}\today}}
    \end{minipage}
}
\vspace{0.75cm}
\normalsize


\section*{Task 1}
The layout in figure 1 resebles a 6T SRAM cell, where data is transmitted through the pink interconnects,
the yellow bar is $V_{dd}$ and the middle part of the bottom green connected to the blue line is $V_{ss}$.
In the middle is the cross-coupled inverters.

\section*{Task 2}

\section*{Task 3}

\section*{Task 4}
\begin{enumerate}
    \item For A. we have that
        \[
            \overline{\overline{(A\bar{B}C)} \cdot \overline{BC(A + \bar{A})}} = A\bar{B}C + BC = C(A\bar{B} + B) = AC + BC
        \]
        which follows the truth table.

        For B. we cannot have that both $\bar{A}\bar{B}\bar{C}$ and $\bar{A}\bar{B}C$ be true at once, so the entire 
        expression evaluates to 0, which does not match the truth table.

        For C. we have 
        \[
            \bar{A}BC + A\bar{B}C + AB\bar{C} + ABC
        \]
        which has all the expression evaluating to 1 in the truth table, pluss a free term $AB\bar{C}$ which does not show up in the truth 
        table. So whether or not it matches depends on the application and whether or not $AB\bar{C}$ is possible to 
        achieve on the inputs.
    \item Yes, you can do it like this. If Q is the bit representing whether the sequence of 111 is present, then
        \[
            Q = M(0, M(0, A, B), C)
        \]
        This may not be the most optimal solution, but M(0, X, Y) acts as an AND gate, so implementing AND3 can be done in this 
        fashion.
\end{enumerate}

\section*{Task 5}

\section*{Task 6}

\section*{Task 7}

\section*{Task 8}
\begin{enumerate}
    \item DRAM memory is a type of memory that stores it's data in capacitors. The basic bitcell in a DRAM technology 
        consists of a capacitor and a gate transistor. The capacitor stores the data, and the gate transistor is used to
        read and write the data. In order to read data the line where the bitcell is connected to is precharged. The gate 
        transistor is then opened. A sense amplifier is connected to this bitline, and if the capacitor had stored a 1, the 
        charge associated with the logical 1 will increase the voltage on the bitline, and the sense amplifier will read out a 1.
        Since the reading of the data destroys the data, the data must be rewritten after reading. 
    \item One problem of DRAM technologies is that it has to be refreshed quite often. This is due to leakage currents in the 
        capacitors. The capacitors are not perfect, and will leak charge over time. This means that the data stored in the 
        capacitors will be lost. We therefore have to periodically read the contents of the DRAM and write it back. This can be 
        time consuming, reduces power efficiency, and may lead to higher response times.
    \item The gate transistor also leaks.
\end{enumerate}

\section*{Task 9}
For dynamic logic, the inputs to the gate must be monotonically rising. The reason for this is because of the nature 
of the gate, being that it precharges the output before evaluating. If the inputs were to change from high to low 
during the evaluation of the function, the output would be affected as the charge stored upon the output would be lost,
therefore producing an incorrect value. This issue is not present when going from LOW to HIGH during the evaluation, as 
the affect of the beginning of the evaluation has already pulled the output low (in an inverter), and therefore going HIGH
does not affect the result.

\section*{Task 10}
\begin{codebox}[verilog]
module d_flip_flop (
    input D,
    input en,
    output Q,
    output Q_bar
);

    // The other stuff
    wire I, R, S;
    not U0 (I, D);
    and U1 (R, I, E);
    and U2 (S, D, E);

    // Cross-coupled NORs
    nor U3 (Q, R, Q_bar);
    nor U4 (Q_bar, S; Q);

endmodule
\end{codebox}

\section*{Task 11}
In order to drive a large capacitive load, it's best to construct a series of inverters, where each inverter
has a transconductance $e$ times bigger than the last. The number of stages $N$ required will then be
\[
    N = \ln{\left(\frac{C_L}{C_1}\right)}
\]
where $C_L$ is the capacitive load we are trying to drive and $C_1$ is the capacitance on the input of the 
inverter chain.

\end{document}
