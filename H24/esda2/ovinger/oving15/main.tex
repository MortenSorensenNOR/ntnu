\documentclass[a4paper,11pt,norsk]{article}
\usepackage{packages}

\newcommand{\nobibparens}[1]{\unskip\cite{#1}}

\begin{document}

%Headingdel:---------------------------------------------
\topmargin -1.5cm
\makebox[\textwidth][s]{
    \begin{minipage}[c]{0.25\textwidth}
        \includegraphics[width=2.0cm]{Bilder/ntnu_logo.png}  
    \end{minipage}
    \begin{minipage}[c]{0.75\textwidth}
        \huge{\textbf{TFE4120  Electromagnetism}} \\
        \Large{Exercise 7  ---  Morten Sørensen, \large{\color{black!75!white}\today}}
    \end{minipage}
}
\vspace{0.75cm}
\normalsize


\section*{Oppgave 1.}
\begin{enumerate}
    \item Vi får følgende analytiske uttrykk for X(f)
        \[
            X(f) = \delta(f - 10) + \delta(f + 10) + \frac{4}{5}\delta\left(f - 20\sqrt(2)\right) + \frac{4}{5}\delta\left(f + 20\sqrt(2)\right) 
        \]
    \item Vi har at
        \begin{align*}
            x(t) &= e^{j \cdot 2\pi \cdot 10 t} + e^{-j \cdot 2\pi \cdot 10 t} + \frac{4}{5}\left(e^{j \cdot 2\pi \cdot 20\sqrt(2) t} + e^{-j \cdot 2\pi \cdot 20\sqrt(2) t}\right) \\
                 &= 2 \cos\left(20 \pi t\right) + \frac{8}{5} \cos\left(40\pi \sqrt{2} t\right)
        \end{align*}

    \item Vi har at $T = \frac{1}{f}$. Vi har da at
        \[
            T_1 = \frac{1}{10}
        \]
        og 
        \[
            T_2 = \frac{1}{20\sqrt{2}}
        \]

        Siden 
        \[
            \frac{T_1}{T_2} = \frac{10}{20\sqrt{2}} = \frac{1}{2\sqrt{2}}
        \]
        ikke er et rasjonalt tall, er signalet ikke periodisk.
\end{enumerate}

\section*{Oppgave 2.}
Vi har at foriertransformen er gitt ved
\[
    \mathscr{F}\{x(t)\} = \int_{-\infty}^{\infty}{x(t) e^{-j\omega t} dt}
\]

Vi har dermed at 
\begin{align*}
    \mathscr{F}\{a x_1(t) + b x_2(t)\} &= \int_{-\infty}^{\infty}{(a x_1(t) + b x_2(t)) e^{-j \omega t} dt} \\
                                       &= a\int_{-\infty}^{\infty}{x_1(t) e^{-j \omega t} dt} + b \int_{-\infty}^{\infty}{x_2(t) e^{-j \omega t} dt} \\
                                       &= a \mathscr{F}\{x_1(t)\} + b \mathscr{F}\{x_2(t)\}
\end{align*}

\section*{Oppgave 3.}
I et kausalt system er $H(t) = 0$ for $t < 0$. Vi har da at
\begin{align*}
    \mathscr{L}\{h(t)\}\Big|_{s = j\omega} &= \int_{-\infty}^{\infty}{h(t) e^{-st} dt}\Big|_{s = j\omega} \\
                                           &= \int_{-\infty}^{\infty}{h(t) e^{-j\omega t} dt} \\
                                           &= \underline{\underline{\mathscr{F}\{h(t)\}}}
\end{align*}

\section*{Oppgave 4.}
\begin{enumerate}
    \item Vi har at 
        \begin{align*}
            h_{LP}(t) &= \mathscr{F}^{-1}\{H(\omega)\} \\
                      &= \frac{1}{2\pi} \int_{-\infty}^{\infty}{H(\omega) e^{j\omega t} d\omega} \\
                      &= \frac{1}{2\pi} \int_{-2\pi B}^{2\pi B}{e^{j\omega t} d\omega} \\
                      &= \frac{1}{2\pi} \left[\frac{e^{j\omega t}}{jt}\right]_{-2\pi B}^{2\pi B} \\
                      &= \frac{1}{2\pi} \left(\frac{e^{j 2\pi B t}}{jt} - \frac{e^{-j 2\pi B t}}{-jt}\right) \\
                      &= \frac{1}{2\pi} \cdot \frac{2j \sin(2\pi B t)}{jt} = \underline{\underline{2B \cdot \text{sinc}(2Bt)}}
        \end{align*}
    \item 
    \item En grunn til at filteret ikke kan realiseres er at den utregnede impulsresponsen er uendelig lang.
\end{enumerate}

\section*{Oppgave 5.}
Dersom vi har at $H_{HP}(\omega) = 1 - H_{LP}(\omega)$, så har vi at
\begin{align*}
    h_{HP}(t) &= \mathscr{F}^{-1}\{H_{HP}(\omega)\} \\
              &= \mathscr{F}^{-1}\{1 - H_{LP}(\omega)\} \\
              &= \mathscr{F}^{-1}\{1\} - \mathscr{F}\{H_{LP}\} \\
              &= \underline{\underline{\delta(t) - h_{LP}(t)}}
\end{align*}

\section*{Oppgave 7.}
\begin{enumerate}
    \item Oscillatoren i figur 4. har en spuriøs-undertrykking på
        \[
            20 \cdot \log\left(\frac{0.5}{1}\right) \approx -6_{\:[\text{dB}]}
        \]
\end{enumerate}

\end{document}
