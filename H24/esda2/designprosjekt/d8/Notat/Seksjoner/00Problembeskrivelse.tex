\section{Problembeskrivelse}
\label{problemBeskrivelse}
Generasjon av \textit{hvit støy} er i mange praktiske sammenhenger et ganske nyttig redskap. Det 
brukes blant annet innen elektronisk musikk, generering av tilfeldige tall og for testing av elektroniske 
system. Dette notatet skal ta for seg hvordan man tilnæremet sett kan generere hvitt støy med et digitalt system,
og ved hjelp av et aktivt filter, båndbegrense det.

Det overordende systemet er illustrert it figur \ref{fig:system_figure}. Systemet består av en 
digital støygenerator som skal genrere et hvitt signal $s(t)$ i det hørbare området fra $\SI{20}{\hertz}$ til 
$\SI{20}{\kilo\hertz}$, og et båndpassfilter som skal båndbegrense støysignalet rundt en frekvens $f_0$, slik at 
utgangssignalet $y(t)$ har en hørbar tonalitet.

Teknologien tatt i bruk i systemet er en Xilinx FPGA på et Digilent Basys3 brett og andre diverse passive og aktive komponenter 
som kondensatorer, opamper og motstander. Notet skal beskrive fremgangsmåten for å utarbeide støygeneratoren og 
båndpassfilteret, gi mål på spektrum av generert støy og båndbegrenset støy og måle responsen, båndbredden og Q-faktoren til filteret.
