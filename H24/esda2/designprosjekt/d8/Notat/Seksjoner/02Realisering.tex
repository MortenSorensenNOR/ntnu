\section{Realisering}
\label{realisering}

\subsection{LFSR}
Realiseringen av systemet er gjort på en Digilent Basys3. Basys3 brettet har en Xilinx Artix A35T FPGA som skal
benyttes for å implementere LFSR systemet. FPGA-en er noe overkill for dette systemet, men utviklingsverktøyene til Xilinx
er gode og det lar oss skrive SystemVerilog kode for å implementere systemet.

For dette notatet er en $M$ verdi på 31 valgt, og i følge Maxfield \cite{maxfield} gir følgende funksjonen 
for $f$ maksimal lengde på utgangssignalet
\[
    f(\{Q_i | i \in {0, \dots, M}\}) = Q_2 \oplus Q_{30}
\]

For å kunne generere klokken som skal utføre shift operasjonen på LFSR-en er det valgt å bruke en teller krets, der 
et register får verdien sin invertert når telleren når en verdi lik FPGA-ens klokke frekvens delt på
ønsket klokkefrekvens delt på 2. Da blir telleren også satt til 0.

Implementeringen av systemet er gjort i SystemVerilog og er relativt kort. 

\begin{verilogcode}
module lfsr #(
    parameter logic [30:0] SEED = 31'h608420dd, // Seed for å starte LFSR-en
    parameter unsigned OVERSAMPLING = 4         // X ganger 20kHz
                                                // for å oppnå Nykvist kriteriet
) (
    input logic clk,
    input logic rst,

    output logic v
);
    // Teller for å generere klokke på 20kHz * OVERSAMPLING
    localparam unsigned NUM_CLKS_PER_SHIFT = 100_000_000 / 
                                            (20_000 * OVERSAMPLING) / 2;
    logic [$clog2(NUM_CLKS_PER_SHIFT)-1:0] clk_cntr = '0;
    logic sr_clk = '0;

    // Generer 80kHz klokken
    always_ff @(posedge clk) begin
        if (rst) begin
            clk_cntr <= '0;
        end else begin
            if (clk_cntr == NUM_CLKS_PER_SHIFT - 1) begin
                clk_cntr <= '0;
                sr_clk <= ~sr_clk;
            end else begin
                clk_cntr <= clk_cntr + 1;
            end
        end
    end

    // Shift registeret
    logic [30:0] SR = SEED;

    // Funksjonen for å finne neste bit
    logic Q_n;
    always_comb begin
        Q_n = SR[2] ^ SR[30];
    end

    // Shift operasjon
    logic sr_clk_r = '0;
    always_ff @(posedge clk) begin
        if (rst) begin
            SR <= SEED;
        end else begin
            sr_clk_r <= sr_clk;

            // Shift kunn på positiv flanke
            if (sr_clk == 1'b1 && sr_clk_r == 1'b0) begin
                SR <= {SR[29:0], Q_n};
            end
        end
    end

    assign v = SR[30];
endmodule    
\end{verilogcode}

I koden over er startverdien til shiftregisteret satt til en konfigurerbar seed. Denne vil bestemme sekvensen som blir generert for $v[n]$. I situasjoner der 
en fixed seed ikke er ønskelig er det også mulig å blant annet sample FPGA-ens on-chip analog til digital omformer dersom den er latt være flytende.

For å kunne synthesize koden over til FPGA-en måtte følgende constraint fil benyttes
\begin{verilogcode}
set_property CONFIG_VOLTAGE 3.3 [current_design]
set_property CFGBVS VCCO [current_design]

# Clock signal
set_property PACKAGE_PIN W5 [get_ports clk]							
    set_property IOSTANDARD LVCMOS33 [get_ports clk]
    create_clock -add -name sys_clk_pin -period 10.00 -waveform {0 5} [get_ports clk]

set_property PACKAGE_PIN T18 [get_ports rst]						
    set_property IOSTANDARD LVCMOS33 [get_ports rst]

set_property PACKAGE_PIN J1 [get_ports {v}]					
    set_property IOSTANDARD LVCMOS33 [get_ports {v}]
\end{verilogcode}

I constaint filen over er utgangssignalet $v$ satt til å være koblet til Basys3 brettets JA PMOD connector sin pin 1.

\subsection{Delyiannis-Friend filter}
For bandpassfilteret har en Q-faktor på 20 blitt valgt. Dette er hovedsakelig for å ta høyde for at filteret trolig ikke vil kunne oppnå denne verdien, da en ønsket verdi for Q faktoren er at 
den oppnår $Q >= 10$ for å kunne få en høybar tone fra det båndbegrensede støysignalet. Senterfrekvens for filteret $f_0$ er spesifisert til $\SI{3320}{\hertz}$. Velger så en kondensatorverdi
$C = \SI{10}{\nano\farad}$ som implementeres ved å parallellkoble 3 $\SI{3.3}{\nano\farad}$ kondensatorer. Til slutt ble det valgt å implementere filteret både med $H_0 = 1$ og 
$H_0 = 10$, da dette kunn har en realistisk endring på valget av $R_1$, som minker med en faktor på 10 når man går fra $H_0 = 1$ til $H_0 = 10$. Valget av kondensatorverdi er primært for å få realistiske verdier 
på motstandene ifht. hva som er tilgjengelig og med hensyn på at opampen ikke er ideell, altså er det ønskelig at motstandsverdien $R_1$ er mellom 1-100 kiloohm.

De valgte verdiene gir da at
\begin{align*}
    R_3 &= \frac{20}{\pi \cdot \SI{3.32}{\kilo\ohm} \cdot \SI{10}{\nano\farad}} \approx \SI{192}{\kilo\ohm} \\
    R_1 &= \frac{\SI{192}{\kilo\ohm}}{2} \approx \SI{95870}{\kilo\ohm} \:\:\: \text{ eller } \:\:\: \SI{9.587}{\kilo\ohm} \\
    R_2 &= \frac{\SI{192}{\kilo\ohm}}{4 \cdot 20^2 - 2} \approx \SI{120}{\ohm} \:\:\: \text{ også for } H_0 = 10
\end{align*}

Filteret ble implementert med den tidligere nevnte LF353 med en forsyningsspenning på $\pm\SI{5}{\volt}$. Dette er trolig noe 
lavt da utgangssignalet fra FPGA-en er på 3.3V, og vil derfor gi begrenset forsterkning ved $H_0 = 10$.

\subsection{Implementert system på breadboard}
Systemet ble realisert på oppkoblingsbrett og på den valgte Basys3 FPGA kortet.
\begin{figure}[H]
    \centering
    \includegraphics[width=0.5\textwidth]{Bilder/system_photo.png}
    \caption{Bilde av systemet fysisk implementert}
\end{figure}
