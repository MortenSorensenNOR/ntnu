\section{Problembeskrivelse}
\label{problemBeskrivelse}

Operasjonsforsterkere er en sentral komponent innen analog og digital elektronikk,
blant annet i forbindelse med signalbehandling og regulering. Et symbol og en overordnet 
modell av en operasjonsforsterker er vist i figur \ref{fig:opamp}.

\begin{figure}[H]
    \centering
    \includegraphics[width=0.5\textwidth]{Bilder/opamp_figur.png}
    \caption{Operasjonsforsterker: (a) Symbol og (b) modell \cite{problem_stilling}}
    \label{fig:opamp}
\end{figure}

En \textit{ideell} operasjonsforsterker har følgende egenskaper:
\begin{enumerate}
    \item inngangsmotstand $R_{i} = \infty$
    \item utgangsmotstand $R_{o} = 0$
    \item og en utgangsspenning $v_{o}$ på formen
        \begin{equation}
            v_{o} = f(v^{+}, v^{-}) = 
            \begin{cases}
                \min{\left\{V, A(v^{+} - v^{-})\right\}} & \text{for } v^{+} - v^{-} > 0 \\
                \max{\left\{-V, A(v^{+} - v^{-})\right\}} & \text{for } v^{+} - v^{-} < 0 \\
            \end{cases}
        \end{equation}
\end{enumerate}

Konstanten $A$ er operasjonsforsterkerens forsterkning under åpen løkke-forsterkning.
I praksis vil en operasjonsforsterker gjerne ha en endelig inngangsmotstand, en endelig utgangsmotstand og 
vil ha en begrenset forsterkning.

Dette designnotatet skal ta for seg et design og realisering av en operasjonsforsterker 
med differensiell inngang. Operasjonsforsterkeren skal operere på $\pm \SI{5}{\volt}$,
og skal evalueres ved 
\begin{itemize}
    \item Forsterkning $A$ ved sinuspåtrykk med frekves $\SI{1}{\kilo\hertz}$
    \item Total harmonisk distorsjon ved sinuspåtrykk med frekvens $\SI{1}{\kilo\hertz}$
\end{itemize}
under forskjellige lastmotstander: $R_L = \SI{100}{\kilo\ohm}$ og $R_L = \SI{100}{\ohm}$.

Designet skal så testes både som en inverterende forsterker med forsterkning $A = 10$,
og under åpen løkke-forsterkning.
