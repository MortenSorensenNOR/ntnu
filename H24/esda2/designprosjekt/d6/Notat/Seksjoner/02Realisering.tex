\section{Realisering}
\label{realisering}

For realiseringen av kretsen har vi valgt å benytte følgende transistorer:
\begin{figure}[H]
    \centering
    \begin{tabularx}{0.5\textwidth}{|X|X|}
        \hline
        \textbf{Transistor} & \textbf{Type}\\
        \hline
        Q1 & BC547 \\
        Q2 & BC547 \\
        Q3 & BC557 \\
        Q4 & BC557 \\
        Q5 & BC547 \\
        Q6 & BC547 \\
        Q7 & BC547 \\
        \hline
    \end{tabularx}
\end{figure}

For bias-strøm motstanden $R$ ble en verdi $R = \SI{67}{\kilo\ohm}$ valgt. Det gir en bias-strøm
på 
\[
    I_{bias} = \frac{\SI{5}{\volt} - \SI{-5}{\volt}}{\SI{67}{\kilo\ohm}} = \SI{149.25}{\micro\ampere}
\]

Det gir en forsterkning $A$ på (dersom det er ingen signal på inngangen)
\[
    A = g_m \cdot r_o = \frac{I_{bias}}{V_T} \cdot \frac{1}{h_{oe}} \approx 96
\]
der $h_{oe}$ er gitt fra databladet til transistoren \cite{bc557}

For bufferen ble kondensatorverdien $C = \SI{1}{\micro\farad}$ valgt for begge kondensatorene, og 
arbeidspunktet til transistoren valgt til $V_{B} = \SI{0}{\volt}$ og følgelig $V_{E} = \SI{-0.7}{\volt}$.
Dette ble ugunstig nok realisert ved å sette $R_{B_1} = \SI{67}{\kilo\ohm}$, $R_{B_2} = \SI{10}{\mega\ohm}$
og $R_{E} = \SI{1}{\kilo\ohm}$. Dette er noe ugunstig grunnet at det gir en uønskelig lav 
inngangsmotstand og en høyere enn ønsket utgangsmotstand. En bedre bufferkrets vil gi bedre resultater og 
vil føre til at operasjonsforsterkeren oppfører seg mer ideelt.

For testing av inverterende opamp med forsterkning $A = 10$ ble $R_{in} = \SI{1}{\kilo\ohm}$ og $R_{f} = \SI{10}{\kilo\ohm}$ valgt.
