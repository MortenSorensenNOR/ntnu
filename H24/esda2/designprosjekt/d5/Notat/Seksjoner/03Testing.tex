\section{Testing}

Testing av systemet ble utført ved hjelp av en Analog Discovery 2 \cite{analog_discovery} oscilloskop og 
signalgenerator. Kilden $v_0$ ble generert som et sinussignal fra ett av oscilloskopets signalgeneratorer 
og ble satt til en frekvens $f = \SI{1.0}{\kilo\hertz}$ og en amplitude på $A_0 = \SI{500}{\milli\volt}$.

\subsection{Test av ideelt system}
Systemet ble først testet med ideelle verdier for kilde- og lastmotstanden, altså
$R_K = 0$ og $R_L = \infty$. I figur \ref{fig:buffer_no_load_plot} er kildespenningen 
$v_0$ plottet sammen med utgangsspenningen til bufferen $v_2$.

\begin{figure}[H]
    \centering 
    \includegraphics[width=0.7\textwidth]{Bilder/buffer_no_load.png}
    \caption{Buffer test med ideell lastmotstand $R_L = \infty$ og kildemotstand $R_K = 0$}
    \label{fig:buffer_no_load_plot}
\end{figure}

Plottet viser tydelig at $v_0 \approx v_2$. De målte VRMS verdiene til de to signalene for denne testen var 
\begin{center}
    $v_0 = \SI{354.7}{\milli\volt}\:\:\:\:$ og $\:\:\:\:v_2 = \SI{349.2}{\milli\volt}$ 
\end{center}

Det gir en dempning på
\[
    A_\text{[dB]} = 20 \cdot \log{\left(\frac{v_2}{v_0}\right)} \approx -0.135 \text{ dB}
\]

Altså fungerer bufferen svært godt under ideelle forhold.

\subsection{Test med realistiske last- og kildemotstander}
Videre er systemet testet med lastmotstand (figur \ref{fig:buffer_load_no_res_plot}) og med last- 
og kildemotstand (figur \ref{fig:buffer_load_source_res_plot}).

\begin{figure}[H]
    \centering 
    \includegraphics[width=0.625\textwidth]{Bilder/buffer_load.png}
    \caption{System test med lastmotstand $R_L = \SI{220}{\ohm}$ og ideell kildemotstand $R_K = \SI{0}{\ohm}$}
    \label{fig:buffer_load_no_res_plot}
\end{figure}

Vi ser nå at vi får et større avvik mellom $v_2$ og den ønskede spenningen $v_0$. 

For systemtesten med kunn lastmotstand $R_L = \SI{220}{\ohm}$ oppnådde systemet følgende VRMS verdier:
\begin{center}
    $v_0 = \SI{353.9}{\milli\volt}\:\:\:\:$ og $\:\:\:\:v_2 = \SI{334.8}{\milli\volt}$ 
\end{center}

Bufferen oppnår da en demping på 
\[
    A_\text{[dB]} = 20 \cdot \log{\left(\frac{v_2}{v_0}\right)} \approx -0.482 \text{ dB}
\]

\begin{figure}[H]
    \centering 
    \includegraphics[width=0.7\textwidth]{Bilder/buffer_load_source.png}
    \caption{System test med lastmotstand $R_L = \SI{220}{\ohm}$ og kildemotstand $R_K = \SI{1.2}{\kilo\ohm}$}
    \label{fig:buffer_load_source_res_plot}
\end{figure}

For systemtesten med både kildemotstanden $R_K = \SI{1.2}{\kilo\ohm}$ og
lastmotstanden $R_L = \SI{220}{\ohm}$

\begin{center}
    $v_0 = \SI{354.0}{\milli\volt}\:\:\:\:$ og $\:\:\:\:v_2 = \SI{316.8}{\milli\volt}$ 
\end{center}

Det gir en demping på
\[
    A_\text{[dB]} = 20 \cdot \log{\left(\frac{v_2}{v_0}\right)} \approx -0.964 \text{ dB}
\]

Vi ser nå at vi taper en del signalstyrke når vi introduserer last- og kildemotstanden. Bedre 
bufferkarakteristikk kan trolig oppnås dersom inngangsmotstanden hadde økt og utgangsmotstanden hadde 
minsket. Dette er trolig mulig, men kan nå begrensninger grunnet karakteristikken til transistoren.
Det er derfor trolig bedre å heller introdusere flere stadier, og dermed også flere BJT-transistorer, noe som 
hadde gitt en mer avansert kretstopologi, slik som en Darlington Emitter Follower \cite{darlington}.

\subsection{Test av maksimal kildeamplitude $A_0$}
Maksimal kildeamplitude $A_0$ ble også testet. Amplituden til sinussignalet generert ble gradvis økt frem til 
forvrengninger i utgangssignalet ble tydelige. For den realiserte bufferkretsen gav amplituder omlag $\SI{1.6}{\volt}$ 
tydlige forvrengninger, som vist i figur \ref{fig:buffer_disturb_850mv_plot}.

\begin{figure}[H]
    \centering 
    \includegraphics[width=0.7\textwidth]{Bilder/buffer_distort_1V6.png}
    \caption{Utgangsspenning $v_2(t)$ ved inngangsspenning $v_0(t)$ lik $\SI{1.6}{\volt}$}
    \label{fig:buffer_disturb_850mv_plot}
\end{figure}

\subsection{Test av frekvensrespons}
Bufferens frekvensrespons ble også testet. Network Analyser funksjonen i 
Analog Discoveryens \textit{Waveforms} applikasjon ble benyttet for spektrumsanalysen. Frekvenser mellom
$\SI{10}{\hertz}$ og $\SI{4}{\mega\hertz}$ ble testet. Frekvensresponsen til systemet er vist i figur \ref{fig:buffer_frekvens_respons}.
\begin{figure}[H]
    \centering 
    \includegraphics[width=0.7\textwidth]{Bilder/buffer_frekvens_respons.png}
    \caption{Frekvensresponsen til buffer systemet}
    \label{fig:buffer_frekvens_respons}
\end{figure}

Systemet oppnådde en -3dB knekkfrekvens $f_0 = \SI{82.2}{\hertz}$. Dette er langt under den forventede 
frekvensen til kildesignalet $f = \SI{1000}{\hertz}$.

\subsection{Inn- og utgangsmotstand til bufferkretsen}
Til slutt ble inn- og utgangsmotstanden til bufferkretsen målt. Oppsettet for måling av inn og utgangsmotstand 
er vist i figur \ref{fig:in_out_res_measure_setup}.

\begin{figure}[H]
    \centering
    % \includegraphics[width=0.6\textwidth]{Bilder/in_out_res_setup.png}
    \caption{Oppsett for måling av inn- og utgangsmotstand til bufferen}
    \label{fig:in_out_res_measure_setup}
\end{figure}

For å finne inngangsmotstanden ser vi i figur \ref{fig:in_out_res_measure_setup} at vi ved spenningsdeling
får at 
\[
    v_1 = v_0 \cdot \frac{R_{inn}}{R_{inn} + R_K}
\]

Løser vi for $R_{inn}$ får vi at 
\[
    R_{inn} = R_K \cdot \frac{v_1}{v_0 - v_1}
\]

Ved måling har vi at
\begin{center}
    $R_K = \SI{1.208}{\kilo\ohm}\:\:\:\:$ $\:\:\:\:v_0 = \SI{354.0}{\milli\volt}\:\:\:\:$ $\:\:\:\:v_1 = \SI{334.6}{\milli\volt}$
\end{center}

Det gir at inngangsmotstanden blir
\[
    R_{inn} = \SI{1.2}{\kilo\ohm} \cdot \frac{\SI{334.6}{\milli\volt}}{\SI{354.0}{\milli\volt} - \SI{334.6}{\milli\volt}} \approx \SI{20.9}{\kilo\ohm}
\]

Følgelig kan vi finne utgangsmotstanden også fra figur \ref{fig:in_out_res_measure_setup} ved spenningsdeling 
da 
\[
    v_2 = v_{ut} \cdot \frac{R_L}{R_{ut} + R_L}
\]

Løser for $R_{ut}$ og får at 
\[
    R_{ut} = R_L \cdot \left(\frac{v_{ut}}{v_2} - 1\right)
\]

De målte verdiene fra tidligere test for $v_2$ med og uten last gav at 
\begin{center}
    $v_{ut} = \SI{349.2}{\milli\volt}\:\:\:\:$ $\:\:\:\:v_{2} = \SI{334.8}{\milli\volt}\:\:\:\:$ $\:\:\:\:R_L = \SI{220}{\ohm}$
\end{center}

Det gir at utgangsmotstanden er lik 
\[
    R_{ut} = 220 \cdot \left(\frac{349.2}{334.8} - 1\right) \approx \SI{9.5}{\ohm}
\]

Vi ser altså at implementeringen har oppnådd en svært god utgangsmotstand, men at inngangsmotstanden 
ikke er helt så ideell som vi skulle ønske. Det skyldes trolig endringene som ble gjort på $R_{B1}$ 
for å kunne oppnå det ønskede arbeidspunktet for basespenningen. Som nevnt tidligere 
ville dette trolig vært et mindre problem dersom transistorene hadde blitt kaskadekoblet eller dersom 
kretstopologien hadde vært mer avansert, slik som for Darlington emitter følgeren.
