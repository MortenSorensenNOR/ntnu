\section{Prinsipiell løsning}
\label{prinsipiellLoesning}

For å oppnå at designets mål at $v_2 \approx v_1 \approx v_0$ må bufferen designes på en sånn måte
at påvirkningen fra kildemotstanden $R_K$ og lastmotstanden $R_L$ blir minst mulig. Vi ønsker altså at mest 
mulig av spenningen $v_0$ skal ligge over inngangen på bufferen, og at minst mulig av det genererte 
spenningsnivået internt i bufferen skal ligge over bufferens utgangsmotstand. Dette er mulig dersom 
vi gjør inngangsmotstanden til bufferen stor, og utgangsmotstanden til bufferen liten.

For å oppnå at $v_2 \approx v_1$ kan en bufferkrets basert op en BJT Emitter follower krets benyttes. 
Kretsens oppbygning er vist i figur \ref{fig:bjt_voltage_follower}.

\begin{figure}[H]
    \centering 
    \includegraphics[width=0.7\textwidth]{Bilder/bjt_follower.png}
    \caption{BJT Emitter Follower}
    \label{fig:bjt_voltage_follower}
\end{figure}

Fra småsignalskjemaet i figur \ref{fig:small_signal} kan vi utlede et uttrykk for bufferens forsterkning/dempning.

\begin{figure}[H]
    \centering
    \includegraphics[width=\textwidth]{Bilder/small_signal_model_buffer.png}
    \caption{Et småsignalskjema for bufferkretsen \cite{hambley}}
    \label{fig:small_signal}
\end{figure}

Vi ser fra småsignalskjemaet at 
\[
    i_e = (1 + \beta) \cdot i_b
\]

Vi kan så skrive om motstandene $R_{B1}$ og $R_{B2}$ til en felles motstand 
\[
    R_B = R_{B1 || B2}
\]

og emitter- og lastmotstanden til en felles motstand
\[
    R'_L = R_{E || L}
\]

Vi kan så finne et uttrykk for inn- og utgangsspenningen til bufferen.
\[
    v_{inn} = r_{\pi} i_b + (1 + \beta)i_b R'_L
\]
og
\[
    v_{ut} = R'_L (1 + \beta)i_b
\]

Det gir følgende forsterkning til bufferen
\[
    A_v = \frac{(1 + \beta)R'_L}{r_{\pi} + (1 + \beta)R'_L} \approx 1
\]

Altså er spenningsforsterkningen til bufferen omtrentlig 1, men noe under grunnet 
base-motstanden $r_{\pi}$. Siden spenningsforstekningen også er positiv, vil bufferen fungere 
som en ikke-inverterende forsterker. Vi kan derfor benytte dette designet som en buffer for systemet 
vi ønsker å implementere.

For at bufferen skal kunne opperere som den skal må et arbeidspunkt bestemmes.
Arbeidspunktet til kretsen er gitt ved verdiene $V_B$, $V_E$ og $I_E$. Vi ønsker å velge et arbeidspunkt 
$V_B$ s.a. bufferen git størst mulighet for inngangssignalet $v_1$ å varriere uten å oppleve at bufferen 
går i mettning. Vi kan derfor velge at $V_B = \frac{V_{CC}}{2}$. Ettersom at en BJT transistor er brukt i kretsen 
vil det være naturlig å anta at emitterspenningen $V_E$ vil ligge omlag $\SI{0.7}{\volt}$ under basespenningen.
Vi har da at $V_E = V_B - \SI{0.7}{\volt}$. 

Til slutt kan en emitterstrøm $I_E$ velges. Emmitterstrømmen må velges utifra databladet til transistoren 
og ønsket emittermotstand. Siden emittermotstanden har stor påvirkining på utgangsmotstanden til 
bufferen $R_{ut}$ er det naturlig å velge en så høy strøm som mulig, da dette vil gi en lavere 
verdi for $R_E$ og dermed også lav utgangsmotstand. Verdiene til motstandene $R_{B1}$ og $R_{B2}$ må velges 
slik at arbeidspunktet til $V_B$ blir oppnådd. Verdiene på de to motstandene vil basere seg blant annet 
på base-motstanden $R_{\pi}$ til transistoren og emittermotstanden $R_E$. Et godt startpunkt for valg 
av motstandsverdier er å sette $R_{B1} = R_{B2}$. Dette vil trolig gi et arbeidspunkt godt under den ønskede 
verdien. Det er da mulig å øke $R_{B2}$ helt til $V_B = V_{CC}/2$ blir oppnådd, evt. også øke 
utgangsmotstanden $R_E$, men dette vil også øke utgangsmotstanden, så små endringer på $R_E$ er ønskelig om
mulig. Størrelsen på $R_{B1}$ og $R_{B2}$ må velges til å være store verdier slik at påvirkningen fra 
kildemotstanden $R_K$ blir liten. Verdier omlag $\SI{100}{\kilo\ohm}$ til $\SI{1}{\mega\ohm}$ er ønskelig.

\newpage
Til slutt kan kondensatorverdiene $C_1$ og $C_2$ velges. Ettersom at kondensatoren i lag med 
inngangsmotstanden $R_{inn}$ danner et høypassfilter, er det naturlig å velge en kondensatorverdi som 
er slik at knekkfrekvensen $f_0 << f$, der $f$ er den forventede frekvensen til inngangssignalet $v_0$.
Større kodensatorverdier har liten negativ påvirkning annet enn at de tar lenger tid å lade opp, og derfor 
forlenger tiden før kretsen oppnår de ønskede egenskapene fra kretsen først blir aktivert.
