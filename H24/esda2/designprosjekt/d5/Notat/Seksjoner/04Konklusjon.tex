\newpage
\section{Konklusjon}
\label{konklusjon}

Dette designnotatet har tatt for seg et design av en buffer fra diskrete komponenter for å kunne 
øke strømleveransen fra et kildesignal til en last. Designet baserte seg på en BJT emitter følger,
og ble realisert ved bruk av en BC547B BJT transistor. 

Deisgnet ble så testet der det ble funnet at at bufferen oppnådde en maksimal dempning på 
utgangssignalet ifht. inngangssignalet på omlag -1 dB for realistiske verdier for kildemotstand 
$R_K = \SI{1.2}{\kilo\ohm}$ og lastmotstand $R_L = \SI{220}{\ohm}$.

Maksimal amplitude for inngangssignalet ble også testet, og det ble funnet at bufferen oppererte ideelt frem til
en inngangsamplitude $A_0$ på omlag $\SI{1.6}{\volt}$, da tydlige forvrengninger i utgangssignalet ble tydelig.
Bufferens frekvensrespons ble så testet, da det ble funnet at bufferen oppfører seg som et høypassfilter med en 
knekkfrekvens på omlag 82 Hz. Til slutt ble inn- og utgangsmotstanden til bufferen testet, da det ble funnet 
at den realiserte kretsen oppnådde en inngangsmotstand på $\SI{20.9}{\kilo\ohm}$ og en utgangsmotstand på 
$\SI{9.5}{\ohm}$.

Mulige forbedringer av kretsen ble diskutert, blant annet en mer avansert kretstopologi kjent som en
Darlington Emitter Follower, som trolig vil gi bedre bufferkarakteristikk enn designet realisert i dette notatet.
Det er altså tydelig at dette designet av en buffer har fungert, men at den er langt dårligere enn 
de ferdigstilte buffer IC-ene som er mulig å ta i bruk.
