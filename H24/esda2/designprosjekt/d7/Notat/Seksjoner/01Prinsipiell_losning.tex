\section{Prinsipiell løsning}
\label{prinsipiellLoesning}

Den prinsipielle løsningen baserer seg rundt et Butterworth lavpass filter med en n-te ordens Sallen-Key struktur (\ref{sallen-key}). 
Grunnen til at Butterworth filtertypen er valgt for anti-aliasing applikasjonen er at filteret kan oppnå maksimal flathet, og derfor 
påvirke signalene vi ønsker å sample under $f_s/2$ minst mulig, som gitt i kravene. En annen mulig filtertype er en type 2 Tsjebysjef filter, da denne typen 
filter ikke har ripples i passbandet slik type 1 Tsjebysjef filtre har. Dette notatet velger likevel Butterworth da det er simpelt, enkelt å regne på, 
og funker bra for denne applikasjonen.

\subsection{Utregning av nødvendig orden}
Før filteret kan designes må filteres orden bestemmes. Vi har gitt at amplituderesponsen til et Butterworth filter 
er gitt ved \cite{lundheim-butterworth-amplitude-respons}
\begin{equation}
    A = |H(j\omega)| = |H(j \cdot 2\pi f)| = \frac{1}{\sqrt{1 + \left(\frac{f}{f_c}\right)^{2n}}}
\end{equation}
hvor $f_c$ er knekkfrekvensen definert ved $f_c = 0.75 \cdot f_s / 2$ som gitt fra problembeskrivelsen.

Løser vi for $n$ får vi at
\begin{equation}
    n = \left\lceil\frac{1}{2}\frac{\ln{\left(A^{-2} - 1\right)}}{\ln{\left(\frac{f}{f_c}\right)}}\right\rceil
\end{equation}

Vi kan så produsere et filter av orden $n$ ved å kaskade koble første og andre ordens Sallen-Key lavpassfiltre, slik at 
summen av alle filterenes orden blir lik $n$.

\subsection{Sallen-Key Struktur}
\label{sallen-key}
Overordnet kan Sallen-Key strukturen for et lavpassfilter av andre orden slik:
\begin{figure}[H]
    \centering
    \includegraphics[width=0.55\textwidth]{Bilder/sallen_key_lowpass.png}
    \caption{Andre ordens lavpass Sallen-Key struktur \cite{sallen-key-lowpass-wikipedia}}
\end{figure}

Vi har så at systemfunksjonen til Sallen-Key strukturen over er gitt ved
\begin{equation}
    H(s) = \frac{1}{s^2 + 2\zeta\omega_0 s + \omega_0^2}
\end{equation}
der $\zeta$ er den relative dempningsfaktoren til filteret, og $\omega_0 = 2\pi f_0 = 2\pi f_c$.

Den ønskede dempningsfaktoren $\zeta$ for hver Sallen-Key strukturen i kaskadekoblingen til det ferdiget 
lavpassfilteret er gitt ved følgende formel
\begin{equation}
    \zeta_i = \begin{cases}
        \cos\left(\frac{\pi}{n}i\right), & \text{ for $n$ odde,} \\
        \cos\left(\frac{\pi}{2n} + (i - 1)\frac{\pi}{n}\right), & \text{ for $n$ jevne} \\
    \end{cases}
    \label{eq:zeta}
\end{equation}
der $\zeta_i$ er dempningsfaktoren til $i$-te Sallen-Key strukturen i kretsen.

Vi kan så gå frem for å regne ut komponentverdiene $R_1$, $R_2$, $C_1$ og $C_2$ som oppnår den ønskede frekvensresponsen.
En metode er ved å velge en motstandsverdi $R$ og la $R_1 = R_2 = R$. Vi har da at de to tidskonstantene blir
\begin{center}
    $\tau_1 = \frac{1}{\omega_0 \zeta_i} = R C_1\:\:\:\:\:\:$ og $\:\:\:\:\:\: \tau_2 = \frac{1}{\omega_0^2\tau_1} = R C_2$
\end{center}

Og vi kan dermed løse for $C_1$ og $C_2$ fra de kjente verdiene
\begin{center}
    $C_1 = \frac{\tau_1}{R}\:\:\:\:\:\:$ og $\:\:\:\:\:\: C_2 = \frac{\tau_2}{R}$
\end{center}
Vi kan på denne måten gå frem for å finne komponentverdiene til alle Sallen-Key strukturene vi trenger til filteret.
Denne metoden garanterer oss ved å benytte formel (4) et maksimalt flatt filter, som er det vi ønsker for å påvirke 
frekvenskomponentene under $f_s/2$ minst mulig.

Ettersom at denne metoden baserer seg på å fastsette en felles motstandsverdi $R$ og fra den regne ut kondensatorverdier,
kan kondensatorverdiene blir litt hårete. Det finnes derfor en annen metode \cite{sallen-key-lowpass-wikipedia} som går ut på 
å heller velge seg to kondensatorverdier $C_1$ og $C_2$ med et gitt forhold $n$ og $1/n$ og heller fra formelen

\[
    Q = \frac{mn}{m^2 + 1}
\]
der $Q = \frac{1}{\zeta}$ er kvalitetsfaktoren regne ut en verdi $m$ under kravet om at vi ønsker maksimal flathet.
Kravet om at vi ønsker minst mulig påvirkning av frekvenser under $f_s/2$ gir at filteret må oppnå maksimal flathet,
og dette får vi ved $Q = \frac{1}{\sqrt{2}}$. Komponentverdiene blir da valgt som

\begin{align*}
    &R_1 = mR \\
    &R_2 = R/m \\
    &C_1 = nC   \\
    &C_2 = C/n
\end{align*}

Dette kan være mer fleksibelt da kondensator verdier kan være vanskeligere å finne rett verdi for. Det finnes også digitale 
verktøy som benytter denne metoden, slik som denne: \url{https://tools.analog.com/en/filterwizard/}
\begin{figure}[H]
    \centering
    \includegraphics[width=0.75\textwidth]{Bilder/analog_filter_wizzard.png}
    \caption{Valg av RC sizing for Sallen-Key struktur komponenter}
    \label{analog-wizzard}
\end{figure}
(\textit{Verdiene vist i kretsen over er urelatert til dette anti-alianseringsfilteret})\\
Denne tillater brukeren å velge forholdet mellom motstandene og kondensatorene i strukturen, som vist ovenfor. Det er gjennom 
dette mulig å velge et forhold som gir kondensator verdier man har for hånd.

