\documentclass[a4paper,11pt,norsk]{article}
\usepackage{packages}

\begin{document}

%Headingdel:---------------------------------------------
\topmargin -1.5cm
\makebox[\textwidth][s]{
    \begin{minipage}[c]{0.25\textwidth}
        \includegraphics[width=2.0cm]{Bilder/ntnu_logo.png}  
    \end{minipage}
    \begin{minipage}[c]{0.75\textwidth}
        \huge{\textbf{TTT4260 ESDA}} \\
        \Large{Øving 3  ---  Morten Sørensen, \large{\color{black!75!white}\today}}
    \end{minipage}
}
\vspace{0.75cm}
\normalsize


\section*{Task 1}
\begin{enumerate}
    \item In each simulation cycle the right-hand side of all expressions 
        (unless they are in a process which does not include it in its sensitivity list) are 
        evaluated. For each expression, if it results in a new output value, the new value is 
        updated after one delta-delay, and if the expression has a delay, the new value assigned will be placed in the 
        event queue. The event queue is then processed in the next delta-cycle, and (if the expression had a delay, and the 
        delay has passed) the new value is assigned to the signal.
    \item If a signal is not assigned explicit delays, the value will be updated after a delta-delay, typically a femtosecond.
        If it is assigned an explicit delay, the new value is placed in the event queue, and assigned to the signal when the event 
        queue has reached the point where that delay has passed.
\end{enumerate}

\section*{Task 2}
\begin{enumerate}
\item The following is a timing diagram for A, B, Q and QN:
\begin{table}[H]
    \centering
    \begin{tabular}{|l|c|c|c|c|}
        \hline
        Time & \textbf{A} & \textbf{B} & \textbf{Q} & \textbf{QN} \\
        \hline
        \textbf{0}          & U     & U     & U     & U     \\
        0 + $\Delta$        & 1     & 0     &       &       \\
        0 + $2\Delta$       &       &       & 0     & X     \\
        0 + $3\Delta$       &       &       & 0     & 1     \\
        0 + $4\Delta$       &       &       &       &       \\
        \textbf{10}         &       &       &       &       \\
        10 + $\Delta$       & 0     & 0     &       &       \\
        10 + $2\Delta$      &       &       & 0     & 1     \\
        \textbf{20}         &       &       &       &       \\
        20 + $\Delta$       & 0     & 1     &       &       \\
        20 + $2\Delta$      &       &       & 0     & 0     \\
        20 + $3\Delta$      &       &       & 1     & (0)   \\
        \textbf{30}         &       &       &       &       \\
        30 + $\Delta$       & 0     & 0     &       &       \\
        30 + $2\Delta$      &       &       & 1     & 0     \\
        \textbf{40}         &       &       &       &       \\
        40 + $\Delta$       & 1     & 1     &       &       \\
        40 + $2\Delta$      &       &       & 0     & 0     \\
        \hline
    \end{tabular}
\end{table}

\item A process will continue until all changes relating to the signals in the sensitivity list are completed. No osillations occured after the last stimuli, 
    so the process will stop after the second delta delay after the last stimuli.

\item The following is an implementation of the process \textbf{Q} and \textbf{QN} with an entity and architecture.
\begin{vhdlcode}
library ieee;
use ieee.std_logic_1164.all;

entity TestCircuit is 
    port(
        A, B : in std_ulogic;
        Q, QN : out std_ulogic
    );
end entity TestCircuit;

architecture Behavioral of TestCircuit is
    signal q_internal, qn_internal : std_ulogic;
begin
    q_internal <= A nor qn_internal;
    Q <= A nor qn_internal;
    qn_internal <= B nor q_internal;
    QN <= B nor q_internal;
end architecture Behavioral;
\end{vhdlcode}
In the above code two new signals \textbf{q\_internal} and \textbf{qn\_internal} had to be created 
because one gets an error if one tries to read the value from a declared output port. Another 
solution would be to change the \textbf{Q} and \textbf{QN} ports to inout ports, but I prefer to 
do it this way.

\begin{vhdlcode}
library ieee;
use ieee.std_logic_1164.all;

entity TestCircuit_tb is
end entity TestCircuit_tb;

architecture Behavioral of TestCircuit_tb is
    signal A, B, Q, QN : std_ulogic;
begin
    DUT : entity work.TestCircuit port map(A, B, Q, QN);
    process begin
        A <= '1'; B <= '0';
        wait for 10 ns;
        A <= '0';
        wait for 10 ns;
        B <= '1';
        wait for 10 ns;
        B <= '0';
        wait for 10 ns;
        B <= '1'; A <= '1';
        wait;
    end process;
end architecture Behavioral;
\end{vhdlcode}
\end{enumerate}

\end{document}
