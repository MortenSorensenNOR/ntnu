\documentclass{article}

% Packages
\usepackage{braket}
\usepackage{amsmath}
\usepackage{amssymb}
\usepackage{amsfonts}
\usepackage{graphicx}
\usepackage{mathtools}
\usepackage[shortlabels]{enumitem}

% Margins
\topmargin=-2.5cm
\evensidemargin=0in
\oddsidemargin=0in
\textwidth=6.5in
\textheight=9.0in
\headsep=0.25in

% Commands
\makeatletter
\newcommand{\listintertext}{\@ifstar\listintertext@\listintertext@@}
\newcommand{\listintertext@}[1]{% \listintertext*{#1}
\hspace*{-\@totalleftmargin}#1}
\newcommand{\listintertext@@}[1]{% \listintertext{#1}
\hspace{-\leftmargin}#1}
\makeatother

\graphicspath{ {./images/} }

\title{\huge{Øving 10\\ Diskret matematikk}}
\author{Morten Sørensen}
\date{\today}

\begin{document}
\maketitle

\subsection*{Oppgave 1.}
\begin{enumerate}[(a)]
    \item {
        \textbf{1.} Ja, siden grafen ikke har noen sykler. \\
        \textbf{2.} Grafen har en eulervei fordi den har nøyaktig to noder av odde grad, har derfor en eulervei 
        fra node 0 til node 4, eller motsatt. \\
        \textbf{3.} Det finnes ingen eulerkrets for denne grafen. Starter man på eulerveien er det ingen måte å komme tilbake til 
        start-noden. \\
        \textbf{4.} Nei. \\
        \textbf{5.} Dersom $f$ være den bijektive funksjonen for en gitt selv-isomorfi av grafen. Da har vi at $f(0)$ har to muligheter,
        nemlig at $f(0) = 0$ eller at $f(0) = 4$. Dersom $f(0) = 0$ så må $f(1) = 1$, $f(2) = 2$, osv. Dersom 
        vi velger $f(0) = 4$, så har vi at $f(1) = 3$, $f(2) = 2$ osv. Altså har grafen to selv-isomorfier.
    }
    \item {
        \textbf{1.} Ja, siden grafen er en sykel med et partall antall noder. \\
        \textbf{2.} Alle nodene i grafen er av partall grad, så grafen har minst en eulervei. \\
        \textbf{3.} Det finnes minst en eulerkrets siden alle nodene er av grad 2. \\
        \textbf{4.} Ja. \\
        \textbf{5.} Vi har først 4 valg for hvor vi skal plassere node 0, så 2 valg for hvor vi skal plassere 
        node 1 slik at vi bevarer kant-relasjonene. Det blir totalt $4 \cdot 2 = 8$ mulige selv-isomorfier.
    }
    \item {
        \textbf{1.} Nei, siden det eksisterer en sykel, f.eks. nodene $\set{0, 1, 2}$, som har et oddetall antall noder. \\
        \textbf{2.} Grafen har flere enn to noder av odde grad, og har derfor ingen eulervei. \\
        \textbf{3.} Nei, fordi grafen ikke har en eulervei. \\
        \textbf{4.} Nei. \\
        \textbf{5.} Grafen har ihvertfall 2 selv-isomorfier, siden $f(0)$ kan både være lik 0 og 3.
    }
    \item {
        \textbf{1.} Nei, siden det finnes sykler av lengde $l \in \left[2, 73\right]$, og derfor flere sykeler med oddetall antall noder. 
        Ergo ingen lovlig fargelegg av grafen med kunn to farger. \\
        \textbf{2.} I en komplett $K_n$ graf er graden til hver node $n-1$, så for $K_{73}$ blir det altså $73-1=72$, som er et partall.
        Grafen har derfor en eulervei. \\
        \textbf{3.} Alle nodene har grad 72, et partall, så det eksisterer en eulerkrets på grafen. \\
        \textbf{4.} Ja. \\
        \textbf{5.} Antallet selv-isomorfier av $K_n$ er lik $n!$, så vi har at $K_{73}$ har $73!$ selv-isomorfier.
    }
    \item {
        \textbf{1.} Ja, siden det er ingen kanter mellom partall, eller noen kanter mellom oddetall, så kan vi fargelegge 
        alle partalls nodene med en farge, og oddetalls nodene med den andre. \\
        \textbf{2.} Alle nodene blir av partall grad (for hvert partall er det 4 oddetall som gir en oddetalls sum),
        så grafen har en eulervei. \\
        \textbf{3.} Alle nodene har grad 4, så det eksisterer en eulerkrets på grafen. \\
        \textbf{4.} Ja. \\
        \textbf{5.} La $f$ være den bijektive funksjonen for en gitt selv-isomorfi. $f(0)$ har 8 valg, så har $f(2)$ 3 valg, $f(4)$ 2 valg og $f(6)$ 1 valg.
        For enhver av disse permuteringene har vi $4!$ permuteringer av de resterende oddetalls nodene. Vi får da at 
        grafen har totalt $8 \cdot 3 \cdot 2 \cdot 4! = 1152$ selv-isomorfier.
    }
\end{enumerate}

\subsection*{Oppgave 2.}
\begin{enumerate}[(a)]
    \item {
        Tilstandsmaskinen aksepterer alle strenger på formen 
        $$m^{*}\,((c \mid b) \mid \Lambda)\,m^{*}\,((c \mid b) \mid \Lambda)\,m^{*} \text{.}$$ 
        Altså, alle strenger der summen av antallet $c$ og $b$ er mindre enn 3.
    }
    \item \hphantom{.}
        \begin{center}
            \includegraphics[width=15cm]{IMG_0011.jpg}
        \end{center}
    \item \hphantom{.}
        \begin{center}
            \includegraphics[width=15cm]{IMG_0012.jpg}
        \end{center}
\end{enumerate}

\end{document}
