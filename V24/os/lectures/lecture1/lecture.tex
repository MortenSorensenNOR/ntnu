\documentclass{article}

% Packages
\usepackage{braket}
\usepackage{xcolor}
% \usepackage{minted}
\usepackage{amsmath}
\usepackage{amssymb}
\usepackage{amsfonts}
\usepackage{graphicx}
\usepackage{mathtools}
\usepackage{tcolorbox}
\usepackage[shortlabels]{enumitem}

\pagecolor[rgb]{0.1,0.1,0.1} %black
\color[rgb]{1,1,1} %grey

% Margins
\evensidemargin=0in
\oddsidemargin=0in
\textwidth=6.5in
\textheight=9.0in
\headsep=0.25in

% Commands
\makeatletter
\newcommand{\listintertext}{\@ifstar\listintertext@\listintertext@@}
\newcommand{\listintertext@}[1]{% \listintertext*{#1}
\hspace*{-\@totalleftmargin}#1}
\newcommand{\listintertext@@}[1]{% \listintertext{#1}
\hspace{-\leftmargin}#1}
\makeatother

\graphicspath{ {./images/} }

\title{\huge{Lecture 01\\ OS}}
\author{Morten Sørensen}
\date{\today}

\begin{document}
\maketitle

\section{Assignments}
\begin{enumerate}
    \item{
        \textbf{8 theoretical assignments - not mandatory}
        \begin{itemize}
            \item Helpful for the final exam.
            \item Usefull to make sure that you are keeping up with the materials.
        \end{itemize}
    }
    \item{
        \textbf{4 practical assignments - Lab sessions}
        \begin{itemize}
            \item Programming tasks, using C language
            \item Have to be submitted on time and passed successfully. Otherwise, one is unqualified for the final exam.
        \end{itemize}
    }
\end{enumerate}

\subsection{Practical assignment}
\begin{enumerate}
    \item {
        \textbf{Purpose} \\
        Hand-on experience with kernel-level programming. Implement some functions within a simplified OS
    }
    \item{
        \textbf{XV6}
        \begin{itemize}
            \item A simplified Unix-like teaching operating systems
            \item Developed at MIT for operating system course since 2006
            \item The lates version is based on RISC-V architecture
            \item More inforation will be introduced on 19th Jan in the practical lecture
        \end{itemize}
    }
    \item {
        The practical assignments can be challenging, because
        \begin{itemize}
            \color{red!45!white}
            \item You are not familiar with C programming
            \item You are not familiar with low-level/system-level programming
            \item You are not familiar with XV6
        \end{itemize}
    }
\end{enumerate}

\section{Administration}
\begin{itemize}
    \item You are free to work from home
    \item Both lectures and practical lectures will be recorded using Panopto and can be watched on Blackboard
    \item The projects will be submitted via Blackboard
\end{itemize}

\section{Textbook}
\begin{enumerate}
    \item The course is mainly based on OSTEP - OS in three easy pieces
    \item Other excellent textbook can be used as reference
\end{enumerate}

\section{Learning goals}
\begin{enumerate}
\item Grasp basic and fundamental knowladge about Operating Systems (\textbf{theory}) and systems-level programming (\textbf{practice})
    \item Important concepts
        \begin{itemize}[-]
            \color{blue!45!white}
            \item \textbf{Process}
            \item \textbf{Scheduling}
            \item \textbf{Memory}
            \item \textbf{Page}
            \item \textbf{Concurrency}
            \item \textbf{Lock}
            \item \textbf{Semaphore}
            \item \textbf{I/O Devices}
            \item \textbf{File systems}
        \end{itemize}
\end{enumerate}

\section{What is an operating system?}
\textit{
    An operating system is a program that manages a computer's hardware. It also provides a basis for application programs and acts as an intermediary between 
    the computer user and the computer hardware. \\ 
    --- \textbf{Operating Systems Concepts}
}

\subsection{What does OS Provide}
\begin{itemize}
    \item \textbf{What is a resource?}
        \begin{itemize}[-]
            \item Anything valuable
        \end{itemize}
    \item \textbf{What abbstaction does modern OS typically provide for each resource?}
        \begin{itemize}[-]
            \item CPU: process and/or thread
            \item Memory: adress space
            \item Disk: files
        \end{itemize}
    \item \textbf{Advantages of OS providing abstraction}
        \begin{itemize}[-]
            \item Seperate interface and underlying hardware implementation
            \item Allow application to reuse common facilities
            \item Make different devices look the same
        \end{itemize}
    \item \textbf{\color{red!45!white}Resource management} -- Share resources in a good manner
        \begin{itemize}[-]
            \item Protect applications from one another
            \item efficient access to resources (cost, time, energy)
            \item Provide fair access to resources
        \end{itemize}
    \item \textbf{\color{blue!45!white} Policies}
        \begin{itemize}[-]
            \item What to do?
            \item E.g. scheduling
        \end{itemize}
    \item \textbf{\color{blue!45!white} Mechenisms}
        \begin{itemize}[-]
            \item How do we do it?
            \item A scheduling algorithm is designed and implemented in a specific way
        \end{itemize}
\end{itemize}

\section{OS Structure}
\begin{itemize}
    \item OS's are becoming \textbf{\color{red!45!white} more complicated} as more functions are added
    \item To manage fthe increasing complexity, some techniques can be deployed:
        \begin{itemize}[-]
            \color{blue!45!white}
            \item Modularity
            \item Abstraction
            \item Layering
            \item Hierarchy
        \end{itemize}
    \item Complex OS's are divided into several \textbf{\color{blue!45!white} several layers}
    \item \textbf{\color{blue!45!white}Kernel} is the core of an OS
    \item It is the software responsible for providing secure access to the hardware and executing programs
        \begin{itemize}[-]
            \color{blue!45!white}
            \item Process management
            \item Device driver
            \item Memory management
            \item System calls
        \end{itemize}
\end{itemize}

\section{System Calls}
\begin{itemize}
    \item \textbf{\color{blue!45!white}System calls} provide the interface between a running program and the \textbf{\color{blue!45!white} the operating system kernel}
        \begin{itemize}[-]
            \item Generally available in routines written in C and C++
            \item Certain low-level tasks may have to be written using \textbf{\color{blue!45!white}assembly language} 
        \end{itemize}

    \item The run-time support system (run-time libraries) provides a system-call interface
        \begin{itemize}[-]
            \item A number assosiated with each system call
            \item System-call interface maintains a table indexed according to these numbers
        \end{itemize}
    \item Major differences in how they are implemented (e.g., Windows vs. Unix)
\end{itemize}

\subsection{OS Structure II}
\begin{itemize}
    \item \textbf{\large{\color{white!60!green}Simple structure}}
        \begin{itemize}[-]
            \item Programs and OS have the same level of privilege
            \item \textbf{\color{blue!45!white}Pros: Efficiency, no switch}
            \item \textbf{\color{red!45!white}Cons: No isolation, security}
            \item \textbf{Example: MS-DOS}  
        \end{itemize}
    \item \textbf{\large{\color{white!60!yellow}Monolithic Kernel}}
        \begin{itemize}[-]
            \item The entire OS works in kernel space
            \item \textbf{\color{blue!45!white}Pros: Good Isolation, security}
            \item \textbf{\color{red!45!white}Cons: Reliability, inflexibility}
            \item \textbf{Example: Linux, Unix} 
        \end{itemize}
    \item \textbf{\large{\color{white!60!purple}Microkernel}}
        \begin{itemize}[-]
            \item Some services or modules are decoupled from the OS kernel and stay in the user space
            \item \textbf{\color{blue!45!white}Pros: Reliability, security, flexibility}
            \item \textbf{\color{red!45!white}Cons: Complexity, communication overhead}
            \item \textbf{Example: Mach, BlackBerry QNX, L4} 
        \end{itemize}
    \item \textbf{\large{\color{white!60!cyan}Hybrid kernel}}
        \begin{itemize}[-]
            \item Combines monolithic kernel and microkernel
            \item \textbf{\color{blue!45!white}A trade-off between the two catergories}
            \item \textbf{\color{red!45!white}Inherits pros and cons from both}
            \item \textbf{Example: Windows, Apple OSs} 
        \end{itemize}
\end{itemize}

\section{Goals of OS}
\begin{itemize}
    \item Primary goals of Operating systems:
        \begin{itemize}[-]
            \item Use the computer hardware in an \textbf{\color{blue!45!white}efficient} manner
            \item Execute user programs and make solving user promblems easier
            \item More stuff, was distracted
        \end{itemize}

    \item Protection
        \begin{itemize}[-]
            \item Between applications, between OS and applications
            \item Isolation: Prevent bad processes to harm others
        \end{itemize}
    \item Reliability
        \begin{itemize}[-]
            \item OS plays a fundamental role for applications' execution
            \item OS runs non-stop
        \end{itemize}
    \item Energy efficiency
        \begin{itemize}[-]
            \item Sustainability
            \item Battery-supplied devices
        \end{itemize}
    \item Security
        \begin{itemize}
            \item More sensitive and private data
        \end{itemize}
    \item Mobility
        \begin{itemize}[-]
            \item IoT devices and embedded devices
        \end{itemize}
\end{itemize}

\end{document}
