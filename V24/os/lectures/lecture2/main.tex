\documentclass[a4paper,11pt,english]{article}
\usepackage{packages}

\begin{document}

%Headingdel:---------------------------------------------
\topmargin -1.5cm
\makebox[\textwidth][s]{
    \begin{minipage}[c]{0.25\textwidth}
        \includegraphics[width=2.0cm]{Bilder/ntnu_logo.png}  
    \end{minipage}
    \begin{minipage}[c]{0.75\textwidth}
        \huge{\textbf{TTT4260 ESDA}} \\
        \Large{Øving 3  ---  Morten Sørensen, \large{\color{black!75!white}\today}}
    \end{minipage}
}
\vspace{0.75cm}
\normalsize


\section*{Introduction to C}
\subsection*{Why C?}
\begin{itemize}
    \item De-facto standard for Systems Programming
    \item It's fast...
    \item High-level low-level programming language
    \item Avoid writing Assembly
\end{itemize}

\subsection*{From Code to Executable}
\begin{itemize}
    \item Pre-process $\rightarrow$ Compile $\rightarrow$ Link $\rightarrow$ Run
\end{itemize}

\subsection*{The Compiler is you Friend}
\begin{itemize}
    \item{Type Checking}
    \item{Use -Wall flag}
    \item{-O2 makes your code faster}
\end{itemize}

\subsection*{Compiling larger projects}
\begin{itemize}
    \item{Manually typing gcc file1.c file2.c file3.c ... is combursome}
    \item{Use Makefile}
    \item{Labs will also use Makefiles}
\end{itemize}

\newpage
\subsection*{Data Types}
\begin{itemize}
    \item {Basic data types
        \begin{itemize}
            \item char
            \item short
            \item int
            \item long
            \item float
            \item double
        \end{itemize}
    }
    \item Data type sizes depend on the machine
    \item Integers: Signed and Unsigned
    \item Boolean does nto exist: use number (0 = False, otherwise True)
    \item Strongly typed
    \item Strings are null terminated char arrays
    \item Arrays dont have a known length
    \item Type Casting
\end{itemize}

\subsection*{Pointer Types in C}
\begin{itemize}
    \item Special Type: Reference to Memory Location
    \item \&: "Adress of" operator
    \item *: "Dereference" operator
    \item int[] = int*
    \item We can also do calculations with pointers - pointer arithmetic
\end{itemize}

\section*{Process}
\subsection*{Overview}
\begin{itemize}
    \item Process concept
    \item Process state
    \item Process API (creation, wait)
    \item Process tree
    \item Limited directed execution
\end{itemize}

\subsection*{Process}
\begin{itemize}
    \item {Programs are a \textbf{\color{blue}static} entity stored on dist (\textbf{\color{blue}executable file}), process is \textbf{\color{blue}active}
            \begin{itemize}
                \item Program becomes a process when the executable file is loaded into memory
            \end{itemize}
        }
    \item {Execution of program started via Graphic User Interface (GUI) mouse clicks, command line entry of its name, etc.
            \begin{itemize}
                \item cat hello.c 
            \end{itemize}
        }
    \item Process is \textbf{\color{blue}an abstraction} of CPU

    \item A program becomes a process when it is selected to execute and loaded into memory
    \item A process has an \textbf{\color{blue}adress space}

    \item \textbf{Consists of:}
        \begin{itemize}
            \item \textbf{\color{blue!60!white}Code:} Instructions
            \item \textbf{\color{blue!60!white}Stack:} Temporary data, e.g., functions parameters, returned adresses, local variables
            \item \textbf{\color{blue!60!white}Registers:} Program counter (PC), general purpose, stack pointer
            \item \textbf{\color{blue!60!white}Data:} Global variables
            \item \textbf{\color{blue!60!white}Heap:} Dynamically allocated
        \end{itemize}
    \item A process is represented by a \textbf{\color{blue}process control block (PCB)}
    \item PCB usuallu stores \textbf{\color{blue}relevant information} of a single process
        \begin{itemize}
            \item Process ID (PID, unique)
            \item State
            \item Parent process
            \item Opened files, etc.
        \end{itemize}
\end{itemize}

\subsection*{Process State}
\begin{itemize}
    \item Process has different states
        \begin{itemize}
            \item \textbf{\color{blue}READY}
                \begin{itemize}
                    \item Ready to run and pending for running
                \end{itemize}
            \item \textbf{\color{blue}RUNNING}
                \begin{itemize}
                    \item Being executed by OS
                \end{itemize}
            \item \textbf{\color{blue}BLOCKED}
                \begin{itemize}
                    \item Suspended due to some other events
                    \item Like I/O requests
                \end{itemize}
            \item \textbf{\color{blue}ZOMBIE}
                \begin{itemize}
                    \item Completed execution but still has an entry in the system 
                \end{itemize}
        \end{itemize}
    \item Process API
        \begin{itemize}
            \item \textbf{\color{blue}CREATE}
                \begin{itemize}
                    \item Create new process, e.g., double click, a command in terminal
                \end{itemize}
            \item \textbf{\color{blue}WAIT}
                \begin{itemize}
                    \item Wait for a process to stop
                    \item Like I/O request
                \end{itemize}
            \item \textbf{\color{blue}DESTROY}
                \begin{itemize}
                    \item Kill the process 
                \end{itemize}
            \item \textbf{\color{blue}STATUS}
                \begin{itemize}
                    \item Obtain information about the process 
                \end{itemize}
            \item \textbf{\color{blue}OTHERS}
                \begin{itemize}
                    \item Suspend and resume a process 
                \end{itemize}
        \end{itemize}
    \item Process Creation
        \begin{itemize}
            \item In Unix-like OSs, a process is created by another process, \textbf{\color{blue}parent process} or \textbf{\color{blue}calling process}
            \item In Unix-like OSs, process creation relies on two system calls
                \begin{itemize}
                    \item \textbf{\color{blue}fork()}
                        \begin{itemize}
                            \item Create a new process and \textbf{\color{blue}clone } 
                        \end{itemize}
                    \item \textbf{\color{blue}exec()}
                        \begin{itemize}
                            \item Overwrite the created process with a new program 
                        \end{itemize}
                \end{itemize}
        \end{itemize}
    \item fork()
        \begin{itemize}
            \item A function without any arguments
                \begin{itemize}
                    \item \textbf{\color{blue}pid = fork()} 
                \end{itemize}
            \item Both \textbf{\color{blue}parent process} and \textbf{\color{blue}child process} continue to execute \textbf{\color{red}the instruction following the fork()}
            \item The return value indicates which process it is (\textbf{\color{blue}parent} or \textbf{\color{red}child})
                \begin{itemize}
                    \item \textbf{\color{blue}Non-zero pid} (pid of child process): return value of the \textbf{\color{blue}parent} process,
                    \item \textbf{\color{red}Zero} : the return value of the new child process
                    \item \textbf{-1} : an error or failure occurs when creating new process
                \end{itemize}

            \item The child process is a \textbf{\color{red}duplicate} of the parent process and has the same 
                \begin{itemize}
                    \item \textbf{\color{blue}Instructions}
                    \item \textbf{\color{blue}Data}
                    \item \textbf{\color{blue}stack}
                \end{itemize}
            \item Child and parents have \textbf{\color{red}different}
                \begin{itemize}
                    \item \textbf{\color{blue}PIDs}
                    \item \textbf{\color{blue}memory adresses}
                \end{itemize}
        \end{itemize}
    \item wait()
        \begin{itemize}
            \item Let the parent process wait for the completion of the child process
                \begin{itemize}
                    \item \textbf{\color{blue}pid = wait()}
                \end{itemize}
            \item \textbf{\color{blue}waitpid()} is an alternative of fork
        \end{itemize}
    \item exec()
        \begin{itemize}
            \item Replace current data and code with new in specified file, i.e., it loads a new program into the current process
                \begin{itemize}
                    \item \textbf{\color{blue}exec(cmd, argv)} 
                \end{itemize}
            \item \textbf{\color{blue}exec()} does not return. It starts to execute the new program. \textbf{\color{red}Return if erros.}
            \item There is a family of \textbf{\color{blue}exec()}, e.g., \textbf{\color{blue}execl(), execvp()}, etc
        \end{itemize}
    \item Why fork() + exec()
        \begin{itemize}
            \item Why dont we directly create a new process instead of "fork+exec"?
                \begin{itemize}
                    \item \textbf{\color{blue}Simple} and \textbf{\color{blue}powerful}
                    \item Import to build Unix Shell (an interface to the Unix system)
                \end{itemize}
            \item Via separating \textbf{\color{blue}fork()} and \textbf{\color{blue}exec()}, we can manipulate various settings just before executing a new program and \textbf{\underline{make the IO redirection and pipe possible}}
                \begin{itemize}
                    \item IO redirection: "cat w3.c > newfile.txt"
                    \item pipe: "echo hello world | wc"
                \end{itemize}
        \end{itemize}
    \item pipe
        \begin{itemize}
            \item A communication method between two processes 
        \end{itemize}
    \item Process Tree
        \begin{itemize}
            \item pstree (to show the process tree)
            \item ps (to show all processes)
        \end{itemize} 
    \item Limited Direct Execution
        \begin{itemize}
            \item The most efficient way to execute a process
                \begin{itemize}
                    \item \textbf{\color{blue}Direct execution (DE)} on CPU
                \end{itemize}
            \item Problems with DE
                \begin{itemize}
                    \item \textbf{\color{red}Restricted Operation}
                        \begin{itemize}
                            \item Protection 
                        \end{itemize}
                    \item \textbf{\color{red}Switching between processes}
                        \begin{itemize}
                            \item Control 
                        \end{itemize}
                \end{itemize}
            \item Solution to \textbf{\color{red}Restricted Operation}
                \begin{itemize}
                    \item Hardware support
                    \item Different execution modes
                \end{itemize}
            \item \textbf{\color{blue}User mode:} restricted, limited operations
                \begin{itemize}
                    \item Process start in user mode
                \end{itemize}
            \item \textbf{\color{red}Kernel mode:} priviliged, not restricted
                \begin{itemize}
                    \item OS starts in kernel mode
                \end{itemize}
            \item What if a process wants to perform some restricted operations?
                \begin{itemize}
                    \item \textbf{\color{blue}System calls:} Allow the kernel services to provide some functionalities to user programs.
                \end{itemize}
            \item A process starts in \textbf{\color{blue}user mode}
            \item If it needs to perform a restricted operation, it calls a system call by executing a \textbf{\color{blue}trap instruction}
            \item The state and registers of the calling process are stored, the system enters \textbf{\color{red}kernel mode}, OS completes the syscal work
            \item \textbf{\color{blue}Return from syscal}, restore the states and registers of the process, and resume the execution of the process
            \item Solution to \textbf{\color{red}Switching Between Processes}
                \begin{itemize}
                    \item \textbf{\color{blue}Cooperative approach}
                    \item \textbf{\color{blue}Non-cooperative approach}
                \end{itemize}
            \item \textbf{\color{blue}Cooperative approach}
                \begin{itemize}
                    \item \textbf{Trust process to relinquish CPU and OS through traps}
                        \begin{itemize}
                            \item \textbf{System calls}
                            \item \textbf{\color{red}Illegal operations, e.g., divided by zero}
                        \end{itemize} 
                    \item \textbf{\color{red}Issue: if no system call}
                \end{itemize}
            \item \textbf{\color{blue}Non-cooperative approach}
                \begin{itemize}
                    \item \textbf{The OS takes control}
                    \item \textbf{OS obtains control periodically, e.g., timer interrupter}
                \end{itemize}
        \end{itemize}
\end{itemize}

\end{document}
