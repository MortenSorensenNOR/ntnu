\documentclass[a4paper,11pt,english]{article}
\usepackage{packages}

\begin{document}

%Headingdel:---------------------------------------------
\topmargin -1.5cm
\makebox[\textwidth][s]{
    \begin{minipage}[c]{0.25\textwidth}
        \includegraphics[width=2.0cm]{Bilder/ntnu_logo.png}  
    \end{minipage}
    \begin{minipage}[c]{0.75\textwidth}
        \huge{\textbf{TTT4260 ESDA}} \\
        \Large{Øving 3  ---  Morten Sørensen, \large{\color{black!75!white}\today}}
    \end{minipage}
}
\vspace{0.75cm}
\normalsize


\section*{Problems of Paging}
\begin{itemize}
    \item Page table is \textbf{\color{red} too big}
    \item A linear page table array for 32-bit address space ($2^{32}$ bytes) and 4KB page ($2^{12}$ bytes)
        \begin{itemize}
            \item How many pages: $2^{20}$ pages
            \item How much memory: \textbf{\color{red} 4MB} assuming each page-table entry is 4 bytes
                \[
                    2^{log_{32}(\SI{4}{\kilo\byte})} \cdot 4 = \SI{4}{\mega\byte}
                \]
            \item One page table for one process:
                \begin{itemize}
                    \item 100 processes: \textbf{\color{red} 400MB}
                \end{itemize}
        \end{itemize}
\end{itemize}

\section*{Smaller Page Table}
\begin{itemize}
    \item Naiive solution:
        \begin{itemize}
            \item \textbf{\color{blue} Bigger page size} -> \textbf{\color{blue} smaller page table}
            \item 32-bit address space: 4KB page size -> 16KB
            \item We can reduce the size by \textbf{\color{blue} 4x} to 1MB per page table
        \end{itemize}
    \item Page size: $2^x$, \textbf{\color{blue} 4KB-1GB}
        \begin{itemize}
            \item getconf PAGESIZE
            \item 16KB for MacOS, 4KB for Linux
        \end{itemize}
    \item \textbf{\color{red} Problem: Internal fragmentation}
        \begin{itemize}
            \item \textbf{Do not use up the whole page}
        \end{itemize}
\end{itemize}

\section*{Variable Page Size}
\begin{itemize}
    \item TLB has \textbf{\color{blue} limited size}
        \begin{itemize}
            \item 16-512
            \item Multiple-level implementation, like cache
        \end{itemize}
    \item Smaller page size -> more TLB entries
        \begin{itemize}
            \item A process of 64KB, \textbf{\color{blue} 4Kb page size}
                \begin{itemize}
                    \item \textbf{16 TLB entries}
                \end{itemize}
            \item \textbf{\color{blue} 1MB page size}
                \begin{itemize}
                    \item 1 TLB entry
                \end{itemize}
        \end{itemize}
    \item Veriable page size
        \begin{itemize}
            \item This depends on hardware and OS
            \item Windows 10 supports 4KB and 2MB
            \item LInux has default page size (4KB) and huge page
        \end{itemize}
\end{itemize}

\section*{Paging+Segmentation}
\begin{itemize}
    \item Hybrid Approach: Paging and Segments
    \item Recall: \textbf{\color{blue} Segmentation}
        \begin{itemize}
            \item Different registers for each segment
        \end{itemize}
    \item Instead of one \textbf{\color{red} single} page table for one process, 
        \textbf{\color{blue} one page table for each logical segment} 
    \item \textbf{\color{blue} Base register} to point to \textbf{\color{blue} the physical address of the page table}
    \item \textbf{\color{blue} Limit register} to determine \textbf{\color{blue} the size of the segment, i.e.,} how many valid pages the segment has
    \item 32-bit virtual address space with 4KB pages
        \begin{itemize}
            \item 12-bit for offset
        \end{itemize}
    \item 3 segments: code, heap, stack 
    \item \textbf{\color{red} Problems: page table waste for sparsely-used heap}
    \item \textbf{\color{red} Probmels: external fragmentation}
\end{itemize}

\section*{Multi-Level paging}
\begin{itemize}
    \item Large page table is \textbf{\color{blue} contiguous} and may have some \textbf{\color{red} unused pages}
    \item Allocate page table in a \textbf{\color{blue} non-contiguous manner}
    \item Break the page table into pages, i.e., page the page tables
    \item Create \textbf{\color{blue} multiple leves of page tables;} outer level "\textbf{\color{blue} page directory}"
        \begin{itemize}
            \item \textbf{\color{blue} Page directory} to track wheter a page of the page table is valid
        \end{itemize}
    \item If an entire page of page table entries is \textbf{\color{red} invalid}, \textbf{\color{red} no allocation}  
    \item A virtual address of 32-bit with 4KB page size is divided into
        \begin{itemize}
            \item a page number consisting of 20 bits
            \item a page offset consisting of 12 bits
        \end{itemize}
    \item A page table entry is 4 bytes
    \item Since the page table is paged, the page number is further divided into
        \[
            \underbrace{[\:\:\:\:p_1\:\:\:\:][\:\:\:\:p_2\:\:\:\:][\:\:\:\:\:\:\:d\:\:\:\:\:\:\:]}_{\text{10 \hphantom{0.5cm} 10 \hphantom{0.55cm} 12 \hphantom{|}}}
        \]
        where $p_1$ is an index into the page directory, and $p_2$ is the apge table index
    \item 16KB address space, 14-bit address
    \item 64-byte ($2^6$) page size
    \item page table entry 4 bytes
    \item Problem with 2 levels: \textbf{\color{red} page directories may not fit in a page}
    \item Split page directories into pieces
    \item Multi-level page directories and each one can fit in a page
    \item 30-bit address space, 512-byte page size, 4 byte PTE
    \item XV6 uses 3 or 4 level MLP (depending on virtual address length)
\end{itemize}

\section*{Page Swapping}
\begin{itemize}
    \item Motivation:
        \begin{itemize}
            \item Processes spend the \textbf{\color{blue} majority of time} in \textbf{\color{blue} small portion of code}
                \begin{itemize}
                    \item The 90/10 rule: approximately \textbf{\color{blue} 90\%} of time is spent in \textbf{\color{blue} 10\%} of the code
                \end{itemize}
            \item Process only uses a \textbf{\color{blue} small amount} of the address space (pages) at any moment
            \item Only small amount of address space (pages) need to be resident in physical memory
        \end{itemize}
    \item Hardware:
        \begin{itemize}
            \item \textbf{\color{blue} Memory: fast}, but \textbf{\color{red} small}, 2-100GB/s
            \item \textbf{\color{red} Disk: slow}, but \textbf{\color{blue} large}, 80-160 MB/s (HDD) 500MB/s (SSD)
        \end{itemize}
    \item Idea:
        \begin{itemize}
            \item Process can run with \textbf{\color{blue} only some of its pages} in memory
            \item Only keep \textbf{\color{blue} the actively used pages} in memory
            \item Keep \textbf{\color{red} unreferenced pages} on disk
        \end{itemize}
    \item Swapping makes it possible for \textbf{\color{blue} the total physical address space of all processes to exceed} the real physical memory of the system
    \item Reserve some space on the disk for moving pages back and forth --- \textbf{\color{blue} Swapping space}
    \item OS keeps track of the swap space, in \textbf{\color{orange} page-sized unit} 
    \item How to know where a page lives?
        \begin{itemize}
            \item \textbf{\color{blue} Present bit/Valid bit}
            \item \textbf{\color{blue} 1 indicates in-memory}
            \item \textbf{\color{red} 0 indicates in-disk}
        \end{itemize}
    \item \textbf{\color{red} Page fault:} if present bit in PTE is 0, when accessing a page
\end{itemize}

\section*{Page Swapping Policies}
\begin{itemize}
    \item The objective of page swapping policies: to minimize the number of \textbf{\color{red} page faults} (\textbf{\color{red} cache misses})
    \item Two decisions:
        \begin{itemize}
            \item \textbf{\color{blue} Page selection}
                \begin{itemize}
                    \item \textbf{\color{blue} When} should a page on disk be brought into memory?
                \end{itemize}
            \item \textbf{\color{blue} Page replacement}
                \begin{itemize}
                    \item \textbf{\color{blue} When} in-memory pages should be evicted to disk?
                \end{itemize}
        \end{itemize}
    \item \textbf{\color{blue} Demand paging:}
        \begin{itemize}
            \item Load page only when it is needed (\textbf{\color{blue} demand})
            \item Less I/O, less memory
            \item Problems: \textbf{\color{red} High page fault cost}
        \end{itemize}
    \item \textbf{\color{blue} Prefetch:}
        \begin{itemize}
            \item Load page before referenced
            \item OS predicts future accessed pages (oracle) and brings them into memory early
            \item Works well for some access patterns, like sequential pages
        \end{itemize}
\end{itemize}

\section*{Copy-On-Write Paging}
\begin{itemize}
    \item \textbf{\color{blue} Copy the page} only if a process writes to it (\textbf{\color{blue} demand})
        \begin{itemize}
            \item Process creation \textbf{\color{blue} fork() + exec()}
        \end{itemize}
\end{itemize}

\section*{Page Replacement}
\begin{itemize}
    \item When does page replacement happen?
    \item \textbf{\color{blue} Lazy approach}
        \begin{itemize}
            \item If memory is \textbf{\color{red} entirely full}, the the OS repalces a page to make room for some other page
            \item This is \textbf{\color{red} unrealistic}
            \item The OS usually needs to \textbf{\color{blue} reserve some room} for the new pages
        \end{itemize}
    \item \textbf{\color{blue} Swap Daemon, Page Daemon}
        \begin{itemize}
            \item There are fewer than \textbf{\color{red} LW (low watermark)} pages available, a background thread that is responsible for freeing memory is activated 
            \item The tread avicts page until there are \textbf{\color{blue} HW (high watermark)} pages available
        \end{itemize}
    \item 
\end{itemize}

\section*{Page Replacement Policies}
\begin{itemize}
    \item \textbf{\color{blue} Average memory access time (AMAT)}
        \[
            AMAT = P_{Hit} \times T_M + P_{Miss} \times T_D 
        \]
    where $P_{Hit} + P_{Miss} = 1$.
    \item Optimal replacement:
        \begin{itemize}
            \item Replacement the page which is not used for the \textbf{\color{blue} longest time in the future}
            \item \textbf{\color{blue} Pros: Minimal number of page faults}
            \item \textbf{\color{red} Cons: impractical, need to predict the future}
            \item \textbf{Can be used as a comparison baseline}
        \end{itemize}
    \item FIFO:
        \begin{itemize}
            \item Replace the page which is loaded into memory first
            \item \textbf{\color{blue} Pros: Fair, easy to implement}
            \item \textbf{\color{red} May evict useful pages}
        \end{itemize}
    \item Least-recently-used (LRU): (Predict using history)
        \begin{itemize}
            \item Replace the page which has not been used for the longest time
            \item \textbf{\color{blue} Pros: Approximate optimal replacement}
            \item \textbf{\color{red} Cons: Difficult to implement}
        \end{itemize}
    \item Other pilicies:
        \begin{itemize}
            \item \textbf{\color{blue} Random (RAND), Least-frequently used (LFU)}
        \end{itemize}
    \item The performance of replacement policies also depenfs on workloads
        \begin{itemize}
            \item mRandom workload: LRU, RAND and FIFO \textbf{\color{red} no difference}
            \item 80-20 workload: \textbf{\color{blue} LRU} is better than RAND and FIFO
            \item Looping sequential workload: \textbf{\color{blue} RAND} is better than LRU and FIFO
        \end{itemize}
    \item What happens to performance, if adding more physical memory
        \begin{itemize}
            \item LRU and RAND have frwer or same number of page faults
            \item FIFO usually has fewer page faults, but Belady's anomaly -> \textbf{\color{red}more page faults}
                \begin{itemize}
                    \item \textbf{Sequence: ABCDABEABCDE}
                    \item \textbf{\color{red} 3 frames vs. 4 frames}
                    \item \textbf{\color{red} 9 misses vs. 10 misses}
                \end{itemize}
        \end{itemize}
    \item How to implement LRU 
        \begin{itemize}
            \item \textbf{\color{red} Software Perfect LRU}: a data structure to track reference time of all pages
            \item \textbf{\color{red} Hardware Perfection LRU}: add a timestamp register to each page
            \item \textbf{\color{blue} Practical LRU}: approximate implementation, find an old page, but not necessarily the oldest one;
        \end{itemize}
\end{itemize}

\end{document}
