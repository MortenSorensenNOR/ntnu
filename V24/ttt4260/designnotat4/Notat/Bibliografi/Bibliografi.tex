\bibitem{dsp_self}
    John G. Proakis, Vinay K. Ingle, et.al,
    \emph{A Self-Study Guide for Digital Signal Processing},
    1st edition,
    Pearson,
    2003.
    ISBN 978-0-13-143239-0.

\bibitem{lf353_opamp}
    Texas Instruments
    \emph{LF353 Wide-Bandwidth JFET-Input Dual Operational Amplifier}
    Mars, 2016
    \url{https://www.ti.com/lit/ds/symlink/lf353.pdf?ts=1709921659081&ref_url=https\%253A\%252F\%252Fwww.ti.com\%252Fproduct\%252FLF353}

\bibitem{SNR}
    E-Mastered Blog
    \emph{ Signal to Noise Ratio: The Ultimate Guide}
    .10 april, 2023
    \url{https://emastered.com/blog/signal-to-noise-ratio}

\bibitem{frekvens_mul}
    Lars Lundheim
    \emph{Frekvensmultiplikator.}
    24. april, 2023

\bibitem{design3}
    Morten Sørensen
    \emph{Designnotat 3: Støyfiltrering.}
    20. mars, 2024

\bibitem{analog_discovery}
    Digilent
    \emph{Analog Discovery 2.}
    \newline
    \url{https://digilent.com/shop/analog-discovery-2-100ms-s-usb-oscilloscope-logic-analyzer-and-variable-power-supply/}
    
\bibitem{waveforms}
    Digilent
    \emph{Waveforms.}
    \newline
    \url{https://digilent.com/reference/software/waveforms/waveforms-3/start}

\bibitem{fir_wiki}
    Wikipedia
    \emph{Finite Impulse Response Filter.}
    \url{https://en.wikipedia.org/wiki/Finite_impulse_response}

\bibitem{nyquist}
    Wikipedia
    \emph{Nyquist rate.}
    \url{https://en.wikipedia.org/wiki/Nyquist_rate}

\bibitem{aliasing}
    Wikipedia
    \emph{Aliasing.}
    \url{https://en.wikipedia.org/wiki/Aliasing}

\bibitem{zturn}
    Digikey
    \emph{MYS-7Z020-V2.}
    \url{https://www.digikey.no/no/products/detail/myir-tech-limited/MYS-7Z020-V2-0E1D-766-C-S/18668544}

\bibitem{fixed_point}
    Wikipedia
    \emph{Fixed-point arithmetic}
    \url{https://en.wikipedia.org/wiki/Fixed-point_arithmetic}

\bibitem{pipelining}
    Wikipedia
    \emph{Pipelining}
    \url{https://en.wikipedia.org/wiki/Pipelining_(DSP_implementation)}

\bibitem{shockley}
    Wikipedia
    \emph{Shockley diode equation}
    \url{https://en.wikipedia.org/wiki/Shockley_diode_equation}
