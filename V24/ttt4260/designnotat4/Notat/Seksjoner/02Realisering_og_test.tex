\section{Realisering}
\label{realiseringOgTest}

Realiseringen av systemet er blitt gjort ved å sende et generert sinussignal 
med frekvens $f = \SI{2600}{\hertz}$ og amplitude $A_1 = \SI{1}{\volt}$ gjennom systemet.
Signalet ble generert av an Analog Discovery 2 \cite{analog_discovery}. 

For evalueringen av hele systemet ble Analog Discovery-ens to oscilloskop brukt. Den første, Channel 1,
ble koblet til inngangssignalet $x_1(t)$. Den andre, Channel 2, ble koblet 
til utgangssignalet $\hat{x}_2(t)$.
For hvert av de underliggende systemene - det ulineære systemet og båndpassfilteret - ble 
Channel 1 koblet til inngangssignalet og Channel 2 koblet til utgangssignalet av systemet.

\subsection{Det ulineære systemet}
For realiseringen av det ulineære systemet har en motstandsverdi $R_1$ på $\SI{100}{\kilo\ohm}$ blitt brukt. Denne komponentverdien 
ble valgt for å forhindre mulige ikke-idealiteter i den valgte dioden med henhold til diodens motstandsverdi.

\subsection{Båndpassfilter}

\subsubsection{RLC båndpass}
For realiseringen av det analoge RLC båndpassfilteret har en induktansverdi på $L = \SI{100}{\milli\henry}$ blitt valgt. To slike spoler ble 
målt, der begge ble målt til en verdi på $L = \SI{100}{\milli\henry} \pm \SI{1}{\milli\henry}$ over et spektrum fra $\SI{1}{\kilo\hertz}$ til $\SI{20}{\kilo\hertz}$. 
Får å oppnå den ønskede senterfrekvensen til systemet får vi at kondensatorverdien til 
RLC filteret blir 
\[
    C = \frac{1}{4\pi^2 \cdot (2f)^2 \cdot L} = \SI{9.37}{\nano\farad}
\]

For realiseringen av systemet har en parallellkobling av to $\SI{4.7}{\nano\farad}$
kondensatorer blitt brukt, slik vist i figur \ref{fig:impl_rcl_sys}.
Parallellkoblingen av kondensatorene ble målt til en kombinert kapasitansverdi på $C = \SI{9.265}{\nano\farad} \pm \SI{0.08}{\nano\farad}$ over et spektrum fra $\SI{1}{\kilo\hertz}$ til $\SI{20}{\kilo\hertz}$.
Dette gir en effektiv senterfrekvens til realiseringen av filteret på $f_c = \SI{5226}{\hertz}$.

\begin{figure}[H]
    \centering
    \includegraphics[width=0.75\textwidth]{Bilder/Analog_sys_val.png}
    \caption{Realisering av et RLC båndpassfilter med senterfrekvens $f_c = \SI{5.2}{\kilo\hertz}$}
    \label{fig:impl_rcl_sys}
\end{figure}

Til slutt er et $\SI{10}{\kilo\ohm}$ potensiometer valgt for $R_2$ for å kunne justere kvalitetsfaktoren
til filteret. Opampen valgt for implementeringen er en LM358P \cite{lf353_opamp} med et operasjonsområde på 
$\pm \SI{15}{\volt}$. Den ble gitt +$\SI{5}{volt}$ på den positive supply inngangen og -$\SI{5}{volt}$ på den negative.
Begge kildene ble supplert fra Analog Discovery-en.

Det endelige systemet ble valgt til å ha et andre ordens filter der to identisk filter til filteret i figur \ref{fig:impl_rcl_sys} 
kaskadekoblet med en buffer i mellom. Implementeringen av systemet er vist i figur \ref{fig:impl_rcl_sys_img}.

\begin{figure}[H]
    \centering
    \includegraphics[width=0.4\textwidth]{Bilder/Analog_sys_img.jpg}
    \caption{Realisering av et andre ordens RLC båndpassfilter med senterfrekvens $f_c = \SI{5.2}{\kilo\hertz}$}
    \label{fig:impl_rcl_sys_img}
\end{figure}

\subsubsection{FIR båndpass}
For realiseringen ble filterkoeffisientene regnet ut med MatLab koden fra pinsippiel løsning,
med følgende parametre: 
\begin{figure}[H]
    \label{fig:fir_filter_params}
    \centering
    \begin{tabular}{|l|r|}
        \hline
        \textbf{Parameter} & \textbf{Verdi} \\
        \hline
        $f_s$ & $\SI{96}{\kilo\hertz}$ \\
        \hline
        $A_\text{stopp}$ & $-80$dB \\
        \hline
        $A_\text{pass}$ & $-0.5$dB \\
        \hline
        $F_\text{stopp1}$ & $\SI{3}{\kilo\hertz}$ \\
        \hline
        $F_\text{pass1}$ & $\SI{4.5}{\kilo\hertz}$ \\
        \hline
        $F_\text{pass2}$ & $\SI{5.5}{\kilo\hertz}$ \\
        \hline
        $F_\text{stopp2}$ & $\SI{7}{\kilo\hertz}$ \\
        \hline
    \end{tabular} 
\end{figure}

Det genererte et filter med orden 323 med amplituderespons plottet i figur \ref{fig:fir_filter_perf}.
\begin{figure}[H]
    \centering 
    \includegraphics[width=0.85\textwidth]{Bilder/fir_filter_perf.png}
    \caption{Amplituderesponsen til det implementerte FIR filteret}
    \label{fig:fir_filter_perf}
\end{figure}

Vi ser altså at vi klarte å oppnå den ønskede dempningen i både båndpass- og båndstoppområdet. Filterkoeffisientene 
ble så eksportert til en fil, slik at den kunne importeres inn i python programmet.

FPGA-en valgt for implementering på hardware var en Xilinx XC7Z020 FPGA på et MYIR Z-Turn V2 kretskort.
\begin{figure}[H]
    \centering 
    \includegraphics[width=0.5\textwidth]{Bilder/Zturn.jpg}
    \caption{MYIR Z-Turn V2 kretskort med en Xilinx XC7Z020 FPGA}
    \label{fig:zturn}
\end{figure}

Ettersom at Xilinx FPGA-en har en innebygd balansert analog til digital omformer, har implementeringen 
valgt å benytte den istedenfor en ekstern ADC. Den har en pressisjon på 12-bit, har en maksimal samplingfrekvens på
$\SI{1}{\mega\hertz}$, og godtar innputspenninger fra $-\SI{0.5}{\volt}$ til $\SI{0.5}{\volt}$. Vi må derfor senke 
amplituden til det originale signalet til $\SI{0.5}{\volt}$, eller sende den målte spenningen gjennom en enkel spenningsdeler.
Denne implementeringen har valgt å endre spenningen til $x_1(t)$, men dette er så klart applikasjonsspesifikt.

Under er kretstegningen for systemet \ref{fig:fir_sys}. Signalet fra bufferen etter det ulineære systemet sendes gjennom et lavpassfilter med 
knekkfrekvens lik $f_s / 2 = \SI{48}{\kilo\hertz}$, og signalet måles differensielt i forhold til et felles ground nivå, s.a. FPGA-ens eget 
groundnivå ikke endrer signalet. Til slutt blir det digitale, filtrerte signalet sendt ut fra Z-Turn V2 kortets GPIO porter til en 
12-bit bipolar - altså at den kan sende ut både negative og positive spenninger - digital til analog omformer. Dessverre var ikke en slik 
DAC tilgjengelig for implementeringen, så hele systemet ble ikke testet.
\begin{figure}[H]
    \centering
    \includegraphics[width=0.9\textwidth]{Bilder/Digital_sys.png}
    \caption{Realisering av system med FIR båndpassfilter og en samplingfrekvens $f_s = \SI{50}{\kilo\hertz}$}
    \label{fig:fir_sys}
\end{figure}

Til slutt ble klokkefrekvensen til systemet valgt til å være $\SI{100}{\kilo\hertz}$, altså større enn samplingfrekvensen til den interne ADC-en.
I figur \ref{fig:vivado_fir_block_diagram} er et blokkdiagram av systemet vist fra Vivado, som er programvaren som ble brukt for syntetisering og 
ruting av RTL koden.
\begin{figure}[H]
    \centering 
    \includegraphics[width=0.7\textwidth]{Bilder/Vivado_fir_filter.png}
    \caption{Blokk diagram fra Vivado; System implementering av FIR og ADC på Xilinx XC7Z020 FPGA}
    \label{fig:vivado_fir_block_diagram}
\end{figure}

Dette designet vil trolig benytte alle de 220 Digital Signal Processing blokkene på XC7Z020 FPGA-en i tillegg til en større andel
av dens 50,000 LUTs.
