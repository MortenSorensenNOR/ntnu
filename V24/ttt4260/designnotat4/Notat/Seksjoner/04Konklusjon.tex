\newpage
\section{Konklusjon}
\label{konklusjon}

Dette designnotatet har tatt for seg to design av en frekvensdobler som 
tar inn et sinus-formet signal $x_1(t)$ med kjent frekvens $f$, og produserer et utgangssignal 
$\hat{x}_2(t)$, der frekvensen til utgangssignalet er $2f$. Kvaliteten på systemet
ble vurdert basert på utgangssignalets signal-til-distorsjonsforhold. Systemet som ble 
presentert bestod av to deler; et ulineært system som produserer overharmoniske frekvenskomponenter ved 
$k \cdot f$, og et båndpassfilter som skal isolere den første harmoniske ved $2f$. To 
filtre ble presentert og implementert; et analogt RLC filter og et digitalt FIR filter.
Gjennom testing av systemet ble det funnet at implementeringen av systemet med et RLC filter 
oppnådde en SDR på 18dB, men at systemet trolig kan oppnå bedre resultater dersom realiserte 
komponentverdier hadde hatt mindre avvik fra de ideelle verdiene, og dersom orderen på kaskadekoblingen 
av filtre hadde vært større. Videre ble det digitale FIR filteret undersøkt. Den oppnådde en SDR 
verdi på 44.60dB, en omtrentlig 457 ganger større enn det analoge filteret. Designnotates mål om å oppnå
er SDR på over 30dB ble dermed nådd ved bruk av et digitalt FIR filter. Filteret ble implementert både 
i python og på en Xilinx XC7Z020 FPGA, men ble kunn testet på python implementeringen grunnet tilgang på 
en digital til analog omformer som kunne omforme til både negative og positive spenningssignal.
