\newpage
\section{Testing}

\subsection{Det ulineære systemet}
For test av det ulineære systemet ble $x_1(t)$ påtrykt systemet med $A_1 = \SI{1}{\volt}$, og 
spekteret til ut- og inngangen av signalet observert. Vi ser i figur \ref{fig:test_ulin_sys} at 
vi har klart å produsere overtoner ved $2f$, $3f$, $4f$, osv.
\begin{figure}[H]
    \centering 
    \includegraphics[width=0.8\textwidth]{Bilder/ulin_sys_spektrum_plot.png}
    \caption{Frekvensspektrum før og etter det ulineære systemet ved $f = \SI{2.6}{\kilo\hertz}$}
    \label{fig:test_ulin_sys}
\end{figure}

Det er verdt å merke at magnituden av utgangssignalet er sterkt svekket i.fht. inngangssignalet, fra omtrent $-5$dBV
til omlag $-15$dBV. Ettersom at båndpassfilteret kunn vil dempe styrken på signalet videre, er det tenkelig 
at en forsterker på utgangen av systemet hadde vært gunstig for praktiske applikasjoner av frekvensdobleren.

\subsection{Båndpassfilter}
\subsubsection{RLC båndpass}
For testig av det analoge RLC filteret ble programvaren Waveforms \cite{waveforms} sin "Nettverk" funksjon 
benyttet for å produsere frekvensresponsen til filteret. I figur \ref{fig:test_analog_filter_perf} er 
amplituderesponsen til første og andre ordens RLC båndpassfilter implementeringene plottet opp mot hverandre.

\begin{figure}[H]
    \centering 
    \includegraphics[width=0.95\textwidth]{Bilder/test_filter_perf.png}
    \caption{Amplituderespons til første og andre ordens RLC båndpassfilter med senterfrekvens $f_c = \SI{5.2}{\kilo\hertz}$}
    \label{fig:test_analog_filter_perf}
\end{figure}

Vi ser en betydelig forbedring i filterkarakteristikk ved andre ordens filteret, med en demping på $|H(f)| = -45$dB ved 
frekvensen til inngangssignalet $f = \SI{2600}{\hertz}$ sammenlignet med $-19$dB for første ordens implementeringen, samtidig som at
båndbredden på filteret gikk fra rundt $\SI{980}{\hertz}$ til rundt $\SI{330}{\hertz}$.

Det er tydelig når vi observerer amplituderesponsen at senterfrekvensen til filteret ikke helt samsvarte med 
$2f$. Dette er trolig grunnet valget av å tilnærme den ønskede kondensatorverdien med parallellkoblingen av to $\SI{4.7}{\nano\farad}$ kodensatorer.
To mulige forbedringer er mulig å implementere for å gjøre filterkarakteristikken bedre: øke orderen av filteret, altså legge til flere filtre i kaskadekoblingen,
og implementere designe med mer nøyaktige komponentverdier. 

Videre kan vi analysere formen på utgangssignalet plottet i figur \ref{fig:test_analog_resulting_waveform}. Vi ser her 
tydelig påvirkningen kaskadekoblingen av filtre hadde på utgangssignalet. 

\begin{figure}[H]
    \centering 
    \includegraphics[width=0.95\textwidth]{Bilder/test_analog_resulting_waveform.png}
    \caption{Utgangssignal fra hele systemet gitt første og andre ordens RLC båndpassfilter vs. inngangssignal $x_1(t)$}
    \label{fig:test_analog_resulting_waveform}
\end{figure}

Ser vi på spekteret til utgangssignalet $\hat{x}_2(t)$ i figur \ref{fig:test_analog_resulting_spectrum} ser vi tydelig at de overharmoniske frekvensene fortsatt er 
tilstede i utgangssignalet. Det er derfor tydelig at systemimplementeringen har nytte av filter-forbedringene diskutert ovenfor.

\begin{figure}[H]
    \centering 
    \includegraphics[width=0.95\textwidth]{Bilder/test_analog_resulting_spectrum.png}
    \caption{Frekvensspektrumet til utgangssignalet $\hat{x}_2(t)$ ved andre ordens RLC filter}
    \label{fig:test_analog_resulting_spectrum}
\end{figure}

Videre får vi følgende målinger for RMS spenningene til det realiserte utgangssignalet $\hat{x}_2(t)$ og 
den ønskede frekvenskoponenten $x_2(t)$
\begin{center}
    $V_{\hat{x}_2} = \SI{75,52}{\milli\volt}$ og $V_{x_2} = \SI{74.95}{\milli\volt}$
\end{center}

Dersom vi beregner SDR for systemet får vi at 
\[
    \text{SDR} = \frac{V_{x_2}^2}{V_{\hat{x}_2}^2 - V_{x_2}^2} = \frac{(\SI{74.95}{\milli\volt})^2}{(\SI{75.52}{\milli\volt})^2 - (\SI{74.95}{\milli\volt})^2} \approx 65.49
\]

og i desibel
\[
    \text{SDR}_\text{{[dB]}} = 10\log{(\text{SDR})} \approx 18.16\text{dB.}
\]

Altså oppnår ikke implementeringen av det analoge RLC filteret den ønskede SDR verdien på 30dB.

\subsubsection{FIR båndpass}
For testingen av FIR båndpassfilteret ble pythonprogrammet i \ref{code:fir_bandpass_python} kjørt med filterkoeffisientene 
funnet og måledata fra det ulineære systemet gjennom RC lavpassfilteret med cutoff frekvens på omlag $\SI{24}{\kilo\hertz}$. 
Signalet ble samplet med samplingfrekvens $f_s = \SI{96}{\kilo\hertz}$ av Analog Discovery-en.
Dette er fordi en 12-bit bipolar DAC ikke var tilgjengelig, og fordi det gir et mer direkte innblikk i hvorvidt filteret 
var godt nok tilpasset reell data. Trolig ville implementeringen på FPGA hatt noe lavere presisjon grunnet fikspunktmultiplikasjon 
versus floatingpoint og høyere antall bits per filterkoeffisient, i tillegg til at DAC trolig ville introdusert noe forstyrrelser i systemet, og 
dermed også utenfor dette designets kontroll.

Ser vi på resultatet av å kjøre python koden får vi følgende spenningsplott for inn og utgangssignalet:
\begin{figure}[H]
    \centering 
    \includegraphics[width=0.95\textwidth]{Bilder/fir_result_signal.png}
    \caption{Plot av utgangssignalet $\hat{x}_2(t)$ ved FIR filter med $f_s = \SI{96}{\kilo\hertz}$}
    \label{fig:fir_result_signal}
\end{figure}

Vi ser tydelig at utgangssignalet $\hat{x}_2(t)$ har dobbelt så stor frekvens som inngangssignalet, og vi kan
observere at utgangssignalet ser ut til å være tilnærmet lik et rent sinussignal.
For å få imperisk data på dette kan vi se på frekvensspektrumet til signalet i figur \ref{fig:fir_result_spectrum}.
\begin{figure}[H]
    \centering 
    \includegraphics[width=0.95\textwidth]{Bilder/fir_result_spectrum.png}
    \caption{Frekvensspektrumet til utgangssignalet $\hat{x}_2(t)$ ved FIR filter med $f_s = \SI{96}{\kilo\hertz}$}
    \label{fig:fir_result_spectrum}
\end{figure}

Vi ser her at frekvenskomponenten ved $f$ er blitt dempet til -96.5dB, en stor forbedring over det det andre ordens 
analoge RLC båndpassfilteret klarte, som var på -56.3dB. Vi ser også at frekvenskomponenten ved $2f$ har blitt nærmest 
urørt av filteret sammenlignet med dBV verdien fra utgangen av det ulineære systemet.

Til slutt har vi at $\text{SDR}_{[\text{dB}]}$ verdien til utgangssignalet ble målt til
\[
    \text{SDR}_{[\text{dB}]} = 44.60\text{dB}
\]
altså godt over ønsket verdi på 30dB. Dette munner ut i et utgangssignal $\hat{x}_2(t)$ som har 26.4dB høyere 
SDR i forhold til det analoge filteret, noe som tilsvarer en økning på omlag 457 ganger høyere SDR.


