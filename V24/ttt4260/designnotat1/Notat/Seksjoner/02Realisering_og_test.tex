\section{Realisering og test}
\label{realiseringOgTest}

%Her kan du skrive om realisering og test
\subsection{Realisering}
For realiseringen av systemet er følgende krafspesifikasjoner gitt:
\begin{align*}
    &A_{min} = -5\text{dB} \\
    &A_{max} = -22\text{dB}
\end{align*}
Setter vi $A_{min}$ og $A_{max}$ inn i likning (2) og (3) får vi følgende likningsett
\begin{align}
    &20 \cdot \lg{\left(\frac{R + R_2}{R + R_1 + R_2}\right)} = -5 \text{dB} \label{eq:eq1} \tag{I} \\
    &20 \cdot \lg{\left(\frac{R_2}{R_1 + R_2}\right)} = -22 \text{dB} \label{eq:eq2} \tag{II}
\end{align}

For å løse likningssettet må vi først velge en verdi for potensiometeret $R$.
Den velger vi til å være $R \approx 10.65k\Omega$.
Vi kan så løse for $R_1$ og $R_2$. Dette gir $R_1 \approx 8885\Omega$ og $R_2 \approx 767\Omega$. Fordi motstandene brukt i realiseringen 
hadde en toleranse på $\pm 5$\% ble følgende kombinasjon av motstander brukt etter eksperimentering og måling
\begin{align*}
    &R_1 = 2 \cdot 3.3k\Omega + 2 \cdot 1k\Omega + 220\Omega + 3 \cdot 33\Omega + 10\Omega = 8929\Omega \pm 5\% \\
    &R_2 = 2 \cdot 330\Omega + 100\Omega + \left(10 \mid\mid 10 \mid\mid 10 \mid\mid 10\right)\Omega = 762\Omega \pm 5\%
\end{align*}

Vi forventer da å få følgende målte $A_{min}$ og $A_{max}$ verdier:
\begin{align*}
    &A_{min} = (-5.0 \pm 0.2) \text{dB} \\
    &A_{max} = (-22.0 \pm 0.8) \text{dB} 
\end{align*}

For implementeringen av bufferet ble en LM353P Op-Amp fra Texas Instruments brukt \cite{lf353_opamp}.

\subsection{Testing}
For analyse av systemet har en Analog Discovery 2 \cite{discovery} blitt brukt. 
Signalgeneratoren ble satt til å generere et sinussignal med amplitude på 1V og 
frekvens $f = 1000$Hz, og hadde en sampling rate på $4 \cdot 10^6$Hz.

\begin{figure}[H]
    \centering
    \includegraphics[width=0.6\textwidth]{Bilder/voltage_plot.png}
    \caption{Utgangsspenning $v_2(t)$ vs. inngangsspenning $v_1(t)$ for $A = A_\text{max}$ og $A = A_\text{min}$.}
    \label{fig:volt_plot}
\end{figure}

Testing av systemet ble utført ved å skru potensiometeret slik at den har minimal motstand for måling av $A = A_{min}$ og slik at den har maksimal 
motstand for måling av $A = A_{max}$. I figur 3. er utgangen av nivåregulatoren plottet over tid sammen med inngangssignalet $v_1(t)$. For $A_{min}$ 
var maks amplitude $0.566\text{V}$ som medfører at $A_{min} = -4.94\text{dB}$, 
og for $A_{max}$ var maks amplitude $0.080 \text{V}$ som tilsvarer at 
$A_{max} = -21.92 \text{dB}$. Altså er avviket fra ønsket verdi 
$\Delta A \leq 0.1\text{dB}$ i begge tilfellene.

\begin{figure}[H]
    \centering
    \includegraphics[width=0.6\textwidth]{Bilder/desibel_plot.png}
    \caption{Dempningsfaktor $A$ som funksjon av tid.}
    \label{fig:gain_plot}
\end{figure}

Ser vi derimot på $A$ som en funksjon av tid som vist i figur \ref{fig:gain_plot} ser vi at når signalet $v_1(t)$ nærmer seg 0V
blir spredningen i dempningsfaktoren $A$ stor, både for $A_\text{max}$ og for $A_\text{min}$.
Det kan være flere grunner for det store spriket i $A$-verdier. Trolig er dette grunnet begrenset presisjon 
i måleverkyøyet. 

\begin{figure}[H]
    \centering
    \includegraphics[width=0.7\textwidth]{Bilder/gauss_plot.png}
    \caption{Gaussfordeling og boksplott av demping av signalet.}
    \label{fig:gaus_plot}
\end{figure}

Dersom vi likevel ser mer analytisk på måledataen, får vi plottet i figur \ref{fig:gaus_plot}. Plottet viser 
et boksplott av alle målte verdier for $A_\text{min}$ og $A_\text{max}$, samt en tilpasset normalfordelingskurve.
Vi ser at gjennomsnittlig målt dempningsverdi for $A_\text{min}$ er $-5.03$dB med et standardavvik på $0.3$dB og for $A_\text{min}$ et gjennomsnitt på
$-22.16$dB med et standardavvik på $1$dB. Vi ser altså at tilpasset verdi for både $A_\text{min}$ og $A_\text{max}$, og at 
de har et godt gyldighetsområde til tross for de store sprikene i målt dempningsverdi rundt $0$V.


