\section{Konklusjon}
\label{konklusjon}

%Her kan du skrive din konklusjon
Dette designnotatet har tatt for seg et design av en variable nivåregulator for demping av 
signaler ved bruk av en variabel spenningsdeler. Designet ble implementert og testet på et signal
med amplitude på 1V og frekvens $f = 1000$Hz, der implementeringen hadde som spesifikasjon å oppnå 
maksimal og minimal demping av signalet på henholdsvis $A_{max} = -22$dB og $A_{min} = -5$dB, med et 
avvik på mindre enn $0.1$dB. Testingen av systemet viste at begge spesifikasjonsmålene ble møtt innenfor 
gode marginer, men at det gurnnet begrenset presisjon i måleverktøyene var vanskelig å verifisere systemets 
effekt på signalet når inngangs- og utgangssignalet lå nærme 0V.
