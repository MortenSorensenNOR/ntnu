\section{Prinsipiell løsning}
\label{prinsipiellLoesning}

Prinsipielt kan systemet for den elektroniske terningen representeres som i figur 2. Terningen har som input en knapp, som enten sender et høyt eller 
et lavt signal til systemet. Når signalet fra knappen er høyt, altså når knappen er trykket inn, itereres en teller. Når tallet i telleren er større 
enn 6, resettes telleren til 1. Når signalet fra knappen går lavt, blir verdien i output registeret satt lik tallet i telleren. Tallet i output 
registeret blir så omgjort til et signal som representerer hvilke lysdioder som skal være aktive for å realisere det korrosponderende mønsteret på 
lysdiodene, før de relevante diodene så får et høyt signal.

I denne løsningen vil altså utgangsverdien av hvert "kast" være tilnærmet deterministisk, men dersom klokkehastigheten til systemet er slik at telleren 
teller fra 1 til 6 flere ganger i tiden det tar fra brukeren slipper knappen til utgansverdien fra knappen går lavt, vil resultatet oppleves tilfeldig.

\begin{figure}[H]
    \centering
    \includegraphics[width=0.65\textwidth]{Bilder/esda_flow.png}
    \caption{Flytdiagram representasjon av en mulig prinsipiell løsning}
\end{figure}

% Lag en kretstegning for hvordan lysdiodene blir koblet opp, og beskriv
