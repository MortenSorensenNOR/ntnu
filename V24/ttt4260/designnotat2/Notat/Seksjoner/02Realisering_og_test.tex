\section{Realisering}
\label{realiseringOgTest}

Realiseringen av den elektroniske terningen er gjort på et Nandland Go Board \cite{goboard} 
og et koblingsbrett. Go Board kortet benytter en Lattice ICE40 FPGA, som er 
blitt brukt for å implementere designet av logikken i terningen. Oppkoblingen av 
LED-ene som danner terningmønsteret er gjort på et oppkoblingsbrett, der LED-ene 
er koblet til Go Board kortets PMOD utganger. I figur 3 er den skjematisk fremstillingen 
av LED oppkoblingen. Orienteringen til LED-ene i figur 3 er lik den fysiske 
realiseringen, og er identisk til fremstillingen i firgur 1. I motsetning til skjemaet var ingen motstander brukt,
ettersom at målt strømforbruk av LED-ene var mellom 20-30 milliampere, som er ordinær driftsområde for slike LED-lys \cite{ledcurrent}.

\begin{figure}[H]
    \centering
    \includegraphics[width=0.65\textwidth]{Bilder/dice_sch.png}
    \caption{Skjematisk fremstilling av LED delen av kretsen.}    
\end{figure}


For realiseringen av systemet har Verilog blitt valgt for implementeringen på FPGA-en. Overordnet består implementeringen av tre komponenter:
en Dice modul, en Debounce modul og en Number\_to\_LED modul. Vi får da inndelingen som er vist i figur 4.
Koden ble kompilert i programmet Lattice iCECube2 med de medfølgende constraint 
filene til Go Board (se appendix A.1 og A.2).
\begin{figure}[H]
    \centering
    \includegraphics[width=0.65\textwidth]{Bilder/esda_flow_mod.jpg}
    \caption{Flytdiagram representasjon av systemet inndelt i moduler.}
\end{figure}


\subsection{Debounce}
Ettersom at hvert "kast" aktiveres av en knapp, og klokkehastigheten til Go Board brettet er $\SI{25}{\mega\hertz}$ \cite{goboard}, vil 
kretsen oppleve "bounce" i det digitale signalet fra knappen. Fenomenet er visualisert i figur 5. I en periode 
på ~$\SI{5}{\milli\s}$ \cite{bouncedur} svinger utgangssignalet fra knappen.
\begin{figure}[H]
    \centering
    \includegraphics[width=0.7\textwidth]{Bilder/Switch_Debounce.jpg}
    \caption{Visualisering av "bounce" i fysiske brytere \cite{bounce}}
\end{figure}
Implementeringen av systemet tar derfor høyde for dette ved å legge til en buffer-periode 
på $\SI{10}{\milli\s}$ før signalet fra knappen går inn i Dice modulen.

\begin{verilogcode}
    module Debounce_Switch
        (input i_Clk,
         input i_Switch,
         output o_Switch);

     parameter  c_DEBOUNCE_LIMIT = 250000; // 10ms ved 25 MHz
     reg        r_State = 1'b0;
     reg [17:0] r_Count = 0;

     always @(posedge i_Clk) begin
         if (i_Switch !== r_State && r_Count < c_DEBOUNCE_LIMIT)
             r_Count <= r_Count + 1; // Counter
         else if (r_Count == c_DEBOUNCE_LIMIT) begin
             r_Count <= 0;
             r_State <= i_Switch;
         end
         else
             r_Count <= 0;
     end
     assign o_Switch = r_State;
    endmodule
\end{verilogcode}

Dette er gjort ved å large en delay-teller som venter i 10 millisekund før den lar 
utgangssignalet være lik inngangssignalet.

\subsection{Number\_to\_LED}
For å visualisere resultatet av et "kast" må vi tilordne resultatet til aktive LEDs. 
Sannhetstabellen for tilordningen står i figur 6. $D_1, \dots, D_7$ refererer til LED-ene i kretsskjemaet i figur 3.

\begin{figure}[H]
    \centering
    \begin{tabular}{| c | c c c c c c c |} 
        \hline
        Resultat av kast & $D_1$ & $D_2$ & $D_3$ & $D_4$ & $D_5$ & $D_6$ & $D_7$ \\
        \hline
        1 & Av & Av & Av & Av & Av & Av & På \\ 
        \hline
        2 & På & Av & Av & Av & Av & På & Av \\ 
        \hline
        3 & På & Av & Av & Av & Av & På & På \\ 
        \hline
        4 & På & På & Av & Av & På & På & Av \\ 
        \hline
        5 & På & På & Av & Av & På & På & På \\ 
        \hline
        6 & På & På & På & På & På & På & Av \\ 
        \hline
    \end{tabular}
    \caption{Sannhetstabell for aktive LEDs for hvert resultat}
\end{figure}

Programmatisk kan vi gjøre tilordningen som et 7-bit tall r\_Dice, der hvert bit 
$b_i$ representerer om lysdiode $i$ skal være av eller på. Vi kan da for hvert resultat 
sette de resulterende bitverdiene i r\_Dice, og returnere dette tallet. 
\begin{verilogcode}
    module Number_To_LED
        (input i_Clk,
        input [2:0] i_Number,
        output [6:0] o_Dice);

     reg [6:0] r_Dice = 7'b0000000;

     always @(posedge i_Clk)
     begin
         case (i_Number) // Se bit-order i Designnotat 
             3'b001 : r_Dice = 7'b0000001;
             3'b010 : r_Dice = 7'b1000010;
             3'b011 : r_Dice = 7'b1000011;
             3'b100 : r_Dice = 7'b1100110;
             3'b101 : r_Dice = 7'b1100111;
             3'b110 : r_Dice = 7'b1111110;
         endcase
     end

     assign o_Dice = r_Dice;
    endmodule
\end{verilogcode}

\subsection{Dice}
For implementeringen av Dice modulen starter vi med å definere et register r\_Number\_Counter,
som representerer verdien til telleren, og et register r\_Number\_Output som definerer 
output registeret til lysene, samt registeret for outputverdien av Debounce modulen.
\begin{verilogcode}
 module Dice
    (input i_Clk, input i_Switch_1,
     output io_PMOD_1, output io_PMOD_2,
     output io_PMOD_3, output io_PMOD_4,
     output io_PMOD_7, output io_PMOD_8,
     output io_PMOD_9);

 wire       w_Switch_1;
 wire [6:0] r_Dice;
 reg        r_Switch_1 = 1'b0;
 reg  [2:0] r_Number_Counter = 3'b000;
 reg  [2:0] r_Number_Output = 3'b110;
   
\end{verilogcode}

\newpage
Modulen instansierer så Debounce modulen for å forhindre bounce på 
inngangssignalet fra knappen. 
\begin{verilogcode}
  Debounce_Switch Debounce_Switch_Inst
    (.i_Clk(i_Clk),
     .i_Switch(i_Switch_1),
     .o_Switch(w_Switch_1));
\end{verilogcode}

Så defineres telleren, som teller fra 1 til 6. Den benytter r\_Number\_Counter definert tidligere,
og itererer den med 1 hver gang klokkesignalet til FPGA-en går høyt.
\begin{verilogcode}
always @(posedge i_Clk)
begin
    if (r_Number_Counter >= 3'b110) begin
        r_Number_Counter = 3'b000;
    end
    r_Number_Counter = r_Number_Counter + 1;
    r_Switch_1 <= w_Switch_1;
     
    if (w_Switch_1 == 1'b0 && r_Switch_1 == 1'b1) begin // Høyt til lavt
        r_Number_Output <= r_Number_Counter;
    end
end
\end{verilogcode}
Så blir r\_Number\_Output omgjort til LED-output verdier med Number\_To\_LED modulen, som så blir tilordnet 
riktig PMOD utgang. Merk at LED1 får tilordet det mest segnifikante bittet, ettersom at 
modulen følger direkte fra tabellen i figur 6.
\begin{verilogcode}
 Number_To_LED Number_To_LED_Inst
    (.i_Clk(i_Clk),
     .i_Number(r_Number_Output),
     .o_Dice(r_Dice));

 assign io_PMOD_1 = r_Dice[6];
 assign io_PMOD_2 = r_Dice[5];
 assign io_PMOD_3 = r_Dice[4];
 assign io_PMOD_4 = r_Dice[3];
 assign io_PMOD_7 = r_Dice[2];
 assign io_PMOD_8 = r_Dice[1];
 assign io_PMOD_9 = r_Dice[0];
endmodule    
\end{verilogcode}

\section{Test}
Ved testing av terningen virder den som den skal. Når knappen slippes er et nytt tall generert,
og tallet vises riktig på LED-ene.
\begin{figure}[H]
    \centering
    % \includegraphics[width=0.45\textwidth]{Bilder/dice_3.jpg}
    \includegraphics[width=0.45\textwidth]{Bilder/dice_5.jpg}
    \includegraphics[width=0.45\textwidth]{Bilder/dice_6.jpg}
    \caption{Resultater fra "kast" av terning.}
\end{figure}

Målt strømmforbruk og effekt for hver av fargene i implementeringen ser man i figur 8.

\begin{figure}[H]
    \centering
    \begin{tabular}{| c | c | c | c |} 
        \hline
        \textbf{LED Farge} & \textbf{$V$} & \textbf{$I$} & \textbf{$W$} \\ 
        \hline
        Gul & $\SI{2.31}{\volt}$ & \SI{31.0}{\milli\ampere} & \SI{71.6}{\milli\watt} \\ 
        \hline
        Grønn & $\SI{2.41}{\volt}$ & \SI{28.5}{\milli\ampere} & \SI{68.7}{\milli\watt} \\ 
        \hline
        Rød & $\SI{2.41}{\volt}$ & \SI{28.5}{\milli\ampere} & \SI{69.8}{\milli\watt} \\ 
        \hline
    \end{tabular}
    \caption{Målt effekt for hver LED farge i implementeringen}
\end{figure}

Her er det antatt at alle portutgangene på PMOD-en gir samme spenningspotensial i.fht. ground,
og at alle LED-ene av samme type har omtrent samme effektforbruk, antatt avvik mindre enn $\SI{0.1}{\milli\watt}$.

Dersom vi med disse antakelsene i tillegg til antakelsen om at alle resultatene i et "kast" er like 
sannsynlige kan vi regne ut forventet effektforbruk til denne implementeringen av en elektronisk terning 
(unntatt strømforbruket til Go Board kortet).

\begin{alignat}{2}
    &E(1) = E(\text{Rød}) &&= \SI{69.8}{\milli\watt} \\
    &E(2) = 2 \cdot E(\text{Gul}) &&= \SI{143}{\milli\watt} \\
    &E(3) = 2 \cdot E(\text{Gul}) + E(\text{Rød}) &&= \SI{213}{\milli\watt} \\
    &E(4) = 4 \cdot E(\text{Gul}) &&= \SI{286}{\milli\watt} \\
    &E(5) = 4 \cdot E(\text{Gul}) + E(\text{Rød}) &&= \SI{356}{\milli\watt} \\
    &E(6) = 4 \cdot E(\text{Gul}) + 2 \cdot E(\text{Grønn}) &&= \SI{424}{\milli\watt}
\end{alignat}

Vi får da at forventet effektbruk for terningen når den viser resultatet blir 
\[
    E = \frac{1}{6} \cdot (E(1) + E(2) + E(3) + E(4) + E(5) + E(6)) \approx \SI{249}{\milli\watt} 
\]

Effektforbruket når terningen blir "kastet" vil i implementeringen over vil vise det forrige resultatet
av et "kast", evt. ingen aktive lys hvis ingen kast har blitt foretatt. Effektforbruket i et "kast" blir derfor 
identisk til effektforbruket når et resultat vises. Det er også mulig å sette r\_Number\_Output til 0 når terningen 
blir "kastet", evt. et annet mønster. I så fall blir effektforbruket $\SI{0}{\milli\watt}$. Dette ble ikke implementert her. 
