\documentclass[a4paper,11pt,norsk]{article}
\usepackage{packages}

\begin{document}

%Headingdel:---------------------------------------------
\topmargin -1.5cm
\makebox[\textwidth][s]{
    \begin{minipage}[c]{0.25\textwidth}
        \includegraphics[width=2.0cm]{Bilder/ntnu_logo.png}  
    \end{minipage}
    \begin{minipage}[c]{0.75\textwidth}
        \huge{\textbf{TTT4260 ESDA}} \\
        \Large{Øving 3  ---  Morten Sørensen, \large{\color{black!75!white}\today}}
    \end{minipage}
}
\vspace{0.75cm}
\normalsize


\section*{Oppgave 1.}
\begin{enumerate}
    \item Starter med å sette opp småsignalskjema til systemet 
        \begin{figure}[H]
            \centering
            \includegraphics[width=0.75\textwidth]{Bilder/1a.png}
            \caption{Småsignalskjema til systemet} 
        \end{figure}
        Finner så $R_i$:
        \begin{align*}
            R_i &= R_{G1} \mid\mid R_{G2} \\
                &= \frac{\SI{3}{\mega\ohm} \cdot \SI{1}{\mega\ohm}}{\SI{4}{\mega\ohm}} \\ 
                &= \SI{0.75}{\mega\ohm}
        \end{align*}
        $R_o$ blir da 
        \[
            R_o = R_D = \SI{2}{\kilo\ohm}
        \]
        Vi får videre at 
        \[
            V_2 = -i_{d}R_D = -g_m v_1 R_D
        \]
        Får til slutt at 
        \begin{align*}
            A &= \frac{V_2}{V_1} = -g_m R_D = -k\left(V_{GS} - V_T\right)R_D \\ 
              &= -32 \cdot \left(\SI{1.5}{\volt} - \SI{1}{\volt}\right) \cdot \SI{2}{\kilo\ohm} \\
              &= -32
        \end{align*}
    \item 
        Vi får at 
        \begin{figure}[H]
            \centering
            \includegraphics[width=0.75\textwidth]{Bilder/1b.png}
            \caption{Småsignalskjema til systemet} 
        \end{figure}
        \[
            V_2 = A v_1 \frac{R_L}{R_o + R_L} = -32 v_1 \cdot \frac{\SI{2}{\kilo\ohm}}{\SI{4}{\kilo\ohm}} = -16 v_1
        \]
        Altså halveres forsterkningen når $R_L$ kobles på.
        Øker vi lastmotstanden til 10 ganger original verdi får vi at
        \[
            V_2 = -32 v_1 \cdot \frac{\SI{20}{\kilo\ohm}}{\SI{22}{\kilo\ohm}} = -29 v_1
        \]
        altså er spenningsforsterkningen større med større lastmotstand.
    \item Skal finne spenningsforsterkningen $A_s$. Finner først $V_1$:
        \[
            V_1 = v_{sig}\frac{R_i}{R_i + R_sig}
        \]
        Finner så $A_s$:
        \begin{align*}
            A_s &= \frac{V_2}{v_{sig}} = \frac{g_m V_1 R_o}{v_{sig}} \\
                &= \frac{g_m R_i R_o}{R_i + R_o} \\
                &= \frac{-k \cdot (V_{GS} - V_T) \cdot R_i \cdot R_o}{R_i + R_o} \\
                &= \frac{-32 \cdot (\SI{1.5}{\volt} - \SI{1}{\volt}) \cdot \SI{0.75}{\mega\ohm} \cdot \SI{2}{\kilo\ohm}}{\SI{0.75}{\mega\ohm} + \SI{2}{\kilo\ohm}} \\
                &\approx -31.9 
        \end{align*}
        $A_s$ er tilnærmet lik $A$. Det er fordi det aller meste av spenningsfallet fra $v_{sig}$ ligger over $R_i$, og ikke $R_{sig}$.
\end{enumerate}

\end{document}
