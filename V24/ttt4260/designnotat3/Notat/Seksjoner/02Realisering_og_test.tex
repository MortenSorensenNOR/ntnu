\section{Realisering og test}
\label{realiseringOgTest}
For spektralanalysen har python blitt brukt. Følgende kode tar inn en .wav fil og kjører en fouriertransformasjon på lyd-dataen i filen. Programmet tar i bruk 
scipy for analyse og datainnlastning.

\begin{pythoncode}
import numpy as np
import matplotlib.pyplot as plt
import scipy

def plot_frequency_spectrum(wav_file):
    sample_rate, data = scipy.io.wavfile.read(wav_file)
    if len(data.shape) > 1:
        data = data[:,0]

    n = len(data)
    audio_fft = scipy.fft.fft(data)
    frequencies = np.linspace(0, sample_rate, n)
    magnitude = np.abs(audio_fft)[:n // 2] * 2 / n

    # Begrenser spektrumet til [0, ~2600]
    plt.figure(figsize=(12, 6))
    plt.plot(frequencies[:n // 16], magnitude[:n//16]) 
    plt.xlabel('Frequency (Hz)')
    plt.ylabel('Magnitude')
    plt.title('Frequency Spectrum')
    plt.grid(True)
    plt.plot()
\end{pythoncode}

Resultatet av å kjøre programmet over på lydfilen er plottet i figur 6.

\begin{figure}[H]
    \centering
    \includegraphics[width=0.8\textwidth]{Bilder/Lydsignal_82.png}
    \caption{Spektralanalyse av lydsignalet over hele tidsperioden til signalet}
\end{figure}

Vi ser et kraftig sprik i spektrumet ved $f = \SI{1480}{\hertz}$, som trolig er frekvensen til forstyrrelsen $w(t)$.

\subsection{RLC båndstopp}
\subsubsection{Realisering}
Når frekvensen er kjent kan vi regne ut komponentverdiene vi trenger for filteret. En spole med induktans målt til $\SI{107}{\milli\henry}$ er bitt valgt,
og vi kan da løse for kapasitansen $C$.

\[
    C = \frac{1}{(2\pi f_w)^2 L} \approx \frac{1}{(2\pi \SI{1480}{\hertz})^2 \SI{107}{\milli\henry}} \approx \SI{108}{\nano\farad}
\]

Den faktiske komponentverdien valgt var $\SI{110}{\nano\farad}$, og ble realisert med en $\SI{100}{\nano\farad}$ i parallell med en $\SI{10}{\nano\farad}$ kondensator.
Velger så et $\SI{10}{\kilo\ohm}$ potensiometer for $R$.

For to RLC filtre i serie har en annen spole målt til $\SI{104}{\milli\henry}$ blitt brukt. Samme kapasitansverdier er blitt valgt for realiseringen. Det samme gjelder 
$\SI{10}{\kilo\ohm}$ potensiometeret.

Op-ampen brukt i bufferet er en Texas Instruments LF353P wide-bandwidth op-amp \cite{lf353_opamp} blitt brukt. LF353 op-apmen godtar forsyningsspenniger mellom 
$3.3-18$V på $V_{CC+}$ og $-3.3-18$V på $V_{CC-}$. 5V og -5V fra en Analog Discovery 2 har blitt brukt.

Både RLC filteret og kaskadefilteret er vist implementert i figur 7.

\begin{figure}[H]
    \begin{subfigure}[c]{.49\linewidth}
        \includegraphics[width=\textwidth]{Bilder/filter_realisering1.jpg}
    \end{subfigure}
    \hfill
    \begin{subfigure}[c]{.49\linewidth}
        \includegraphics[width=\textwidth]{Bilder/filter_realisering2.jpg}
    \end{subfigure}
    \caption{Realisering av et RLC filter}
\end{figure}

\subsubsection{Test}
For testing av filteret ble en Analog Discovery 2 brukt. Den har mulighet til både å måle frekvensresponsen til systemet og muligheten til å spille av en lydfil og 
kjøre spektralanalyse på utgangen fra systemet.

Responsen til systemet ble målt ved forskjellige motstandsverdier $R_{pot}$, ved lav, $\sim \SI{1}{\kilo\ohm}$, halv, $\sim \SI{5}{\kilo\ohm}$ og maks motstand, $\sim \SI{10}{\kilo\ohm}$.
Kaskadefilteret ble så målt ved begge potensiometerene ved $R = \SI{5}{\kilo\ohm}$ for sammenligning med kunn et RLC filter.
\begin{figure}[H]
    \centering
    \includegraphics[width=0.90\textwidth]{Bilder/spectrum_filter_low.png}
    \caption{Frekvensrespons for RLC filteret ved lav potensiometer verdi}
\end{figure}

\begin{figure}[H]
    \centering
    \includegraphics[width=0.90\textwidth]{Bilder/spectrum_filter_half.png}
    \caption{Frekvensrespons for RLC filteret ved halv potensiometer verdi, $\sim \SI{5}{\kilo\ohm}$}
\end{figure}

\begin{figure}[H]
    \centering
    \includegraphics[width=0.90\textwidth]{Bilder/spectrum_filter_max.png}
    \caption{Frekvensrespons for RLC filteret ved maks potensiometer verdi, $\sim \SI{10}{\kilo\ohm}$}
\end{figure}

\begin{figure}[H]
    \centering
    \includegraphics[width=0.90\textwidth]{Bilder/spectrum_filter2.png}
    \caption{Frekvensrespons for to RLC filtre i serie ved halv potensiometer verdi, hver $\sim \SI{5}{\kilo\ohm}$}
\end{figure}

Vi observerer at systemet oppnår maksimal amplituderespons ved $f_w$ som ønsket, men at filteret ved $R = \SI{10}{\kilo\ohm}$ har en båndbredde 
på $\sim \SI{20}{\kilo\ohm}$, fra omlag $\SI{100}{\hertz}$ til $\SI{20}{\kilo\hertz}$, altså omtrent lik hele bredden på spekteret vi er interessert i.
Dette er åpenbart ikke ideelt. Ved kaskadekobling av to RLC filtre ser vi at båndbredden blir mindre, altså er det tenkelig at å legge til flere 
filtre i kaskadekoblingen vil føre til et bedre resulterende filter for vår applikasjon.

Når vi observerer spektrumet til systemet når lydsignalet $x(t)$ blir sendt gjennom kan vi observere at systemet klarer å dempe forstyrrelsen $f_w$ ved $\SI{1480}{\hertz}$,
og at kaskadekoblingen demper de omliggende frekvensen mindre enn det enestående RLC filteret gjør, altså at $\hat{s}(t)$ beholder mer av det opprinnelige signalet $s(t)$.

\begin{figure}[H]
    \centering
    \includegraphics[width=0.90\textwidth]{Bilder/spectrum_analysis_no_filter.png}
    \caption{Spektralanalyse av utgangssignalet $\hat{s}(t)$ i blått og inngangssignalet $x(t)$ i gult ved ingen filtrering}
\end{figure}

\begin{figure}[H]
    \centering
    \includegraphics[width=0.90\textwidth]{Bilder/spectrum_analysis_filter1.png}
    \caption{Spektralanalyse av utgangssignalet $\hat{s}(t)$ i blått og inngangssignalet $x(t)$ i gult filtrert gjennom et RLC filter}
\end{figure}

\begin{figure}[H]
    \centering
    \includegraphics[width=0.90\textwidth]{Bilder/spectrum_analysis_filter2.png}
    \caption{Spektralanalyse av utgangssignalet $\hat{s}(t)$ i blått og inngangssignalet $x(t)$ i gult filtrert gjennom to RLC filter i serie med buffer}
\end{figure}

En mulig løsning for å forhindre den store båndbredden er ved å enten kaskadekoble flere filtre, eller eventulet benytte et aktivt filter, ettersom at de kan kompansere 
for uønsket dempning av de andre frekvensene i spekteret.

\subsection{Python IIR filter}
Det digitale IIR filteret har også blitt implementert ved hjelp av scipy sitt signals bibliotek. Følgende kode sette opp et IIR notch filter ved en gitt notch frekvens,
kvalitetsfaktor og samplerate, og benytter lfilter funksjonen for å påføre notchfilteret på innputt signalet $x(t)$.
\begin{pythoncode}
import numpy as np
from scipy.io import wavfile
from scipy.signal import iirnotch, lfilter

b, a = iirnotch(notch_freq, quality_factor, sample_rate)
filtered_data = np.apply_along_axis( \ 
                    lambda d: lfilter(b, a, d), axis=0, arr=data)
\end{pythoncode}

Spektralanalysen av det filtrerte signalet $\hat{s}(t)$ plottet viser 
at magnituden til spektrumet ved $f = \SI{1480}{\hertz}$ er blitt satt til 0, og vi har derfor til en mye større grad enn de analoge filtrene oppnådd den ønskede filter, karakteristikken.
\begin{figure}[H]
    \centering
    \includegraphics[width=0.90\textwidth]{Bilder/Filtered_spectrum.png}
    \caption{Spektralanalyze av det filtrerte lydklippet}
\end{figure}
